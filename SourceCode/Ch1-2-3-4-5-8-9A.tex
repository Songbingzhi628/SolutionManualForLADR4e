% Copyright (C) 2024 Songbingzhi628. This work is licensed under Creative Commons Attribution-NonCommercial-ShareAlike 4.0 International License.
% Email: 13012057210@163.com

\ChDecl{Ch1B}{1$\cdot$B}{}

\vspace{2pt}

%\ProblemN{\Anchor{1B1}{1}}{
%	\vspace{2pt}\TextA{Prove $\forall v\in V,-\Par{{-v}} = v$.}
%}$-\Par{{-v}}=\Par{{-1}}\BigPar{\Par{{-1}}v}=\BigPar{\Par{{-1}}\Par{{-1}}}v=1\cdot v=v.$\parSol{\vspace{4pt}}
%\Or Becs $-\Par{{-v}}+\Par{{-v}}=0$ 又 $v+\Par{{-v}}=0.$ Now by the uniqnes of add inv.\PfEnd
%\SepLine
%
%\ProblemN{\Anchor{1B2}{2}}{
%	\TextA{Supp $a\in\Fbb,v\in V$, and $av = 0$. Prove $a = 0$ or $v = 0$.}
%}Supp $a\neq 0,\,\exists\,a^{-1}\in\Fbb,\,a^{-1}a=1,$ hence $v=1\cdot v=\Par{a^{-1}a}v=a^{-1}\Par{av}=a^{-1}\cdot 0=0.$\PfEnd
%\SepLine
%
%\ProblemN{\Anchor{1B3}{3}}{
%	\TextA{Supp $v, w\in V$. Explain why $\exists\,!\,x\in V,v + 3x = w.$}
%}$v + 3x = w \;\Leftrightarrow\; 3x = w − v \;\Leftrightarrow\; x = \frac{\;1\;}{3}\Par{w-v}.$\PfEnd\vspace{4pt}\quad
%\Or \hspace{1pt}\;{\Existns} \;Let $x=\Frac{\;1\;}{3}\Par{w-v}.$\vspace{2pt}\par\quad
%\Blind{\Or}{\Uniqnes} \;If $v+3x_1=w,$(I)$\,\,v+3x_2=w\,$(II). Then (I) $-$ (II) $: 3\Par{x_1-x_2}=0\Rightarrow x_1=x_2.$\PfEnd
%\SepLine

%\Anchor{1B5}\ProblemN[]{5}{
%	\TextA{Show in the def of a vecsp, the add inv cond can be replaced by [1.29].}
%	\TextA{\large{\tgsc Hint\tgbf:} Supp $V$ satisfies all conds in the def, except we've replaced the add inv cond with [1.29].}
%	\TextA{\large\Blind{{\tgsc Hint\tgbf:}} Prove the add inv is true.\quad\tgnr $0v=0\Longleftrightarrow\Sbra{1+\Par{{-1}}}v=1\cdot v+\Par{{-1}}v=v+\Par{{-v}}=0,$ by [1.31].\vspace{-3pt}}
%}\SepLine

%\ProblemN{\Anchor{1B6}{6}}{
%	\TextA{Let $\infty$ and $-\infty$ denote two disti objects that are not in $\Rbb$.}
%	\TextA{Define the natural add and scalar multi on $\Rbb\cup\Bra{\infty, -\infty},$ that is, for each $t\in\Rbb,$\vspace{4pt}}
%	\TextA{\;\FontNorm$t\infty=\hMath{l}{\left\{}{\;\right|}{
%		-\infty\,$ if $t<0,\\
%		\Blind{-}0\,$\,\, if $t=0,\\
%		\hfill\infty\,\,\,$if $t>0,}\,\,\,t\Par{{-\infty}}=\hMath{l}{\left\{}{\;\right|}{
%		-\infty\,$ if $t>0,\\
%		\Blind{-}0\,$\,\, if $t=0,\\
%		\hfill\infty\,\,\,$if $t<0,}$\tgnr$\:\hText{$
%		(a) $t + \infty = \infty + t = \infty + \infty = \infty,\\$
%		(b) $t + \Par{{-\infty}} = \Par{{-\infty}} + t = \Par{{-\infty}} + \Par{{-\infty}} = -\infty,\\$
%		(c) $\infty + \Par{{-\infty}} = \Par{{-\infty}} + \infty = 0.}$\vspace{4pt}}
%	\TextA{Is $\Rbb\cup\Bra{\infty, -\infty}$ a vecsp over $\Rbb$? Explain.}
%}No. Becs the add and scalar multi is not assoc and distr.\parSol{}
%By Assoc: $\Par{a+\infty}+\Par{{-\infty}}\neq a+\Par{\infty+\Par{{-\infty}}}.$\parSol{}
%\Or By Distr: $\infty=\Par{2+\Par{{-1}}}\infty\neq 2\infty+\Par{{-\infty}}=\infty+\Par{{-\infty}}=0.$\PfEnd
%\SepLine

\Anchor{1BN1}\ProblemBX[]{\NoteFor{Fields}}{
	\TextB{Many choices.\;\;\FontNorm\Sbra[3pt]{{\tgsl Req Multi Inv Uniq}}\vspace{2pt}}
	\TextB{\AExa $\Zbb_m=\Bra{K_0,K_1,\dots,K_{m-1}}$ is a field $\Longleftrightarrow m\in\Nbp$ is a prime.}
}\SepLine

%\ProblemB{
%	\TextB{Let $\mS$ be the collec of all subsets of $\Fbb.$ Define the add and scalar multi on $\mS$ by}
%	\TextB{$A+B=\Bra{a+b:a\in A,b\in B},\lambda A=\Bra{\lambda a:a\in A},$ and $A+\emptySet=\emptySet=A,\lambda\emptySet=\emptySet.$}
%	\TextB{Is $\mS$ a vecsp over $\Fbb$? Which req in [1.19] does $\mS$ not hold ?}
%}
%\SepLine

\Anchor{1B4e7}\ProblemBnoor{4E 1.B.7}{
	\TextB{Supp $V\neq\varnothing$ and $W$ is a vecsp. Let $W^V=\Bra{\:f:V\rightarrow W}.$\vspace{1pt}}
	(a) \TextB{Define a natural add and scalar multi on $W^V.$ \;{\tgnr\large(b)} Prove $W^V$ is a vecsp with these defs.}
}\par\quad
(a) $W^V\ni f+g: x\rightarrow f\Par{x}+g\Par{x};$ where $f\Par{x}+g\Par{x}$ is the vec add on $W.$\par\quad\Ha
$W^V\ni\lambda f: x\rightarrow\lambda f\Par{x};$ where $\lambda f\Par{x}$ is the scalar multi on $W.$\par\quad
(b) Commu, Assoc, Distr are omitted.\par\quad\Hb
%Commu: $\Par{\,f+g}\Par{x}=f\Par{x}+g\Par{x}=g\Par{x}+f\Par{x}=\Par{g+f}\Par{x}.$\par\quad\Hb
%Assoc: $\BigPar{\Par{\,f+g}+h}\Par{x}=\BigPar{\,{f}\Par{x}+{g}\Par{x}}+{h}\Par{x}$\par\quad\Hb
%\Blind{Assoc: $\BigPar{\Par{\,f+g}+h}\Par{x}$} $={f}\Par{x}+\BigPar{{g}\Par{x}+{h}\Par{x}}=\BigPar{\,f+\Par{g+h}}\Par{x}.$\par\quad\Hb
%Add Id: $\Par{\,f+0}\Par{x}=f\Par{x}+0\Par{x}=f\Par{x}+0=f\Par{x}.$\par\quad\Hb
Add Inv: $\Par{\,f+g}\Par{x}={f}\Par{x}+g\Par{x}=f\Par{x}+\BigPar{{-f\Par{x}}}=0=0\Par{x}.$\par\quad\Hb
Multi Id: $\Par{1f}\Par{x}=1f\Par{x}=f\Par{x}.$\par\quad\Hb
%Distr: $\BigPar{a\Par{f+g}}\Par{x}=a\Par{f+g}\Par{x}=a\BigPar{f\Par{x}+g\Par{x}}$\par\quad\Hb
%\Blind{Distr: $\BigPar{a\Par{f+g}}\Par{x}=a\Par{f+g}\Par{x}$} $=a\,f\Par{x}+ag\Par{x}=\Par{a\,f}\Par{x}+\Par{ag}\Par{x}=\Par{a\,f+ag}\Par{x}.$\par\quad\Hb
%\Blind{Distr: }Simlr, $\BigPar{\Par{a+b}f}\Par{x}=\Par{a\,f+b\,f}\Par{x}.$\par\quad\Hb
We must have used the same properties in $W.$ \;\Sbra{{\tgsc If $W^V$ is a vecsp, then $W$ must be a vecsp.}}\PfEnd
\SepLine


\vfill\ChDecl{Ch1C}{1$\cdot$C}{\quad{\small\textbf{注意\,:}\;这里我将3.E积空间的定义前置;仅涉及概念。}}

\vspace{2pt}

\Anchor{1CNE5}\BulletPointX\NoteForSmall{Exe (5)}\;\;$\Cbb=\Rbb\oplus\Bra{c\i:c\in\Rbb}=\Bra{a+b\i:a,b\in\Rbb}$ if we let $\Fbb=\Rbb$ and $\i^2=-1.$\vspace{-4pt}
\SepLine

\Anchor{1CNE6}\BulletPointX\NoteForSmall{Exe (6)}\;\;Supp $V$ is a vecsp over $\Rbb.$ Then $V$ is not a vecsp over $\Cbb.$\vspace{-4pt}
%\BulletPointX\AComm Supp $V$ is a vecsp over $\Cbb$ of dim $n.$ Then $V$ is also a vecsp over $\Rbb$ of dim $2n.$
\SepLine

%\ProblemN[]{\Anchor{1C7}\Anchor{1C8}{7, 8}}{
%	\TextA{Give a non-trivial $U\subseteq\Rbb^2,$ $U$ is}
%	\PrePa\TextA{closd taking add invs and add, but is not a subsp of $\Rbb^2.$\hfill\tgnr\FontNorm Let $U=\Zbb^2$ or $\Qbb^2,$ with $0\in U.$}
%	\PrePb\TextA{scalar multi, but is not a subsp of $\Rbb^2.$\hfill\tgnr\FontNorm Let $U=\Bra{\Par{x,y}\in\Rbb^2:x=0\vee y=0}.$}
%}\SepLine

\BulletPointX\Largesl{Supp $U,W,V_{\!1},V_{\!2},V_{\!3}$ are subsps of $V.$}\par
\Anchor{1C15}\Anchor{1C16}\Onumber{15}\quad{$U+U\ni u+w\in U.$} \qquad \Onumber{16}\quad{$U+W\ni u+w=w+u\in W+U.$}\PfEnd
\Onumber{\Anchor{1C17}{17}}\quad{$\Par{V_{\!1}+V_{\!2}}+V_{\!3}\ni\Par{v_1+v_2}+v_3=v_1+\Par{v_2+v_3}\in V_{\!1}+\Par{V_{\!2}+V_{\!3}}.$}\PfEnd\Anchor{1C'1}
\BulletPointX{ \quad}{$\Par{U+W}{_{\Cbb}}\ni \Par{u_1+w_1}+\i\Par{u_2+w_2}=\Par{u_1+\i u_2}+\Par{w_1+\i w_2}\in U_\Cbb+W_{\!\Cbb}.$}\PfEnd
\BulletPointX{ \quad}$\Par{U\cap W}{_\Cbb}\ni u_1+\i u_2=w_1+\i w_2\in U_\Cbb\cap W_\Cbb.$\PfEnd
\BulletPointX{ \quad}$U_\Cbb=W_\Cbb\Longleftrightarrow U=W.$ \;Supp $U_\Cbb\ni u+\i v\in W_\Cbb.$ Then $U\ni u,v\in W.$\PfEnd
\BulletPointX{ \quad}${V_{\!1}}_\Cbb\times\cdots\times{V_{\!m}}_\Cbb=\Par{V_{\!1}\times\cdots\times V_{\!m}}{_\Cbb}.$\PfEnd
\SepLine

\ProblemN{\Anchor{1C18}{18}}{
	\TextA{Does the add on the subsps of $V$ have an add id? Which subsps have add invs?}
}Supp $\Omega$ is the uniq add id.\par\quad
(a) For any subsp $U$ of $V,$ $\Omega\subseteq U+\Omega=U\Rightarrow\Omega\subseteq U$. Let $U=\zeroSubs,$ then $\Omega=\zeroSubs.$\par\quad
(b) Supp $U+W=\Omega.$ Becs $U+W\supseteq U,W\Rightarrow \Omega\supseteq U,W\Rightarrow U=W=\Omega=\zeroSubs.$\PfEnd
\SepLine

%\ProblemN{\Anchor{1C11}{11}}{
%	\TextA{Prove the intersec of every collec of subsps of $V$ is a subsp of $V$.}
%}Supp $\Bra{U_{\alpha}}{_{\alpha\in\Gamma}}$ is a collec of subsps of $V$; here $\Gamma$ is an index set.\vspace{2pt}\parSol{}
%We show $\bigcap_{\alpha\in\Gamma}U_\alpha,$ which equals the set of vecs in each $U_\alpha,$ is a subsp of $V$.\vspace{2pt}\parSol{}
%(a) $0\in\bigcap_{\alpha\in\Gamma}U_\alpha.$ Nonempty.\parSol{}
%(b) $u,v\in\bigcap_{\alpha\in\Gamma}U_\alpha\Rightarrow u+v\in U_\alpha,\,\,\forall\alpha\in\Gamma\Rightarrow u+v\in\bigcap_{\alpha\in\Gamma}U_\alpha$. Closd add.\parSol{}
%(c) $u\in\bigcap_{\alpha\in\Gamma}U_\alpha,\lambda\in\Fbb\Rightarrow\lambda u\in U_\alpha,\,\,\forall\alpha\in\Gamma\Rightarrow\lambda u\in\bigcap_{\alpha\in\Gamma}U_\alpha$. Closd scalar multi.\vspace{4pt}\parSol{}
%Thus $\bigcap_{\alpha\in\Gamma}U_\alpha$ is nonempty subset of $V$ that is closd add and scalar multi.\PfEnd
%\SepLine

\Anchor{1CN1.45}\BulletPointX\NoteForSmall{[1.45]}\;\;Another proof: Supp $\forall v\in V,\exists\,!\,\Par{u,w}\in U\times W,\,v=u+w.$\TextB{}
\Blind{\NoteForSmall{[1.45]}\;\;}Asum non0 $v\in U\cap W.$ Then the $\Par{u,w}$ can be $\Par{v,0}$ or $\Par{0,v},$ ctradic the uniqnes.\vspace{-2pt}
\SepLine

%\Anchor{1CT1}\ProblemBX[]{\TipsN{1}}{
%	Supp $U,W\subseteq V.$ And $U,W,V$ are vecsps $\Rightarrow U,W$ are subsps of $V.$\TextA{}
%	Then $U+W$ is also a subsp of $V.$ Becs $\forall u\in U,w\in U,u+w\in V$ since $u,w\in V.$\TextA{\vspace{-3pt}}
%}\SepLine

%\Anchor{1CN1.38}\ProblemB{
%	\TextB{Supp $U=\Bra{\Par{x,x,y,y}},W=\Bra{\Par{x,x,x,y}}\subseteq\FbbP{4}.$ Prove $U+W=\Bra{\Par{x,x,y,z}}.$\vspace{2pt}}
%}Let $T$ denote $\Bra{\Par{x,x,y,z}}.$ By def, $U+W\subseteq T.$\par\quad
%And $T\ni\Par{x,x,y,z}\Rightarrow\Par{0,0,y-x,y-x}+\Par{x,x,x,-y+x+z}\in U+W$. Hence $T\subseteq U+W.$\PfEnd
%\SepLine

%\ProblemN{\Anchor{1C23}{23}}{
%	\TextA{Give an exa of vecsps $V_{\!1},V_{\!2},U$ suth $V_{\!1}\oplus U=V_{\!2}\oplus U$, but $V_{\!1}\neq V_{\!2}$.\vspace{0pt}}
%}$V=\FbbP{2}$,  $U=\Bra{\Par{x,x}}$, $V_{\!1}=\Bra{\Par{x,0}}$, $V_{\!2}=\Bra{\Par{0,x}}$.\par
%\SepLine

%\ProblemN{\Anchor{1C21}{21}}{
%	\TextA{Supp $U=\Bra{\Par{x,y,x+y,x-y,2x}}$. Find a $W$ suth $\FbbP{5}=U\oplus W$.}
%}Let $W=\Bra{\Par{0,0,z,w,u}}$. Then $U\cap W=\zeroSubs$.\parSol{}
%And $\FbbP{5}\ni\Par{x,y,z,w,u}\Rightarrow\Par{x,y,x+y,x-y,2x}+\Par{0,0,z-x-y,w-x-y,u-2x}\in U+W.$\par
%\SepLine
%
%\ProblemN{\Anchor{1C22}{22}}{
%	\TextA{Supp $U=\Bra{\Par{x,y,x+y,x-y,2x}\in\FbbP{5}}$.}
%	\TextA{Find non0 subsps $W_1,W_2,W_3$ of $\,\FbbP{5}$ suth $\FbbP{5} = U\oplus W_1\oplus W_2\oplus W_3 $.}\vspace{-2pt}
%}\par\quad
%Let $W_1=\Bra{\Par{0,0,z,0,0}\in\FbbP{5}}\Rightarrow W_1\cap U=\zeroSubs.$ \hfill Now $U\oplus W_1=\Bra{\Par{x,y,z,x-y,2x}\in\FbbP{5}}=U_1$.\par\quad
%Let $W_2=\Bra{\Par{0,0,0,w,0}\in\FbbP{5}}\Rightarrow W_2\cap U_1=\zeroSubs.$ \hfill Now $U_1\oplus W_2=\Bra{\Par{x,y,z,w,2x}\in\FbbP{5}}=U_2$.\par\quad
%Let $W_3=\Bra{\Par{0,0,0,0,u}\in\FbbP{5}}\Rightarrow W_3\cap U_2=\zeroSubs.$ \hfill Now $U_2\oplus W_3=\Bra{\Par{x,y,z,w,u}\in\FbbP{5}}=U_3$.\par\quad
%Thus $\FbbP{5}=\Sbra{\Par{U\oplus W_1}\oplus W_2}\oplus W_3.$\PfEnd
%\SepLine

\Anchor{1CNC}\BulletPointX\NoteForSmall{\Large $“\complement_V U \cup \zeroSubs”$}\;\;$“\complement_V U \cup \zeroSubs”$ is supposed to be a subsp $W$ suth $V=U\oplus W$.\TextB{\vspace{3pt}}
But if we let $u\in U\nonzero$ and $w\in W\nonzero$, then $\MathRightBrace{l}{w\in\complement_V U \cup \zeroSubs\\ u\pm w\in\complement_V U \cup \zeroSubs }\Rightarrow u\in\complement_V U \cup \zeroSubs.$ Ctradic.\vspace{3pt}\TextB{}
To fix this, {\FontLarge denote the set $\Bra{W_{\hspace{-1pt}1},W_{\hspace{-1pt}2},\cdots}$ by $\Scom{V}{U}$,} {where each $W_{\!i}\oplus U=V.$}
\SepLine

\Anchor{1C'2}\ProblemB[]{
	\TextB{Supp $V_{\!1},V_{\!2},U_1,U_2$ are vecsps, $V_{\!1}\oplus U_1=V_{\!2}\oplus U_2,\;V_{\!1}\subseteq V_{\!2},\;U_2\subseteq U_1.$}
	\TextB{Give a countexa\hspace{1pt}$:$ $V_{\!1}=V_{\!2},\;U_1=U_2.$\qquad\FontNorm\tgnr Let $U_2=\zeroSubs\Rightarrow V_2=V_1\oplus U_1.$}
}\par\vspace{-38pt}\quad
\hspace{450pt}\includegraphics[width=2.4cm,height=1.2cm,scale=0.22]{diagram1C-1.png}\vspace{-2pt}
\SepLine\pagebreak

\Anchor{1CT1}\ProblemBX{\TipsN{1}}{
	\TextA{Supp $V_{\!1}\subseteq V_{\!2}$ and $V_{\!1}\oplus U=V_{\!2}\oplus U.$ Prove $V_{\!1}=V_{\!2}.$}
}Becs the subset $V_{\!1}$ of vecsp $V_{\!2}$ is closd add and scalar multi, $V_{\!1}$ is a subspace of $V_{\!2}.$\parSol{}
Supp $W$ is suth $V_{\!2}=V_{\!1}\oplus W.$ Now $V_{\!2}\oplus U=\Par{V_{\!1}\oplus W}\oplus U=\Par{V_{\!1}\oplus U}\oplus W=V_{\!1}\oplus U.$\parSol{}
If $W\neq\zeroSubs,$ then $V_{\!1}\oplus U\subsetneq\Par{V_{\!1}\oplus U}\oplus W$, ctradic. Hence $W=\zeroSubs,V_{\!1}=V_{\!2}.$\PfEnd
\SepLine

%\ProblemB{
%	\TextA{Supp the intersec of any two of the vecsps $U,W,X,Y$ is $\zeroSubs.$}
%	\TextA{Give an exa that $\BigPar{X\oplus U}\,${\LARGE$\cap$}$\,\BigPar{Y\oplus W}\neq\zeroSubs.$}
%}\Sbra{ Using notats in Chapter 2. } \;Let $B_X=\Par{e_1},B_U=\Par{e_2-e_1},B_Y=\Par{\,},B_W=\Par{e_2}.$
%\SepLine

\Anchor{1CT2}\ProblemBX{\TipsN{2}}{
	\TextA{Supp $V=X\oplus Y,$ and $Z$ is a subsp of $V.$ Show $X\subseteq Z\Rightarrow Z=X\oplus\Par{Y\cap Z}.$}
}$\forall z\in Z,\exists\,!\,\Par{x,y}\in X\times Y,\,z=x+y.$\parSol{}
Becs $x\in Z\Rightarrow z-x=y\in Z\Rightarrow z\in X+\Par{Y\cap Z}.$ 又 $X\cap\Par{Y\cap Z}\subseteq X\cap Y.$\PfEnd
\SepLine

\Anchor{1CT3}\ProblemBX{\TipsN{3}}{
	\TextA{Let $V=U+W,\;I=U\cap W,\,U=I\oplus X,\,W=I\oplus Y.$ Prove $V=I\oplus\Par{X\oplus Y}.$}
}We show $X\cap Y=U\cap Y=W\cap X=\zeroSubs$ by ctradic.\parSol{}
$X\cap Y=\Delta\neq\zeroSubs\Rightarrow I=U\cap W\supseteq\Delta\Rightarrow I\cap X\neq\zeroSubs, I\cap Y\neq\zeroSubs.$\parSol{}
$U\cap Y=\Delta\neq\zeroSubs\Rightarrow I=U\cap W\supseteq\Delta\Rightarrow I\cap Y\neq\zeroSubs.$ Simlr for $W\cap X.$\parSol{\vspace{2pt}}
Thus $I+\Par{X+Y}=\Par{I\oplus X}\oplus Y=I\oplus\Par{X\oplus Y}.$\parSol{\vspace{2pt}}
Now we show $V=I+\Par{X+Y}.$ \;$\forall v\in V,v=u+w,\exists\,\Par{u,w}\in U\times W$\parSol{}
$\Rightarrow\exists\,\Par{i_u,x_u}\in I\times X,\,\Par{i_w,y_w}\in I\times Y,\;v=\Par{i_u+i_w}+x_u+y_w\in I+\Par{X+Y}.$\PfEnd
\SepLine

\ProblemN{\Anchor{1C12}{12}}{
	\TextA{Supp $U,W$ are subsps of $V.$ Prove $U\cup W$ is a subsp of $V\Longleftrightarrow U\subseteq W$ or $W\subseteq U$.}
}(a) Supp $U\subseteq W$. Then $U\cup W=W$ is a subsp of $V$.\parSol{}
(b) Supp $U\cup W$ is a subsp of $V$. Asum $U\not\subseteq W,\;U\not\supseteq W$ \BigPar{ $U\cup W\neq U$ and $W$ }.\parSol{\Hb}
Then $\forall a\in U\wedge a\not\in W,\forall b\in W\wedge b\not\in U,$ we have $a+b\in U\cup W$.\parSol{\vspace{2pt}\Hb}
\!\!\!$\MathRightMid{l}{$
$a+b\in U\Rightarrow b=\Par{a+b}+\Par{{-a}}\in U$, ctradic $\Rightarrow W\subseteq U.\\ $
$a+b\in W\Rightarrow a=\Par{a+b}+\Par{{-b}}\in W$, ctradic $\Rightarrow U\subseteq W.}\hText{$
Ctradic asum.$\\\;}$\PfEnd[-14pt]
\SepLine

\ProblemN{\Anchor{1C13}{13}}{
	\TextA{Supp $U_1,U_2,U_3$ are subsps of $V,$ and the union $U_1\cup U_2\cup U_3=\mathcal{U}$ is a subsp of $V.$}
	\TextA{Prove one of the subsps contains the other two.}
	\TextA{\large This exe is not true if we \uline{replace $\Fbb$ with a field containing only two elems.}}
}\uline{\AExa Let $\Fbb=\Zbb_2.$} $U_1=\Bra{u,0},U_2=\Bra{v,0},U_3=\Bra{v+u,0}.$ While $\mathcal{U}=\Bra{0,u,v,v+u}$ is a subsp.\vspace{4pt}\par\quad
\NOTICE that, $U\cup W=V$ is vecsp $\notRightarrow U,W$ are subsps of $V.$\par\quad
This trick is invalid: $\Par{A\cup B}\cup\Par{B\cup C}=\Par{A\cup C}\cup\Par{B\cup C}=\Par{A\cup B}\cup\Par{A\cup C}.$\vspace{2pt}\par\quad
(I) If any $U_j$ is contained in the union of the other two, say $U_1\subseteq U_2\cup U_3,$ then $\mathcal{U}=U_2\cup U_3.$\par\quad\HI
By applying Exe (12) we conclude that one $U_j$ contains the other two. Thus done.\vspace{2pt}\par\quad\EndI
(II) \envFontLarge{Asum no one is contained in the union of other two, and no one contains the other two.}\par\quad\HII
{\vspace{4pt}Say $U_1\not\subseteq U_2\cup U_3$ and $U_1\not\supseteq U_2\cup U_3.$}\par\quad
{\Large\vspace{4pt}$\exists\,u\in U_1\wedge u\not\in U_2\cup U_3;\;v\in U_2\cup U_3\wedge v\not\in U_1.$ Let $W=\Bra{v+\lambda u:\lambda\in\Fbb}\,\subseteq\mathcal{U}.$}\par\quad
{\Large\vspace{4pt}Note that $W\cap U_1=\emptySet,$ for if any $v+\lambda u\in W\cap U_1$ then $v+\lambda u-\lambda u=v\in U_1$.}\par\quad
{\Large\vspace{4pt}Now $W\subseteq U_1\cup U_2\cup U_3\Rightarrow W\subseteq U_2\cup U_3.$ $\forall v+\lambda u\in W,v+\lambda u\in U_i,i=2,3.$}\par\quad
{\Large\vspace{4pt}If $U_2\subseteq U_3$ or $U_2\supseteq U_3,$ then $\mathcal{U}=U_1\cup U_i,i=2,3.$} {By Exe (12) done.}\par\quad
{\Large\vspace{4pt}Othws, {\FontNorm both $U_2,U_3\neq\zeroSubs.$} Becs \tgsl$W\subseteq U_2\cup U_3$ has at least three disti elems.}\par\quad
{\Large\vspace{4pt}There must be some $U_i$ that contains at least two disti elems of $W.$}\par\quad
{\Large$\exists\,\lambda_1\neq\lambda_2,\;v+\lambda_1 u$ \,and \,$v+\lambda_2 u$ \,both in $U_2$ or $U_3\Rightarrow u\in U_2\cap U_3,$ ctradic.}\FontNorm\PfEnd
\SepLine

%\ChEnd\pagebreak

\ChDecl{Ch2A}{2$\cdot$A}{}

\vspace{6pt}

\ProblemN{\Anchor{2A1}{1}}{
	\TextA{Prove [P] $\Par{v_1,v_2,v_3,v_4}$ spans $V\Longleftrightarrow\BigPar{v_1-v_2,v_2-v_3,v_3-v_4,v_4}$ also spans V [Q].}
}Note that $V=\Span{v_1,\dots,v_n}\Longleftrightarrow\forall v\in V,\exists\,a_1,\dots,a_n\in\Fbb,v=a_1v_1+\dots+a_nv_n.$\par\quad
Asum $\forall v\in V,\exists\,a_1,\dots,a_4,b_1,\dots,b_4\in\Fbb,$ \BigPar{ that is, if $\exists\,a_i,$ then we are to find $b_i,$ vice versa }\par\quad
$v=a_1 v_1+a_2 v_2+a_3 v_3+a_4 v_4=b_1\Par{v_1-v_2}+b_2\Par{v_2-v_3}+b_3\Par{v_3-v_4}+b_4 v_4$\par\quad
$\Blind{v}=b_1 v_1+\Par{b_2-b_1}v_2+\Par{b_3-b_2}v_3+\Par{b_4-b_3}v_4$\par\quad
$\Blind{v}=a_1\Par{v_1-v_2}+\Par{a_1+a_2}\Par{v_2-v_3}+\Par{a_1+a_2+a_3}\Par{v_3-v_4}+\Par{a_1+\dots+a_4}v_4.$\PfEnd
\SepLine

\Anchor{2A4e3}\Anchor{2A4e14}\ProblemBnoor{4E 3, 14}{
	\TextB{Supp $\Par{v_1,\dots,v_m}$ is a list in $V$. For each $k$, let $w_k=v_1+\dots+v_k$.}
	(a) \TextB{Show $\Span{v_1,\dots,v_m}=\Span{w_1,\dots,w_m}.$}
	(b) \TextB{Show [P] $\Par{v_1,\dots,v_m}$ is liney indep $\Longleftrightarrow$ $\Par{w_1,\dots,w_m}$ is liney indep [Q].}
}\par\quad
(a) Asum $a_1v_1+\dots+a_mv_m=b_1w_1+\dots+b_mw_m=b_1v_1+\dots+b_k\Par{v_1+\dots+v_k}+\dots+b_m\Par{v_1+\dots+v_m}.$\par\quad\Ha
Then $a_k=b_k+\dots+b_m;\;\;a_{k+1}=b_{k+1}+\dots+b_m\Rightarrow b_k=a_k-a_{k+1};\;b_m=a_m.$ Simlr to Exe (1).\vspace{4pt}\par\quad
(b) $P\Rightarrow Q:\;\;b_1 w_1+\dots+b_m w_m=0=a_1 v_1+\dots+a_m v_m,\text{\;where\;}0=a_k=b_k+\dots+b_m.$\par\quad\Hb
$Q\Rightarrow P:\;\;a_1 v_1+\dots+a_m v_m=0=b_1 w_1+\dots+b_m w_m=0, \text{\;where\;} 0=b_m=a_m,\;0=b_k=a_k-a_{k+1}.$\vspace{4pt}\par\quad\Hb
\Or By (a), let $W=\Span{v_1,\dots,v_m}=\Span{w_1,\dots,w_m}.$ Supp $\Par{v_1,\dots,v_m}$ is liney dep.\par\quad\Hb
By [2.21](b), a list of len $\Par{m-1}$ spans $W.$ 又 By [2.23], $\Par{w_1,\dots,w_m}$ liney indep $\Rightarrow m\leqslant m-1.$\par\quad\Hb
Thus $\Par{w_1,\dots,w_m}$ is liney dep. Now rev the roles of $v$ and $w.$\PfEnd
\SepLine

\ProblemN{\Anchor{2A2}{2}}{
	(a) \TextA{[P]\hspace{17pt}{\; A list $\Par{v}$ of len $1$ in $V$ is liney indep $\Longleftrightarrow v\neq 0.$} \hfill[Q]}
	(b) \TextA{[P]{\; A list $\Par{v,w}$ of len $2$ in $V$ is liney indep $\Longleftrightarrow\forall\lambda,\mu\in\Fbb,v\neq \lambda w,w\neq \mu v.$} \hfill[Q]\vspace{4pt}}
}(a) $Q\Rightarrow P:$ $v\neq 0\Rightarrow$ if $av=0$ then $a=0\Rightarrow\Par{v}$ liney indep.\parSol{\Ha}
$P\Rightarrow Q:$ $\Par{v}$ liney indep $\Rightarrow v\neq 0$, for if $v=0,$ then $av=0\notRightarrow a=0.$\parSol{\Ha}
${}^\neg Q\Rightarrow{}^\neg P:$ $v=0\Rightarrow av=0$ while we can let $a\neq 0\Rightarrow\Par{v}$ is liney dep.\parSol{\Ha}
${}^\neg P\Rightarrow{}^\neg Q:$ $\Par{v}$ liney dep $\Rightarrow av=0$ while $a\neq 0\Rightarrow v=0.$\parSol{\vspace{4pt}}
(b) $P\Rightarrow Q:$ $\Par{v,w}$ liney indep $\Rightarrow$  if $av+bw=0,$ then $a=b=0\Rightarrow$ no scalar multi.\parSol{\Hb}
$Q\Rightarrow P:$ no scalar multi $\Rightarrow$ if $av+bw=0,$ then $a=b=0\Rightarrow\Par{v,w}$ liney indep.\parSol{\Hb}
${}^\neg P\Rightarrow{}^\neg Q:$ $\Par{v,w}$ liney dep $\Rightarrow$ if $av+bw=0,$ then $a$ or $b\neq 0\Rightarrow$ scalar multi.\parSol{\Hb}
${}^\neg Q\Rightarrow{}^\neg P:$ scalar multi $\Rightarrow$ if $av+bw=0,$ then $a$ or $b\neq 0\Rightarrow$ liney dep.\PfEnd
\SepLine

\ProblemN{\Anchor{2A10}{10}}{
	\TextA{Supp $\Par{v_1,\dots,v_m}$ is liney indep in $V$ and $w\in V$.}
	\TextA{Prove if $\BigPar{v_1+w,\dots,v_m+w}$ is liney dep, then $w\in\Span{v_1,\dots,v_m}$.}
}\par\quad
Note that $a_1\Par{v_1+w}+\dots+a_m\Par{v_m+w}=0\Rightarrow a_1 v_1+\dots+a_m v_m=-\Par{a_1+\dots+a_m}w.$\par\quad
Then $a_1+\dots+a_m\neq 0$, for if not, $a_1 v_1+\dots+a_m v_m=0$ while $a_i\neq 0$ for some $i$, ctradic.\par\quad
\Or We prove the ctrapos: Supp $w\not\in\Span{v_1,\dots,v_m}.$ Then $a_1+\dots+a_m=0.$\par\quad
Thus $a_1v_1+\dots+a_mv_m=0\Rightarrow a_1=\dots=a_m=0.$ Hence $\BigPar{v_1+w,\dots,v_m+w}$ is liney indep.\PfEnd\vspace{2pt}\quad
\Or $\exists\,j\in\;\!\!\Bra{1,\dots,m},v_j+w\in\Span{v_1+w,\dots,v_{j-1}+w}.$ If $j=1$ then $v_1+w=0$ and done.\par\quad
If $j\geqslant 2,$ then $\exists\,a_i\in\Fbb,v_j+w=a_1\Par{v_1+w}+\dots+a_{j-1}\Par{v_{j-1}+w}\Longleftrightarrow v_j+\lambda w=a_1 v_1+\dots+a_{j-1}v_{j-1}.$\par\quad
Where $\lambda=1-\Par{a_1+\dots+a_{j-1}}.$ Note that $\lambda\neq 0,$ for if not, $v_j+\lambda w=v_j\in\Span{v_1,\dots,v_{j-1}},$ ctradic.\par\quad
Now $w=\lambda^{-1}\Par{a_1 v_1+\dots+a_{j-1}v_{j-1}-v_j}\Rightarrow w\in\Span{v_1,\dots,v_m}.$\PfEnd
\SepLine

\ProblemN{\Anchor{2A11}{11}}{
	\TextA{Supp $\Par{v_1,\dots,v_m}$ is liney indep in $V$ and $w\in V$.}
	\TextA{Show [P] $\Par{v_1,\dots,v_m,w}$ is liney indep $\Longleftrightarrow w\not\in\Span{v_1,\dots,v_m}$ [Q].}
}Equiv to $\Par{v_1,\dots,v_m,w}$ liney dep $\Longleftrightarrow w\in\Span{v_1,\dots,v_m}.$ Using [2.21]. Obviously.\PfEnd
\ANote (a) Supp $\Par{v_1,\dots,v_m,w}$ is liney indep. Then $\Par{v_1,\dots,v_m}$ liney indep $\Longleftrightarrow w\not\in\Span{v_1,\dots,v_m}.$\parNot
(b) Supp $\Par{v_1,\dots,v_m,w}$ is liney dep. Then $\Par{v_1,\dots,v_m}$ liney indep $\Longleftrightarrow w\in\Span{v_1,\dots,v_m}.$
\SepLine

\ProblemN{\Anchor{2A14}{14}}{
	\TextA{Prove [P] $V$ is infinide $\Longleftrightarrow\exists$ seq $\Par{v_1,v_2,\dots}$ in $V$ suth each $\Par{v_1,\dots,v_m}$ liney indep. [Q]}
}$P\Rightarrow Q:\;$ Supp $V$ is infinide, so that no list spans $V.$ Define the desired seq recurly via:\parSol{}
\Blind{$P\Rightarrow Q:\;$} {\tgbf Step 1}\;\; Pick a $v_1\neq 0,\Par{v_1}$ liney indep.\parSol{}
\Blind{$P\Rightarrow Q:\;$} {\tgbf Step m}\; Pick a $v_m\not\in\Span{v_1,\dots,v_{m-1}},$ by Exe (11), $\Par{v_1,\dots,v_m}$ is liney indep.\vspace{4pt}\parSol{}
${}^\neg P\Rightarrow{}^\neg Q:\;$ Supp $V$ is finide and $V=\Span{w_1 , ..., w_m}$.\parSol{}
\Blind{${}^\neg P\Rightarrow{}^\neg Q:\;$} Let $\Par{v_1 , v_2 , \dots}$ be a seq in $V$, then $\Par{v_1,v_2,\dots,v_{m+1}}$ must be liney dep.\vspace{4pt}\parSol{}
\Or\; $Q\Rightarrow P:\;$ Supp there is such a seq.\parSol{}
\Blind{\Or\; $Q\Rightarrow P:\;$} Choose an $m$. Supp a liney indep list $\Par{v_1,\dots,v_m}$ spans $V$.\parSol{}
\Blind{\Or\; $Q\Rightarrow P:\;$} Simlr to [2.16]. $\exists\,v_{m+1}\in V\Backslash\Span{v_1,\dots,v_m}.$ Hence no list spans $V.$\PfEnd
\SepLine

\ProblemN{\Anchor{2A17}{17}}{
	\TextA{Prove $\Par{p_0,p_1,\dots,p_m}$ cannot be liney indep in $\PoF{m}$ with each $p_k\Par{2}=0.$}
}\par\quad
Supp $\Par{p_0,p_1,\dots,p_m}$ is liney indep. Define $p\in\PoF{m}$ by $p\Par{z}=z.$\par\quad
\NOTICE that $\forall a_i\in\Fbb,z\neq a_0 p_0\Par{z}+\dots+a_m p_m\Par{z},$ for if not, let $z=2.$ Thus $z\not\in\Span{p_0,p_1,\dots,p_m}$.\par\quad
Then $\Span{p_0,p_1,\dots,p_m}\subsetneq\PoF{m}$ while the list $\Par{p_0,p_1,\dots,p_m}$ has len $\Par{m+1}$.\par\quad
Hence $\Par{p_0,p_1,\dots,p_m}$ is liney dep. For if not, then becs $\Par{1,z,\dots,z^m}$ of len $\Par{m+1}$ spans $\PoF{m},$\par\quad
by the steps in [2.23] trivially, $\Par{p_0,p_1,\dots,p_m}$ of len $\Par{m+1}$ spans $\PoF{m}.$ Ctradic.\PfEnd\vspace{6pt}\par\quad
\Or Becs $\Par{1,z,\dots,z^m}$ of len $\Par{m+1}$ spans $\PoF{m}.$ Then $\Par{p_0,p_1,\dots,p_m,z}$ of len $\Par{m+2}$ is liney dep.\par\quad
As shown above,  $z\not\in\Span{p_0,p_1,\dots,p_m}.$ And hence by [2.21](a), $\Par{p_0,p_1,\dots,p_m}$ is liney dep.\PfEnd
\SepLine
\ChEnd

\vfill\ChDecl{Ch2B}{2$\cdot$B}{}

\vspace{4pt}

\Anchor{2BN2.34}\BulletPointX\NoteFor{{\tgsc liney indep seq and }[2.34]}\;\;$“V=\Span{v_1,\dots,v_n,\dots}”$ is an invalid expr.\TextB{}
If we allow using $“$infini list$”$, then we must assure that $\Par{v_1,\dots,v_n,\dots}$ is a spanning $“$list$”$\TextB{}
suth $\forall v\in V,\exists$ smallest $n\in\Nbp,\;v=a_1 v_1+\dots+a_n v_n.$ Moreover, given a list $\Par{w_1,\cdots,w_n,\cdots}$ in $W,$\TextB{}
we can prove $\exists\,!\,T\in\Lm{V,W}$ with each $Tv_k=w_k,$ which has less restr than [3.5].\TextB{}
But the key point is, how can we assure that such a $“$list$”$ exis ? \Sbra{{\FontSmall\tgsl See higher courses}}
\SepLine

\ProblemN{\Anchor{2B1}{1}}{
	\TextA{Find all vecsps on whatever $\Fbb$ that have exactly one bss.}
}The trivial vecsp $\zeroSubs$ will do. Indeed, the only bss of $\zeroSubs$ is the empty list $\Par{\,}$.\parSol{}
Now consider the field $\Bra{0,1}$ containing only the add id and multi id,\parSol{}
with $1+1=0.$ Then the list $\Par{1}$ is the uniq bss. Now the vecsp $\Bra{0,1}$ will do.\parSol{}
\AComm All vecsp on such $\Fbb$ of dim $1$ will do.\parSol{}
Consider other $\Fbb.$ Note that this $\Fbb$ contains at least and strictly more than $0$ and $1.$ Failed.\PfEnd
\SepLine\pagebreak

%\ProblemN{\Anchor{2B7}{7}}{
%	\TextA{Prove or give a countexa\hspace{1pt}$:$ If $\Par{v_1,v_2,v_3,v_4}$ is a bss of $V$ and $U$ is a subsp of $V$}
%	\TextA{suth $v_1,v_2\in U$ and $v_3\not\in U$ and $v_4\not\in U$, then $\Par{v_1,v_2}$ is a bss of U.}
%}A countexa\hspace{1pt}: Let $V=\Rbb^4$ and $B_V=\Par{e_1,e_2,e_3,e_4}$ be std bss.\par\quad
%Let $v_1=e_1,v_2=e_2,v_3=e_3+e_4,v_4=e_4.$ Then $\Par{v_1,\dots,v_4}$ is a bss of $\Rbb^4.$\par\quad
%Let $U=\Span{e_1,e_2,e_3}=\Span{v_1,v_2,v_3-v_4}.$ Then $v_3\not\in U$ and $\Par{v_1,v_2}$ is not a bss of $U.$\PfEnd\quad
%\AComm Let $W=\Span{v_4-v_1}.$ Then $v_4\in V=U\oplus W$ but $W\not\ni v_4\not\in U.$
%\SepLine

\Anchor{2B4e9}\ProblemBnoor{{4E 9}}{
	\TextA{Supp $\Par{v_1,\dots,v_m}$ is a list in $V$. For $k\in\;\!\!\Bra{1,\dots,m}$, let $w_k=v_1+\dots+v_k.$}
	\TextA{Show [P] $B_V=\Par{v_1,\dots,v_m}\Longleftrightarrow B_V=\Par{w_1,\dots,w_m}.$ [Q]}
}\NOTICE that $B_U=\Par{u_1,\dots,u_n}\Longleftrightarrow\forall u\in U,\exists\,!\,a_i\in\Fbb,u=a_1 u_1+\dots+a_n u_n.$\par\quad
$P\Rightarrow Q:\forall v\in V,\exists\,!\,a_i\in\Fbb,\;v=a_1 v_1+\dots+a_m v_m\Rightarrow v=b_1 w_1+\dots+b_m w_m,\exists\,!\,b_k=a_k-a_{k+1},b_m=a_m.$\vspace{2pt}\par\quad
$Q\Rightarrow P:\forall v\in V,\exists\,!\,b_i\in\Fbb,\;v=b_1w_1+\dots+b_mw_m\Rightarrow v=a_1v_1+\dots+a_mv_m,\exists\,!\,a_k=\sum_{j=k}^m b_j.$\PfEnd\vspace{3pt}
\AComm \Or Using \Sbra{3.C {\NOTEFOR} [3.30, 32](a)}.
\SepLine

%\Anchor{2B4e5}\ProblemBnoor{{4E 5}}{
%	\TextA{Supp $U,W$ are finide, $V=U+W,\;B_U=\Par{u_1,\dots,u_m},\;B_W=\Par{w_1,\dots,w_n}.$}
%	\TextA{Prove $\exists\,B_V$ consisting of vecs in $U\cup W$.\vspace{-4pt}}
%}$V=\Span{u_1,\dots,u_m}+\Span{w_1,\dots,w_n}=\Span[\BigPar]{\overbrace{u_1,\dots,u_m,w_1,\dots,w_n}^{\text{Reduce}}}.$ By [2.31].\PfEnd
%\SepLine

\ProblemN{\Anchor{2B8}{8}}{
	\TextA{Supp $B_U=\Par{u_1,\dots,u_m},B_W=\Par{w_1,\dots,w_n}.$}
	\TextA{Prove $V=U\oplus W\Longleftrightarrow B_V=\BigPar{u_1,\dots,u_m,w_1,\dots,w_n}.$}
}$\forall v\in V,\exists\,!\,u\in U,w\in W\Rightarrow\exists\,!\,a_i,b_i\in\Fbb,v=u+w=\sum_{i=1}^ma_iu_i+\sum_{i=1}^nb_iw_i.$\parSol{}
\Or\;$V=\Span{u_1,\dots,u_m}\oplus\Span{w_1,\dots,w_n}=\Span[\BigPar]{u_1,\dots,u_m,w_1,\dots,w_n}.$\parSol{\vspace{2pt}}
\Blind{\Or\;}Note that $\sum_{i=1}^ma_iu_i+\sum_{i=1}^nb_iw_i=0\Rightarrow\sum_{i=1}^ma_iu_i=-\sum_{i=1}^nb_iw_i\in U\cap W=\zeroSubs.$\PfEnd
\SepLine

\Anchor{9A3}\Anchor{9A4}\Anchor{2B4e11}\ProblemBnoor{9.A.3,4 \OR 4E 11}{
	\TextB{Supp $V$ is on $\Rbb,$ and $v_1,\dots,v_n\in V.$ Let $B=\Par{v_1,\dots,v_n}.$\vspace{2pt}}
	(a) \TextB{Show [P] $B$ is liney indep in $V$ $\Longleftrightarrow B$ is liney indep in $V_{\!\Cbb}.$ [Q]}
	(b) \TextB{Show [P] $B$ spans $V$ $\Longleftrightarrow B$ spans $V_{\!\Cbb}.$ [Q]\vspace{2pt}}
}{\FontSmall\par\quad
(a) $P\Rightarrow Q:$ \;Note that each $v_k\in V_{\!\Cbb}.$ Supp $\lambda_1v_1+\dots+\lambda_nv_n=0$ with $\Fbb=\Cbb.$\vspace{-1pt}\par\quad\Ha
\Blind{$P\Rightarrow Q:$ \;}Then $\Par{\Real\lambda_1}v_1+\dots+\Par{\Real\lambda_n}v_n=0\Rightarrow$ each $\Real\lambda_i=0,$ simlr for $\Imaginary\lambda_i.$\par\quad\Ha
$Q\Rightarrow P:$ \;If $\lambda_k\in\Rbb$ with $\lambda_1v_1+\dots+\lambda_nv_n=0,$ then each $\Real\lambda_k=\lambda_k=0.$\par\quad\Ha
${}^\neg P\Rightarrow{}^\neg Q:$ \;$\exists\,v_j=a_{j-1}v_{j-1}+\dots+a_1v_1\in V_{\!\Cbb}.$\par\quad\Ha
${}^\neg Q\Rightarrow{}^\neg P:$ \;$\exists\,v_j=\lambda_{j-1}v_{j-1}+\dots+\lambda_1v_1\in V\Rightarrow v_j=\Par{\Real\lambda_{j-1}}v_{j-1}+\dots+\Par{\Real\lambda_1}v_1\in V.$\par\vspace{3pt}\quad
(b) $P\Rightarrow Q:$ \;$\forall u+\i v\in V_{\!\Cbb},\;u,v\in V\Rightarrow\exists\,a_i,b_i\in\Rbb,u+\i v=\sum_{i=1}^n\Par{a_i+\i b_i}v_i.$\par\quad\Hb
$Q\Rightarrow P:$ \;$\forall v\in V,\exists\,\!\,a_i+\i b_i\in\Cbb,\;v+\i 0=\BigBigPar{{\sum_{i=1}^na_iv_i}}+\i\,\BigBigPar{{\sum_{i=1}^nb_iv_i}}\envFontSmall\Rightarrow v\in\Span{v_1,\dots,v_m}.$\par\quad\Hb
${}^\neg P\Rightarrow{}^\neg Q:$ \;$\exists\,v\in V,\,v\not\in\spn B$ with $\Fbb=\Rbb\Rightarrow v+\i 0\not\in\spn B$ with $\Fbb=\Cbb.$\par\quad\Hb
${}^\neg Q\Rightarrow{}^\neg P:$ \;$\exists\,u+\i v\in V_{\!\Cbb},u+\i v\not\in\spn B\Rightarrow\Par{\Real1}u+\Par{\Real\i}v=u$ or $\Par{\Imaginary1}u+\Par{\Imaginary\i}v=v\not\in\spn B.$
}\PfEnd\vspace{-4pt}
\SepLine

\Anchor{2BTips}\ProblemBX[]{\Tips}{
	\TextA{Supp $\dim V=n,$ and $U$ is a subsp of $V$ with $U\neq V.$}
	\TextA{Prove $\exists\,B_V=\Par{v_1,\dots,v_n}$ suth each $v_k\not\in U.$}
}\TextB{\vspace{0pt}}
Note that $U\neq V\Rightarrow n\geqslant 1.$ We will construct $B_V$ via the following process.\TextB{}
{\tgbfx Step 1.} $\exists\,v_1\in V\Backslash U\Rightarrow v_1\neq 0.$ If $\Span{v_1}=V$ then we stop.\TextB{}
{\tgbfx Step k.} Supp $\Par{v_1,\dots,v_{k-1}}$ is liney indep in $V,$ each of which belongs to $V\Backslash U.$\TextB{}
\Blind{\tgbfx Step k.} Note that $\Span{v_1,\dots,v_{k-1}}\neq V.$ And if $\Span{v_1,\dots,v_{k-1}}\cup U=V,$ then by (1.C.12),\TextB{}
\Blind{\tgbfx Step k.} \Sbra{ becs $\Span{v_1,\dots,v_{k-1}}\not\subseteq U,$ } $U\subseteq \Span{v_1,\dots,v_{k-1}}\Rightarrow\Span{v_1,\dots,v_{k-1}}=V.$\TextB{}
\Blind{\tgbfx Step k.} Hence becs $\Span{v_1,\dots,v_{k-1}}\neq V,$ it must be case that $\Span{v_1,\dots,v_{k-1}}\cup U\neq V.$\TextB{}
\Blind{\tgbfx Step k.} Thus $\exists\,v_k\in V\Backslash U$ suth $v_k\not\in\Span{v_1,\dots,v_{k-1}}.$\TextB{}
\Blind{\tgbfx Step k.} By (2.A.11), $\Par{v_1,\dots,v_k}$ is liney indep in $V$. If $\Span{v_1,\dots,v_k}=V,$ then we stop.\TextB{}
Becs $V$ is finide, this process will stop after $n$ steps.\PfEnd\vspace{4pt}\TextB{}
\Or Supp $U\neq\zeroSubs.$ Let $B_U=\Par{u_1,\dots,u_m}.$ Extend to a bss $\Par{u_1,\dots,u_n}$ of $V.$\TextB{}
\Blind{\Or}Then let $B_V=\Par{u_1-u_k,\dots,u_m-u_k,u_{m+1},\dots,u_k,\dots,u_n}.$\PfEnd
\SepLine\ChEnd
\pagebreak

\ChDecl{Ch2C}{2$\cdot$C}{}

\vspace{4pt}

%\ProblemB{
%	\TextB{Supp $V=\FbbP{S},$ and $S\neq\emptySet.$ Prove $\dim V=\card S.$}
%}
%\SepLine

%\ProblemN{\Anchor{2C15}{15}}{
%	\TextA{Supp $\dim V=n\geqslant 1.$ Prove $\exists\,1$-\hspace{1pt}dim subsps $V_{\!1},\dots,V_{\!n}$ suth $V=V_{\!1}\oplus \dots\oplus V_{\!n}.$}
%}Supp $B_V=\Par{v_1,\dots,v_n}.$ Let each $V_{\!i}=\Span{v_i}.$\parSol{}
%Then $\forall v\in V,\exists\,!\,a_i\in\Fbb,v=a_1 v_1+\dots+a_n v_n\Rightarrow\exists\,!\,u_i\in V_{\!i},v=u_1+\dots+u_n$\PfEnd
%\SepLine

\Anchor{2CN15}\ProblemBX{\NoteForSmall{Exe (15)}}{
	\TextB{Supp $v\in V\nonzero.$ Prove $\exists\,B_V=\Par{v_1,\dots,v_n},v=v_1+\dots+v_n.$}
}If $n=1$ then let $v_1=v$ and done. Supp $n>1.$\parSol{}
Extend $\Par{v}$ to a bss $\Par{v,v_1,\dots,v_{n-1}}$ of $V.$ Let $v_n=v-v_1-\dots-v_{n-1}.$\parSol{}
又 $\Span{v,v_1,\dots,v_{n-1}}=\Span{v_1,\dots,v_n}.$ Hence $\Par{v_1,\dots,v_n}$ is also a bss of $V.$\PfEnd\vspace{4pt}
\AComm Let $B_V=\Par{v_1,\dots,v_n}$ and supp $v=u_1+\dots+u_n,$ where each $u_i=a_i v_i\in V_{\!i}.$\parCom
But $\Par{u_1,\dots,u_n}$ might not be a bss, becs there might be some $u_i=0.$\vspace{-2pt}
\SepLine

%\Anchor{2C1}\ProblemNnoor{1}{{\COROLLARY} for [2.38,39]}{
%	\TextA{Supp $U$ is a subsp of $V$ suth $\dim V=\dim U$. Then $V=U.$}
%}Let $B_U=\Par{u_1,\dots,u_m}.$ Then $m=\dim V.$ 又 $u_i\in V.$ By [2.39], $B_V=\Par{u_1,\dots,u_m}.$\PfEnd\vspace{-2pt}
%\SepLine[-2pt]

\Anchor{2C'1}\ProblemB[]{
	Let $v_1,\dots,v_n\in V$ and $\dim\Span{v_1,\dots,v_n}=n.$ Then $\Par{v_1,\dots,v_n}$ is a bss of $\Span{v_1,\dots,v_n}.$\TextB{}
	{\tgsl Notice that $\Par{v_1,\dots,v_n}$ is a spanning list of $\Span{v_1,\dots,v_n}$ of len $n=\dim\Span{v_1,\dots,v_n}.$}\TextB{\vspace{-4pt}}
}\SepLine

\ProblemN{\Anchor{2C9}{9}}{
	\TextA{Supp $\Par{v_1,\dots,v_m}$ is liney indep in $V,w\in V.$ Prove $\dim\Span[\BigPar]{v_1+w,\dots,v_m+w}\geqslant m-1$.}
}Using (2.A.10, 11).\par\quad
Note that each $v_i-v_1=\Par{v_i+w}-\Par{v_1+w}\in\Span{v_1+w,\dots,v_n+w}.$\par\quad
$\Par{v_1,\dots,v_m}$ liney indep $\Rightarrow$ $\BigPar{v_1,v_2-v_1,\dots,v_m-v_1}$ liney indep $\Rightarrow$ $\underbrace{\BigPar{v_2-v_1,_{}\dots,v_m-v_1}}_{\text{of len }\SmallPar{m-1}}$ liney indep.\vspace{-15pt}\par\quad
又 If $w\not\in\Span{v_1,\dots,v_m}.$ Then $\BigPar{v_1+w,\dots,v_m+w}$ is liney indep.\par\quad
Hence $m\geqslant\dim\Span[\BigPar]{v_1+w,\dots,v_m+w}\geqslant m-1.$\PfEnd
\SepLine

\Anchor{2C4e16}\ProblemBnoor{{4E 16}}{
	\TextA{Supp $V$ is finide, $U$ is a subsp of $V$ with $U\neq V$. Let $n=\dim V,m=\dim U$.}
	\TextA{Prove $\exists\,\Par{n-m}$ subsps  $U_1,\dots,U_{n-m}$, each of dim $\Par{n-1}$, suth $\bigcap\limits_{i=1}^{n-m}U_i=U$.\vspace{-4pt}}
}Let $B_U=\Par{v_1,\dots,v_m},\;B_V=\BigPar{v_1,\dots,v_m,u_1,\dots,u_{n-m}}$.\parSol{}
Define each $U_i=\Span[\BigPar]{v_1,\dots,v_m,u_1,\dots,u_{i-1},u_{i+1},\dots,u_{n-m}}\Rightarrow U\subseteq U_i.$\vspace{4pt}\parSol{}
And becs $\forall v\in \bigcap\limits_{i=1}^{n-m}U_i,v=v_0+b_1 u_1+\dots+b_{n-m} u_{n-m}\in U_i\Rightarrow$ each $b_i=0\Rightarrow v\in U.$\vspace{-4pt}\parSol{}
Hence $\bigcap\limits_{i=1}^{n-m}U_i\subseteq U.$\PfEnd
%\vspace{10pt}
%\AExa {Supp $\dim V=6,\dim U=3$.\par\vspace{2pt}\quad
%	$\Par{\overbrace{\underbrace{v_1,v_2,v_3}_\text{Bss of U},v_4,v_5,v_6}^\text{Bss of V}},$ define $\hMath[0pt]{l}{\left|}{\right\}}{$
%		$U_1=\Span{v_1,v_2,v_3}\oplus\Span{v_5,v_6}\\ $
%		$U_2=\Span{v_1,v_2,v_3}\oplus\Span{v_4,v_6}\\ $
%		$U_3=\Span{v_1,v_2,v_3}\oplus\Span{v_4,v_5}$
%		$}\Rightarrow\dim U_i=6-1,\,\,i=\underbrace{1,2,3}_{6-3=3}.$}
%\PfEnd[-10pt]\vspace{-4pt}
\SepLine

\ProblemN{\Anchor{2C14}{14}}{
	\TextA{Supp $V_{\!1},\dots,V_{\!m}$ are finide. Prove $\dim\!\Par{V_{\!1}+\dots+V_{\!m}}\leqslant\dim V_{\!1}+\dots+\dim V_{\!m}$.}
}For each $V_{\!i}\,,$ let $B_{V_{\!i}}=\mE_i.$ Then $V_{\!1}+\dots+V_{\!m}=\Span[\BigPar]{\mE_1\cup\cdots\cup \mE_m};\,\,\,\dim V_{\!i}=\card\mE_i.$\par\quad
Now $\dim\!\Par{V_{\!1}+\dots+V_{\!m}}=\dim\Span[\BigPar]{\mE_1\cup\cdots\cup \mE_m}\leqslant\Card\BigPar{\mE_1\cup\cdots\cup \mE_m}\leqslant\card \mE_1+\cdots+\card \mE_m.$\par\vspace{2pt}\Anchor{2C16}
\ACoro $V_{\!1}+\dots+V_{\!m}$ is direct\parCor
$\Longleftrightarrow$ For each $k\in\;\!\!\Bra{1,\dots,m-1},$ $\BigPar{V_{\!1}\oplus\dots\oplus V_{\!k}}\cap V_{k+1}=\zeroSubs,\BigPar{\mE_1\cap\dots\cap\mE_{k-1}}\cap\mE_k=\emptySet$\parCor
$\Longleftrightarrow\dim\Span[\BigPar]{\mE_1\cup\dots\cup\mE_m}=\Card\BigPar{\mE_1\cup\dots\cup\mE_m}=\card\mE_1+\dots+\card\mE_m$\parCor
$\Longleftrightarrow \dim\!\Par{V_{\!1}\oplus\dots\oplus V_{\!m}}=\dim V_{\!1}+\dots+\dim V_{\!m}.$\PfEnd
\SepLine

\Anchor{2C'2}\ProblemB{
	\TextB{Supp $\mC$ is a collec of $k$\hspace{1pt}-\hspace{1pt}dim subsps of $V$ with any two of them have a $\Par{k-1}$\hspace{1pt}-\hspace{1pt}dim intersec.}
	\TextB{Prove either all contain a $\Par{k-1}$\hspace{1pt}-\hspace{1pt}dim intersec, or all contained in a $\Par{k+1}$\hspace{1pt}-\hspace{1pt}dim subsp.}
}If $V$ is finide and $\dim V=k,$ then $\mC=\Bra{V},$ done. We use induc on $k.$ (i) $k=1.$ Immed.\parSol{}
(ii) $k>1.$ Asum it holds for $k-1.$ If $\exists$ common $\Par{k-1}$\hspace{1pt}-\hspace{1pt}dim intersec, then done.\parSol{\Hii}
Othws, we show all $X\in\mC$ are contained in a $\Par{k+1}$\hspace{1pt}-\hspace{1pt}dim subsp.\parSol{\Hii}
Supp $U,W\in\mC\Rightarrow\Dim\Par{U+W}=k+1.$ Then for $X\in\mC,\;X\cap U,X\cap W$ are $\Par{k-1}$\hspace{1pt}-\hspace{1pt}dim.\parSol{\Hii}
Now by asum, $\Dim\Par{{X\cap U}+{X\cap W}}=k\Rightarrow X=\Par{X\cap U}+\Par{X\cap W}\Rightarrow X\subseteq U+W.$\PfEnd
\SepLine

\pagebreak

\ProblemN{\Anchor{2C17}{17}}{
	\TextA{Supp $V_{\!1},V_{\!2},V_{\!3}$ are subsps of a finide vecsp. Explain and give a countexa\hspace{1pt}$:$}
	\TextA{{\FontNorm$\Dim\Par{V_{\!1}+V_{\!2}+V_{\!3}}=\dim V_{\!1}+\dim V_{\!2}+\dim V_{\!3}$\par\rightline{${}-\Dim\Par{V_{\!1}\cap V_{\!2}}-\Dim\Par{V_{\!1}\cap V_{\!3}}-\Dim\Par{V_{\!2}\cap V_{\!3}}+\Dim\Par{V_{\!1}\cap V_{\!2}\cap V_{\!3}}$.\qquad}}\vspace{-16pt}}
}\par\quad
(1) $\aXMid{\Par{A\cup B}\cup C}=\cMid{A\cup B}+\cMid{C}-\aXMid{\Par{A\cup B}\cap C}.$\par\quad
(2) $\aXMid{\Par{A\cup B}\cap C}=\aXMid{\Par{A\cap C}\cup\Par{B\cap C}}=\cMid{A\cap C}+\cMid{B\cap C}-\cMid{A\cap B\cap C}.$\par\quad
Thus $\aXMid{\Par{A\cup B}\cup C}=\cMid{A}+\cMid{B}+\cMid{C}+\cMid{A\cap B\cap C}-\cMid{A\cap B}-\cMid{A\cap C}-\cMid{B\cap C}.$\par\vspace{4pt}\quad
Becs $\Par{V_{\!1}+V_{\!2}}+V_{\!3}=V_{\!1}+\Par{V_{\!2}+V_{\!3}}=\Par{V_{\!1}+V_{\!3}}+V_{\!2}.$\par\quad
$\Dim\Par{V_{\!1}+V_{\!2}+V_{\!3}}=\Dim\Par{V_{\!1}+V_{\!2}}+\Dim\Par{V_{\!3}}-\Dim\BigPar{\Par{V_{\!1}+V_{\!2}}\cap V_{\!3}}\quad(1)\hspace{114pt}$\par\quad
$\Blind{\Dim\Par{V_{\!1}+V_{\!2}+V_{\!3}}}=\Dim\Par{V_{\!2}+V_{\!3}}+\Dim\Par{V_{\!1}}-\Dim\BigPar{\Par{V_{\!2}+V_{\!3}}\cap V_{\!1}}\quad(2)$\par\quad
$\Blind{\Dim\Par{V_{\!1}+V_{\!2}+V_{\!3}}}=\Dim\Par{V_{\!1}+V_{\!3}}+\Dim\Par{V_{\!2}}-\Dim\BigPar{\Par{V_{\!1}+V_{\!3}}\cap V_{\!2}}\quad(3).$\par\vspace{3pt}\quad
Generally, $\Par{X+Y}\cap Z\neq\Par{X\cap Z}+\Par{Y\cap Z}.$ \;\AExa $X=\Bra{\Par{x,0}},Y=\Bra{\Par{0,y}},Z=\Bra{\Par{z,z}}\subseteq\FbbP{2}.$\par\vspace{2pt}\quad
\AComm If $X\subseteq Y,$ then $\Par{X+Y}\cap Z=Y\cap Z;$ \;$\Dim\Par{X+Y+Z}=\dim{Y}+\dim{Z}-\Dim\Par{Y\cap Z},$\parCom\quad
and the wrong formula holds. Simlr for $Y\subseteq Z,\,X\subseteq Z,$ and $X,Y\subseteq Z.$\par\vspace{4pt}\quad
\ANote However, it's true that $\Par{X+Y}${\Large${}\cap{}$}$Z\supseteq\Par{X\cap Z}+\Par{Y\cap Z}=\BigPar{X+\Par{Y\cap Z}}${\Large${}\cap{}$}$Z.$\parNot\quad
Becs $\Par{X\cap Z}+\Par{Y\cap Z}\ni v=x+y=z_1+z_2\in\BigPar{X+\Par{Y\cap Z}}${\Large${}\cap{}$}$Z\Rightarrow v\in\Par{X+Y}${\Large${}\cap{}$}$Z.$\par
%\BulletPointX\ACoro $\Dim\Par{V_{\!1}+V_{\!2}+V_{\!3}}=\dim V_{\!1}+\dim V_{\!2}+\dim V_{\!3}$\parCor{\IndentB}
%$\Blind{\Dim\Par{V_{\!1}+V_{\!2}+V_{\!3}}}{}-\Sbra{\Dim\Par{V_{\!1}\cap V_{\!2}}+\Dim\Par{V_{\!1}\cap V_{\!3}}+\Dim\Par{V_{\!2}\cap V_{\!3}}}\Big/3$\parCor{\IndentB}
%$\Blind{\Dim\Par{V_{\!1}+V_{\!2}+V_{\!3}}}{}-\Sbra{\Dim\BigPar{\Par{V_{\!1}+V_{\!2}}\cap V_{\!3}}+\Dim\BigPar{\Par{V_{\!1}+V_{\!3}}\cap V_{\!2}}+\Dim\BigPar{\Par{V_{\!2}+V_{\!3}}\cap V_{\!1}}}\Big/3.$
\SepLine

\ProblemBX[]{\Tips}{
	Becs $\dim \Par{V_{\!1}\cap V_{\!2}\cap V_{\!3}}=\dim V_{\!1}+\Dim\Par{V_{\!2}\cap V_{\!3}}-\Dim\BigPar{V_{\!1}+\Par{V_{\!2}\cap V_{\!3}}}.$\TextA{}
	And $\Dim\Par{V_{\!2}\cap V_{\!3}}=\dim V_{\!2}+\dim V_{\!3}-\Dim\Par{V_{\!2}+V_{\!3}}.$ We have $\Par{1},$ and $\Par{2},\Par{3}$ simlr.\TextA{}
	$\Par{1}\;\Dim\Par{V_{\!1}\cap V_{\!2}\cap V_{\!3}}=\dim V_{\!1}+\dim V_{\!2}+\dim V_{\!3}-\Dim\Par{V_{\!2}+V_{\!3}}-\Dim\BigPar{V_{\!1}+\Par{V_{\!2}\cap V_{\!3}}}.$\TextA{}
	$\Par{2}\;\Dim\Par{V_{\!1}\cap V_{\!2}\cap V_{\!3}}=\dim V_{\!1}+\dim V_{\!2}+\dim V_{\!3}-\Dim\Par{V_{\!1}+V_{\!3}}-\Dim\BigPar{V_{\!2}+\Par{V_{\!1}\cap V_{\!3}}}.$\TextA{}
	$\Par{3}\;\Dim\Par{V_{\!1}\cap V_{\!2}\cap V_{\!3}}=\dim V_{\!1}+\dim V_{\!2}+\dim V_{\!3}-\Dim\Par{V_{\!1}+V_{\!2}}-\Dim\BigPar{V_{\!3}+\Par{V_{\!1}\cap V_{\!2}}}.$\TextA{\vspace{-3pt}}
}\SepLine

\Anchor{2C4e14}\Anchor{2C4e15}\ProblemB[]{
	\TextB{Supp $V_{\!1},V_{\!2},V_{\!3}$ are subsps of $V$ with}
	\PrePa\TextB{$\dim V=10,\;\dim V_{\!1} = \dim V_{\!2} = \dim V_{\!3} = 7$. Prove $V_{\!1}\cap V_{\!2}\cap V_{\!3}\neq\zeroSubs$.}
	\Blind{\PrePa}By {\TIPS}, $\Dim\Par{V_{\!1}\cap V_{\!2}\cap V_{\!3}}\geqslant\dim V_{\!1}+\dim V_{\!2}+\dim V_{\!3}\uline{{}-2\dim V}>0.$\TextB{\vspace{4pt}}
	\PrePb\TextB{$\dim V_{\!1}+\dim V_{\!2}+\dim V_{\!3} > 2\dim V$. Prove $V_{\!1}\cap V_{\!2}\cap V_{\!3}\neq\zeroSubs$.}
	\Blind{\PrePb}By {\TIPS}, $\Dim\Par{V_{\!1}\cap V_{\!2}\cap V_{\!3}}\uuline{{}>2\dim V}-\Dim\Par{V_{\!2}+V_{\!3}}-\Dim\Par{V_{\!1}+\Par{V_{\!2}\cap V_{\!3}}}\geqslant 0.$\PfEnd\vspace{14pt}
}\SepLine
\ChEnd


\vfill\ChDecl{Ch3A}{3$\cdot$A}{\quad{\small\textbf{注意\,:}\;这里我将3.B的值空间、\!零空间、\!单满射、\!和3.D的可逆性定义前置;仅涉及概念。}}

\vspace{4pt}

\Anchor{3AT1}\BulletPointX\TipsN{1} \,\,\,$T:V\rightarrow W$ is liney $\Longleftrightarrow\MathLeftrightMid{l}{\hspace{-4pt}$
(一) $\forall v,u\in V,T\Par{v+u}=Tv+Tu;\\\hspace{-4pt} $
(二) $\forall v,u\in V,\lambda\in\Fbb,T\Par{\lambda v}=\lambda\Par{Tv}.$
$}\Longleftrightarrow T\Par{v+\lambda u}=Tv+\lambda Tu.$\vspace{6pt}\TextB{}
\ANote Supp $V$ is a vecsp. For $U\subseteq V,$ $U$ is a subsp of $V\Longleftrightarrow\forall u_1,u_2\in U,\lambda\in\Fbb,u_1+\lambda u_2\in U.$\par\vspace{4pt}
\Anchor{3E1}\ProblemBnoor[]{3.E.1}{
	\TextB{A function $T:V\rightarrow W$ is liney $\Longleftrightarrow$ The graph of $T$ is a subspace of $V\times W.$\vspace{-6pt}}
}\SepLine

\Anchor{3A4e10}\ProblemBnoor[]{4E 10}{
	\ANote Composition and product are not the same in $\PoFi.$\TextB{\vspace{-3pt}}
}\SepLine\pagebreak

\ProblemN{\Anchor{3A11}{11}}{
	\TextA{Supp $U$ is a subsp of $V$ and $S\in\Lm{U,W}$.}
	\TextA{Prove $\exists\,T\in\Lm{V,W},Tu = Su,\forall u\in U$. {\FontNorm\tgnr\BigPar{ \Or $\exists\,T\in\Lm{V,W},T\mmid_U=S.$ }}}
	\TextA{\large In other words, every liney map on a subsp of $V$ can be {\tgsc extended} to a liney map on the entire $V$.}
}Supp $W$ is suth $V=U\oplus W.$ Then $\forall v\in V,\exists\,!\,u_v\in U,w_v\in W,v=u_v+w_v.$\parSol{}
Define $T\in\Lm{V,W}$ by $T\Par{u_v+w_v}=Su_v.$\PfEnd\parSol{\vspace{0pt}}
\Or \Sbra[3pt]{{\tgsl Finide Req}} \;Define by $T\BigBigPar{{\sum_{i=1}^m a_i u_i}}=\sum_{i=1}^n a_i Su_i.$ \,Let $B_V=\Par{\overbrace{u_1,\dots,u_n}^{B_U},\dots,u_m}.$\PfEnd
\SepLine

\Anchor{3ANR}\ProblemB[]{
	\NoteForSmall{Restr}\;\;$U$ is a subsp of $V.$ (a) $\left.\Par{T+\lambda S}\right|{_U}=T\mmid_U+\lambda S\mmid_U.$ (b) $\left.\Par{ST}\right|{_U}=ST\mmid_U.$\vspace{-2pt}\TextB{}
}\SepLine

\Anchor{3AT2}\ProblemBX[]{\TipsN{2}}{
	$T\in\Lm{V,W}.$ (a) If $U$ is a subsp of $W.$ Then $\range T\subseteq U\Longleftrightarrow T\in\Lm{V,U}\subseteq\Lm{V,W}.$\TextA{}
	\Blind{$T\in\Lm{V,W}.$} (b) If $U$ is a subsp of $V.$ Then $U\subseteq\null T\Longleftrightarrow T\mmid_U=0.$\TextA{\vspace{-2pt}}
}\SepLine

\Anchor{4O4e3}\ProblemBnoor{{4E 4.3}}{
	\TextB{Supp $\Fbb=\Rbb,\,T\in\Lm{V,W},\,S=\REAL\circ T_{\!\Cbb}.$ \,Show $T_{\!\Cbb}=S-\i\,S\circ\i\,I.$}
}$T_{\!\Cbb}=S+\i\,\IMAGINARY\,T_{\!\Cbb}.$ 又 $\REAL\circ\Par{T_{\!\Cbb}\:\i\, I}=\REAL\circ\Par{\i\,T_{\!\Cbb}}=-\IMAGINARY\circ T_{\!\Cbb}=S\circ\i\,I.$\PfEnd
\AComm $\REAL,\IMAGINARY:\Cbb\mapsto\Rbb$ is not liney, while they have the add.
\SepLine

\Anchor{3ANLC}\ProblemBX[]{\NoteForSmall{Complex of Liney Maps}}{
	Supp $V,W$ are vecsps over $\Rbb.$ Then $\Lm{V,W}{_\Cbb}=\Lm{V_{\!\Cbb},W_\Cbb}.$\TextB{}
	For $S,T\in\Lm{V,W},$ $\Par{S+\lambda T}{_\Cbb}=S_\Cbb+\lambda T_{\!\Cbb}.$ For $S\in\Lm{V,W},T\in\Lm{U,V},$ $\Par{ST}{_\Cbb}=S_\Cbb T_{\!\Cbb}.$\TextB{}
	For $T\in\Lm{V,W},$ $\Null\Par{T_{\!\Cbb}}=\Par{\null T}{_\Cbb},\,\Range\Par{T_{\!\Cbb}}=\Par{\range T}{_\Cbb}.$\vspace{-3pt}\TextA{}
}\SepLine

\Anchor{9A17}\ProblemBnoor{9.A.17}{
	\TextA{Supp $\Fbb=\Rbb,T\in\Lm{V}$ suth $T^2=-I.$ Define complex scalar multi on $V$ as}
	\TextA{$\Par{a+b\i}v=av+bTv.$ Then $V$ itself is already a complex vecsp with these defs.}
	\TextA{Show the dim of $V$ as a complex vecsp is half of the dim of $V$ as the usual real vecsp.}
}Supp $V\neq\zeroSubs.$ Let $N=\dim V$ as real vecsp. We construct a real $B_V$ via a $\Par[0pt]{N-1}$\hspace{1pt}-\hspace{1pt}step process.\par\quad
Let $\Par{v_1,Tv_1}$ be liney indep in $V$ as real vecsp. Let $v_2\not\in\Span{v_1,Tv_1}\Rightarrow\Par{v_1,Tv_1,v_2}$ liney indep.\vspace{1pt}\par\quad
{\tgbf Step 1.} \;We show $\Par{v_1,Tv_1,v_2,Tv_2}$ liney indep in $V$ as real vecsp. Asum $Tv_2=a_1v_1+b_1Tv_1+a_2v_2.$\par\quad
\Blind{{\tgbf Step 1.} \;}Then $-v_2=a_1Tv_1-b_1v_1+a_2Tv_2.$ Note that $a_2\neq0$ and $a_2^2=-1$ while $a_2\in\Rbb,$ ctradic.\vspace{2pt}\par\quad
{\tgbf Step k.} \;\!\Sbra{$k\leqslant N-1$} We show $\Par{v_1,Tv_1,\dots,v_k,Tv_k,v_{k+1},Tv_{k+1}}$ liney indep in $V$ as real vecsp. Simlr.\par\quad
\Blind{{\tgbf Step k.} \;}Asum $Tv_{k+1}=a_1v_1+b_1Tv_1+\dots+a_{k+1}v_{k+1}.$ Then $-v_{k+1}=a_1Tv_1-b_1v_1+\dots+a_{k+1}Tv_{k+1}.$\PfEnd
\SepLine

%\Anchor{9A2}\Anchor{9A6}\Anchor{3B4e33}\ProblemBnoor{{9.A.2,6 \OR 4E 3.B.33}}{
%	\TextB{Supp $V,W$ are on $\Rbb,$ and $T\in\Lm{V,W}.$ Show}
%	{\tgnr\large(a)} $T_{\!\Cbb}\in\Lm{V_{\!\Cbb},W_{\!\Cbb}}.$ \;{\tgnr\large(b)} $\Null\Par{T_{\!\Cbb}}=\BigPar{\null T}{_\Cbb},\,\Range\Par{T_{\!\Cbb}}=\BigPar{\range T}{_\Cbb}.$ \;{\tgnr\large(c)} $T_{\!\Cbb}$ is inv $\Leftrightarrow T$ is inv.\TextB{}
%}{(a) $T_{\!\Cbb}\BigPar{\Par{u_1+\i v_1}+\Par{x+\i y}\Par{u_2+\i v_2}}=T\Par{u_1+xu_2-yv_2}+\i T\Par{v_1+xv_2+yu_2}$\parSol{\Ha}
%	$=T_{\!\Cbb}\Par{u_1+\i v_1}+\Par{x+\i y}T_{\!\Cbb}\Par{u_2+\i v_2}.$\parSol{}
%	(b) $u+\i v\in\Null\Par{T_{\!\Cbb}}\Longleftrightarrow u,v\in\null T\Longleftrightarrow u+\i v\in\BigPar{\null T}{_\Cbb}.$\parSol{\Hb}
%	$w+\i x\in\Range\Par{T_{\!\Cbb}}\Longleftrightarrow w,x\in\range T\Longleftrightarrow w+\i x\in\BigPar{\range T}{_\Cbb}.$\parSol{}
%	(c) $\forall w,x\in W,\exists\,!\,u,v\in V,T_{\!\Cbb}\Par{u+\i v}=w+\i x\Longleftrightarrow Tu=w,Tv=x.$ \Or By (b).}\large
%\PfEnd
%\SepLine

%\ProblemN{\Anchor{3A7}{7}}{
%	\TextA{Supp $\dim V = 1$ and $T\in\Lm{V}.$ Prove $\exists\,\lambda\in\Fbb,Tv = \lambda v,\forall v\in V$.}
%}Let $u$ be a non0 vec in $V\Rightarrow B_V=\Par{u}$. Becs $Tu\in V\Rightarrow Tu=\lambda u$ for some $\lambda$.\parSol{}
%Supp $v\in V\Rightarrow v=au,\,\exists\,!\,a\in\Fbb$. Then $Tv=T\Par{au}=\lambda au=\lambda v.$\PfEnd
%\SepLine

%\ProblemN{\Anchor{3A3}{3}}{
%	\TextA{Supp $T\in\Lm{\FbbP{n},\FbbP{m}}.$ Prove $\exists\,A_{j,k}\in\Fbb,\:T\Par{x_1,\dots,x_n}=\Big({{\sum_{i=1}^nA_{1,i}\,x_i,\cdots,\sum_{i=1}^nA_{m,i}\,x_i}}\Big).$\vspace{2pt}}
%}Let $T\Par{1,0,0,\dots,0,0}=\Par{A_{1,1},\dots,A_{m,1}},$\parSol{}
%\Blind{Let }$T\Par{0,1,0,\dots,0,0}=\Par{A_{1,2},\dots,A_{m,2}},$\vspace{-4pt}\parSol{}
%\Blind{Let $T\Par{1,0,0,}$}\!$\vdots$\vspace{-8pt}\parSol{}
%\Blind{Let }$T\Par{0,0,0,\dots,0,1}=\Par{A_{1,n},\dots,A_{m,n}}.$\PfEnd
%\SepLine

%\ProblemN{\Anchor{3A8}{8}}{
%	\TextA{Give a map $\varphi:\Rbb^2\rightarrow\Rbb$ suth $\forall a\in\Rbb,v\in\Rbb^2,\varphi\Par{av} = a\varphi\Par{v}$ but $\varphi$ is not liney.\vspace{4pt}}
%}Define $T\Par{x,y}=\MathLeftBrace{l}{x+y,\;\;\text{if }\Par{x,y}\in\Span{3,1},\\ 0,\hfill\text{othws}.
%}$ \quad\Or Define $T\Par{x,y}=\sqrt[3]{\BigPar{x^3+y^3}}.$\PfEnd
%\SepLine

%\ProblemN{\Anchor{3A9}{9}}{
%	\TextA{Give a map $\varphi:\Cbb\rightarrow\Cbb$ suth $\forall w,z\in\Cbb,\varphi\Par{w + z} = \varphi\Par{w} + \varphi\Par{z}$ but $\varphi$ is not liney.}
%}Define $\varphi\Par{u+\i v}=u=\REAL\Par{u+\i v}$\quad\Or Define $\varphi\Par{u+\i v}=v=\IMAGINARY\Par{u+\i v}.$\PfEnd
%\SepLine

%\Anchor{3A4e10}\ProblemB{
%	\TextB{Prove if $q\in\PoRi$ and $T:\PoRi\rightarrow\PoRi$ is defined by $\underbrace{Tp=q\circ p}_{composition},$ then $T$ is not liney.\vspace{-8pt}}
%}{\tgsc Composition and product are not the same in $\PoFi.$}\par\quad
%\NOTICE that $\Par{p\circ q}\Par{x}=p\BigPar{q\Par{x}},$ while $\Par{pq}\Par{x}=p\Par{x}q\Par{x}=q\Par{x}p\Par{x}.$\par\quad
%Becs in general, $\Sbra{q\circ \Par{p_1+\lambda p_2}}\Par{x}=q\BigPar{p_1\Par{x}+\lambda p_2\Par{x}}\neq \Par{qp_1}\Par{x}+\lambda \Par{qp_2}\Par{x}.$\par\quad
%\AExa Let $q$ be defined by $q\Par{x}=x^2,$ then $q\circ\BigPar{1+\Par{{-1}}}=0\neq q\Par{1}+q\Par{{-1}}=2.$\PfEnd
%\SepLine

%\ProblemN{\Anchor{3A10}{10}}{
%	\TextA{Supp $U$ is a subsp of $V$ with $U\neq V,$ and $S\in\Lm{U,W}$ is non0.\vspace{-2pt}}
%	\TextA{Prove $T$ is not liney, where we define $T: V\rightarrow W$ by $Tv={}${\FontNorm$\MathLeftBrace{l}{Sv,\;\text{if }v\in U,\\ 0,\Blind{S}\;\text{if }v\in V\Backslash U.}$}}
%}Asum $T$ is liney. Supp $v\in V\Backslash U,$ and $u\in U$ suth $Su\neq 0.$ Asum $v+u\in U\Rightarrow v\in U,$ ctradic.\parSol{}
%Now $v+u\in V\Backslash U\Rightarrow T\Par{v+u}=0=Tv+Tu=0+Su\Rightarrow Su=0.$ Ctradic.\PfEnd
%\SepLine

%\Anchor{3B7}\ProblemBnoor{3.B.7}{
%	\TextA{Supp $2\leqslant\dim V=n\leqslant m=\dim W$, if $W$ is finide.}
%	\TextA{Show $U=\Bra{T\in\Lm{V,W}:T\text{ is not inje}}$ is not a subsp of $\Lm{V,W}$.}
%}{\tgsl The set of all inje $T\in\Lm{V,W}$ is a not subsp either.}\par\quad
%Let $\Par{v_1,\dots,v_n}$ be a bss of $V$, $\Par{w_1,\dots,w_m}$ be liney indep in $W$. \Sbra{ $2\leqslant n\leqslant m.$ }\par\hspace{0pt}
%$\MathRightMid{l}{$
%	Define $T_{\!1}\in\Lm{V,W}$ as $T_{\!1}:\,\,\,\,v_1\mapsto 0,$\qquad$v_2\mapsto w_2,$\qquad$v_i\mapsto w_i$.$\\$
%	Define $T_{\!2}\in\Lm{V,W}$ as $T_{\!2}:\,\,\,\,v_1\mapsto w_1,$ \quad\,$v_2\mapsto 0,$\,\,\,\qquad$v_i\mapsto w_i$,\quad $i=3,\dots,n.}$ Thus $T_{\!1}+T_{\!2}\not\in U.$\PfEnd[-23pt]\vspace{10pt}
%\AComm If $\dim V=0,$ then $V=\zeroSubs=\Span{\,}$. $\forall T\in\Lm{V,W}, T$ is inje. Hence $U=\emptySet$.\parCom
%If $\dim V=1,$ then $V=\Span{v_0}.$ Thus $U=\Span{T_0},$ where $\forall v\in V,T_0 v=0\Rightarrow T_0=0.$\par
%\SepLine
%
%\Anchor{3B8}\ProblemBnoor{3.B.8}{
%	\TextA{Supp $2\leqslant\dim W=m\leqslant\dim V$, if $V$ is finide.}
%	\TextA{Show $U=\Bra{T\in\Lm{V,W}:T\text{ is not surj}}$ is not a subsp of $\Lm{V,W}$.}
%}{\tgsl The set of all surj $T\in\Lm{V,W}$ is not a subsp either. \tgsc Using the generalized version of [3.5].}\par\quad
%Let $\Par{v_1,\dots,v_n}$ be liney indep in $V$, $\Par{w_1,\dots,w_m}$ be a bss of $W$. \Sbra{  $n\in\;\!\!\Bra{m,m+1,\dots}$; $2\leqslant m\leqslant n.$ }\par\quad
%Define $T_{\!1}\in\Lm{V,W}$ as $T_{\!1}:\,\,\,\,v_1\mapsto 0,$\qquad$v_2\mapsto w_2,$\qquad$v_j\mapsto w_j,$\qquad$v_{m+i}\mapsto 0.$\par\quad
%Define $T_{\!2}\in\Lm{V,W}$ as $T_{\!2}:\,\,\,\,v_1\mapsto w_1,$\,\,\,\quad$v_2\mapsto 0,$\,\,\,\qquad$v_j\mapsto w_j,$\qquad$v_{m+i}\mapsto 0.$\par\quad
%\BigPar{ For each $j=2,\dots,m;\,\,i=1,\dots,n-m$, if $V$ is finide, othws let $i\in\Nbp.$ } Thus $T_{\!1}+T_{\!2}\not\in U.$\PfEnd\vspace{2pt}
%\AComm If $\dim W=0,$ then $W=\zeroSubs=\Span{\,}$. $\forall T\in\Lm{V,W}, T$ is surj. Hence $U=\emptySet.$\parCom
%If $\dim W=1,$ then $W=\Span{w_0}.$ Thus $U=\Span{T_0},$ where each $T_0v_i=0\Rightarrow T_0=0.$\SepLine

%\Anchor{3B'1}\ProblemB{
%	\TextB{Supp $S,T\in\Lm{V}.$ Prove or give a countexa\hspace{1pt}$:$}
%	\PrePa\TextB{$\null S\subseteq\null T\Rightarrow\range T\subseteq\range S;$ \; {\tgnr\large(b)} $\range T\subseteq\range S\Rightarrow\null S\subseteq\null T.$}
%}Let $B_V=\Par{v_1,v_2,v_3}.$ Countexas:\par{\hspace{0pt}}
%$\hText{$(a) Let $\\\,}\hText{S:v_1\mapsto 0;\;\;v_2\mapsto 0;\;\;v_3\mapsto v_2.\\T:v_1\mapsto 0;\;\;v_2\mapsto 0;\;\;v_3\mapsto v_3.}\MathLeftMid{l}{$
%	\!\!Then $\null S=\null T,$ but$\\$
%	\!\!$\range T=\Span{v_3}\nsubseteq\Span{v_2}=\null T.}$\par{\hspace{0pt}}
%$\hText{$(b) Let $\\\,}\hText{S:v_1\mapsto v_2;\;\;v_2\mapsto v_2;\;\;v_3\mapsto v_2.\\T:v_1\mapsto 0;\;\;v_2\mapsto 0;\;\;v_3\mapsto v_2.}\MathLeftMid{l}{$
%	\!\!Then $\range T=\range S,$ but$\\$
%	\!\!$\null S=\Span{v_1-v_2,v_2-v_3,v_3-v_1}\nsubseteq\Span{v_1,v_2}=\null T.}$\par\vspace{8pt}
%\SepLine

%\ProblemN{\Anchor{3A4}{4}}{
%	\TextA{Supp $T\in\Lm{V,W},$ $\Par{Tv_1,\dots,Tv_m}$ liney indep in $W$. Prove $\Par{v_1,\dots,v_m}$ liney indep.}
%}Let $a_1 v_1+\dots+a_m v_m=0\Rightarrow a_1 Tv_1+\dots+a_m Tv_m=0\Rightarrow a_1=\dots=a_m=0.$\PfEnd
%\SepLine

\Anchor{3ANFS}\ProblemBX{\NoteFor{$\FbbP{S}$}}{
	\TextB{\vspace{1pt}}
	\TextB{Supp $S\neq\emptySet,\,C_S=\Bra{\,f\in\FbbP{S}:\text{\ensuremath{\exists} finily many \ensuremath{x}, suth \ensuremath{f\Par{x}\neq0}}}.$ Then $C_S$ is a subsp of $\FbbP{S}.$\vspace{2pt}}
	\PrePa\TextB{If $S=\Bra{x_1,\dots,x_n}.$ Find a bss of $\FbbP{S}$ and conclude $\FbbP{S}=C_S.$\hfill\tgnr\FontNorm{$\FbbP{S}$ infinide $\Rightarrow S$ infini.}\vspace{2pt}}
	\PrePb\TextB{If $S$ has infily many elem. Prove $\FbbP{S}$ is infinide.\hfill\tgnr\FontNorm{$\FbbP{S}$ finide $\Rightarrow S$ fini.}\vspace{2pt}}
	\PrePc\TextB{Supp $V$ is on $\Fbb.$ Prove $\exists$ surj $T\in\Lm{C_V,V}.$\vspace{2pt}}
}(a) Define each $\,f_i\Par{x_j}=\delta_{i,j}.$ Supp $\,f\in C_S,$ let each $y_k=\,f\Par{x_k}=\BigPar{y_1\,f_1+\dots+y_n\,f_n}\Par{x_k}.$\parSol{\Ha}
Then $\,f=y_1\,f_1+\dots+y_n\,f_n\in\Span{\,f_1,\dots,f_n}.$ 又 If $\,f=0,$ then each $y_k=0.$\vspace{3pt}\parSol{}
(b) Let $S=\Bra{x_1,\cdots,x_n,\cdots}.$ Define each $\,f_i\Par{x_j}=\delta_{i,j}\Rightarrow f_i\in C_S.$ 又 $\Par{\,f_1,\cdots,f_n,\cdots}$ liney indep.\parSol{\Hb}
\ACoro $S$ fini $\Longleftrightarrow\FbbP{S}$ finide.\vspace{3pt}\parSol{}
(c) Define $T:C_V\rightarrow V$ by $T\Par{\,f}=\sum\,f\Par{x}x.$ Note that $f\Par{x}\neq0$ for finily many $x\in V.$\parSol{\Hc}
\uline{Becs for any $v\in V,\exists$ liney indep $\Par{v_1,\dots,v_n}$ suth $v=a_1v_1+\dots+a_nv_n.$} \Sbra{{\FontSmall\tgsl See higher courses}}\parSol{\Hc}
Define each $\:\!f\Par{v_k}=a_k$ and $f\Par{x}=0$ for $x\not\in\;\!\!\Bra{v_1,\dots,v_n}.$ Then $T\Par{\,f}=v.$\PfEnd
\SepLine

%\ProblemN{\Anchor{3A12}{12}}{
%	\TextA{Supp non0 $V$ is finide and $W$ is infinide. Prove $\Lm{V,W}$ is infinide.}
%}Let $B_V=\Par{v_1,\dots,v_n}.$ Let $\Par{w_1,\dots,w_m}$ be liney indep in $W$ for any $m\in\Nbp.$\par\vspace{-6pt}\quad
%Define $T_{x,y}:V\rightarrow W$ by \,$T_{x,y}\Par{v_z}=\delta_{z,x} w_y$, $\forall x\in\;\!\!\Bra{1,\dots,n},y\in\;\!\!\Bra{1,\dots,m}$, \,where $\delta_{z,x}=\MathLeftBrace{l}{
%	0,\quad z\neq x,\\
%	1,\quad z=x.}$\vspace{-5pt}\par\quad
%{\normalsize$\forall v=\sum_{i=1}^n a_iv_i,\;u=\sum_{i=1}^n b_iv_i,\;\lambda\in\Fbb,T_{x,y}\Par{v+\lambda u}=\Par{a_x+\lambda b_x}w_y=T_{x,y}\Par{v}+\lambda T_{x,y}\Par{u}.$}\vspace{2pt}\par\quad
%Linity checked. Now supp $a_1 T_{x,1}+\dots+a_m T_{x,m}=0$.\par\quad
%Then $\BigPar{a_1 T_{x,1}+\dots+a_m T_{x,m}}\Par{v_x}=0=a_1 w_1+\dots+a_m w_m\Rightarrow a_1=\dots=a_m=0.$ 又 $m$ arb.\par\quad
%Thus $\Par{T_{x,1},\dots,T_{x,m}}$ is a liney indep list in $\Lm{V,W}$ for any $x$ and len $m$. Hence by (2.A.14).\PfEnd
%\SepLine

\ProblemN{\Anchor{3A13}{13}}{
	\TextA{Supp $\Par{v_1,\dots,v_m}$ is liney dep in $V$ and $W\neq\zeroSubs$.}
	\TextA{Prove $\exists\,w_1,\dots,w_m\in W,\not\exists\,T\in\Lm{V,W}$ suth $Tv_k=w_k,\forall k = 1,\dots,m$.}
}\par\quad
We prove by ctradic. By liney dep lemma, $\exists\,j\in\;\!\!\Bra{1,\dots,m},v_j\in\Span{v_1,\dots,v_{j-1}}$.\par\quad
Supp $a_1 v_1+\dots+a_m v_m=0,$ where $a_j\neq 0.$ \;Now let $w_j\neq 0$, while $w_1=\dots=w_{j-1}=w_{j+1}=w_m=0.$\par\quad
Define $T\in\Lm{V,W}$ with each $Tv_k=w_k$. Then $T\BigPar{a_1 v_1+\dots+a_m v_m}=0=a_1 w_1+\dots+a_m w_m.$\par\quad
And \,$0=a_j w_j$ while $a_j\neq 0$ and $w_j\neq 0.$ Ctradic.\PfEnd\vspace{4pt}\quad
\Or We prove the ctrapos\hspace{1pt}: \,Supp $\forall w_1,\dots,w_m\in W,\exists\,T\in\Lm{V,W},$ each $Tv_k=w_k.$\par\quad
{Now we show $\Par{v_1,\dots,v_n}$ is liney indep. Supp {$\exists\,a_i\in\Fbb,a_1 v_1+\dots+a_n v_n=0$}.}\vspace{2pt}\par\quad
{Choose one $w\in W\nonzero.$ By asum, for {$\BigPar{\overline{a_1}w,\dots,\overline{a_m}w},\exists\,T\in\Lm{V,W},$ each $Tv_k=\overline{a_k}w.$}}\vspace{2pt}\par\quad
{Now we have {$ 0=T\BigBigPar{{\sum_{k=1}^m a_k v_k}}=\sum_{k=1}^m a_k Tv_k=\sum_{k=1}^m a_k\overline{a_k}w=\BigBigPar{{\sum_{k=1}^m \aMidsq{a_k}}}w$}.}\vspace{2pt}\par\quad
{Then {$\sum_{k=1}^m\aMidsq{a_k}=0.$ Thus $a_1=\dots=a_m=0.$} Hence $\Par{v_1,\dots,v_n}$ is liney indep.}\PfEnd
\SepLine

\Anchor{3A4e11}\ProblemBnoor{4E 11}{
	\TextA{Supp $V$ is finide, $T\in\Lm{V}$ is suth $\forall S\in\Lm{V},ST=TS$. Prove $\exists\,\lambda\in\Fbb,T=\lambda I$.}
}Asum $\exists\,v\in V,\Par{v,Tv}$ is liney indep. Let $B_V=\Par{v,Tv,u_1,\dots,u_n}$.\parSol{}
Define $S\in\Lm{V}$ by $S\Par{av+bTv+c_1 u_1+\dots+c_n u_n}=bv\Rightarrow S\Par{Tv}=v=T\Par{Sv}=0.$\parSol{}
Asum $V\neq\zeroSubs$ and $\forall v\in V,\Par{v,Tv}$ is liney dep, then $\exists\,\lambda_v\in\Fbb,Tv=\lambda_v v.$\parSol{}
To prove $\lambda_v$ is indep of $v$, we discuss in two cases:\vspace{2pt}\parSol{}
\!\!\!$\MathRightBrace{l}{$($-$) If $\Par{v,w}$ is liney indep, $\lambda_{v+w}\Par{v+w}=T\Par{v+w}=Tv+Tw=\lambda_v v+\lambda_w w\\$($=$) Othws, supp $w=cv$, $\lambda_w w=Tw=cTv=c\lambda_v v=\lambda_v w\Rightarrow\Par{\lambda_w-\lambda_v}w}\Rightarrow \lambda_w=\lambda_v.$\PfEnd\vspace{12pt}\quad
\Or \;\envFontLarge{\Large\vspace{4pt}Let $B_V=\Par{v_1,\dots,v_m}.$ Define {\Large$\varphi\in\Lm{V,\Fbb}$} by $\varphi\Par{v_1}=\cdots=\varphi\Par{v_m}=1.$}\par\quad
\Blind{\Or \;}{\Large\vspace{4pt}Supp $v\in V.$ Define $S_v\in\Lm{V}$ by $S_v \Par{u}=\varphi\Par{u}v.$}\par\quad
\Blind{\Or \;}{\Large Then $Tv=T\Par{\varphi\Par{v_1}v}=T\Par{S_v v_1}=S_v\Par{Tv_1}=\varphi\Par{Tv_1}v=\lambda v.$}\PfEnd\vspace{10pt}\quad
\Or \;{\vspace{5pt}Define {\Large$S_k\BigBigPar{{\sum_{i=1}^n a_i v_i}}=a_k v_k$}. Then {\Large$S_kv=v\Longleftrightarrow\exists\,!\,a_k\in\Fbb,v=a_kv_k$}.}\par\quad
\Blind{\Or \;}{\vspace{7pt}Hence {\Large$S_k\Par{Tv_k}=T\Par{S_kv_k}=Tv_k\Rightarrow Tv_k=a_k v_k$}. \;Becs $v_k$ is arb. Simlr to above. Done.}\par\quad
\Blind{\Or \;}{\vspace{3pt}\Or Define {\Large$A^{\SmallPar[1.5pt]{j,k}}\in\Lm{V}$} by {\Large$A^{\SmallPar[1.5pt]{j,k}}v_j=v_k,\;A^{\SmallPar[1.5pt]{j,k}}v_k=v_j,\;A^{\SmallPar[1.5pt]{j,k}}v_x=0,x\neq j,k$}.}\par\quad
\Blind{\Or \;}{\hspace{-2.7pt}$\hText{$Then$\\[6pt]\;}$ {\Large$\hMath{r}{\left|\;\;}{\;\right\}}{A^{\SmallPar[1.5pt]{j,k}}Tv_j=TA^{\SmallPar[1.5pt]{j,k}}v_j=Tv_k=a_kv_k\\A^{\SmallPar[1.5pt]{j,k}}Tv_j=A^{\SmallPar[1.5pt]{j,k}}a_jv_j=a_jA^{\SmallPar[1.5pt]{j,k}}v_j=a_jv_k}\Rightarrow a_k=a_j.$}} {Hence $a_k$ is indep of $v_k.$}\PfEnd
\SepLine

\Anchor{3A4e17}\ProblemBnoor{{4E 17}}{
	\TextB{Supp $V$ is finide. Show all two-sided ideals of $\Lm{V}$ are $\zeroSubs$ and $\Lm{V}$.}
	\TextB{{\FontNorm A subsp $\mE$ of $\Lm{V}$ is called a two-sided ideal of $\Lm{V}$ if $TE\in\mE,ET\in\mE,\,\,\forall E\in\mE,T\in\Lm{V}$.}}
}{If $\mE=\zeroSubs$, then done. Supp $0\neq S\in\mE,$ a two-sided ideal of $\Lm{V}$. Let $B_V=\Par{v_1,\dots,v_n}.$}\par\quad
Define $R_{x,y}\in\Lm{V}:\,v_x\mapsto v_y,\;v_z\mapsto 0\,\Par{z\neq x}$. \Or {$R_{x,y}v_z=\delta_{z,x}v_y$}. Asum each $R_{x,y}\in\mE$.\par\quad
Then $\BigPar{R_{1,1}+\dots+R_{n,n}}v_j=v_j\Rightarrow\sum_{r=1}^n R_{r,r}=I\Rightarrow\Lm{V}\ni T=I\circ T=T\circ I\in\mE.$\par\quad
\Or Let each $Tv_j=w_j=A_{1,j}v_1+\dots+A_{n,j}v_n\Rightarrow T=\sum_{x=1}^n\sum_{y=1}^nA_{y,x}R_{x,y}\in\mE.$  Now we prove the asum.\vspace{2pt}\par\quad
Supp $Sv_i\neq 0$ and $Sv_i=a_1 v_1+\dots+a_n v_n$, where $a_k\neq 0.$ We show $R_{k,y}SR_{x,i}=a_k R_{x,y}\in\mE.$\par\quad
Becs $SR_{x,i}=a_1R_{x,1}+\dots+a_kR_{x,k}+\dots+a_nR_{x,n}\in\mE,$ for all $x\in\;\!\!\Bra{1,\dots,n}.$\par\quad
\Or $\Par{R_{k,y}S}{v_i}=a_k v_y\Rightarrow\BigPar{\Par{R_{k,y}S}\circ R_{x,i}}{v_z}=\delta_{z,x}\Par{a_k v_y},$ for all $y\in\;\!\!\Bra{1,\dots,n}.$ Immed.\PfEnd\vspace{3pt}
\AComm Not true if infinide. Consider the subsp $X=\Bra{T\in\Lm{V}:\range T\text{ is finide}}.$\par\quad
For any $T\in X,\;\forall E\in\Lm{V},\range TE\subseteq\range T;\;\range ET=\Span{Ew_1,\dots,Ew_n}\Rightarrow TE,ET\in X.$
\SepLine

\Anchor{3B4e32}\ProblemBnoor{{4E 3.B.32}}{
	\TextA{Supp $\dim V=n.$ Supp $\varphi:\Lm{V}\rightarrow\Fbb$ is liney.}
	\TextA{Show if $\forall S,T\in\Lm{V},\varphi\Par{ST} = \varphi\Par{S}\cdot\varphi\Par{T}$, then $\varphi = 0$.}
}Using notats in (4E 17) and {\NOTEFOR} [3.60].\parSol{\vspace{2pt}}
Supp $\varphi\neq 0\Rightarrow\exists\,i,j\in\;\!\!\Bra{1,\dots,n},\,\varphi\Par{R_{i,j}}\neq 0$. \envFontLarge Becs {\Large\vspace{4pt}$R_{i,j}=R_{x,j}\circ R_{i,x},\,\,\forall x=1,\dots,n$}\parSol{}
{\Large\vspace{4pt}$\Rightarrow\varphi\Par{R_{i,j}}=\varphi\Par{R_{x,j}}\cdot\varphi\Par{R_{i,x}}\neq 0\Rightarrow\varphi\Par{R_{x,j}}\neq 0$ {\large and} $\varphi\Par{R_{i,x}}\neq 0.$}\parSol{}
{\vspace{4pt}Again, becs {\Large$R_{i,x}=R_{y,x}\circ R_{i,y},\,\,\forall y=1,\dots,n.$} \;Thus {\Large$\varphi\Par{R_{y,x}}\neq 0,\;\forall x,y=1,\dots,n$}.}\parSol{}
{Let $k\neq i,j\neq l$ and then {\Large\vspace{4pt}$\varphi\Par{R_{i,j}\circ R_{l,k}}=\varphi\Par{R_{l,k}\circ R_{i,j}}=\varphi\Par{0}=0=\varphi\Par{R_{l,k}}\cdot\varphi\Par{R_{i,j}}$}}\parSol{}
{\vspace{4pt}{\Large$\Rightarrow\varphi\Par{R_{l,k}}=0$} or {\Large$\varphi\Par{R_{i,j}}=0$}. Ctradic.\PfEnd}\parSol{\vspace{4pt}}
\FontNorm\Or {Becs $\exists\,S,T\in\Lm{V},ST-TS\neq 0.$ While $\varphi\Par{ST-TS}=\varphi\Par{S}\varphi\Par{T}-\varphi\Par{T}\varphi\Par{S}=0.$}\parSol{}
\Blind{\Or }{Note that $\forall E\in\null\varphi,T\in\Lm{V},\varphi\Par{ET}=\varphi\Par{TE}=0\Rightarrow ET,TE\in\null\varphi.$ By (4E 17).}\PfEnd
\SepLine

%\Anchor{3A5}\ProblemN[]{5}{
%	\TextA{Becs $\Lm{V,W}=\Bra{T:V\rightarrow W\;{\envFontHuge\envFontA\mmid}\;T\text{ is liney}}$ is a subsp of $W^V,$ $\Lm{V,W}$ is a vecsp.}
%}\SepLine

\Anchor{3A'1}\ProblemB{
	\TextB{Given the fact that $\Lm{V,W}$ is a vecsp. Prove or give a countexa\hspace{1pt}$:$ $V,W$ are vecsps.}
	\TextB{\FontNorm By [3.2], the add and homo imply that $V$ is closd add and scalar multi. While $W^V$ might not be a vecsp.}
}We can assure that $\zeroSubs\subseteq\Lm{V,W},\zeroSubs\subseteq V,\zeroSubs\subseteq W.$\par\quad
(I) If $W^V=\zeroSubs.$ Then $\Lm{V,W}=\zeroSubs.$\par\quad\HI
And $W=\zeroSubs,$ for if not, $\exists\,w\in W\nonzero,$ define a map $f$ by $f\Par{x}=w,\forall x\in V.$\par\quad\HI
And $V$ might not be a vecsp. \AExa Let $V=\Rbb,$ but with the scalar multi defined by $a\odot v=0.$\par\quad\EndI
(II) If $W^V$ is a non0 vecsp $\Longleftrightarrow W$ is a non0 vecsp.\par\quad\HII
(a) If $\Lm{V,W}=\zeroSubs,$ then by Exa (I), $V$ might not be vecsp.\par\quad\HII
(b) If not, then $\exists\;T\in\Lm{V,W},\,T\neq 0.$ Which means $\exists\,v\in V,Tv\neq 0\Rightarrow v\neq 0.$\colorbox{yellow}{TODO}\par\quad\HII\Hb
Then both $W$ and $V$ have a non0 elem.\par\quad\HII\Hb
(i) If $\exists$ inje $T\in\Lm{V,W},$ then $T\Par{u+v}=T\Par{v+u}\Rightarrow u+v=v+u.$ etc. Hence $V$ is a vecsp.\par\quad\HII\Ha\Endi
(ii) If not, then we cannot guarantee that $V$ is a vecsp. Exa: ???\par\quad\EndII
(III) If $W^V$ is not a vecsp $\Longleftrightarrow W$ is not a vecsp.\par\quad\HIII
(a) If $\Lm{V,W}=\zeroSubs,$ then by Exa (I), $V$ might not be vecsp.\par\quad\HIII
(b) If not.\PfEnd
\SepLine
\ChEnd\pagebreak


\ChDecl{Ch3B}{3$\cdot$B}{\quad{\small\textbf{注意\,:}\;这里我将3.D可逆性、同构部分前置。}}

\vspace{4pt}

%\Anchor{3B3}\ProblemN[]{3}{
%	\TextA{Supp $\Par{v_1,\dots,v_m}$ in V. Define $T\in\Lm{\FbbP{m}, V}$ by $T\Par{z_1,\dots,z_m}=z_1 v_1+\dots+z_m v_m.$\vspace{2pt}}
%	\PrePa\TextA{The surj of $T$ corres to $\Par{v_1,\dots,v_m}$ spanning $V$.\hfill\FontNorm$\range T=\Span{v_1,\dots,v_m}=V.$}
%	\PrePb\TextA{The inje of $T$ corres to $\Par{v_1,\dots,v_m}$ being liney indep.\hfill\FontNorm $\Par{v_1,\dots,v_m}$ liney indep $\Longleftrightarrow T$ inje.}
%}
%\AComm Let $\Par{e_1,\dots,e_m}$ be std bss of $\FbbP{m}.$ Then $T{e_k}=v_k.$
%\SepLine
%
%\ProblemN{\Anchor{3B9}{9}}{
%	\TextA{Supp $\Par{v_1,\dots,v_n}$ is liney indep. Prove for any inje $T,\,\Par{Tv_1,\dots,Tv_n}$ is liney indep.}
%}$a_1 Tv_1+\dots+a_n Tv_n=0=T\BigBigPar{{\sum_{i=1}^n a_i v_i}}\Longleftrightarrow \sum_{i=1}^n a_i v_i=0\Longleftrightarrow a_1=\dots=a_n=0.$\PfEnd
%\SepLine

%\ProblemN{\Anchor{3B10}{10}} {
%	\TextA{Supp $\Span{v_1,\dots,v_n}=V$. Show $\Span{Tv_1,\dots,Tv_n}=\range T$.}
%}(a) $\range T=\Bra{Tv:v\in\Span{v_1,\dots,v_n}}\Rightarrow Tv_1,\dots,Tv_n\in\range T.$ By [2.7].\parSol{\Ha}
%\Or $\Span{Tv_1,\dots,Tv_n}\ni a_1 Tv_1+\dots+a_n Tv_n=T\Par{a_1 v_1+\dots+a_n v_n}\in\range T.$\parSol{}
%(b) $\forall w\in\range T,w=Tv,\exists\,v\in V\Rightarrow\exists\,a_i\in\Fbb,v=\sum_{i=1}^na_i v_i,w=a_1 Tv_1+\dots+a_n Tv_n.$\PfEnd
%\SepLine

%\ProblemN{11}{
%	\TextB{Supp $S_1,\dots,S_n$ are liney and inje suth $S_1S_2\cdots S_n$ makes sense. Prove $S_1S_2\cdots S_n$ inje.}
%}$S_1S_2\cdots S_nv=0\Rightarrow S_2\cdots S_nv=0\Rightarrow\cdots\Rightarrow S_nv=0\Rightarrow v=0.$\PfEnd
%\SepLine

%\Anchor{5A4e33}\ProblemBnoor{4E 5.A.33}{
%	\TextB{Supp $T\in\Lm{V},m\in\Nbp.$ Prove $T$ inje $\Longleftrightarrow T^m$ inje, and $T$ surj $\Longleftrightarrow$ $T^m$ surj.}
%}(a) $T^m$ inje $\Rightarrow$ if $Tv=0,$ then $T^{m-1} Tv=T^m v=0\Rightarrow v=0,$ thus $T$ inje. \,Convly immed.\vspace{2pt}\parSol{}
%(b) $T^m$ surj $\Rightarrow\forall u\in V,\exists\,v\in V\Rightarrow\exists\,w=T^{m-1}v,\;T^m v=u=Tw.$\parSol{\Hb}
%$T$ surj $\Rightarrow\forall u\in V,\exists\,v_1,\dots,v_m\in V,\,T\Par{v_1}=T^2v_2=\cdots=T^m v_m=u.$\PfEnd
%\SepLine

%\ProblemN{\Anchor{3B16}{16}}{
%	\TextA{Supp $T\in\Lm{V}$ suth $\null T,\range T$ are finide. Prove $V$ is finide.}
%}Let $B_{\range T}=\Par{Tv_1,\dots,Tv_n},B_{\null T}=\Par{u_1,\dots,u_m}.$\parSol{}
%$\forall v\in V,\exists\,!\,a_i\in\Fbb,T\Par{v-a_1 v_1-\dots-a_n v_n}=0\Rightarrow\exists\,!\,b_i\in\Fbb,v-\sum_{i=1}^na_iv_i=\sum_{i=1}^mb_iu_i.$\PfEnd
%\SepLine
%
%\ProblemN{\Anchor{3B17}{17}}{
%	\TextA{Supp $V,W$ are finide. Prove $\exists$ inje $T\in\Lm{V,W}\Longleftrightarrow\dim V\leqslant\dim W$.}
%}(a) Supp $\exists$ inje $T$. Then $\dim V=\dim\range T\leqslant\dim W$.\parSol{}
%(b) Supp $\dim V\leqslant\dim W.$ Let $B_V=\Par{v_1,\dots,v_n},B_W=\Par{w_1,\dots,w_m}.$ Define each $Tv_i=w_i.$\PfEnd
%\SepLine
%
%\ProblemN{\Anchor{3B18}{18}}{
%	\TextA{Supp $V,W$ are finide. Prove $\exists$ surj $T\in\Lm{V,W}\Longleftrightarrow\dim V\geqslant\dim W$.}
%}(a) Supp $\exists$ surj $T$. Then $\dim V=\dim W+\dim\null T\Rightarrow\dim W\leqslant\dim V$.\parSol{}
%(b) Supp $\dim V\geqslant\dim W.$ Let $B_V=\Par{v_1,\dots,v_n},B_W=\Par{w_1,\dots,w_m}.$\parSol{\Hb}
%Define $T\in\Lm{V,W}$ by $T\BigPar{a_1 v_1+\dots+a_m v_m+\dots+a_n v_n}=a_1 w_1+\dots+a_m w_m.$\PfEnd
%\SepLine
%
%\ProblemN{\Anchor{3B19}{19}}{
%	\TextA{Supp $V,W$ are finide, $U$ is a subsp of $V$.}
%	\TextA{Prove $\exists\,T\in\Lm{V,W},\,\,\null T = U\Longleftrightarrow\underbrace{\dim U}_m\geqslant\underbrace{\dim V}_{m\,+\,n}-\underbrace{\dim W}_{p}.$}\vspace{-10pt}
%}\par\quad
%(a) Supp $\exists\,T\in\Lm{V,W},\null T=U.$ Then $\dim U+\dim\range T=\dim V\leqslant\dim U+\dim W.$\par\quad
%(b) Let $B_U=\Par{u_1,\dots,u_m},B_V=\Par{u_1,\dots,u_m,v_1,\dots,v_n},B_W=\Par{w_1,\dots,w_p}.$ Supp $p\geqslant n.$\par\quad\Hb
%Define $T\in\Lm{V,W}$ by $T\Par{a_1 v_1+\dots+a_n v_n+b_1 u_1+\dots+b_m u_m}=a_1 w_1+\dots+a_n w_n.$\PfEnd
%\SepLine

\Anchor{3BT1}\ProblemBX[]{\TipsN{1}}{
	\TextA{Supp $U$ is a subsp of $V.$ Then for $T\in\Lm{V,W},\:U\cap\null T=\null T\mmid_U.$\vspace{0pt}}
}%Note that $U\cap\null T\subseteq\null T\mmid_U.$ On the other hand, supp $u\in\null T\mmid_U\subseteq U.$\parSol{}
%Then $T\mmid_U\Par{u}=0$ makes sense and equals $Tu.$ Now $Tu=0\Rightarrow u\in\null T.$\PfEnd

\Anchor{3BT2}\ProblemBX{\TipsN{2}}{
	\TextA{Supp $T\in\Lm{V,W}.$ Let $V=M+N,\;U=X+Y.$\vspace{0pt}}
	\TextA{Then $\range T=\range T\mmid_M+\range T\mmid_N,\;\range T\mmid_U=\range T\mmid_X+\range T\mmid_Y.$\vspace{1pt}}
	(a) \TextA{If $T\mmid_U$ is inje. Show $U=X\oplus Y\Longleftrightarrow\range T\mmid_U=\range T\mmid_X\oplus\range T\mmid_Y.$\vspace{1pt}}
	(b) \TextA{Give an exa suth $V=M\oplus N,\;\range T\neq\range T\mmid_M\oplus\range T\mmid_N.$}
}Supp $U=X\oplus Y.$ Asum for some $u\in U,$ there exis two disti pairs $\Par{x_1,y_1},\Par{x_2,y_2}$ in $X\times Y$\parSol{}
suth $Tu=Tx_1+Ty_1=Tx_2+Ty_2.$ Becs $\forall u\in U,\exists\,!\,\Par{x,y}\in X\times Y,v=x+y.$\parSol{}
Now $T\Par{x_1+y_1}=T\Par{x_2+y_2}\Longrightarrow x_1+y_1=x_2+y_2\Longrightarrow\Par{x_1,y_1}=\Par{x_2,y_2},$ ctradic.\parSol{}
Thus $\forall u\in U,\exists\,!\,\Par{Tx,Ty}\in\range T\mmid_X\times\range T\mmid_Y,Tu=Tx+Ty.$ \quad Convly, becs $T$ inje.\PfEnd\vspace{2pt}
\AExa Let $B_V=\Par{v_1,v_2,v_3},B_W=\Par{w_1,w_2},\;T:v_1\mapsto 0,\;v_2\mapsto w_1,\;v_3\mapsto w_2.$\parExa
Let $B_M=\Par{v_1-v_2,v_3},B_N=\Par{v_2}.$ Then $\range T\mmid_M=\Span{w_1,w_2},\range T\mmid_N=\Span{w_1}$\par\vspace{2pt}
\AComm Also $\null T\mmid_M=\null T\mmid_N=\zeroSubs.$ Hence $\null T\neq\null T\mmid_M\oplus\null T\mmid_N.$
\SepLine

\ProblemN{\Anchor{3B12}{12}}{
	\TextA{Prove $\forall T\in\Lm{V,W},\exists$ subsp $U$ of $V$ suth}
	\TextA{{\large$U\cap\null T$}${}=\null T\mmid_U=\zeroSubs,\;\range T={}${\FontNorm$\Bra{Tu:u\in U}$}${}=\range T\mmid_U.$}
	\TextA{\FontNorm Which is equiv to $T\mmid_U:U\rightarrow\range T$ being iso.}
}By [2.34] \BigPar{note that $V$ can be infinide}, $\exists$ subsp $U$ of $V$ suth $V=U\oplus\null T$.\parSol{}
$\forall v\in V,\exists\,!\,w\in\null T,u\in U, v=w+u.$ Then $Tv=T\Par{w+u}=Tu\in\;\!\!\Bra{Tu:u\in U}.$\PfEnd\vspace{6pt}
\ACoro {\FontLarge\tgsl[P] \quad{$T\mmid_U:U\rightarrow\range T$ is iso $\Longleftrightarrow U\oplus\null T=V.$}\quad [Q]}\vspace{2pt}\parCor
We have shown $Q\Rightarrow P.$ Now we show $P\Rightarrow Q$ to complete the proof.\parCor
$\forall v\in V,Tv\in\range T=\range T\mmid_U\Rightarrow \exists\,!\,u\in U,Tv=Tu\Rightarrow v-u\in\null T.$\parCor Thus $v=\Par{v-u}+u\in U+\null T.$ 又 $U\cap\null T=\null T\mmid_U.$\PfEnd\vspace{4pt}\parCor
\Or ${}^\neg Q\Rightarrow{}^\neg P:$ \;Becs $U\oplus\null T\subsetneq V.$ We show $\range T\neq\range T\mmid_U$ by ctradic.\parCor
Let $X\oplus\Par{U\oplus\null T}=V.$ Now $\range T=\range T\mmid_X\oplus\range T\mmid_U.$ And $X$ is non0.\parCor
Asum $\range T=\range T\mmid_U.$ Then $\range T\mmid_X=\zeroSubs.$ While $T\mmid_X$ is inje. Ctradic.\parCor
\Or $\range T\mmid_X\subseteq\range T\mmid_U\Rightarrow \forall x\in X,Tx\in\range T\mmid_U,\exists\,u\in U,Tu=Tx\Rightarrow x=0.$\vspace{4pt}\parCor
Also, ${}^\neg P\Rightarrow{}^\neg Q:$ \;(a) $\range T\mmid_U\subsetneq\range T;\;$ {\OR} (b) $U\cap\null T\neq\zeroSubs.$\parCor
For (a), $\exists\,x\in V\Backslash U,Tx\neq 0\Longleftrightarrow x\not\in\null T.$ Thus $U+\null T\subsetneq V.$ For (b), immed.\PfEnd\vspace{4pt}
\AComm If $T\mmid_U:U\rightarrow\range T$ is iso. Let $R\oplus U=V.$ Then $R$ might not be $\null T.$\parCom
\Or Extend $B_U$ to $B_V=\Par{u_1,\dots,u_n,r_1,\dots,r_m},$ then $\Par{r_1,\dots,r_m}$ might not be a $B_{\null T}.$\vspace{-2pt}
\SepLine

%\BulletPointX\AComm Extend $B_U=\Par{u_1,\dots,u_n}$ to $B_V=\Par{u_1,\dots,u_n,v_1,\dots,v_m}.$\parCom\IndentB{}
%Then $\null T$ might not equal to $\Span{v_1,\dots,v_m}.$\par
%\SepLine

\Anchor{3BT3}\BulletPointX\TipsN{3}\,\,\,{Supp $T\in\Lm{V,W}$ and $U$ is a subsp suth $V=U\oplus\null T.$ Let $\null T=X\oplus Y.$}\TextB{}
Now $\forall v\in V,\exists\,!\,u_v\in U,\Par{x_v,y_v}\in X\times Y,v=u_v+x_v+y_v.$ Define $i\in\Lm{V,U\oplus X}$ by $i\Par{v}=u_v+x_v.$\TextB{}
\uline{Then $T=T\circ i.$} Becs $\forall v\in V,T\Par{v}=T\Par{u_v+x_v+y_v}=T\Par{u_v}=T\Par{u_v+x_v}=T\BigPar{i\Par{v}}=\Par{T\circ i}\Par{v}.$\par
\SepLine

\Anchor{3BT4}\BulletPointX\TipsN{4}\,\,\,Let $B_{\range T}=\Par{Tv_1,\dots,Tv_n}\Rightarrow R=\Par{v_1,\dots,v_n}$ is liney indep in $V.$ Let $\spn R=U.$\TextB{\vspace{3pt}}
\IndentTipsN{4}(a) $T{\BigBigPar{{\sum_{i=1}^n a_i v_i}}}=0\Longleftrightarrow\sum_{i=1}^n a_i Tv_i=0\Longleftrightarrow a_1=\dots=a_n=0.$ Thus $U\cap\null T=\zeroSubs.$\TextB{\vspace{4pt}}
\IndentTipsN{4}(b) $Tv=\sum_{i=1}^n a_i Tv_i\Longleftrightarrow v-\sum_{i=1}^n a_i v_i\in\null T\Longleftrightarrow v=\BigBigPar[0pt]{v-\sum_{i=1}^n a_i v_i}+\BigBigPar{{\sum_{i=1}^n a_i v_i}}.$\par\vspace{2pt}\IndentB{}\Hb{}
\IndentTipsN{4}Thus $U+\null T=V.$ \;\Or $\range T=\Bra{Tu:u\in U}=\range T\mmid_U.$ Using Exe (12).\PfEnd\vspace{4pt}
\IndentTipsN{4}\ACoro Convly if $U\oplus\null T=V$ and $B_U=\Par{v_1,\dots,v_n},$ then $B_{\range T}=\Par{Tv_1,\dots,Tv_n}.$
%\parCor
%\IndentTipsN{4}Becs $\range T=\range T\mmid_U=\Span{Tv_1,\dots,Tv_n},$ 又 $T\mmid_U$ is inje.
\SepLine

\Anchor{3BT5}\ProblemBX[]{\TipsN{5}}{
	Supp $S\in\Lm{U,V}$ is surj. Define $\mB\in\Lm[\BigPar]{\Lm{V,W},\Lm{U,W}}$ by $\mB\Par{T}=TS.$\TextA{}
	Then $\mB$ is inje. Becs {$\mB\Par{T}=TS=0\Longleftrightarrow T\mmid_{\range S}=0.$}\TextA{\vspace{-3pt}}
}\SepLine

%\ProblemB{
%	\TextB{Supp $V$ is finide, $T\in\Lm{V,W},B_{\range T}=\BigPar{Tv_1,\dots,Tv_n},B_V=\BigPar{v_1,\dots,v_n,u_1,\dots,u_m}$.\vspace{3pt}}
%	\TextB{Prove or give a countexa\hspace{1pt}$:$ $\Par{u_1,\dots,u_m}$ is a bss of $\null T$.}
%}{{\tgsl Always notice that $\Scom{V}{\Span{v_1,\dots,v_n}}=\Bra{U_1,\cdots,\null T,\cdots,U_n,\cdots}.$}}\par\quad
%A countexa: Let $\dim V=3, Tv_1=Tv_2=Tv_3=w_1.$ Then $\Span{Tv_1,Tv_2,Tv_3}=\Span{w_1}$.\par\quad
%Extend $\Par{v_i}$ to $\Par{v_1,v_2,v_3}$ for each $i$. But none of $\Par{v_1,v_2},\Par{v_1,v_3},\Par{v_2,v_3}$ is a bss of $\null T.$\PfEnd
%\SepLine

%\Anchor{3D4e15}\ProblemBnoor{4E 3.D.15}{
%	\TextB{Supp $T\in\Lm{V}$ and $V=\Span{Tv_1,\dots,Tv_m}$. Prove $V=\Span{v_1,\dots,v_m}$.}
%}Becs $V=\Span{Tv_1,\dots,Tv_m}\Rightarrow T$ surj $\Rightarrow T,T^{-1}$ inv.\parSol{}
%$\forall v\in V,\exists\,a_i\in\Fbb, v=\sum_{i=1}^ma_i Tv_i\Rightarrow T^{-1}v=\sum_{i=1}^ma_i v_i\Rightarrow \range T^{-1}\subseteq\Span{v_1,\dots,v_m}.$\vspace{2pt}\parSol{}
%\Or Reduce to $B_V=\Par{Tv_{\:\!\!\alpha_1},\dots,Tv_{\:\!\!\alpha_n}}.$\PfEnd
%\SepLine

\Anchor{3B4e27}\Anchor{5BI4}\ProblemBnoor{{4E 27}}{
	\TextB{Supp $P\in\Lm{V}$ and $P^2 = P$. Prove $V=\null P\oplus\range P$.\vspace{0pt}}
}(a) If $v\in\null P\cap\range P\Rightarrow Pv=0,$ and $\exists\,u\in V,v=Pu.$ Then $v=Pu=P^2 u=Pv=0.$\parSol{}
(b) Note that $\forall v\in V,v=Pv+\Par{v-Pv}$ and $P\Par{v-Pv}=0\Rightarrow v-Pv\in\null P.$\parSol{\Hb}
\Or Becs $\dim V=\dim\null P+\dim\range P=\Dim\BigPar{\null P\oplus\range P}.$\PfEnd{\vspace{3pt}}\parSol{}
\Or Becs $P\mmid_{\range P}:Pv\mapsto Pv^2=Pv\Rightarrow P\mmid_{\range P}=I$ is iso. By {\COROLLARY} in Exe (12).\PfEnd
\SepLine

\Anchor{3B4e21}\ProblemBnoor{{4E 21}}{
	\TextA{Supp $V$ is finide, $T\in\Lm{V,W},$ $Y$ is a subsp of $W$. Let $\mathcal{K}_{\!Y}=\Bra{v\in V: Tv\in Y}.$\vspace{1.5pt}}
	\TextA{Then $\mathcal{K}_{\!Y}$ is a subsp. Prove $\mathcal{K}_{\!Y}=\dim\null T +\Dim\BigPar{Y\cap\range T}.$}
}Define the range-restr map $R$ of $T$ by $R=T\mmid_{\mathcal{K}_{\!Y}}\in\Lm{\mathcal{K}_{\!Y},Y}.$ Now $\range R=Y\cap \range T.$\parSol{}
And $v\in\null T\Longleftrightarrow Tv=0\in Y\Longleftrightarrow Rv=0\in\range T\Longleftrightarrow v\in\null R.$ By [3.22].\PfEnd\vspace{2pt}
\AComm Now $\Span{v_1,\dots,v_m}\oplus\null T=\mathcal{K}_{\!Y}.$ {Where $B_{Y\,\cap\,\range T}=\Par{Tv_1,\dots,Tv_m}.$}\vspace{0pt}\parCom
In particular, $\dim\mathcal{K}_{\!\range T}=\dim\null T+\dim\range T\Longrightarrow\mathcal{K}_{\!\range T}=V.$
\SepLine

\Anchor{3B4e31}\ProblemBnoor{{4E 31}}{
	\TextA{Supp $V$ is finide, $X$ is a subsp of $V$, and $Y$ is a finide subsp of $W$.}
	\TextA{Prove if $\dim X + \dim Y = \dim V$, then $\exists\,T\in\Lm{V,W},\null T = X,\range T = Y$.}
}Let $V=U\oplus X,B_U=\Par{v_1,\dots,v_m}.$ \,Then $\forall v\in V,\exists\,!\,a_i\in\Fbb,x\in X,\;v=\sum_{i=1}^m a_i v_i+x.$\parSol{}
Let $B_Y=\Par{w_1,\dots,w_m}.$ Define $T\in\Lm{V,W}$ with each $T{v_i}=w_i,\,Tx=0.$\parSol{}
Now $v\in\null T\Longleftrightarrow Tv=a_1w_1+\dots+a_mw_m=0\Longleftrightarrow v=x\in X.$ \;Hence $\null T=X.$\parSol{}
And $Y\ni w=a_1 w_1+\dots+a_m w_m=a_1 Tv_1+\dots+a_m Tv_m\in\range T.$ \;Hence $\range T=Y.$\parSol{}
\Or \NOTICE that $V=U\oplus\null T.$ By Exe (12), $\range T=\range T\mmid_U.$\parSol{}
{\Blind{\Or}}又 $\dim\range T\mmid_U=\dim U=\dim Y;\;\range T\subseteq Y.$\par\quad
\Or Let $B_X=\Par{x_1,\dots,x_n}.$ Now $\range T=\Span[\BigPar]{Tv_1,\dots,Tv_m,Tx_1,\dots,Tx_n}=\Span{w_1,\dots,w_m}=Y.$\PfEnd
\SepLine

%\ProblemN{\Anchor{3B31}{31}}{
	%	\TextA{Prove $\exists\,T_{\!1},T_{\!2}\in\Lm{\Rbb^5,\Rbb^2},\null T_{\!1}=\null T_{\!2}$ and $T_{\!1}\neq cT_{\!2},\forall c\in\Fbb$.}
	%}Let $B_{\Rbb^5}=\Par{v_1,\dots,v_5},B_{\Rbb^2}=\Par{w_1,w_2}.$ Define $T,S\in\Lm{V,W}$ by\parSol{\vspace{2pt}}
%\;\,$\MathRightBrace{l}{Tv_1=w_1,\quad Tv_2=w_2,\quad\,\, Tv_3=Tv_4=Tv_5=0\\\,Sv_1=w_1,\quad Sv_2=2w_2,\quad\, Sv_3=Sv_4=Sv_5=0}\Rightarrow\null T=\null S.$\parSol{\vspace{4pt}}
%Supp $T=\lambda S$. Then $w_1=Tv_1=\lambda Sv_1=\lambda w_1\Rightarrow \lambda=1$.\parSol{\quad\qquad\qquad\qquad\,}
%While $w_2=Tv_2=\lambda Sv_2=2\lambda w_2\Rightarrow \lambda=\frac{\;1\;}{2}$. Ctradic.\PfEnd
%\SepLine

\Anchor{3B20}\Anchor{3B21}\ProblemN{20, 21}{
	(a) \TextA{Prove if $ST=I\in\Lm{V},$ then $T$ is inje and $S$ is surj.}
	(b) \TextA{Supp $T\in\Lm{V,W}.$ Prove if $T$ is inje, then $\exists$ surj $S\in\Lm{W,V},\;ST=I$.}
	(c) \TextA{Supp $S\in\Lm{W,V}.$ Prove if $S$ is surj, then $\exists$ inje $T\in\Lm{V,W},\;ST=I.$\vspace{-6pt}}
}\par\quad
(a) $Tv=0\Rightarrow S\Par{Tv}=0=v.$ \Or $\null T\subseteq\null ST=\zeroSubs.$\par\quad\Ha
$\forall v\in V,ST\Par{v}=v\in\range S.$ \;\Or $V=\range ST\subseteq\range S.$\par\vspace{2pt}\quad
(b) Define $S\in\Lm{\range T,V}$ by $Sw=T^{-1}w,$ {where $T^{-1}$ is the inv of $T\in\Lm{V,\range T}.$}\par\quad\Hb
Then extend to $S\in\Lm{W,V}$ by (3.A.11). Now $\forall v\in V,STv=T^{-1}Tv=v.$\par\vspace{3pt}\quad\Hb
\Or \Sbra[3pt]{{\tgsl Req $V$ Finide}} \;Let $B_{\range T}=\Par{Tv_1,\dots,Tv_n}\Rightarrow B_V=\Par{v_1,\dots,v_n}.$ Let $U\oplus\range T=W$.\par\quad\Hb
Define $S\in\Lm{W,V}$ with each $S\Par{Tv_i}=v_i,\,Su=0$ for $u\in U.$ Thus $ST=I.$\par\vspace{6pt}\quad
(c) By Exe (12), $\exists$ subsp $U$ of $W,W=U\oplus\null S,\;\range S=\range S\mmid_U=V.$\par\quad\Hc
Note that $S\mmid_U:U\rightarrow V$ is iso. Define $T=\BigPar{S\mmid_U}{^{-1}},$ where $\BigPar{S\mmid_U}{^{-1}}:V\rightarrow U.$\par\quad\Hc
Then $ST=S\circ\BigPar{S\mmid_U}{^{-1}}=S\mmid_U\circ\BigPar{S\mmid_U}{^{-1}}=I_V.$\par\vspace{4pt}\quad\Hc
\Or \Sbra[3pt]{{\tgsl Req $V$ Finide}} \;Let $B_{\range S}=B_{V}=\Par{Sw_1,\dots,Sw_n}\Rightarrow\Span{w_1,\dots,w_n}\oplus\null S=W.$\par\quad\Hc
Define $T\in\Lm{V,W}$ by $T\Par{Sw_i}=w_i.$ Now $ST\BigPar{a_1Sw_1+\dots+a_nSw_n}=\Par{a_1Sw_1+\dots+a_nSw_n}.$\PfEnd
\SepLine\pagebreak

\ProblemN{\Anchor{3B22}{22}}{
	\TextA{Supp $U,V$ are finide, $S\in\Lm{V,W},T\in\Lm{U,V}$.}
	\TextA{Prove $\dim\null ST=\dim \null S\mmid_{\range T} + \dim \null T$.\vspace{2pt}}
}Becs (a) \uline{$\range T\mmid_{\null ST}=\range T\cap\null S=\null S\mmid_{\range T}\,,$}\parSol{}
\Blind{Becs} (b) \uline{$\null T\mmid_{\null ST}=\null T\cap\null ST=\null T.$} By [3.22]\PfEnd\parSol{\vspace{6pt}}
%(a) $Tu\in\range T\mmid_{\null ST}\Longleftrightarrow S\Par{Tu}=0\Longleftrightarrow Tu\in\null S\cap\range T=\null S\mmid_{\range T}.$\par\quad
%(b) $u\in\null T\mmid_{\null ST}=\null T\cap\null ST\Longleftrightarrow Tu=0\Longleftrightarrow u\in\null T.$\PfEnd\vspace{6pt}\quad
\Or \NOTICE that \uline{$u\in\null ST\Longleftrightarrow S\Par{Tu}=0\Longleftrightarrow Tu\in\null S.$}\parSol{}
\Blind{\Or}Thus \uline{$\Bra{u\in U:Tu\in\null S}=\mathcal{K}_{\!\null S\;\cap\;\range T}=\null ST.$} By Exe (4E 21).\PfEnd\vspace{6pt}
\ACoro (1) $T$ surj $\Rightarrow\dim\null ST=\dim\null S+\dim\null T.$\parCor
(2) $T$ inv $\Rightarrow\dim\null ST=\dim\null S,\;\null ST=\null T.$\parCor
(3) $S$ inje $\Rightarrow\dim\null ST=\dim\null T.$
\SepLine

\ProblemN{\Anchor{3B23}{23}}{
	\TextA{Supp $V$ is finide, $S\in\Lm{V,W},T\in\Lm{U,V}$.}
	\TextA{Prove $\dim\range ST\leqslant\min\!\Bra{{\dim \range S, \dim\range T}}$.\vspace{2pt}}
}\NOTICE that \uline{$\range ST=\Bra{Sv:v\in\range T}=\range S\mmid_{\range T}.$}\parSol{}
Let $\range ST=\Span[\BigPar]{Su_1,\dots,Su_{\dim\range T}},$ where $B_{\range T}=\BigPar{u_1,\dots,u_{\dim\range T}}.$\parSol{}
$\dim\range ST\leqslant \dim\range T$
又 $\dim\range ST\leqslant \dim\range S.$\PfEnd\parSol{\vspace{6pt}}
\Or \uline{$\dim\range ST=\dim\range S\mmid_{\range T}=\dim\range T-\dim\null S\mmid_{\range T}$}${}\leqslant \range T.$\PfEnd\parSol{}
\AComm $\dim\range ST=\dim U-\dim\null ST=\dim \range T\mmid_{U}-\dim\range T\mmid_{\null ST}.$\par\vspace{6pt}
\ACoro (1) $S\mmid_{\range T}$ inje $\Longleftrightarrow\dim\range ST=\dim\range T.$\parCor
(2) Let $X\oplus\null S=V.$ Then $X\subseteq\range T\Longleftrightarrow\range ST=\range S.$\vspace{-2pt}\parCor
\Blind{(2)} And $T$ is surj $\Rightarrow\range ST=\range S.$\parCor
(3) $\dim U=\dim V\Rightarrow\dim\null ST\geqslant\dim V-\dim\min\!\Bra{{\dim\range S,\dim\range T}}$\parCor
\Blind{(3) $\dim U=\dim V\Rightarrow\dim\null ST$}${}=\dim\max\!\Bra{{\dim V-\dim\range S,\dim V-\dim\range T}}.$\par\vspace{4pt}
\AExa Let $U=W=\Rbb,V=\Rbb^2.$ Define $T\in\Lm{\Rbb,\Rbb^2}:x\mapsto\Par{x,0},$ and $S\in\Lm{\Rbb^2,\Rbb}:\Par{x,y}\mapsto ax,a\neq0.$\parExa
Now $\dim\null S=1,\dim\null T=0,$ and $\dim\null S\mmid_{\range T}=0.$
\SepLine

\Anchor{3B4e24}\ProblemB{
	%\PrePa\TextB{Supp $\dim V = 5,$ and $ST = 0$ where $S,T\in\Lm{V}$. Prove $\dim\range TS\leqslant 2.$\vspace{2pt}}
	\PrePa\TextB{Supp $\dim V=n,$ \,$ST=0$ where $S,T\in\Lm{V}.$ Prove $\dim\range TS\leqslant\Big\lfloor${\LARGE$\frac{\,n\,}{2}$}$\Big\rfloor.$}
	\PrePb\TextB{Give an exa of such $S, T$ with $n=5$ and $\dim\range TS = 2$.\vspace{1pt}}
}\par\quad
Note that $\dim\range TS\leqslant\min\!\BigBra{\!\dim\range T,\dim\range S}.$ We prove by ctradic.\par\vspace{2pt}\quad
%(a) $\dim\range TS\leqslant\min\!\Bra{\overbrace{\dim \range S}^{5\,-\,\dim\null T}, \overbrace{\dim\range T}^{5\,-\,\dim\null S}}.$\par\quad\Ha
%We show $\dim\range TS\leqslant 2$ by ctradic. Asum $\dim\range TS\geqslant 3$.\par\quad\Ha
%Then $\min\!\Bra{5-\dim\null T,5-\dim\null S}\geqslant 3\Rightarrow\max\!\Bra{{\dim\null T,\dim\null S}}\leqslant 2$.\par\quad\Ha
%又 $\dim\null ST=5\leqslant\dim\null S+\dim\null T\leqslant 4.$ Ctradic.\vspace{6pt}\par\quad\Ha
%\Or $\MathRightBrace{l}{\dim\null S=5-\dim\range S\\ \dim\range TS\leqslant\dim\range S}\Rightarrow\dim\null S\leqslant 5-\dim\range TS.$\par\vspace{2pt}\quad\Ha
%And $ST=0\Rightarrow\range T\subseteq\null S\Rightarrow\dim\range TS\leqslant\dim\range T\leqslant\dim\null S.$\PfEnd\vspace{6pt}\quad
Asum $\dim\range TS\geqslant \Big\lfloor${\Large$\frac{\,n\,}{2}$}$\Big\rfloor+1.$ \,Then $\min\!\BigBra{n-\dim\null T,n-\dim\null S}\envFontDefault\geqslant \Big\lfloor${\Large$\frac{\,n\,}{2}$}$\Big\rfloor+1$\vspace{3pt}\par\quad
又 $\dim\null ST=n\leqslant\dim\null S+\dim\null T\;\Big|\hspace{-0.7pt}\Rightarrow\max\!\BigBra{\!\dim\null T,\dim\null S}\envFontDefault\leqslant \Big\lceil${\Large$\frac{\,n\,}{2}$}$\Big\rceil-1.$\par\vspace{3pt}\quad
Thus \;$n\leqslant 2\BigBigPar{\Big\lceil${\Large$\frac{\,n\,}{2}$}$\Big\rceil-1}\Rightarrow{}${\Large$\frac{\,n\,}{2}$}${}\leqslant\Big\lceil${\Large$\frac{\,n\,}{2}$}$\Big\rceil-1$. \;Ctradic.\PfEnd\vspace{8pt}\quad
\Or $\dim\null S=n-\dim\range S\leqslant n-\dim\range TS.$ \;又 $ST=0\Rightarrow\range T\subseteq\null S.$\par\quad
$\dim\range TS\leqslant\dim\range T\leqslant\dim\null S\leqslant n-\dim\range TS.$ \;Thus \;$2\dim\range TS\leqslant n.$\PfEnd\vspace{8pt}\quad
\Or Becs $\dim\range TS\leqslant\Big\lfloor${\Large$\frac{\,n\,}{2}$}$\Big\rfloor,$ and $\Big\lfloor${\Large$\frac{\,n\,}{2}$}$\Big\rfloor+\Big\lceil${\Large$\frac{\,n\,}{2}$}$\Big\rceil=n.$\par\vspace{3pt}\quad
We show $\dim\null TS\geqslant\Big\lceil${\Large$\frac{\,n\,}{2}$}$\Big\rceil.$ Note that $\dim\null S+\dim\null T\geqslant n.$\par\vspace{3pt}\quad
$\dim\null S+\dim\null T\mmid_{\range S}=\dim\null TS.$ \;If $\dim\null S\geqslant\Big\lceil${\Large$\frac{\,n\,}{2}$}$\Big\rceil.$ Then done.\par\vspace{3pt}\quad
Othws, $\dim\null S\leqslant\Big\lceil${\Large$\frac{\,n\,}{2}$}$\Big\rceil-1\Rightarrow\dim\null T\geqslant n-\dim\null S\geqslant n-\Big\lceil${\Large$\frac{\,n\,}{2}$}$\Big\rceil+1=\Big\lfloor${\Large$\frac{\,n\,}{2}$}$\Big\rfloor+1\geqslant\Big\lceil${\Large$\frac{\,n\,}{2}$}$\Big\rceil.$\par\vspace{3pt}\quad
Thus $\dim\null TS\geqslant\max\!\Bra{{\dim\null S,\dim\null T}}=\Big\lceil${\Large$\frac{\,n\,}{2}$}$\Big\rceil.$\PfEnd\vspace{6pt}\quad
\AExa Define $T: \; v_1\mapsto 0, \;\;\, v_2\mapsto 0, \;\;\; v_i\mapsto v_i\;;$ \;\; $S: \; v_1\mapsto v_4, \; v_2\mapsto v_5, \;\; v_i\mapsto 0\;;\;\; i=3,4,5.$
\SepLine\pagebreak

\ProblemN{\Anchor{3B24}{24}}{
	\TextA{Supp {\FontNorm $S\in\Lm{V,M},T\in\Lm{V,W},$} and $\null S\subseteq\null T.$ Prove $\exists\,E\in\Lm{M,W},T=ES$.}
}\par\quad
%Define $E:\range S\rightarrow W$ by \uline{$E:Sv\mapsto Tv.$} \,Extend $E\in\Lm[\BigPar]{\range S,W}$ to $E\in\Lm{W}.$\PfEnd\vspace{2pt}\quad
%\Or Let \uline{$V=U\oplus\null S\Rightarrow S\mmid_{U}:U\rightarrow\range S$ is iso.} \,Extend $T\BigPar{S\mmid_{U}}{^{-1}}$ to $E\in\Lm{W}.$\PfEnd\vspace{2pt}\quad
\!\!\!$\hText{$
	Let \uline{$V=U\oplus\null S$}$\\$
	\uline{$\Rightarrow S\mmid_{U}:U\rightarrow\range S$ is iso.}$\\$
	Extend $T\BigPar{S\mmid_{U}}{^{-1}}$ to $E\in\Lm{M,W}.}\hspace{-20pt}\small\left.\begin{tikzcd}
	{\range T} & U \\[4pt] & {\range S}
	\arrow["\;surj\;T"', from=1-2, to=1-1]
	\arrow["S"{pos=0.4}, from=1-2, to=2-2]
	\arrow["inv"'{pos=0.4}, from=1-2, to=2-2]
	\arrow["surj\;E\!\!"{pos=0.4}, from=2-2, to=1-1]
\end{tikzcd}\right|
\large\hText{$
	\Or Define $E:\range S\rightarrow W$ by \uuline{$E:Sv\mapsto Tv.$}$\\$
	Extend $E\in\Lm[\BigPar]{\range S,W}$ to $E\in\Lm{M,W}.$
$}$\PfEnd\vspace{6pt}\quad
\AComm Let $\Delta\oplus\null S=\null T,\;U_{\Delta}\oplus\Par{\Delta\oplus\null S}=V=U_{\Delta}\oplus\null T.$ \;Redefine $U=U_{\Delta}\oplus \Delta.$\par\vspace{2pt}\quad
%\Or Let $U_{\Delta}\oplus\null T\mmid_U=U.$ Let $\Delta=\null T\mmid_U.$ Then $U_{\Delta}\oplus\null T=V,\;\Delta\oplus\null S=\null T.$\parCom\quad
$\hText{$\includegraphics[width=88pt]{diagram3B-2}$}\hText{$
	\Blind{$\range S\xlongleftarrow{S}{}$}$U_{\Delta}\xlongrightarrow{T}\range T\\[-7pt]$
	$\range S\xlongleftarrow{S}\oplus\\[-6pt]$
	\Blind{$\range S\xlongleftarrow{S}{}$}$\Delta\xlongrightarrow{T}\zeroSubs}\MathLeftMid{l}{$
	Becs $\Delta=\null T\mmid_U=\null T\cap\Range\BigPar{S\mmid_U}{^{-1}}.\\$
	Thus \,$E=T\BigPar{S\mmid_{U}}{^{-1}}$ is not inje $\Longleftrightarrow\Delta\neq\zeroSubs.\\$
	In other words, $\range S\mmid_\Delta=\null E,\\$
	while $E\mmid_{\cdots}:\range S\mmid_{U_\Delta}\rightarrow\range T$ is iso.$}$\par\vspace{6pt}\quad
\AComm Let $E_1\in\Lm{U_{\Delta}\oplus\null T,U_{\Delta}},$ and $E_2$ be an iso of $\range S\mmid_{U_{\Delta}}$ onto $\range T.$\parCom\quad
Define $E_1\mmid_{U_{\Delta}}=I\mmid_{U_{\Delta}},$ and $E_2=T\BigPar{S\mmid_{U_{\Delta}}}{^{-1}}.$ \;Then $T=E_2SE_1.$\par\vspace{4pt}\quad
\ACoro If $\null S=\null T.$ Then $\Delta=\zeroSubs,U_{\Delta}=U.$ \Sbra[3pt]{{\tgsl Req $W$ Finide}} \;By (3.D.3),\parCor\quad
we can extend inje $T\BigPar{S\mmid_{U}}{^{-1}}\in\Lm[\BigPar]{\range S,W}$ to inv $E\in\Lm{M,W}.$\vspace{6pt}\par\quad
\Or \Sbra[3pt]{{\tgsl Req $\range S$ Finide}} \;Let $B_{\range S}=\Par{Sv_1,\dots,Sv_n}$. Then \uline{$V=\Span{v_1,\dots,v_n}\oplus\null S$.}\par\quad
Define $E\in\Lm{\range S,W}$ by $E\Par{Sv_i}=Tv_i.$ \;Extend to $E\in\Lm{M,W}.$\par\quad
Hence $\forall v=\sum_{i=1}^na_iv_i+u\in V,\,\,$\uline{$\BigPar{\exists\,!\,u\in\null S\subseteq\null T\,},\,Tv=\sum_{i=1}^na_iTv_i+0$}${}=E\BigBigPar{{\sum_{i=1}^na_iSv_i+0}}.$\PfEnd\Anchor{3D4}\vspace{6pt}\quad
\ACoro \Sbra[3pt]{{\tgsl Req $W$ Finide}} \;Supp $\null S=\null T.$ We show $\exists$ inv $E\in\Lm{M,W},T=ES.$\par\quad
Redefine $E\in\Lm{M,W}$ by $E\Par{Tv_i}=Sv_i,\;E\Par{w_j}=x_j,$ for each $Tv_i$ and $w_j.$ Where:\par\quad
Let $B_{\range T}=\Par{Tv_1,\dots,Tv_m},B_W=\Par{Tv_1,\dots,Tv_m,w_1,\dots,w_n},B_U=\Par{v_1,\dots,v_m}.$\par\quad
Now $V=U\oplus\null T=U\oplus\null S\Rightarrow B_{\range S}=\BigPar{Sv_1,\dots,Sv_m}.$ Let $B_M=\Par{Sv_1,\dots,Sv_m,x_1,\dots,x_n}.$\PfEnd 
\SepLine

\ProblemN{\Anchor{3B25}{25}}{
	\TextA{Supp {\FontNorm$S\in\Lm{Y,W},T\in\Lm{V,W},$} and $\range T\subseteq\range S.$ Prove $\exists\,E\in\Lm{V,Y},T = SE$.}
}Let $Y=U\oplus\null S$\vspace{-4pt}\parSol{}
$\Rightarrow S\mmid_U:U\rightarrow\range S$ is iso. Becs $\Par{S\mmid_U}{^{-1}}:\overset{\range T\:\subseteq}{{\range S}}\rightarrow U.$\par\quad
\uline{Define $E=\BigPar{S\mmid_U}{^{-1}}T=\BigPar{S\mmid_U}{^{-1}}\Big|{_{\range T}}T\in\Lm{V,U}\subseteq\Lm{V,Y}.$}\PfEnd\vspace{-20pt}\quad
\!\!\!$\hText{\\[6pt]$
	\AComm Let $U_1=U.$ Let $U_2\oplus\null T=V.\\$
	Let $U_{1\Delta}=\Range\BigPar{S\mmid_{U_1}}{^{-1}}\Big|{_{\range T}}\subseteq U_1=\Delta\oplus U_{1\Delta}.\\$
	\Or Let $U_{1\Delta}=\range E\mmid_{U_2}.$ \,Let $\Delta\oplus\range E\mmid_{U_2}=U_1.$
%	Thus $U_1\oplus\null S=U_{1\Delta}\oplus{}$\uline{$\Par{\Delta\oplus\null S}$}${}=U_2\oplus{}$\uline{$\null T$}$.\\[-10pt]$
%	\Blind{Thus $U_1\oplus\null S=U_{1\Delta}\oplus{}$}$\qquad\text{\tiny|}\qquad\qquad\qquad\qquad\text{\tiny|}\\[-20pt]$
%	$\hspace{164.2pt}\underset{\text{iso, \;if finide.}}{\uline{\;\qquad\qquad\qquad\qquad\hspace{-2pt}}}$
$}\qquad\qquad\qquad\quad$\FontSmall$\hText{U_1\xlongrightarrow[S]{inv}\range S\\[-8pt]\;\text{\small|\:|}\qquad\qquad\text{\small|\:|}\\[-6pt]\;\Delta\,\xlongrightarrow[S]{inv}\range S\mmid_{\Delta}\\[-8pt]\;\oplus\qquad\quad\;\;\oplus\\[-6pt]\,U_{1\Delta}\xlongrightarrow[S]{inv}\range T\xlongleftarrow[T]{inv}U_2\\[-8pt]\,\;\uparrow\qquad\qquad\qquad\qquad\quad|\\[-19pt]\,\;\,\underset{inv\;\;E\def\envFontA{\normalsize}\mmid_{U_2}}{{\uline{\qquad\qquad\qquad\qquad\qquad\!\!}}}}$\FontNorm\par\vspace{2pt}\quad
%\ACoro If $\Delta=\zeroSubs,$ then $U_1=U_{1\Delta}\Rightarrow\range S=\range T.$ 又 $\null S,\null T$ are iso.\par\quad
%By (3.D.3), we can re-extend inje $E\mmid_{U_2}\in\Lm[\BigPar]{U_2,U_1\oplus\null S}$ to inv $E\in\Lm[\BigPar]{U_2\oplus\null T,U_1\oplus\null S}.$\par\vspace{4pt}\quad
%Thus we have $\Delta\neq\zeroSubs\Longleftrightarrow E\mmid_{U_2}\in\Lm{U_2,V}$ cannot be re-extended to inv $E\in\Lm{V}$ freely.
\Sbra[3pt]{{\tgsl Req $\range T$ Finide}} \;Let $B_{\range T}=\Par{Tv_1,\dots,Tv_n}.$ Now $B_{U_2}=\Par{v_1,\dots,v_n}.$\par\quad
\uline{Let $S\Par{u_i}=Tv_i$ for each $Tv_i.$} \;Define $E$ with each $Ev_i=u_i,Ex=0$ for $x\in\null T.$\PfEnd\vspace{4pt}\quad
\AComm \Sbra[3pt]{{\tgsl Req $V$ Finide}} \;Note that $\dim U_2\leqslant\dim U_1\Longrightarrow\dim\null T=p\geqslant q=\dim\null S.$\parCom\quad
Let $B_{\null T}=\Par{x_1,\dots,x_p},B_{\null S}=\Par{y_1,\dots,y_q}.$ Redefine $E:v_i\mapsto u_i,\;x_k\mapsto y_k,\;x_j\mapsto 0,$\parCom\quad
for each $i\in\;\!\!\Bra{1,\dots,\dim U_2},k\in\;\!\!\Bra{1,\dots,\dim\null S}=K,j\in\;\!\!\Bra{1,\dots,\dim\null T}\Backslash[\Big]K.$\parCom\quad
Note that $\Par{u_1,\dots,u_n}$ is liney indep. Let $X=\Span{x_1,\dots,x_q}\oplus\Span{v_1,\dots,v_n}.$\parCom\quad
Now $E\mmid_X$ is inje, but cannot be re-extend to inv $E\in\Lm{V,Y}$ suth $T=SE.$\par\Anchor{3D5}\vspace{4pt}\quad
\ACoro \Sbra[3pt]{{\tgsl Req $V$ Finide}} \;If $\range T=\range S,$ then $\dim\null T=\dim\null S=p.$\parCor\quad
Redefine $E$ by $Ev_i=u_i,\;Ex_j=y_j$ for each $v_i$ and $x_j.$ Then $E\in\Lm{V,Y}$ is inv.\PfEnd
\SepLine

\Anchor{3B'1}\BulletPointX\ANote $\null T=\null S\Rightarrow E:Sv\mapsto Tv$ and $E^{-1}:Tv\mapsto Sv$ well-defined $\Rightarrow\range T,\range S$ iso.\parNot{\IndentB}
While $\range T=\range S\notRightarrow\null T,\null S$ iso. \;\AExa Backwd shift optor and id optor on $\FbbP{\infty}.$
\SepLine

\Anchor{3D6}\ProblemBnoor{3.D.6}{
	\TextA{Supp $V,W$ are finide, and $S,T\in\Lm{V,W},$ and $\dim\null S=\dim\null T=n.$}
	\TextA{Prove $S=E_2 T E_1,\exists$ inv $ E_1\in\Lm{V}, E_2\in\Lm{W}.$}
}$\text{Define}\;\,E_1:\,\,\,\,\, v_i\mapsto r_i\,\,;\quad u_j\mapsto s_j;$\quad for each $i\in\;\!\!\Bra{1,\dots,m},j\in\;\!\!\Bra{1,\dots,n}.$\parSol{}
$\text{Define}\;\,E_2:Tv_i\mapsto Sr_i\,\,;\;\;x_j\mapsto y_j;$\quad for each $i\in\;\!\!\Bra{1,\dots,m},j\in\;\!\!\Bra{1,\dots,n}.$
Where:\parSol{\vspace{2pt}}
$\MathLeftrightMid{l}{$
	Let $B_{\range T}=\Par{Tv_1,\dots,Tv_m};\;B_{\range S}=\Par{Sr_1,\dots,Sr_m}.\\ $
	Let $B_W=\Par{Tv_1,\dots,Tv_m,x_1,\dots,x_p};\;B_W'=\Par{Sr_1,\dots,Sr_m,y_1,\dots,y_p}.\\ $
	Let $B_{\null T}=\Par{u_1,\dots,u_n};\;B_{\null S}=\Par{s_1,\dots,s_n}.\\ $
	Thus $B_V=\Par{v_1,\dots,v_m,u_1,\dots,u_n};\;B_V'=\Par{r_1,\dots,r_m,s_1,\dots,s_n}.}\hMath{c}{}{}{\therefore\;E_1,E_2$ are inv$ \\ $and $S=E_2 TE_1.}$\PfEnd
\SepLine

\ProblemN{\Anchor{3B28}28}{
	\TextA{Supp $T\in\Lm{V,W}.$ Let $\Par{Tv_1,\dots,Tv_m}$ be a bss of $\range T$ and each $w_i=Tv_i.$\vspace{2pt}}
	\PrePa\TextA{Prove $\exists\,\varphi_1,\dots,\varphi_m\in\Lm{V,\Fbb}$ suth $\forall v\in V,Tv=\varphi_1\Par{v}\:\!w_1+\dots+\varphi_m\Par{v}\:\!w_m$.\vspace{4pt}}
	\PrePb{\FontSmall\Sbra{4E 3.F.5}} \TextA{$\forall v\in V,\exists\,!\,\varphi_i\Par{v}\in\Fbb,Tv=\varphi_1\Par{v}\:\!w_1+\dots+\varphi_m\Par{v}\:\!w_m.$\vspace{2pt}}\Anchor{3F4e5}
	\Blind{{\FontSmall\Sbra{4E 3.F.5}}} \TextA{Thus defining each $\varphi_i:V\rightarrow\Fbb.$ \;Show each $\varphi_i\in\Lm{V,\Fbb}.$}
}{\FontSmall\tgsl The solus to (b) with (b) itself is another solus to (a).}\par\quad
(a) $\Span{v_1,\dots,v_m}\oplus\null T=V\Rightarrow\forall v\in V,\exists\,!\,a_i\in\Fbb,u\in\null T,\;v=\sum_{i=1}^m a_i v_i+u.$\par\quad\Ha
Define $\varphi_i\in\Lm{V,\Fbb}$ by $\varphi_i\Par{v_j}=\delta_{i,j},\;\varphi_i\Par{u}=0$ for all $u\in\null T.$\par\quad\Ha
Linity: $\forall v,w\in V\,\Sbra{\exists\,!\,a_i,b_i\in\Fbb},\lambda\in\Fbb,\varphi_i\Par{v+\lambda w}=a_i+\lambda b_i=\varphi\Par{v}+\lambda\varphi\Par{w}.$\PfEnd\vspace{4pt}\quad
(b) $\sum_{i=1}^m\varphi_i\Par{u+\lambda v}w_i=T\Par{u+\lambda v}=Tu+\lambda Tv={\BigBigPar{{\sum_{i=1}^m\varphi_i\Par{u}w_i}}}+\lambda{\BigBigPar{{\sum_{i=1}^m\varphi_i\Par{v}\:\!w_i}}}.$\PfEnd
\SepLine

%\ProblemN{\Anchor{3B29}{29}}{
%	\TextA{Supp $\varphi\in\Lm{V,\Fbb}.$ Supp $\varphi\Par{u}\neq 0.$ Prove $V = \null\varphi\oplus\Bra{au :a\in\Fbb}.$\FontSmall\hfill By \TIPSN{4}, immed.}
%}(a) $v=cu\in\null\varphi\cap\Span{u}\Rightarrow c\varphi\Par{u}=0\Rightarrow v=0.$ \;Now $\null\varphi\cap\Span{u}=\zeroSubs.$\parSol{}
%(b) For $v\in V,$ let $a_v=\varphi\Par{v}.$ Then $v=\Sbra{v-\Par{a_v\big/a_u}u}+\Par{a_v\big/a_u}u\Rightarrow V=\null\varphi+\Span{u}.$\PfEnd\vspace{-3pt}
%\SepLine

\ProblemN{\Anchor{3B30}{30}}{
	\TextA{Supp $\varphi,\beta\in\Lm{V,\Fbb}$ and $\null\varphi=\null\beta=\eta.$ Prove $\exists\,c\in\Fbb,\varphi=c\beta.$}
}If $\eta=V$, then $\varphi=\beta=0,$ done. Now by Exe (29),\parSol{}
$\varphi\Par{u}\neq 0\Longleftrightarrow V=\null\varphi\oplus\Span{u}\Longleftrightarrow V=\null\beta\oplus\Span{u}\Longleftrightarrow\beta\Par{u}\neq 0.$\parSol{}
\hspace{-5pt}$\hText{$
	Note that $\forall v\in V,\exists\,!\,u_0\in\eta,\;a_v\in\Fbb,v=u_0+a_vu\\$
	$\Rightarrow\varphi\Par{u_0+a_vu}=a_v\varphi\Par{u},\;\beta\Par{u_0+a_vu}=a_v\beta\Par{u}.}\;\MathLeftMid{l}{$Let $c={}${\Large\envFontSmall$\frac{\varphi\Par{u}}{\beta\Par{u}}$}${}\in\nonzeroFbb.}$\PfEnd%\vspace{6pt}
%\AComm Convly, if $\varphi=c\beta,\exists\,c\in\nonzeroFbb.$ \,Then $\null\beta=\Null\Par{c\varphi}=\null\varphi.$\parCom
%Note that $c\varphi=E\circ\varphi,$ where $E\in\Lm{\Fbb}:x\rightarrow cx,$ and $E$ is inv.
\SepLine

\Anchor{3F4e6}\ProblemBnoor{{4E 3.F.6}}{
	\TextB{Supp $\varphi,\beta\in\Lm{V,\Fbb}.$ Prove $\null \beta\subseteq\null\varphi\Longleftrightarrow\varphi=c\beta,\exists\,c\in\Fbb.$}
	\ACoro $\null \varphi=\null\beta\Longleftrightarrow\varphi=c\beta,\;\exists\,c\in\nonzeroFbb.$\TextB{}
}Using Exe (29) and (30).\par\quad
(a) If $\varphi=0,$ then done. Othws, supp $u\not\in\null\varphi\supseteq\null \beta.$\vspace{-4pt}\par\quad\Ha
Now $V=\null \varphi\oplus\Span{u}=\null \beta\oplus\Span{u}.$ By \Sbra{1.C \TIPSN{1}}, $\null \varphi=\null \beta.$ \;Let $c={}${\Large\envFontSmall$\frac{\varphi\Par{u}}{\beta\Par{u}}$}.\vspace{2pt}\par\quad\Ha
\Or We discuss in two cases. If $\null \beta=\null \varphi$, or if $\varphi=0,$ then done. Othws,\par\quad\Ha
$\exists\,u\apostrophe\in\null \varphi\Backslash[\Big]\null \beta,\,\exists\,u\not\in\null\varphi\supsetneq\null\beta\Rightarrow V=\null \beta\oplus\Span{u\apostrophe}=\null \beta\oplus\Span{u}.$\par\quad\Ha
\hspace{-5pt}$\hText{$
	$\forall v \in V, v=w+au=w\apostrophe+bu\apostrophe,\,\exists\,!\,w, w\apostrophe\in\null \beta\\$
	Thus $\varphi\Par{w+au}=a\varphi\Par{u},\,\,\beta\Par{w\apostrophe+bu}=b\beta\Par{u\apostrophe}.}\;\MathLeftMid{l}{$ Let $c={}${\Large\envFontSmall$\frac{a\varphi\Par{u}}{b\beta\Par{u\apostrophe}}$}${}\in\nonzeroFbb.$ \;Done.$}$\vspace{6pt}\par\quad\Ha
\NOTICE that by (b) below, we have $\null \varphi\subseteq\null \beta,$ ctradic the asum.\vspace{6pt}\par\quad
(b) If $c=0$, then $\null \varphi=V\supseteq\null \beta$, done. Othws, becs $v\in\null \beta\Longleftrightarrow v\in\null \varphi.$\PfEnd\vspace{4pt}\quad
\Or By Exe (24), $\null \beta\subseteq\null \varphi\Longleftrightarrow\exists\,E\in\Lm{\Fbb},\varphi=E\circ\beta.$ \Sbra{ If $E$ is inv. Then $\null \beta=\null \varphi.$ }\par\quad
Now $\exists\,E\in\Lm{\Fbb},\varphi=E\circ\beta\Longleftrightarrow\exists\,c=E\Par{1}\in\Fbb,\varphi=c\beta.$ \Sbra{ $E$ is inv $\Longleftrightarrow E\Par{1}\neq 0\Longleftrightarrow c\neq 0.$ }\PfEnd
\SepLine
\ChEnd\pagebreak

\ChDecl{Ch3C}{3$\cdot$C}{\quad{\small\textbf{注意\,:}\;这里我将原书中3.D后半节正文前置。}}

\vspace{8pt}

\Anchor{3CN3.3032}\BulletPointX\NoteFor{[3.30, 32]}\;\;{\tgsl matrix of span}\TextB{}
Supp $L_{\alpha}=\Par{\alpha_1,\dots,\alpha_n}$ and $L_{\beta}=\Par{\beta_1,\dots,\beta_m}$ are in a vecsp $V.$\TextB{}
Let each $\alpha_k=A_{1,k}\beta_1+\dots+A_{m,k}\beta_m,$ forming $A=\Mt{\spn L_\beta\supseteq L_{\alpha}}\in\FbbP{m,n}.$\TextB{\vspace{3pt}}
Which is {\tgsl the matrix of span}. \;Then ${\normalsize\begin{pmatrix}\beta_1 &\hspace{-6pt} \cdots &\hspace{-6pt} \beta_m\end{pmatrix}\begin{pmatrix}
	A_{1,1} &\hspace{-6pt} \cdots &\hspace{-6pt} A_{1,n}\\[-4pt]
	\vdots	&\hspace{-6pt} \ddots &\hspace{-6pt} \vdots \\[-4pt]
	A_{m,1} &\hspace{-6pt} \cdots &\hspace{-6pt} A_{m,n}
\end{pmatrix}}={\normalsize\begin{pmatrix}\alpha_1 &\hspace{-6pt} \cdots &\hspace{-6pt} \alpha_n\end{pmatrix}}.$\TextB{\vspace{6pt}}
(a) Supp $m=n.$ If $\Par{A_{\cdot,1},\dots,A_{\cdot,n}}$ is a bss of $\FbbP{n,1}.$ We show $L_{\alpha}$ liney indep $\Longleftrightarrow L_{\beta}$ liney indep.\TextB{}
\Ha ($\Leftarrow$) Immed. ($\Rightarrow$) Asum $L_{\beta}$ is liney dep and $\beta_j=c_1\beta_1+\dots+c_{j-1}\beta_{j-1}.$ By ctradic.\PfEnd\vspace{2pt}\TextB{}
(b) Supp $m\geqslant n.$ If $L_{\beta}$ liney indep. We show $\Par{A_{\cdot,1},\dots,A_{\cdot,n}}$ liney indep $\Longleftrightarrow L_{\alpha}$ liney indep.\TextB{}
\Hb ($\Rightarrow$) Immed. ($\Leftarrow$) By ctradic.\PfEnd\TextB{}
\Hb\ANote $\Mt{\spn L_{\beta}\supseteq L_\alpha}=\Mt{I,L_{\alpha},L_{\beta}}\Longleftrightarrow L_{\alpha},L_{\beta}$ liney indep $\Rightarrow\Par{A_{\cdot,1},\dots,A_{\cdot,n}}$ liney indep.\parNot{\Hb\IndentB}
Where $I$ is the id optor restr to $\spn L_{\alpha}\subseteq\spn L_{\beta}.$
\vspace{3pt}\TextB{}
(c) Supp $m<n.$ Then $\Par{A_{\cdot,1},\dots,A_{\cdot,n}}$ is liney dep, so is $L_{\alpha}.$\TextB{\vspace{5pt}}
Supp $T\in\Lm{V,W}$ and $B_V=\Par{v_1,\dots,v_m},B_W=\Par{w_1,\dots,w_n}.$\TextB{}
Then $\Mt{T,B_V\hspace{1pt},B_W}=\Mt[\BigPar]{\spn B_W\supseteq\Par{Tv_1,\dots,Tv_m}}.$ \;See also (4E 3.D.23).
\SepLine

\def\fT{\mT}\def\fC{\mC}\def\fR{\mR}\def\fP{\mE}

\Anchor{3CNTrspose}\Anchor{3F33}\BulletPointX\NoteFor{Trspose}\;\;{\FontSmall\Sbra{3.F.33}}\;Define $\fT:A\rightarrow A^t.$ By [3.111], $\fT$ is liney. Becs $\Par{A^t}{^t}=A.$ \TextB{}
$\fT^2=I,\,\fT=\fT^{-1}\Rightarrow \fT$ is iso of $\FbbP{m,n}$ onto $\FbbP{n,m}.$ \;Define $\fC_k:A\rightarrow A_{\cdot,k},\;\fR_j:A\rightarrow A_{j,\cdot},\;\fP_{j,k}:A\rightarrow A_{j,k}.$\TextB{}
Now we show (a) \uline{$\fT\fR_j=\fC_j\fT,$} \,(b) \uline{$\fT\fC_k=\fR_k\fT,$} \,and (c) \uline{$\fT\fP_{j,k}=\fP_{k,j}\fT.$}\TextB{\vspace{2pt}}
So that $\fT\fC_k\fT=\fR_k,\;\fT\fR_j\fT=\fC_j,$ \,and $\fT\fP_{j,k}\fT=\fP_{k,j}.$\TextB{\vspace{2pt}}
Let $A={}{\normalsize\begin{pmatrix}A_{1,1} &\hspace{-6pt} \cdots &\hspace{-6pt} A_{1,n}\\[-4pt] \vdots &\hspace{-6pt} \ddots &\hspace{-6pt} \vdots\\[-4pt] A_{m,1} &\hspace{-6pt} \cdots &\hspace{-6pt} A_{m,n}\end{pmatrix}}\Rightarrow A^t={}{\normalsize\begin{pmatrix}A_{1,1} &\hspace{-6pt} \cdots &\hspace{-6pt} A_{m,1}\\[-4pt] \vdots &\hspace{-6pt} \ddots &\hspace{-6pt} \vdots\\[-4pt] A_{1,n} &\hspace{-6pt} \cdots &\hspace{-6pt} A_{m,n}\end{pmatrix}}.$ \;$\MathLeftMid{l}{\!\!$
Note that $\Par{A_{j,k}}{^t}=A_{j,k}=\Par{A^t}_{k,j}.$ Thus (c) holds.$\\$
\!\!And $\Par{A_{\cdot,k}}{^t}={}{\small\begin{pmatrix}A_{1,k} &\hspace{-6pt} \cdots &\hspace{-6pt}A_{m,k}\end{pmatrix}}{}={}{\small\begin{pmatrix}A^t_{k,1} &\hspace{-6pt} \cdots &\hspace{-6pt}A^t_{k,m}\end{pmatrix}}{}=\Par{A^t}{_{k,\cdot}}\\$
\!\!$\Longrightarrow$ (b) holds. Simlr for (a).$}$\par\vspace{10pt}
\SepLine

\Anchor{3CN3.47}\BulletPointX\NoteForSmall{[3.47]}\,\, $\BigPar{AC}{_{j,k}}=\sum_{r=1}^n A_{j,r}C_{r,k}=\sum_{r=1}^n \BigPar{A_{j,\cdot}}_{1,r}\BigPar{C_{\cdot,k}}_{r,1}=\BigPar{A_{j,\cdot}C_{\cdot,k}}_{1,1}=A_{j,\cdot}C_{\cdot,k}$\PfEnd\vspace{7pt}
\Anchor{3CN3.49}\BulletPointX\NoteForSmall{[3.49]} ${\BigSbra{\BigPar{AC}_{\cdot,k}}}_{j,1}=\BigPar{AC}_{j,k}=\sum_{r=1}^n A_{j,r}C_{r,k}=\sum_{r=1}^n A_{j,r}\BigPar{C_{\cdot,k}}_{r,1}=\BigPar{AC_{\cdot,k}}_{j,1}$\PfEnd\vspace{8pt}
\Anchor{3C10}\BulletPointX\Exercise{10}\;\;${\BigSbra{\BigPar{AC}_{j,\cdot}}}_{1,k}=\BigPar{AC}_{j,k}=\sum_{r=1}^n A_{j,r}C_{r,k}=\sum_{r=1}^n \BigPar{A_{j,\cdot}}_{1,r}C_{r,k}=\BigPar{A_{j,\cdot}C}_{1,k}$\PfEnd\vspace{10pt}
\BulletPointX\ANote Let $C=\Mt{T}\in\FbbP{n,\:\!p},A=\Mt{S}\in\FbbP{m,n}$ wrto std bses, where $U=\FbbP{p},V=\FbbP{n},W=\FbbP{m}.$\vspace{3pt}\parNot{}\IndentB{}
For [3.49], $\Mt{Te_k,B_V}=C_{\cdot,k}\Rightarrow\Mt[\BigPar]{S\Par{Te_k},B_W}=AC_{\cdot,k}\,,$ \;又 $\Mt[\BigBigPar]{\Par{ST}\Par{e_k},B_W}=\Par{AC}{_{\cdot,k}}$\PfEnd\vspace{5pt}\parNot{}\IndentB{}
For Exe (10), $\BigPar{AC}{_{j,\cdot}}=\BigSbra{\BigBigPar{\BigPar{AC}{^t}}{_{\cdot,j}}}{^t}=\BigPar{C^t\Par{A^t}{_{\cdot,j}}}{^t}=\BigPar{\Par{A^t}{_{\cdot,j}}}{^t}C=A_{j,\cdot}C$\PfEnd
\SepLine

\Anchor{3CN4e3.51}\ProblemBnoor[]{4E 3.51}[\Sbra]{
	Supp $C\in\FbbP{m,c}.$ \hfill (a) For $k=1,\dots,p,$\; $\BigPar{CR}{_{\cdot,k}}=C_{\cdot,\cdot}R_{\cdot,k}=\sum_{r=1}^c C_{\cdot,r}R_{r,k}=R_{1,k}C_{\cdot,1}+\dots+R_{c,k}C_{\cdot,c}$\TextA{\vspace{3pt}}
	\Blind{Supp} $R\in\FbbP{c,\:\!p}.$\:\: (b) For $j=1,\dots,m,$\; $\BigPar{CR}{_{j,\cdot}}=C_{j,\cdot}R_{\cdot,\cdot}=\sum_{r=1}^c C_{j,r}R_{r,\cdot}=C_{j,1}R_{1,\cdot}+\dots+C_{j,c}R_{c,\cdot}$\TextA{\vspace{0pt}}
}\SepLine

\Anchor{3C9}\BulletPointX\NoteForSmall{[3.52]}\;\;$A\in\FbbP{m,n},c\in\FbbP{n,1}\Rightarrow Ac\in\FbbP{m,1}.$\hfill By \Sbra{4E 3.51(a)}, $\BigPar{Ac}{_{\cdot,1}}=c_1A_{\cdot,1}+\dots+c_nA_{\cdot,n}.$\Blind{\quad}\PfEnd\vspace{4pt}\quad
\Or $\because\;\BigPar{Ac}{_{j,1}}=\sum_{r=1}^n A_{j,r}c_{r,1}=\BigSbra{{\sum_{r=1}^n\BigPar{A_{\cdot,r}c_{r,1}}}}_{j,1}=\BigPar{c_1 A_{\cdot,1}+\dots+c_n A_{\cdot,n}}_{j,1}$\vspace{2pt}\par\quad
\Blind{\Or}$\therefore\;\,Ac=A_{\cdot,\cdot}c_{\cdot,1}=\sum_{r=1}^n A_{\cdot,r}c_{r,1}=c_1 A_{\cdot,1}+\dots+c_n A_{\cdot,n}$\;\;\Or $\BigPar{Ac}{_{j,1}}=\BigPar{Ac}{_{j,\cdot}}=A_{j,\cdot}c\in\Fbb.$\PfEnd\vspace{3pt}\quad
\Or Let $B_V=\Par{v_1,\dots,v_n}.$ Now $Ac=\Mt{Tv,B_W}=\Mt[\BigPar]{T\Par{c_1v_1+\dots+c_nv_n}}=c_1A_{\cdot,1}+\dots+c_nA_{\cdot,n}.$\PfEnd
\SepLine\pagebreak

\Anchor{3C11}\BulletPointX\Exercise{11}\;\;$a\in\FbbP{1,n},C\in\FbbP{n,\:\!p}\Rightarrow aC\in\FbbP{1,\:\!p}.$\hfill By \Sbra{4E 3.51(b)}, $\BigPar{aC}{_{1,\cdot}}=a_1C_{1,\cdot}+\dots+a_nC_{n,\cdot}.$\Blind{\quad}\PfEnd\vspace{4pt}\quad
\Or $\because\;\BigPar{aC}{_{1,k}}=\sum_{r=1}^n a_{1,r}C_{r,k}=\BigSbra{{\sum_{r=1}^n a_{1,r}\BigPar{C_{r,\cdot}}}}_{1,k}=\BigPar{a_1 C_{1,\cdot}+\dots+a_n C_{n,\cdot}}_{1,k}$\vspace{2pt}\par\quad
\Blind{\Or}$\therefore\;\,aC=a_{1,\cdot}C_{\cdot,\cdot}=\sum_{r=1}^n a_{1,r}C_{r,\cdot}=a_1 C_{1,\cdot}+\dots+a_n C_{n,\cdot}$\;\;\Or $\BigPar{aC}{_{1,k}}=\BigPar{aC}{_{\cdot,k}}=aC_{\cdot,k}\in\Fbb.$\PfEnd\vspace{4pt}\quad
\Or \;$aC=\BigBigPar{\Par{aC}{^t}}{^t}=\BigPar{C^ta^t}{^t}=\Sbra{a^t_1\Par{C^t}{_{\cdot,1}}+\dots+a^t_n\Par{C^t}{_{\cdot,n}}}{^t}=a_1C_{1,\cdot}+\dots+a_nC_{n,\cdot}.$\PfEnd
\SepLine

\Anchor{3CNCRFact}\ProblemB{
	\hspace{0pt}\textsc{\Large CR Factoriz}\quad\vspace{3pt}Supp non0 $A\in\FbbP{m,n}.$ \TextB{Prove, with $p$ below, that $\exists\,C\in\FbbP{m,\:\!p},R\in\FbbP{p,n},A=CR.$\vspace{3pt}}
	\PrePa\TextB{Supp $\col A=\Span[\BigPar]{A_{\cdot,1},\cdots,A_{\cdot,n}}\subseteq\FbbP{m,1},\dim\col A=c,\text{ the col rank}.$ Let $p=c.$\vspace{3pt}}
	\PrePb\TextB{Supp $\row A=\Span[\BigPar]{A_{1,\cdot},\cdots,A_{m,\cdot}}\subseteq\FbbP{1,n},\dim\row A=r,\text{ the row rank}.$ Let $p=r.$\vspace{3pt}}
}Using [4E 3.51]. Notice that $A\neq 0\Rightarrow c,r\geqslant 1.$\vspace{2pt}\par\quad
(a) Reduce to bss $B_C=\BigPar{C_{\cdot,1},\cdots,C_{\cdot,c}},$ forming $C\in\FbbP{m,c}$. Then $\forall k\in\;\!\!\Bra{1,\dots,n},$\vspace{2pt}\par\quad\Ha
$A_{\cdot,k}=R_{1,k}C_{\cdot,1}+\dots+R_{c,k}C_{\cdot,c}=\BigPar{CR}{_{\cdot,k}}\;,\exists\,!\,R_{1,k},\cdots,R_{c,k}\in\Fbb,$ forming $R\in\FbbP{c,n}.$ Thus $A=CR.$\vspace{4pt}\par\quad
(b) Reduce to bss $B_R=\BigPar{R_{1,\cdot},\cdots,R_{r,\cdot}},$ forming $R\in\FbbP{r,n}$. Then $\forall j\in\;\!\!\Bra{1,\dots,m},$\vspace{2pt}\par\quad\Hb
$A_{j,\cdot}=C_{j,1}R_{1,\cdot}+\dots+C_{j,r}R_{r,\cdot}=\BigPar{CR}{_{j,\cdot}}\;,\exists\,!\,C_{j,1},\dots,C_{j,r}\in\Fbb,$ forming $C\in\FbbP{m,r}.$ Thus $A=CR.$\PfEnd
\SepLine

\BulletPointX\AExa ${\small\begin{pmatrix} 10 &\hspace{-4pt} 7 &\hspace{-4pt} 4 &\hspace{-4pt} 1 \\[-2pt] 26 &\hspace{-4pt} 19 &\hspace{-4pt} 12 &\hspace{-4pt} 5\\[-2pt] 46 &\hspace{-4pt} 33 &\hspace{-4pt} 20 &\hspace{-4pt} 7\end{pmatrix}}\xlongequal{\SmallPar{\text{I}}}{\small\begin{pmatrix} 1 &\hspace{-4pt} 0\\[-2pt] 0 &\hspace{-4pt} 1\\[-2pt] 2 &\hspace{-4pt} 1\end{pmatrix}\begin{pmatrix} 10 &\hspace{-4pt} 7 &\hspace{-4pt} 4 &\hspace{-4pt} 1\\[-2pt] 26 &\hspace{-4pt} 19 &\hspace{-4pt} 12 &\hspace{-4pt} 5\end{pmatrix}}={\small\begin{pmatrix} 1 &\hspace{-4pt} 0\\[-2pt] 0 &\hspace{-4pt} 1\\[-2pt] 2 &\hspace{-4pt} 1\end{pmatrix}\begin{pmatrix} 4 &\hspace{-4pt} 1\\[-2pt] 12 &\hspace{-4pt} 5\end{pmatrix}\begin{pmatrix} 3 &\hspace{-4pt} 2 &\hspace{-4pt} 1 &\hspace{-4pt} 0\\[-2pt] -2 &\hspace{-4pt} -1 &\hspace{-4pt} 0 &\hspace{-4pt} 1\end{pmatrix}}$\parExa{\IndentB}
$\Blind{{\small\begin{pmatrix} 10 &\hspace{-4pt} 7 &\hspace{-4pt} 4 &\hspace{-4pt} 1 \\[-2pt] 26 &\hspace{-4pt} 19 &\hspace{-4pt} 12 &\hspace{-4pt} 5\\[-2pt] 46 &\hspace{-4pt} 33 &\hspace{-4pt} 20 &\hspace{-4pt} 7\end{pmatrix}}}\xlongequal{\SmallPar{\text{II}}}{\small\begin{pmatrix} 7 &\hspace{-4pt} 4\\[-2pt] 19 &\hspace{-4pt} 12\\[-2pt] 33 &\hspace{-4pt} 20\end{pmatrix}\begin{pmatrix} 2 &\hspace{-4pt} 1 &\hspace{-4pt} 0 &\hspace{-4pt} -1\\[-2pt] -1 &\hspace{-4pt} 0 &\hspace{-4pt} 1 &\hspace{-4pt} 2\end{pmatrix}}={\small\begin{pmatrix} 1 &\hspace{-4pt} 0\\[-2pt] 0 &\hspace{-4pt} 1\\[-2pt] 2 &\hspace{-4pt} 1\end{pmatrix}\begin{pmatrix} 7 &\hspace{-4pt} 4\\[-2pt] 19 &\hspace{-4pt} 12\end{pmatrix}\begin{pmatrix} 2 &\hspace{-4pt} 1 &\hspace{-4pt} 0 &\hspace{-4pt} -1\\[-2pt] -1 &\hspace{-4pt} 0 &\hspace{-4pt} 1 &\hspace{-4pt} 2\end{pmatrix}}.$\vspace{8pt}
%(I) {\small$\begin{pmatrix} 46 &\hspace{-4pt} 33 &\hspace{-4pt} 20 &\hspace{-4pt} 7 \end{pmatrix}=2\begin{pmatrix} 10 &\hspace{-4pt} 7 &\hspace{-4pt} 4 &\hspace{-4pt} 1\end{pmatrix}+\begin{pmatrix} 26 &\hspace{-4pt} 19 &\hspace{-4pt} 12 &\hspace{-4pt} 5\end{pmatrix}=\begin{pmatrix}2 &\hspace{-4pt} 1\end{pmatrix}\begin{pmatrix} 10 &\hspace{-4pt} 7 &\hspace{-4pt} 4 &\hspace{-4pt} 1\\[-2pt] 26 &\hspace{-4pt} 19 &\hspace{-4pt} 12 &\hspace{-4pt} 5\end{pmatrix}$}, using [4E 3.51(b)].\vspace{3pt}\par\quad\HI
%{\footnotesize$\begin{pmatrix} 46 &\hspace{-4pt} 33 &\hspace{-4pt} 20 &\hspace{-4pt} 7 \end{pmatrix}$}${}\in\Span{A_{1,\cdot},A_{2,\cdot}},$ and $\Par{A_{1,\cdot},A_{2,\cdot}}$ is liney indep. Thus $B_R=\BigPar{A_{1,\cdot},A_{2,\cdot}}.$\par\vspace{6pt}\quad\EndI
%(II) {\small$\begin{pmatrix} 10\\[-2pt] 26\\[-2pt] 46\end{pmatrix}=2\begin{pmatrix} 7\\[-2pt] 19\\[-2pt] 33\end{pmatrix}-\begin{pmatrix} 4\\[-2pt] 12\\[-2pt] 20\end{pmatrix}; \quad \begin{pmatrix} 1\\[-2pt] 5\\[-2pt] 7\end{pmatrix}=-\begin{pmatrix} 7\\[-2pt] 19\\[-2pt] 33\end{pmatrix}+2\begin{pmatrix} 4\\[-2pt] 12\\[-2pt] 20\end{pmatrix}$}. \;Thus $B_C=\BigPar{A_{\cdot,2},A_{\cdot,3}}.$\vspace{10pt}\par
\SepLine

\Anchor{3CNCrankEqRrank}\BulletPointX\textsc{Col Rank = Row Rank}\quad Using CR Factoriz. Let $A=CY$ by (a) and $A=XR$ by (b).\TextB{}
(a) $A_{j,\cdot}=\BigPar{CY}{_{j,\cdot}}=C_{j,\cdot}Y=C_{j,1}Y_{1,\cdot}+\dots+C_{j,c}Y_{c,\cdot}\in\row A=\Span[\BigPar]{A_{1,\cdot},\cdots,A_{n,\cdot}}=\Span[\BigPar]{Y_{1,\cdot},\cdots,Y_{c,\cdot}}.$\TextB{}
(b) $A_{\cdot,k}=\BigPar{XR}{_{\cdot,k}}=XR_{\cdot,k}=R_{1,k}X_{\cdot,1}+\dots+R_{r,k}X_{\cdot,r}\in\col A=\Span[\BigPar]{A_{\cdot,1},\cdots,A_{\cdot,m}}=\Span[\BigPar]{X_{\cdot,1},\cdots,X_{\cdot,r}}.$\TextB{}
Thus (a) $\dim\row A=r\leqslant c=\dim\col A,$ and (b) $\dim\col A=c\leqslant r=\dim\row A.$\vspace{2pt}\PfEnd\TextB{}
\Or Apply (a) to $A^t\in\FbbP{n,m}\Rightarrow\dim\row A^t=\dim\col A=c\leqslant r=\dim\row A=\dim\col A^t.$\PfEnd
\SepLine

\Anchor{3C4e16}\ProblemBnoor{4E 16}{
	\TextB{Supp $A\in\FbbP{m,n}\nonzero.$ Prove $\rank A=1\Rightarrow\exists\,c_j,d_k\in\Fbb,$ each $A_{j,k}=c_j\cdot d_k.$\vspace{6pt}}
}Let $c_j=\Frac{A_{j,1}}{A_{1,1}}=\Frac{A_{j,2}}{A_{1,2}}=\dots=\Frac{A_{j,n}}{A_{1,n}},\;\;d_k'=\Frac{A_{1,k}}{A_{1,1}}=\Frac{A_{2,k}}{A_{2,1}}=\dots=\Frac{A_{m,k}}{A_{m,1}}.$\vspace{4pt}\parSol{}
{$\Rightarrow A_{j,k}=d_k' A_{j,1}=c_j A_{1,k}=c_j d_k' A_{1,1}=c_j d_k,$ where $d_k=d_k' A_{1,1}.$}\quad\Or Using CR Factoriz, immed.\PfEnd
\SepLine

\ProblemN{\Anchor{3C5}{5}}{
	\TextA{Supp $B_W=\Par{w_1,\dots,w_n}$ and $V$ is finide. Supp $T\in\Lm{V,W}$.\vspace{2pt}}
	\TextA{Prove $\exists\,B_V=\Par{v_1,\dots,v_m},\;\Mt[\BigPar]{T,B_V,B_W}{_{1,\cdot}}=\normalsize\begin{pmatrix}0&\cdots&0\end{pmatrix}$ or $\normalsize\begin{pmatrix}1&0&\cdots&0\end{pmatrix}$.}
}\par\quad
Let $\Par{u_1,\dots,u_n}$ be a bss of $V$. Let $A=\Mt[\BigPar]{T,\Par{u_1,\dots,u_n},B_W}.$\par\quad
If $A_{1,\cdot}=0,$ then $B_V=\Par{u_1,\dots,u_n}$ and done. Othws, supp $A_{1,k}\neq 0.$\par\quad
Let $v_1=\Frac{u_k}{A_{1,k}}\Rightarrow Tv_1=1w_1+\Frac{A_{2,k}}{A_{1,k}}w_2+\dots+\Frac{A_{n,k}}{A_{1,k}}w_n.$\;$\MathLeftMid{l}{$
	\!\!Let $v_{j+1}=u_{j}-A_{1,j}v_1$ for each $j\in\;\!\!\Bra{1,\dots,k-1}.\\$
	\!\!Let $v_i=u_i-A_{1,i} v_1$ for $i\in\;\!\!\Bra{k+1,\dots,n}.}$\vspace{4pt}\par\quad
\NOTICE that $Tu_i=A_{1,i}w_1+\dots+A_{n,i}w_n.$ 又 Each $u_i\in\Span{v_1,\dots,v_n}=V.$ Let $B_V=\Par{v_1,\dots,v_n}.$\PfEnd\vspace{6pt}\quad
\Or Using Exe (4). Let $B_{W\apostrophe}$ be the $B_V.$ Now $\exists\,B_{V\apostrophe},\,\Mt{T\apostrophe,B_{W\apostrophe}\,,B_{V\apostrophe}}{_{\cdot,1}}={\small\begin{pmatrix}1&0&\cdots&0\end{pmatrix}}{^t}$ or ${\small\begin{pmatrix}0&\cdots&0\end{pmatrix}}{^t}.$\par\quad
Which is equiv to $\exists\,B_V$ \Sbra{Using (3.F.31)} suth $\Mt{T,B_V,B_W}{_{1,\cdot}}={\small\begin{pmatrix}1&0&\cdots&0\end{pmatrix}}$ or ${\small\begin{pmatrix}0&\cdots&0\end{pmatrix}}.$\PfEnd
\SepLine\pagebreak

\ProblemN{\Anchor{3C6}{6}}{
	\TextA{Supp $V,W$ are finide and $T\in\Lm{V,W}.$ Supp $\dim\range T=1.$}
	\TextA{Prove $\exists\,B_V,B_W,$ all ent of $A=\Mt[\BigPar]{T,B_V,B_W}$ equal $1$.}
}Let $B_{\null T}=\Par{u_2,\dots,u_n}.$ Extend to a bss $\Par{u_1,u_2,\dots,u_n}$ of $V.$\parSol{}
Extend to $\Par{Tu_1,w_2,\dots,w_m}$ a bss of $W.$ Let $w_1=Tu_1-w_2-\dots-w_m\Rightarrow B_W=\Par{w_1,\dots,w_m}.$\parSol{}
Let \,$v_1=u_1,\;v_i=u_1+u_i\Rightarrow B_V=\Par{v_1,\dots,v_n}.$\PfEnd\vspace{3pt}\parSol{}
\Or Supp $B_{\range T}=\Par{w}.$ \,By {\NOTEFOR} (2.C.15), $\exists\,B_W=\Par{w_1,\dots,w_m},\;w=w_1+\dots+w_m.$\parSol{}
By \Sbra{2.C {\TIPS}}, $\exists$ a bss $\Par{u_1,\dots,u_n}$ of $V$ suth each $u_k\not\in\null T.$\parSol{}
Now each $Tu_k\in\range T=\Span{w}\Rightarrow Tu_k=\lambda_k w,\exists\,\lambda_k\in\nonzeroFbb.$ Let each $v_k=\lambda_k^{-1}u_k.$\PfEnd
\SepLine

%\ProblemN{\Anchor{3C1}{1}}{
%	\TextA{Show $A=\Mt{T}$ has at least $n=\dim\range T$ non0 ent wrto any $B_V,B_W.$}
%}\par\quad
%Let $U\oplus\null T=V;\;B_U=\Par{v_1,\dots,v_n},B_V=\Par{v_1,\dots,v_m}.$\par\quad
%Each $Tv_k\neq 0\Longleftrightarrow A_{\cdot,k}\neq 0.$ Hence every such $A_{\cdot,k}$ has at least one non0 ent.\PfEnd\vspace{4pt}\par\quad
%\Or We prove by ctradic. Supp $A$ has at most $\Par{n-1}$ non0 ent.\par\quad
%Then by Pigeon Hole Principle, at least one of $A_{\cdot,1},\dots,A_{\cdot,n}$ equals $0$.\par\quad
%Thus there are at most $\Par{n-1}$ non0 vecs in $Tv_{1},\dots,Tv_n.$\par\quad
%又 $\range T=\Span{Tv_{1},\dots,Tv_n}\Rightarrow\dim\range T=\dim\Span{Tv_{1},\dots,Tv_n}\leqslant n-1.$ Ctradic.\PfEnd
%\SepLine

\Anchor{10A3}\Anchor{3D4e19}\ProblemBnoor{10.A.3, \OR 4E 3.D.19}{
	\TextB{Supp $V$ is finide and $T\in\Lm{V}$.}
	\TextB{Supp $A=\Mt{T}$ wrto any $B_V.$ \;Prove $T=\lambda{I},\exists\,\lambda\in\Fbb.$}
}We show $\Par{v,Tv}$ liney dep by ctradic. Let $B_V=\Par{v,Tv,u_1,\dots,u_m},B_V'=\Par{2v,Tv,u_1,\dots,u_m}.$\parSol{}
Then $Tv=1\cdot Tv=2\cdot Tv=T\Par{2v}\Rightarrow Tv=0.$ Ctradic. Thus $Tv=\lambda_v v.$ \;Simlr to (4E 3.A.11).\PfEnd\vspace{2pt}\quad
%Supp $\forall B_V\neq B_V',\;\Mt{T,B_V}=\Mt{T,B'_V}.$ If $T=0$, then done.\vspace{1pt}\parSol{}
%Supp $T\neq 0,$ and $v\in V\nonzero.$ Asum $\Par{v,Tv}$ is liney indep.\vspace{1pt}\parSol{}
%Extend $\Par{v,Tv}$ to $B_V=\Par{v,Tv,u_3,\dots,u_n}.$ Let $B=\Mt{T,B_V}.$\vspace{1pt}\parSol{}
%$\Rightarrow Tv=B_{1,1} v+B_{2,1}\Par{Tv}+B_{3,1}u_3+\dots+B_{n,1}u_n\Rightarrow B_{2,1}=1,B_{i,1}=0,\,\forall i\neq 2.$\vspace{1pt}\parSol{}
%By asum, $A=\Mt{T,B_V'}=B,\forall B_V'=\Par{v,w_2,\dots,w_{n}}$. Then $A_{2,1}=1,A_{i,1}=0,\forall i\neq 2.$\parSol{}
%$\Rightarrow Tv=w_2,$ which is not true if $w_2=u_3,\;w_3=Tv,\;w_j=u_j,\forall j\in\;\!\!\Bra{4,\dots,n}.$ Ctradic.\parSol{}
%Hence $\Par{v,Tv}$ is liney dep $\Rightarrow \forall v\in V,\exists\,\lambda_v\in\Fbb,Tv=\lambda_v v.$ \;Simlr to (4E 3.A.11).\PfEnd\vspace{4pt}\quad
\Or {\FontSmall Fix one $B_V=\Par{v_1,\dots,v_m}$ and then $\Par{v_1,\dots,\frac{\;1\;}{2}v_k,\dots,v_m}$ is also a bss for any given $k\in\;\!\!\Bra{1,\dots,m}.$}\par\quad
{\FontSmall Fix one $k.$ Now \,$T\Par{\frac{\;1\;}{2}v_k}=A_{1,k}v_1+\dots+A_{k,k}\Par{\frac{\;1\;}{2}v_k}+\dots+A_{m,k}v_m$}\par\quad
{\FontSmall$\Rightarrow Tv_k=2A_{1,k}v_1+\dots+A_{k,k}v_k+\dots+2A_{m,k}v_m=A_{1,k}v_1+\dots+A_{k,k}v_k+\dots+A_{m,k}v_m.$}\par\quad
{\FontSmall Then $A_{j,k}=0$ for all $j\neq k.$ Thus each $Tv_k=A_{k,k}v_k.$} \;We show $A_{k,k}=A_{j,j}$ for all $j\neq k.$ Fix one such pair.\vspace{1pt}\par\quad
Consider $B_V'=\Par{v\apostrophe{\hspace{-3pt}_1},\dots,v\apostrophe{\hspace{-3pt}_j},\dots,v\apostrophe{\hspace{-3pt}_k},\dots,v\apostrophe_{\hspace{-3pt}_m}},$ where $v\apostrophe{\hspace{-3pt}_j}=v_k,\,\,v\apostrophe{\hspace{-3pt}_k}=v_j$ and $v\apostrophe{\hspace{-3pt}_i}=v_i$ for all $i\in\;\!\!\Bra{1,\dots,m}\Backslash[\Big]\Bra{\,j,k}.$\vspace{1pt}\par\quad
Now $T\Par{v\apostrophe{\hspace{-3pt}_k}}=A_{1,k}v\apostrophe{\hspace{-3pt}_1}+\dots+A_{k,k}v\apostrophe{\hspace{-3pt}_k}+\dots+A_{m,k}v\apostrophe{\hspace{-3pt}_m}=A_{k,k}v\apostrophe{\hspace{-3pt}_k}=A_{k,k}v_j,$ while $T\Par{v\apostrophe{\hspace{-3pt}_k}}=T\Par{v_j}=A_{j,j}v_j.$\PfEnd
%\Or ??? \Sbra{ There must be another solution using theorems and facts given in (3.D). }\par
\SepLine

\Anchor{3CT1}\ProblemBX{\TipsN{1}}{
	\TextA{Supp $p\in\PoFx{\,}{\FbbP{n}}.$ Prove $\Mt[\BigPar]{p\Par{T_{\!1},\dots,T_{\!n}}}=p\BigPar{\Mt{T_{\!1}},\dots,\Mt{T_{\!n}}}.$}
	\TextA{\FontNorm Where the liney maps $T_{\!1},\dots,T_{\!n}$ are suth $p\Par{T_{\!1},\dots,T_{\!n}}$ makes sense. \tgnr See [5.16,17,20].}
}Supp the poly $p$ is defined by $p\Par{x_1,\dots,x_n}=\sum_{k_1,\dots,k_n}\!\!\alpha_{k_1,\dots,k_n}\prod_{i=1}^n x_i^{k_i}.$\parSol{\vspace{4pt}}
Note that $\Mt{T^x S^y}=\Mt{T}{^x}\Mt{S}{^y};\;\Mt{T^x+S^y}=\Mt{T}{^x}+\Mt{S}{^y}.$\parSol{\vspace{4pt}}
Then $\Mt[\BigPar]{ p\Par{T_{\!1},\dots,T_{\!n}}}={\Mt[\BigBigPar]{{\sum_{k_1,\dots,k_n}\!\!\alpha_{k_1,\dots,k_n}\prod_{i=1}^n T_{\!i}^{k_i}}}}$\parSol{\vspace{4pt}}
\Blind{Then $\Mt[\BigPar]{ p\Par{T_{\!1},\dots,T_{\!n}}}$} $={\sum_{k_1,\dots,k_n}\!\!\alpha_{k_1,\dots,k_n}\prod_{i=1}^n\Mt[\BigPar]{T_{\!i}^{k_i}}}={p\BigBigPar{{\Mt{T_{\!1}},\dots,\Mt{T_{\!n}}}}}.$\PfEnd\vspace{6pt}
\BulletPointX\ACoro Supp $\tau$ is an algebraic property. Then $\tau$ holds for liney maps $\Longleftrightarrow \tau$ holds for matrices.\parCor{\IndentB}
Supp $\alpha_1,\dots,\alpha_n$ are disti with each $\alpha_k\in\;\!\!\Bra{1,\dots,n}.$\parCor{\IndentB}
Now $p\Par{T_{\!1},\dots,T_{\!n}}=p\Par{T_{\alpha_1},\dots,T_{\alpha_n}}\Longleftrightarrow p\BigBigPar[0pt]{{\Mt{T_{\!1}},\dots,\Mt{T_{\!n}}}}=p\BigBigPar[0pt]{{\Mt{T_{\alpha_1}},\dots,\Mt{T_{\alpha_n}}}}.$\SepLine

\Anchor{3CT2}\ProblemBX{\TipsN{2}}{
	\TextA{Supp $T\in\Lm{V,W},\;B_V=\Par{v_1,\dots,v_n},B_W=\Par{w_1,\dots,w_m}.$\vspace{-1pt}}
	{\tgsl Let $L=\BigPar{Tv_{\:\!\!\alpha_1},\dots,Tv_{\:\!\!\alpha_k}},\;L_{\mM}=\BigPar{A_{\cdot,\,\alpha_1},\cdots,A_{\cdot,\,\alpha_k}},$ where each $\alpha_i\in\;\!\!\Bra{1,\dots,n}.$}\TextA{\vspace{1pt}}
	\PrePa\TextA{Show [P] $L$ is liney indep $\Longleftrightarrow L_{\mM}$ is liney indep. [Q]\vspace{2pt}}
	\PrePb\TextA{Show [P] $\spn L=W\Longleftrightarrow\spn L_{\mM}=\FbbP{m,1}.$ [Q]\hfill\FontNorm\Sbra{ Let $A=\Mt{T,B_V,B_W}.$\hspace{1pt}}\vspace{3pt}}
}(a) Note that $\mM\!:\,Tv_k\rightarrow A_{\cdot,k}$ is iso. of $\spn L$ onto $\spn L_{\mM}.$ By (3.B.9).\parSol{}
(b) Reduce to liney indep lists. By (a) and [2.39].\PfEnd\vspace{4pt}\quad
\Or\;$c_1Tv_{\:\!\!\alpha_1}+\dots+c_kTv_{\:\!\!\alpha_k}=c_1\BigPar{A_{1,\,{\alpha_1}}w_1+\dots+A_{m,\,{\alpha_1}}w_m}+\dots+c_k\BigPar{A_{1,\,{\alpha_k}}w_1+\dots+A_{m,\,{\alpha_k}}w_m}$\par\vspace{2pt}\quad
\Blind{\Or\;}$\Blind{c_1Tv_{\:\!\!\alpha_1}+\dots+c_kTv_{\:\!\!\alpha_k}}=\BigPar{c_1A_{1,\,{\alpha_1}}+\dots+c_kA_{1,\,{\alpha_k}}}w_1+\dots+\BigPar{c_1A_{m,\,{\alpha_1}}+\dots+c_kA_{m,\,{\alpha_k}}}w_m.$\par\vspace{4pt}\quad
\Blind{\Or\;}And \;$c_1A_{\cdot,{\alpha_1}}+\dots+c_kA_{\cdot,{\alpha_k}}=c_1{\normalsize\begin{pmatrix}A_{1,\,{\alpha_1}}\\[-4pt]\vdots\\[-10pt]A_{m,\,{\alpha_1}}\end{pmatrix}}+\dots+c_k{\normalsize\begin{pmatrix}A_{1,\,{\alpha_k}}\\[-4pt]\vdots\\[-10pt]A_{m,\,{\alpha_k}}\end{pmatrix}}{}={}\normalsize\begin{pmatrix}c_1 A_{1,\,{\alpha_1}}+\dots+c_kA_{1,\,{\alpha_k}}\\[-4pt]\vdots\\[-4pt]c_1A_{m,\,{\alpha_1}}+\dots+c_kA_{m,\,{\alpha_k}}\end{pmatrix}$.
\pagebreak\par\vspace{6pt}\quad
(a) $P\Rightarrow Q:$\,\;Supp $\;c_1A_{\cdot,{\alpha_1}}+\dots+c_kA_{\cdot,{\alpha_k}}=0.$ \;Let $v=c_1v_{\:\!\!\alpha_1}+\dots+c_kv_{\:\!\!\alpha_k}.$\par\quad\Ha
\Blind{$Q\Rightarrow P:$\,\;}Then $Tv=\BigPar{c_1A_{1,\,{\alpha_1}}+\dots+c_kA_{1,\,{\alpha_k}}}w_1+\dots+\BigPar{c_1A_{m,\,{\alpha_1}}+\dots+c_kA_{m,\,{\alpha_k}}}w_m=0w_1+\dots+0w_m.$\par\quad\Ha
\Blind{$Q\Rightarrow P:$\,\;}Now $c_1Tv_{\:\!\!\alpha_1}+\dots+c_kTv_{\:\!\!\alpha_k}=0.$ Then each $c_i=0\Rightarrow L_{\mM}$ liney indep.\vspace{4pt}\par\quad\Ha
$Q\Rightarrow P:$\,\;Becs $c_1Tv_{\:\!\!\alpha_1}+\dots+c_kTv_{\:\!\!\alpha_k}=0.$ For each $i\in\;\!\!\Bra{1,\dots,m},\;c_1A_{i,\,{\alpha_1}}+\dots+c_kA_{i,\,{\alpha_k}}=0.$\par\quad\Ha
\Blind{$Q\Rightarrow P:$\,\;}Which is equiv to $c_1A_{\cdot,{\alpha_1}}+\dots+c_kA_{\cdot,{\alpha_k}}=0.$ \;Thus each $c_i=0\Rightarrow L$ liney indep.\par\vspace{4pt}\quad\Ha
\Or\;$\exists\,A_{\cdot,{\alpha_j}}=c_1A_{\cdot,{\alpha_1}}+\dots+c_{j-1}A_{\cdot,{\alpha_{j-1}}}$\par\quad\Ha
\Blind{\Or\;}$\Longleftrightarrow$ For each $i\in\;\!\!\Bra{1,\dots,m},\;A_{i,\,{\alpha_j}}=c_1A_{i,\,{\alpha_1}}+\dots+c_{j-1}A_{i,\,{\alpha_{j-1}}}$\par\quad\Ha
\Blind{\Or\;}$\Longleftrightarrow Tv_{\:\!\!\alpha_j}=A_{1,\,{\alpha_j}}w_1+\dots+A_{m,\,{\alpha_j}}w_m$\par\vspace{2pt}\quad\Ha
\Blind{\Or\;}$\Blind{\Longleftrightarrow Tv_{\:\!\!\alpha_j}}=\BigPar{c_1A_{1,\,{\alpha_1}}+\dots+c_{j-1}A_{1,\,{\alpha_{j-1}}}}w_1+\dots+\BigPar{c_1A_{m,\,{\alpha_1}}+\dots+c_{j-1}A_{m,\,{\alpha_{j-1}}}}w_m$\par\vspace{2pt}\quad\Ha
%\Blind{\Or}$\Blind{\Longleftrightarrow Tv_{\:\!\!\alpha_j}}=c_1\BigPar{A_{1,\,\alpha_1}w_1+\dots+A_{m,\alpha_1}w_m}+\dots+c_{j-1}\BigPar{A_{1,\,\alpha_{j-1}}w_1+\dots+A_{m,\alpha_{j-1}w_m}}$\par\quad\Ha
\Blind{\Or\;}$\Longleftrightarrow\exists\,Tv_{\:\!\!\alpha_j}=c_1Tv_{\:\!\!\alpha_1}+\dots+c_{j-1}Tv_{\:\!\!\alpha_{j-1}}.$\par\vspace{6pt}\quad
(b) Note that each $\Mt{Tv_{\:\!\!\alpha_i}}=A_{\cdot,\,\alpha_i}$\par\quad\Hb
$P\Rightarrow Q:$\,\;Supp each $w_i=Iw_i=J_{1,i}Tv_{\:\!\!\alpha_1}+\dots+J_{k,\,i}Tv_{\:\!\!\alpha_k}.$\par\quad\Hb
%Then fix one $J=\Mt{I,B_W,L}\in\FbbP{k,m}.$
\Blind{$Q\Rightarrow P:$\,\;}$\forall a\in\FbbP{m,1},\exists\,!\,w=a_1w_1+\dots+a_mw_m\in W,\;a=\Mt{w,B_W}.$\par\quad\Hb
\Blind{$Q\Rightarrow P:$\,\;}Becs $w=a_1\BigPar{J_{1,1}Tv_{\:\!\!\alpha_1}+\dots+J_{k,1}Tv_{\:\!\!\alpha_k}}+\dots+a_m\BigPar{J_{1,m}Tv_{\:\!\!\alpha_1}+\dots+J_{k,m}Tv_{\:\!\!\alpha_k}}$\par\vspace{2pt}\quad\Hb
\Blind{$Q\Rightarrow P:$\,\;Becs} $\Blind{w}=\BigPar{a_1J_{1,1}+\dots+a_mJ_{1,m}}Tv_{\:\!\!\alpha_1}+\dots+\BigPar{a_1J_{k,1}+\dots+a_mJ_{k,m}}Tv_{\:\!\!\alpha_k}.$\par\vspace{2pt}\quad\Hb
\Blind{$Q\Rightarrow P:$\,\;}Apply $\mM$ to both sides, $a=c_1A_{\cdot,\,\alpha_1}+\dots+c_kA_{\cdot,\,\alpha_k},$ where each $c_i=a_1J_{i,1}+\dots+a_mJ_{i,m}.$\par\vspace{6pt}\quad\Hb
$Q\Rightarrow P:$\,\;$\forall w\in W,\exists\,\!\,a=\Mt{w,B_W}\Rightarrow\exists\,c_k\in\Fbb,a=c_1A_{\cdot,{\alpha_1}}+\dots+c_kA_{\cdot,\,\alpha_k}\in\FbbP{m,1}$\par\quad\Hb
\Blind{$Q\Rightarrow P:$\,\;}$\Rightarrow w=\BigPar{c_1A_{1,\,{\alpha_1}}+\dots+c_kA_{1,\,{\alpha_k}}}w_1+\dots+\BigPar{c_1A_{m,\,{\alpha_1}}+\dots+c_kA_{m,\,{\alpha_k}}}w_m=c_1Tv_{\:\!\!\alpha_1}+\dots+c_kTv_{\:\!\!\alpha_k}.$\vspace{6pt}\par\quad\Hb
${}{^\neg}Q\Rightarrow{}{^\neg}P:$\,\;$\exists\,w\in W,\exists\,a\in\FbbP{m,1},\Mt{w,B_W}=a,$ but $\nexists\,\Par{c_1,\dots,c_k}\in\FbbP{k},a=c_1A_{\cdot,\,\alpha_1}+\dots+c_kA_{\cdot,\,\alpha_k}$\par\quad\Hb
\Blind{${}{^\neg}Q\Rightarrow{}{^\neg}P:$\,\;}$\Rightarrow\nexists\,\Par{c_1,\dots,c_k}\in\FbbP{k},\;w=c_1Tv_{\:\!\!\alpha_1}+\dots+c_kTv_{\:\!\!\alpha_k}.$ For if not, ctradic.\PfEnd\vspace{6pt}
\ANote Let $L=\BigPar{Tv_1,\dots,Tv_n},\;L_{\mM}=\BigPar{A_{\cdot,1},\cdots,A_{\cdot,n}}.$\parNot
Then (a*) By \Sbra{3.B.9, \TIPSN{4}}, $T$ is inje $\Longleftrightarrow L$ is liney indep, so is $L_{\mM}$.\parNot
And (b*) $T$ is surj $\Longleftrightarrow\spn L=W\Longleftrightarrow\spn L_{\mM}=\FbbP{m,1}.$\Anchor{3C4e17}\parNot
\ACoro $B_{\FbbP{n,1}}=\BigPar{A_{\cdot,1},\cdots,A_{\cdot,n}}\Longleftrightarrow T$ is inje and surj $\Longleftrightarrow B_{\FbbP{1,n}}=\BigPar{A_{\cdot,1},\cdots,A_{\cdot,n}}.$\parNot
\AComm If $T$ is inv. Then by (a*, c) or (b*, d), we have another proof of \COROLLARY.\Anchor{3F32}\parNot\IndentComment
\Or If $m=n.$ Then by [3.118] and one of (a*, b*, c, d). Yet another proof.\parNot
(c) $T$ surj $\Longleftrightarrow T\apostrophe$ inje $\Longleftrightarrow\BigPar{T\apostrophe\Par{\psi_1},\dots,T\apostrophe\Par{\psi_m}}$ liney indep\parNot\Hc
\Blind{$T$ surj }$\xLongleftrightarrow{\text{(a)}}{\BigBigPar{{\Par{A^t}{_{\cdot,1}},\cdots,\Par{A^t}{_{\cdot,m}}}}}$ liney indep in $\FbbP{n,1},$ so is $\BigPar{A_{1,\cdot},\cdots,A_{m,\cdot}}$ in $\FbbP{1,n}.$\vspace{4pt}\parNot
(d) $T$ inje $\Longleftrightarrow T\apostrophe$ surj $\Longleftrightarrow V\apostrophe=\Span[\BigPar]{T\apostrophe\Par{\psi_1},\dots,T\apostrophe\Par{\psi_m}}$\parNot\Hd
\Blind{$T$ inje }$\xLongleftrightarrow{\text{(b)}}\FbbP{n,1}=\Span[\BigBigPar]{{\Par{A^t}{_{\cdot,1}},\cdots,\Par{A^t}{_{\cdot,m}}}}\Longleftrightarrow\FbbP{1,n}=\Span[\BigBigPar]{A_{1,\cdot},\cdots,A_{m,\cdot}}.$
\SepLine\ChEnd


%\ProblemN{\Anchor{3C15}{15}}{
	%	\TextA{Supp $A\in\FbbP{n,n},j,k\in\;\!\!\Bra{1,\dots,n}$. Show $\BigPar{A^3}_{j,k}=\sum_{p=1}^n \sum_{r=1}^n A_{j,\,p}A_{p,r}A_{r,k}$.\vspace{6pt}}
	%}\vspace{-6pt}\AlignEq{}{\BigPar{AAA}_{j,k} &= \BigPar{AA}_{j,\cdot}\hspace{1pt}A_{\cdot,k}=\textstyle\sum_{p=1}^n \BigPar{A_{j,\,p}A_{p,\cdot}}A_{\cdot,k}=\textstyle\sum_{p=1}^n \sum_{r=1}^n A_{j,\,p}A_{p,r}A_{r,k}.\hspace{60pt}\\
	%	\text{\Or}\;\;\BigPar{AAA}_{j,k} &=\textstyle\sum_{r=1}^n\BigPar{AA}_{j,r}A_{r,k} =\textstyle\sum_{r=1}^n\XPar{{\sum_{p=1}^n A_{j,\,p}A_{p,r}}}\hspace{1pt}A_{r,k}\\&=\uline{\textstyle\sum_{r=1}^n\left[A_{j,1}\BigPar{A_{1,r}A_{r,k}}+\dots+A_{j,n}\BigPar{A_{n,r}A_{r,k}}\right]}\\&=\textstyle A_{j,1}\sum_{r=1}^n A_{1,r}A_{r,k}+\dots+A_{j,n}\textstyle\sum_{r=1}^n A_{n,r}A_{r,k}=\textstyle\sum_{p=1}^n \sum_{r=1}^n A_{j,\,p}A_{p,r}A_{r,k}.}\PfEnd[-25pt]
%\SepLine

\pagebreak

\ChDecl{Ch3D}{3$\cdot$D}{}

\vspace{4pt}

\Anchor{3E2}\ProblemBnoor{{3.E.2}}{
	\TextB{Supp $V_{\!1}\times\dots\times V_{\!m}$ is finide. Prove each $V_{\!j}$ is finide.}
}Define each $S_k\in\Lm{V_{\!1}\times\dots\times V_{\!m},V_{\!k}}$ by $S_k\Par{v_1,\dots,v_m}=v_k.$ By [3.22], $\range S_k=V_{\!k}$ is finide.\PfEnd\vspace{3pt}\parSol{}
\Or Denote $V_{\!1}\times\cdots\times V_{\!m}$ by $U$. Denote $\zeroSubs\times\cdots\times\zeroSubs\times V_{\!i}\times\zeroSubs\cdots\times\zeroSubs$ by $U_i$.\parSol{}
We show each $U_i$ is iso to $V_{\!i}.$ Then $U$ is finide $\Longrightarrow$ its subsp $U_i$ is  finide, so is $V_{\!i}.$\parSol{\vspace{2pt}}
$\!\!\!\MathRightBrace{l}{$
	Define $R_i\in\Lm{V_{\!i},U_i}$ by $R_i\Par{u_i}=\Par{0,\dots,0,u_i,0,\dots,0}\\ $
	Define $S_i\in\Lm{U,V_{\!i}}$ by $S_i\Par{u_1,\dots,u_i,\dots,u_m}=u_i$
	$}\Rightarrow\MathLeftBrace{l}{R_iS_j\mmid_{U_j}=\delta_{i,j}I_{U_j},\\S_iR_j=\delta_{i,j}I_{V_{\!j}}.}$\PfEnd\vspace{2pt}\parSol{}
\AComm The key tool in solus of (3.E.4,5).
\SepLine

%\Anchor{3E4}\ProblemBnoor{{3.E.4}}{
%	\TextB{Prove $\Lm{V_{\!1}\times\cdots\times V_{\!m},W}$ and $\Lm{V_{\!1},W}\times\cdots\times\Lm{V_{\!m},W}$ are iso.}
%}Using notat in (3.E.2): $R_i:u_i\mapsto\Par{0,\dots,u_i,\dots,0};\;S_i:\Par{u_1,\dots,u_m}\mapsto u_i.$\par\quad
%Note that $T\Par{u_1,\dots,u_m}=T\Par{u_1,0,\dots,0}+\dots+T\Par{0,\dots,u_m}.$\par{\hspace{0pt}}
%$\MathRightBrace{l}{$
%	Define $\varphi:T\mapsto\Par{T_{\!1},\dots,T_{\!m}}$ by $\varphi\Par{T}=\Par{TR_1,\dots,TR_m}.\\ $
%	Define $\psi:\Par{T_{\!1},\dots,T_{\!m}}\mapsto T$ by $\psi\Par{T_{\!1},\dots,T_{\!m}}=T_{\!1} S_1+\dots+T_{\!m} S_m.$
%	$}\Rightarrow\psi=\varphi^{-1}.$\PfEnd
%\SepLine
%
%\Anchor{3E5}\ProblemBnoor{{3.E.5}}{
%	\TextB{Prove $\Lm{V,W_1\times\cdots\times W_m}$ and $\Lm{V,W_1}\times\cdots\times\Lm{V,W_m}$ are iso.}
%}Using notat in (3.E.2): $R_i:u_i\mapsto\Par{0,\dots,u_i,\dots,0};\;S_i:\Par{u_1,\dots,u_m}\mapsto u_i.$\par{\hspace{0pt}}
%$\hText{$Note that $T_{\!i}:v\mapsto w_i,\\\;T:v\mapsto\Par{w_1,\dots,w_m}.}\hMath{l}{\left|}{\right\}}{$
%	Define $\varphi:T\mapsto \Par{T_{\!1},\dots,T_{\!m}}$ by $\varphi\Par{T}=\Par{S_1 T,\dots,S_m T}.\\ $
%	Define $\psi:\Par{T_{\!1},\dots,T_{\!m}}\mapsto T$ by $\psi\Par{T_{\!1},\dots,T_{\!m}}=R_1T_{\!1}+\dots+R_mT_{\!m}.$
%	$}\Rightarrow\psi=\varphi^{-1}.$\PfEnd[-8pt]\vspace{-8pt}
%\SepLine

\ProblemN{\Anchor{3D18}{18}}{
	\TextA{Show $V$ and $\Lm{\Fbb, V}$ are iso vecsps.}
}Define $\Psi\in\Lm[\BigPar]{V,\Lm{\Fbb, V}}$ by $\Psi\Par{v}=\Psi_{\!v};$ \;where $\Psi_{\!v}\in\Lm{\Fbb, V}$ and $\Psi_{\!v}\Par{\lambda}=\lambda v.$\parSol{}
(a) $\Psi\Par{v}=\Psi_{\!v}=0\Rightarrow \forall \lambda\in\Fbb,\Psi_{\!v}\Par{\lambda}=\lambda v=0\Rightarrow v=0.$ Now $\Psi$ inje.\parSol{}
(b) $\forall T\in\Lm{\Fbb,V},$ let $v= T\Par{1}\Rightarrow T\Par{\lambda}=\lambda v=\Psi_{\!v}\Par{\lambda},\forall\lambda\in\Fbb\Rightarrow T=\Psi\BigPar{T\Par{1}}\in\range \Psi.$\PfEnd\vspace{4pt}\parSol{}
\Or Define $\Phi\in\Lm[\BigPar]{\Lm{\Fbb,V}, V}$ by $\Phi\Par{T}=T\Par{1}.$\parSol{}
(a) Supp $\Phi\Par{T}=0=T\Par{1}=\lambda T\Par{1}=T\Par{\lambda},\forall\lambda\in\Fbb\Rightarrow T=0.$ Now $\Phi$ inje.\parSol{}
(b) For any $v\in V,$ define $T\in\Lm{\Fbb,V}$ by $T\Par{\lambda}=\lambda v.$ Then $\Phi\Par{T}=T\Par{1}=v\in\range\Phi.$\PfEnd\vspace{2pt}
\AComm $\Phi=\Psi^{-1}.$ \;{\FontSmall This is a countexa of the stmt that $\Lm{V,W}$ and $\Lm{W,V}$ are iso if infinde. See (3.F).}
\SepLine

\Anchor{3E6}\ProblemBnoor{{3.E.6}}{
	\TextB{Supp $m\in\Nbp.$ Prove $V^m$ and $\Lm{\FbbP{m},V}$ are iso.\tgnr\FontNorm\hfill By (3.D.18, 3.E.4), immed.}
}\Or Define $T:\Par{v_1,\dots,v_m}\rightarrow\varphi$, where $\varphi:\Par{a_1,\dots,a_m}\mapsto a_1 v_1+\dots+a_m v_m.$\par\vspace{2pt}\quad
(a) Supp $T\Par{v_1,\dots,v_m}=0.$ Then $\forall\Par{a_1,\dots,a_n}\in\FbbP{m}$, $\varphi\Par{a_1,\dots,a_m}=a_1 v_1+\dots+a_m v_m=0$\par\quad\Ha
For each $k,$ let $a_k=1,a_j=0$ for all $j\neq k.$ Then each $v_k=0\Rightarrow\Par{v_1,\dots,v_m}=0.$ Thus $T$ is inje.\par\vspace{2pt}\quad
(b) Supp $\psi\in\Lm{\FbbP{m},V}.$ Let $\Par{e_1,\dots,e_m}$ be std bss of $\FbbP{m}$. Then $\forall\Par{b_1,\dots,b_n}\in\FbbP{m}$,\vspace{3pt}\par\quad\Hb
$\XSbra{T\BigBigPar{\psi\Par{e_1},\dots,\psi\Par{e_m}}}\Par{b_1,\dots,b_m}=b_1\psi\Par{e_1}+\dots+b_m\psi\Par{e_m}=\psi\BigPar{b_1 e_1+\dots+b_m e_m}=\psi\Par{b_1,\dots,b_m}$.\vspace{3pt}\par\quad\Hb
Thus $T\BigPar{\psi\Par{e_1},\dots,\psi\Par{e_m}}=\psi$. Hence $T$ is surj.\PfEnd
\SepLine

\Anchor{3E3}\ProblemBnoor{{3.E.3}}{
	\TextA{Give an exa of a vecsp $V$ and its two subsps $U_1,U_2$ suth}
	\TextA{$U_1\times U_2$ and $U_1+U_2$ are iso but $U_1+U_2$ is not a direct sum.\FontNorm\hfill\Sbra{$V$ must be infinide.}}
}{\NOTE} {\tgsl that at least one of $U_1,U_2$ must be infinide.\;\; Both can be infinide. \Sbra{Req Other Courses.}}\par\quad
Let $V=\FbbP{\infty}=U_1,\;U_2=\Bra{\Par{x,0,\cdots}\in\FbbP{\infty}:x\in\Fbb}.$ Then $V=U_1+U_2$ is not a direct sum.\par{\hspace{0pt}}
$\MathRightBrace{l}{$
	Define $T\in\Lm{U_1\times U_2,U_1+U_2}$ by $T\BigPar{\Par{x_1,x_2,\cdots},\Par{x,0,\cdots}}=\Par{x,x_1,x_2,\cdots}\\ $
	Define $S\in\Lm{U_1+U_2,U_1\times U_2}$ by $S\Par{x,x_1,x_2,\cdots}=\BigPar{\Par{x_1,x_2,\cdots},\Par{x,0,\cdots}}$
	$}\Rightarrow S=T^{-1}.$\PfEnd%\vspace{10pt}\quad
%\Or Let $V=\FbbP{\infty},U_1=\Bra{\Par{x_i}^\infty_{i=1}\in\FbbP{\infty}:\sum_{i=1}^\infty \Par{{-1}}^{i}x_i=0},U_2=\Bra{\Par{y_1,y_2,\cdots}\in\FbbP{\infty}:y_1+y_2+\dots=0}.$\par\quad
%$\MathRightBrace{l}{$
	%	Define $T\in\Lm{U_1\times U_2,U_1+U_2}$ by $T\BigPar{\Par{x_1,x_2,\cdots},\Par{y_1,y_2,\cdots}}=\Par{x_1,y_1,x_2,y_2,\cdots}\\$
	%	Define $S\in\Lm{U_1+U_2,U_1\times U_2}$ by $S\Par{z_1,z_2,z_3,z_4,\cdots}=\BigPar{\Par{z_1,z_3,\cdots},\Par{z_2,z_4,\cdots}}}\Rightarrow S=T^{-1}.$\PfEnd
\SepLine

\Anchor{3D'1}\ProblemB{
	\TextB{Supp $T\in\Lm{V}.$ Prove $\exists$ inv $T_{\!1},T_{\!2}\in\Lm{V}$ suth $T=T_{\!1}+T_{\!2}.$}
}Let $U\oplus\null T=V,\;W\oplus\range T=V.$ Let $S:\null T\rightarrow W$ be an iso.\parSol{}
\hspace{-6pt}$\MathRightBrace{l}{$Define $T_{\!1}\in\Lm{V}$ by $T_{\!1}\Par{u}=\frac{\;1\;}{2}Tu,T_{\!1}\Par{w}=Sw\\$Define $T_{\!2}\in\Lm{V}$ by $T_{\!2}\Par{u}=\frac{\;1\;}{2}Tu,T_{\!2}\Par{w}=-Sw}\Rightarrow T=T_{\!1}+T_{\!2}$ and $T_{\!1},T_{\!2}$ inv.\PfEnd
\SepLine

\Anchor{3D'2}\ProblemB{
	\TextB{Supp $A,B\in\Lm{V}$ and $A+B,\,A-B$ are inv. Supp $C,D\in\Lm{V}.$}
	\TextB{Prove $\exists\,X,Y\in\Lm{V}$ suth $AX+BY=C,BX+AY=D.$}
}Asum $AX+BY=C,BX+AY=D.$ Then $\Par{A\pm B}\Par{X\pm Y}=C\pm D.$\parSol{}
Let $S=\Par{A+B}{^{-1}}\Par{C+D},T=\Par{A-B}{^{-1}}\Par{C-D}.$ Now $X={}${\Large$\frac{\:1\:}{2}$}$\Par{S+T},Y={}${\Large$\frac{\:1\:}{2}$}$\Par{S-T}.$\PfEnd
\SepLine

%\Anchor{3DNE2}\BulletPointX\NoteForSmall{Exe (2)}\;\;The set of inv optors on $V$ of dim $\geqslant2$ is also not a subsp.\TextB{}
%\Blind{\NoteForSmall{Exe (2)}\;\;}Although multi id/inv, and commu for vec multi hold.
%\SepLine

\ProblemN{\Anchor{3D3}{3}}{
	\TextA{Supp $V$ and $W$ are finide, $U$ is a subsp of $V$, and $S\in\Lm{U,W}.$}
	\TextA{Prove $\exists$ inv $T\in\Lm{V,W},Tu = Su,\forall u\in U$ $\Longleftrightarrow$ $S$ is inje. \hfill{\FontNorm\Sbra{ See also (3.A.11). }}}
}(a) $\forall u\in U,u=T^{-1}Su\Rightarrow T^{-1}S=I\in\Lm{U}.$ \Or $\null S=\null T\mmid_U=\null T\cap U=\zeroSubs.$\parSol{}
(b) Get a $B_U,$ apply $S,$ then extend to $B_V,B_W.$\PfEnd\vspace{2pt}
\AExa Let $V=W=\FbbP{\infty}.$ Define $S\Par{x_1,x_2,\cdots}=\Par{0,x_1,x_2,\cdots}\Rightarrow S$ inje.\parExa
Asum $\exists$ inv $T\in\Lm{V,W}$ suth $T\mmid_V=S.$ Then $T=S$ while $S$ is not surj.\SepLine

\ProblemN{\Anchor{3D8}{8}}{
	\TextA{Supp $T\in\Lm{V,W}$ is {\tgsc surj}. Prove $\exists$ subsp $U$ of $V,$ \;$T\mmid_U:U\rightarrow W$ is iso.}
}By (3.B.12). Note that $\range T=W.$ \; \Or \Sbra[3pt]{{\tgsl Req $\range T$ Finide}} \;By \Sbra{3.B \TIPSN{4}}.\PfEnd\vspace{-2pt}
%\vspace{4pt}\AComm See (3.B.12), (4E 3.B.21), (3.B \TIPS).
\SepLine

\Anchor{3DT1}\ProblemBX{\TipsN{1}}{
	\TextA{Supp $V=U\oplus X=W\oplus X.$ Prove $U,W$ are iso.}
}$\forall u\in U,\exists\,!\,\Par{w,x_1}\in W\times X,u=w+x_1.$ While $\exists\,!\,\Par{u\apostrophe,x_2}\in U\times X,w=u\apostrophe+x_2.$\parSol{}
Now $x_1=-x_2,\:u=u\apostrophe.$ Thus $\pi:U\rightarrow W$ defined by $\pi\Par{u}=w,$ \,is inje.\parSol{\vspace{3pt}}
$\forall w\in W,\exists\,!\,\Par{u,x_1}\in U\times X,w=u+x_1.$ While $\exists\,!\,\Par{w\apostrophe,x_2}\in W\times X,u=w\apostrophe+x_2.$\parSol{}
Now $x_1=-x_2,\:w=w\apostrophe.$ Thus $\pi:U\rightarrow W$ defined by $\pi\Par{u}=w,$ \,is surj.\PfEnd\vspace{4pt}
\AComm Let $V=\FbbP{\infty}.$ Let $X=\FbbP{\infty},Y=\Bra{\Par{0,x_1,x_2,\cdots}\in\FbbP{\infty}}.$ Now $X,Y$ are iso subsps of $V.$\parCom
But $\nexists$ iso subsps $M,N$ of $V,$ suth $V=M\oplus X=N\oplus Y.$
\SepLine

%\BulletPointX\NoteForSmall{[3.69]}\;\;Supp $V,W$ are finide and iso, $T\in\Lm{V,W}.$ Then $T$ inv $\Longleftrightarrow$ inje $\Longleftrightarrow$ surj.
%\SepLine

%\Anchor{3D'2}\BulletPointX\AComm If $S\in\Lm{V}$ is iso, $T\in\Lm{U,W}$ is iso, and $W\subsetneq V,$ then $ST=S\mmid_{W}T$ is merely inje.\vspace{-2pt}
%\SepLine

\Anchor{3D9}\ProblemN{9}{
	\TextA{Supp $U,V,W$ are finide, while $S\in\Lm{V,W},T\in\Lm{U,V},$ and $ST$ inv.}
	\TextA{Prove $S,T$ are inv.\hfill\FontSmall\tgnr\ANote Not true if $U,V,W$ infinide. Exa: Forwd and backwd shift.}
}Let $R=\Par{ST}{^{-1}}.$ Becs $R\Par{ST}=\Par{RS}T=I_U$ \OR $\Par{ST}R=S\Par{TR}=I_W,$ $T$ inje and $S$ surj.\PfEnd\parSol{}
\Or $\dim W=\dim\range ST\leqslant\min\!\Bra{\range S,\range T}\Rightarrow S,T$ surj.\PfEnd
\SepLine

\ProblemN{\Anchor{3D10}{10}}{
	\TextA{Supp $V,W$ are finide and $T\in\Lm{V,W},S\in\Lm{W,V}.$ Prove $ST=I\Longleftrightarrow TS=I.$}
}Supp $ST=I\Rightarrow S,T$ inv, by (3.B.20,21). Again, $TS_1=I.$ Becs $STS_1=S_1=S.$\PfEnd\parSol{}
\Or $S\BigPar{\Par{TS}w}=ST\Par{Sw}=Sw\Rightarrow\Par{TS}w=w.$ \;\Or {\FontSmall$S^{-1}=T$ 又 $S=S\Rightarrow TS=S^{-1}S=I$.}\PfEnd
\SepLine

\Anchor{3DT2}\BulletPointX\TipsN{2} \,\,\,Supp each $S_k\in\Lm{V_{\!k},W_k},W_k\subseteq V_{\!k+1}\Rightarrow S_m\circ S_{m-1}\circ\cdots\circ S_2\circ S_1$ makes sense.\TextB{}
(a) By the ctrapos of (3.B.11), $S_m\circ\cdots\circ S_1$ not inje $\Rightarrow\exists\,S_k$ not inje. Convly not true unless $k=1.$\TextB{}
(b) By Exe (9), if all $V_{\!k}$ finide and iso to each other, then $S_m\circ\cdots\circ S_1$ inje $\Rightarrow$ inv, so are all $S_k.$\TextB{}
(c) $\null S_1\subseteq \Null\Par{S_2S_1}\subseteq\cdots\subseteq\Null\Par{S_m\cdots S_2S_1};$ \;$S_m\circ\cdots\circ S_1$ inje $\Rightarrow$ each $S_k\circ\cdots\circ S_1$ inje.\vspace{2pt}\TextB{}
Supp each $W_k=V_{\!k+1},$ for if $W_k\subsetneq V_{\!k+1},$ then $S_1,S_2$ surj $\notRightarrow S_2\circ S_1\in\Lm{V_{\!1},W_2}$ surj.\TextB{}
(d) Each $S_k$ surj $\Rightarrow S_m\circ\cdots\circ S_1$ surj. Convly not true unless all $V_{\!k}$ finide and iso to each other.\TextB{}
(e) $\range S_m\supseteq\Range\Par{S_mS_{m-1}}\supseteq\cdots\supseteq\Range\Par{S_mS_{m-1}\cdots S_1};$ \;$S_m\circ\cdots\circ S_1$ surj $\Rightarrow$ each $S_m\circ\cdots\circ S_k$ surj.
%	\BulletPointX Define $X_p=\Bra{T\in\Lm{V}:p\Par{T}\text{ holds}};\;P_{\!p}:X_p$ is closd vec multi;\;$Q_p:X_p$ is a group.\TextB{}
%	(1) $S$ surj $\Leftarrow$ each $S_k$ surj. \;$P_{\!surj}$ holds. \; (2) $S$ inje $\Leftarrow$ each $S_k$ inje. \;$P_{\!inje}$ holds.\TextB{}
%	(3) $P_{\!inv}$ and $Q_{inv}$ hold. \;\; (4) $Q_{p}$ in (1) and (2) holds $\Longleftrightarrow V$ is finide.\TextB{}
%	(5) $P_{\!inje\;or\;surj}$ holds $\Longleftrightarrow V$ is finide $\Longleftrightarrow Q_{inje\;or\;surj}$ holds.\TextB{\vspace{-4pt}}
\SepLine

%\ProblemN{\Anchor{3D13}{13}}{
%	\TextA{Supp $U,V,W,X$ are iso and finide, $R\in\Lm{W,X}, S\in\Lm{V,W}, T\in\Lm{U,V}.$}
%	\TextA{Supp $RST$ is surj. Prove $S$ is inje.}
%}Using Exe (9). Notice that $U,X$ are finide, so that $RST$ inv.\vspace{4pt}\par\quad
%Let $X=\Par{RST}{^{-1}}\,\left|\hspace{-4pt}\begin{array}{l}$ $Tv=0\Rightarrow v=X\Par{RSTv}=0\Rightarrow T$ inje.$\\ $
%	$\forall v\in V,v=\Par{RST}Xv\in\range R\Rightarrow R$ surj.$\\\end{array}\right\}\Rightarrow S=R^{-1}\Par{RST}T^{-1}.$\PfEnd\vspace{8pt}\quad
%\Or $\Par{RST}{^{-1}}=\BigPar{\Par{RS}T}{^{-1}}=T^{-1}\Par{RS}{^{-1}}=T^{-1}S^{-1}R^{-1}.$\PfEnd
%\SepLine

%\ProblemN{\Anchor{3D1}{1}}{
%	\TextA{Supp $T\in\Lm{U,V},S\in\Lm{V,W}$ are inv. Prove $ST$ is inv and $\Par{ST}{^{-1}} = T^{-1}S^{-1}$.\vspace{8pt}}
%}$\MathRightBrace{l}{\Par{ST}\Par{T^{-1}S^{-1}}=STT^{-1}S^{-1}=I\in\Lm{W}\\\Par{T^{-1}S^{-1}}\Par{ST}=T^{-1}S^{-1}ST=I\in\Lm{V}}\Rightarrow \Par{ST}{^{-1}} = T^{-1}S^{-1}$, by the uniqnes of inv.\PfEnd
%\SepLine

%\ProblemN{\Anchor{3D11}{11}}{
%	\TextA{Supp $V$ is finide, $S, T, U\in\Lm{V}$ and $STU = I$. Show $T$ is inv and $T^{-1} = US$.}
%}Using Exe (9) and (10). {\tgsl\normalsize This result can fail without the hypo that $V$ is finide.}\parSol{}
%$\Par{ST}U=U\Par{ST}=\Par{US}T=I\Rightarrow T^{-1}=US.$\parSol{}
%\Or $\Par{ST}U=S\Par{TU}=I\Rightarrow U,S$ inv $\Rightarrow TU=S^{-1}.$ 又 $U^{-1}=U^{-1}\Rightarrow T=S^{-1}U^{-1}.$\PfEnd\vspace{2pt}\Anchor{3D12}
%\AExa $V=\Rbb^\infty,S\Par{a_1,a_2,\cdots}=\Par{a_2,\cdots};\;T\Par{a_1,a_2,\cdots}=\Par{0,a_1,a_2,\cdots};\;U=I\Rightarrow STU=I$ but $T$ is not inv.
%\SepLine

%\ProblemN{\Anchor{3D15}{15}}{
%	\TextA{Prove if $T\in\Lm{\FbbP{n,1},\FbbP{m,1}}$, then $\exists\,A\in\FbbP{m,n},Tx = Ax,\,\forall x\in\FbbP{n,1}$.}
%}Let $B_1=\Par{E_1,\dots,E_n},B_2=\Par{R_1,\dots,R_m}$ be std bses. Each $T\Par{E_{k}}=A_{1,k}R_1+\dots+A_{m,k}R_m.$\PfEnd\vspace{2pt}\parSol{} %$A={}${\normalsize$\begin{pmatrix} A_{1,1} &\hspace{-6pt} \cdots &\hspace{-6pt} A_{1,n}\\[-4pt] \vdots &\hspace{-6pt} \ddots &\hspace{-6pt} \vdots\\[-4pt] A_{m,1} &\hspace{-6pt} \cdots &\hspace{-6pt} A_{m,n}\end{pmatrix}.$}\parSol{\vspace{-6pt}}
%\Or Let $A=\Mt{T,B_1,B_2}$. Note that $\Mt{x,B_1}=x,\Mt{Tx,B_2}=Tx.$\parSol{}
%Hence $Tx=\Mt{Tx,B_2}=\Mt{T,B_1,B_2}\Mt{x,B_1}=Ax,$ by [3.65].\PfEnd
%\SepLine

%\Anchor{10A1}\Anchor{3D4e22}\ProblemBnoor{4E 22, \OR {10.A.1}}{
%	\TextB{Supp $T\in\Lm{V}.$ Prove $\Mt{T,\alpha\rightarrow\beta}$ is inv $\Longleftrightarrow T$ itself is inv.}
%}Notice that $\mM\!:T\mapsto\Mt{T,\alpha\rightarrow\beta}$ is iso. And that $\Mt{T}\Mt{S}=\Mt{TS}.$\vspace{2pt}\par\quad
%(a) $T^{-1}T=TT^{-1}=I\Rightarrow\Mt{T^{-1}}\Mt{T}=\Mt{I}=\Mt{T}\Mt{T^{-1}}\Rightarrow\Mt{T^{-1}}=\Mt{T}{^{-1}}.$\vspace{2pt}\par\quad
%(b) $\Mt{T}\Mt{T}{^{-1}}=\Mt{T}{^{-1}}\Mt{T}=I,\;\exists\,!\,S\in\Lm{V}$ suth $\Mt{T}{^{-1}}=\Mt{S}$\par\quad\Hb
%$\Rightarrow\Mt{TS}=\Mt{T}\Mt{S}=I=\Mt{S}\Mt{T}=\Mt{ST}$\par\quad\Hb
%$\Rightarrow\Mneg\Mt{TS}=\Mneg\Mt{ST}=I=TS=ST\Rightarrow S=T^{-1}.$\PfEnd\vspace{2pt}
%\ACoro Supp $A\in\FbbP{n,n}.$ Then $A$ is inv $\Longleftrightarrow\exists$ inv $T\in\Lm{\FbbP{n}}$ suth $\Mt[\BigPar]{T,\Par{e_1,\dots,e_n},\Par{f_1,\dots,f_n}}=A.$
%\SepLine

%\Anchor{10A2}\Anchor{3D4e24}\ProblemBnoor{4E 24, \OR 10.A.2}{
%	\TextA{Supp $A,B\in\FbbP{n,n}$. Prove $AB = I\Longleftrightarrow BA=I.$\hfill\small\envFontSmall\Sbra{Using Exe (10, 15).}}
%}Define $T,S\in\Lm{\FbbP{n,1}}$ by $Tx=Ax,Sx=Bx$ for all $x\in\FbbP{n,1}.$ Now $\Mt{T}=A,\Mt{S}=B.$\parSol{}
%$AB=I\Leftrightarrow A\Par{Bx}=x\Longleftrightarrow T\Par{Sx}=x\Leftrightarrow TS=I\Longleftrightarrow ST=I\Longleftrightarrow \Mt{S}\Mt{T}=BA=I.$\parSol{}
%\Or Becs $\mM\!:\Lm{\FbbP{n,1},\FbbP{n,1}}\rightarrow\FbbP{n,n}$ is iso. $\Mneg\Par{AB}=TS=ST=\Mneg\Par{BA}=I.$\PfEnd
%\SepLine

\Anchor{10A4}\Anchor{3D4e23}\ProblemBnoor{4E 23, \OR 10.A.4}{
	\TextB{Supp $\Par{\beta_1,\dots,\beta_n}$ and $\Par{\alpha_1,\dots,\alpha_n}$ are bses of $V$.}
	\TextB{Let $T\in\Lm{V}$ be suth each $T\alpha_k = \beta_k.$ Prove $A=\Mt{T,\alpha\rightarrow\alpha}=\Mt{I,\beta\rightarrow\alpha}=B.$}
}Each $I\beta_k=\beta_k=B_{1,k}\alpha_1+\dots+B_{n,k}\alpha_n=T\alpha_k=A_{1,k}\alpha_1+\dots+A_{n,k}\alpha_n\Rightarrow A=B.$\PfEnd\vspace{2pt}\parSol{}
\Or $\Mt{T,\alpha\rightarrow\alpha}=\Mt{I,\beta\rightarrow\alpha}$\uline{$\Mt{T,\alpha\rightarrow\beta}$}${}=\Mt{I,\beta\rightarrow\alpha}.$\PfEnd\vspace{2pt}\parSol{}
\Or $\Mt{T,\alpha\rightarrow\alpha}=\Mt{I,\alpha\rightarrow\beta}{^{-1}}\BigSbra{$\uline{$\Mt{T,\beta\rightarrow\beta}\Mt{ I,\alpha\rightarrow\beta}$}$}=\Mt{I,\beta\rightarrow\alpha}.$\PfEnd\vspace{4pt}
\AComm $\Mt{T,\beta\rightarrow\beta}=\Mt{T,\alpha\rightarrow\beta}\Mt{I,\beta\rightarrow\alpha}=B.$ \;\Or Let $A\apostrophe=\Mt{T,\beta\rightarrow\beta}.$\parCom
Simlr. Now each $T\beta_k=T\Par{B_{1,k}\alpha_1+\dots+B_{n,k}\alpha_n}=A'_{1,k}\beta_1+\dots+A'_{n,k}\beta_n\Rightarrow A\apostrophe=B.$
\SepLine

%\BulletPointX\NewNotation\;\;{\FontNorm For ease of nota, let $\Mt[\BigPar]{T,\alpha\rightarrow\beta}=\Mt[\BigPar]{T,\Par{\alpha_1,\dots,\alpha_n},\Par{\beta_1,\dots,\beta_n}}.$}
%\SepLine

\Anchor{3DN3.62}\BulletPointX\NoteForSmall{[3.62]}\;\;$\Mt{v}=\Mt[\BigPar]{I,\Par{v},B_V}.$ Here $I$ is restr to $\Span{v},$ and $\Par{v}=\Par{\,}$ if $v=0.$\par\vspace{2pt}
\Anchor{3DN3.65}\BulletPointX\NoteForSmall{[3.65]}\;\;$\Mt{Tv}=\Mt[\BigPar]{I,\Par{Tv},B_W}=\Mt{T,B_V,B_W}\Mt[\BigPar]{I,\Par{v},B_V}=\Mt[\BigPar]{T,\Par{v},B_W}.$\vspace{-2pt}
\SepLine

\Anchor{3DNE15}\BulletPointX\NoteForSmall{Exe (15)}\;\;\uline{$Tx=\Mt{Tx,B_2}$}${}=\Mt{I,B_2',B_2}\Mt{Tx,B_2'}=\Mt{I,B_2',B_2}\Mt{T,B_1',B_2'}\Mt{x,B_1'}$\TextB{}
\Blind{\NoteForSmall{Exe (15)}\;\;$Tx$}$=\Mt{I,B_2',B_2}\Mt{T,B_1',B_2'}\Mt{I,B_1,B_1'}$\uline{$\Mt{x,B_1}$}${}=\Mt{T,B_1,B_2}$\uline{$x$}${}=Ax.$\vspace{2pt}\TextB{}
Where $B_1,B_2$ are std bses, and $B_1',B_2'$ are arb bses. Now $A$ is uniq.\TextB{}
Convly, $\forall A\in\FbbP{m,n},\exists\,!\,T\in\Lm{\FbbP{n,1},\FbbP{m,1}},Tx=Ax=\Mt{T}x\Rightarrow\Mt{T}=A.$\TextB{}
Hence $\range A=\range T$ and $\null A=\range T$ are well-defined becs the def of $T$ is indep of bses.\vspace{2pt}\TextB{}
Let $P_{\!\!\!\:1}\in\Lm{\FbbP{n,1}},P_{\!\!\!\:2}\in\Lm{\FbbP{m,1}}$ be suth $\Mt{P_{\!\!\!\:1},B_1}=\Mt{I,B_1',B_1},\Mt{P_{\!\!\!\:2},B_2}=\Mt{I,B_2,B_2'}.$\TextB{}
Now $\Mt{T,B_1,B_2}$ inje $\Longleftrightarrow T$ inje $\Longleftrightarrow P_{\!\!\!\:2}TP_{\!\!\!\:1}$ inje $\Longleftrightarrow\Mt{T,B_1',B_2'}=\Mt{P_{\!\!\!\:2}TP_{\!\!\!\:1},B_1,B_2}$ inje.\vspace{2pt}\TextB{}
Supp $S\in\Lm{V,W},A_1=\Mt{S,B_V,B_W}=\Mt{T,B_1,B_2},$\TextB{}
$C_V=\Mt{I,B_V',B_V},C_W=\Mt{I,B_W,B_W'},A_2=\Mt{S,B_V',B_W'}=C_WA_1C_V.$\TextB{}
Let $P_{\!\!\!\:1}\in\Lm{\FbbP{n,1}},P_{\!V}\in\Lm{V},P_{\!\!\!\:2}\in\Lm{\FbbP{m,1}},P_{\!W}\in\Lm{W}$ and bses $B_1',B_2'$ be suth\TextB{}
$\Mt{P_{\!\!\!\:1},B_1}=\Mt{I,B_1',B_1}=C_V=\Mt{P_{\!V},B_V},\Mt{P_{\!\!\!\:2},B_2}=\Mt{I,B_2,B_2'}=C_W=\Mt{P_{\!W},B_W}.$\TextB{}
Now $A_1$ inje $\Longleftrightarrow T$ inje $\Longleftrightarrow$ the cols of $A_1$ liney indep $\Longleftrightarrow S$ inje. Simlr for surj.\TextB{}
And $\Mt{S,B_V,B_W}$ inje $\Longleftrightarrow T$ inje $\Longleftrightarrow P_{\!\!\!\:2}TP_{\!\!\!\:1}$ inje $\Longleftrightarrow C_WA_1C_V=\Mt{S,B_V',B_W'}$ inje.\TextB{}
Thus $S$ inje $\Longleftrightarrow\Mt{S}$ inje, wrto any bses. Simlr for surj and inv.
\SepLine

\Anchor{3DT3}\ProblemBX[]{\TipsN{3}}{
	Identify $\FbbP{m,n}$ with $\Lm{\FbbP{n,1},\FbbP{m,1}},$ due to $Tx=Ax;$ or with $\Lm{\FbbP{1,n},\FbbP{1,m}},$ due to $Tx=xA^t.$\TextB{}
	Details about the latter: $x=\Mt{x}{^t}\Rightarrow x\Mt{T}{^t}=\Mt{Tx}{^t}=\Mt{xA^t}{^t}=xA^t\Rightarrow\Mt{T}=A.$\TextB{}
	\ANote If we define $\Mt{v}=\Par{c_1\:\:\cdots\:\:c_n}.$Then [3,64]: $\Mt{T}{_{k,\cdot}}=\Mt{v_k}\Mt{T}=\Mt{Tv_k}.$\TextB{}
	\IndentNote Note that $A=\Mt[\BigPar]{T,\Par{v_1,\dots,v_n},\Par{w_1,\dots,w_m}}\in\FbbP{n,m},$ and each $Tv_k=A_{k,1}w_1+\dots+A_{k,m}w_m.$\TextB{}
	\IndentNote [3.65]: $\Mt{Tv}=c_1\Mt{Tv_1}+\dots+c_n\Mt{Tv_n}=c_1\Mt{T}{_{1,\cdot}}+\dots+c_n\Mt{T}{_{n,\cdot}}=\Mt{v}{\Mt{T}}.$\TextB{}
	\IndentNote [3.41]: $C=\Mt[\BigPar]{T,\Par{u_1,\dots,u_p},\Par{v_1,\dots,v_n}}\in\FbbP{p,n},\,A=\Mt[\BigPar]{S,\Par{v_1,\dots,v_n},\Par{w_1,\dots,w_m}}\in\FbbP{n,m}.$\TextB{}
	\IndentNote\Blind{[3.41]:} $\sum_{j=1}^m\!\Par{AC}{_{k,j}}w_j=STu_k=\sum_{j=1}^m\!\BigPar{{\sum_{r=1}^nC_{k,r}A_{r,j}}}w_j\Rightarrow\Par{AC}{_{k,j}}=\sum_{r=1}^nC_{k,r}A_{r,j}.$\vspace{4pt}\TextB{}
	\IndentNote Exactly in trspose with the original. Now Exe (15): $T\in\Lm{\FbbP{1,n},\FbbP{1,m}}:Tx=xA.$\TextB{}
	\IndentNote Becs $Tx=\Mt{Tx}=\Mt{x}\Mt{T}=x\Mt{T}$ wrto std bses.\TextB{\vspace{-4pt}}
}\SepLine

\Anchor{3DT4}\ProblemBX[]{\TipsN{4}}{
	You must first declare bses and the purpose when using $\Mneg\!:\FbbP{n,1}\mapsto v,$ \,or $\FbbP{m,n}\mapsto\Lm{V,W}.$\TextB{\vspace{-4pt}}
}\SepLine

\Anchor{3DNE34e22}\BulletPointX\NoteForSmall{Exe (3, 4E 22)}\;\;$T\in\Lm{V,W}$ inv $\Longleftrightarrow\Mt{T}$ inv, wrto any\Par{$\Leftrightarrow$ some} $B_V,B_W.$\TextB{}
Supp $\Mt{T}$ wrto some $B_V,B_W$ is inv. Let $S\in\Lm{W,V}$ be suth $\Mt{S,B_W,B_V}=\Mt{T,B_V,B_W}{^{-1}}.$\TextB{}
Then $\Mt{TS,B_W}=\Mt{T,B_V,B_W}\Mt{S,B_W,B_V}=I=\Mt{ST,B_V}.$ Apply $\Mneg\Rightarrow S=T^{-1}.$\vspace{2pt}\TextB{}
\ACoro $\Mt{T,B_V,B_W}{^{-1}}=\Mt{T^{-1},B_W,B_V},$ if $T$ or $\Mt{T}$ inv.
\SepLine

\Anchor{3DN3.60}\BulletPointX\NoteFor{[3.60]}\;\;Supp $B_V=\Par{v_1,\dots,v_n},\;B_W=\Par{w_1,\dots,w_m}.$\TextB{}
Define $E_{i,j}\in\Lm{V,W}$ by $E_{i,j}\Par{v_x}=\delta_{i,x}w_j.$ Define $R_{i,j}\in\Lm{W,V}:w_x\mapsto\delta_{i,x}v_j.$\vspace{2pt}\TextB{}
Define $G_{i,j}=R_{x,j}E_{i,x},$ and $Q_{i,j}=E_{x,j}R_{i,x}.$ \;Let $\mEnt{j,\,i}=\Mt{E_{i,j}},$ simlr for $\mRnt{j,\,i},\mGnt{j,\,i}.$\vspace{2pt}\TextB{}
%Let $\mRnt{j,\,i}=\Mt{R_{i,j}},\,\mGnt{j,\,i}=\Mt{G_{i,j}},\,\mQnt{j,\,i}=\Mt{Q_{i,j}}.$\vspace{0pt}\TextB{}
Now $R_{l,k}E_{i,j}=\delta_{j,\,l}G_{i,k},\;\mRnt{k,\,l}\mEnt{j,\,i}=\delta_{l,j}\mGnt{k,\,i}.$ Simlr for $Q_{i,k}$ and $\mQnt{k,\,i}.$\vspace{4pt}\TextB{}
\hspace{-5pt}$\hText{$
Becs $\mM\!:\Lm{V,W}\rightarrow\FbbP{m,n}$ is iso.$\\$
$E_{i,j}=\Mneg\mEnt{j,\,i}.$ \,By [2.42] and [3.61]:$\\[0pt]$
{\tgsl\normalsize We can rewrite the matrix multi in (3.C).}$\\[20pt]\,}\hspace{0pt}\hText{B_{\!\Lm[\SmallPar]{V,\,W}}={}${\normalsize${\begin{pmatrix} \underset{\;,}{E_{1,1}}, &\hspace{-6pt} \cdots &\hspace{-6pt} ,\underset{\!,}{E_{n,1}},\\[-2pt] \vdots &\hspace{-6pt} \ddots &\hspace{-6pt} \vdots\\[-2pt] \overset{\;,}{E_{1,m}}, &\hspace{-6pt} \cdots &\hspace{-6pt} ,\overset{\!\!\!,}{E_{n,m}}\end{pmatrix}}$}$;\quad B_{\,\FbbP{m,n}}={}${\normalsize$\envFontSmall[\scriptsize]{\begin{pmatrix} \underset{\;,}{\mE^{\Par{1,1}}}, &\hspace{-6pt} \cdots &\hspace{-6pt} ,\underset{\!\!,}{\mE^{\Par{1,n}}},\\[-2pt] \vdots &\hspace{-6pt} \ddots &\hspace{-6pt} \vdots\\[-2pt]\overset{\;,}{\mE^{\Par{m,1}}}, &\hspace{-6pt} \cdots &\hspace{-6pt} ,\overset{\!\!\!,}{\mE^{\Par{m,n}}}\end{pmatrix}}$}.$\\[2pt]\,\\\,}$\par\vspace{-30pt}
\SepLine
\pagebreak

\BulletPointX\Tips \,\,\,Let $B_{\range T}=\Par{Tv_1,\dots,Tv_p},B_V=\BigPar{v_1,\dots,v_p,\dots,v_n}.$ Let each $w_k=Tv_k.$\TextB{\vspace{2pt}}\IndentTips{}
Extend to $B_W=\BigPar{w_1,\dots,w_p,\dots,w_m}.$ Then $T=E_{1,1}+\dots+E_{p,\,p},\;\Mt{T}=\mEnt{1,\,1}+\dots+\mEnt{p,\,p}.$
\SepLine

%Now $\BigPar{\mEnt{j,\,i}}{_{l,k}}=\delta_{j,\,l}\delta_{i,k},$ and $E_{l,k}E_{i,j}=\delta_{j,\,l}E_{i,k}\Rightarrow\mEnt{k,\,l}\mEnt{j,\,i}=\delta_{l,j}\mEnt{k,\,i}.$\vspace{2pt}\TextB{}


\Anchor{3D4e17}\ProblemBnoor{4E 17}{
	\TextB{Supp $U,V,W$ finide, $S\in\Lm{V,W},\mA\in\Lm[\BigPar]{\Lm{U,V},\Lm{U,W}}:T\mapsto ST.$\vspace{1pt}}
	\TextB{Show $\dim\null\mA=\BigPar{{\dim U}}\BigPar{{\dim\null S}},\,\dim\range\mA=\BigPar{{\dim U}}\BigPar{{\dim\range S}}$.\vspace{2pt}}
}(a) $\forall T\in\Lm{U,V},$ $ST=0\Longleftrightarrow\range T\subseteq\null S.$ Thus $\null\mA=\Lm{U,\null S}.$\parSol{\vspace{2pt}}
(b) $\forall R\in\Lm{U,W},$ $\range R\subseteq\range S \Longleftrightarrow\exists\,T\in\Lm{U,V},R=ST,$ by (3.B 25).\parSol{\Hb}
Thus $\range\mA=\Lm{U,\range S}.$\PfEnd\vspace{4pt}\quad
\Or Let $B_{\range S}=\Par{w_1,\dots,w_s}$ with each $w_i=Sv_i.$ Let $B_W=\Par{w_1,\dots,w_n},B_{\null S}=\Par{v_{s+1},\dots,v_p}.$\par\vspace{0pt}\quad
Let $B_U=\Par{u_1,\dots,u_m}.$ Define $E_{i,j}\in\Lm{V,W}:v_x\mapsto\delta_{i,x}w_j.$ Now $S=E_{1,1}+\dots+E_{s,s}.$\par\vspace{0pt}\quad%$\Mt{S,v\rightarrow w}={\footnotesize\begin{pmatrix} \mathbb{1} &\hspace{-6pt}  &\hspace{-6pt} 0 &\hspace{-6pt}  &\hspace{-6pt} 0\\[-8pt]&\hspace{-6pt} \ddots &\hspace{-6pt}  &\hspace{-6pt} \ddots &\hspace{-6pt} \\[-8pt]0 &\hspace{-6pt}  &\hspace{-6pt} \mathbb{1} &\hspace{-6pt}  &\hspace{-6pt} 0\\[-8pt]&\hspace{-6pt} \ddots &\hspace{-6pt}  &\hspace{-6pt} \ddots &\hspace{-6pt} \\[-8pt]0 &\hspace{-6pt}  &\hspace{-6pt} 0 &\hspace{-6pt}  &\hspace{-6pt} 0\end{pmatrix}}.$\par\vspace{-6pt}\quad
Define $R_{i,j}\in\Lm{U,V}:u_x\mapsto\delta_{i,x}v_j.$ \;Let $E_{k,j}R_{i,k}=Q_{i,j}:u_x\mapsto\delta_{i,x}w_j.$\par\vspace{-20pt}\quad
%Where $E_{i,k}:v_x\mapsto\delta_{i,x}w_k,\;$ $Q_{i,k}:w_x\mapsto\delta_{i,x}w_k,\;$ and $G_{i,k}:v_x\mapsto\delta_{i,x}v_k.$\par\vspace{-19pt}\quad
For any $T\in\Lm{V},\exists\,!\,A_{i,j}\in{}\Fbb,\,T=\textstyle\sum_{j=1}^p\sum_{i=1}^m A_{j,\,i}R_{i,j}\Longrightarrow\Mt{T,u\rightarrow v}={\small\begin{pmatrix}
		A_{1,1} &\hspace{-6pt} \cdots &\hspace{-6pt} A_{1,s} &\hspace{-6pt} \cdots &\hspace{-6pt} A_{1,m}\\[-2pt]
		\vdots  &\hspace{-6pt} \ddots &\hspace{-6pt} \vdots  &\hspace{-6pt} \ddots &\hspace{-6pt} \vdots\\[-4pt]
		A_{s,1} &\hspace{-6pt} \cdots &\hspace{-6pt} A_{s,s} &\hspace{-6pt} \cdots &\hspace{-6pt} A_{s,m}\\[-2pt]
		\vdots  &\hspace{-6pt} \ddots &\hspace{-6pt} \vdots  &\hspace{-6pt} \ddots &\hspace{-6pt} \vdots\\[-4pt]
		A_{p,1} &\hspace{-6pt} \cdots &\hspace{-6pt} A_{p,s} &\hspace{-6pt} \cdots &\hspace{-6pt} A_{p,m}
\end{pmatrix}}$\par\vspace{-20pt}\quad
$\Longrightarrow\mA\Par{T}=ST=\textstyle\XPar{{\sum_{k=1}^sE_{k,k}}}\XPar{{\textstyle\sum_{j=1}^p\sum_{i=1}^m A_{j,\,i}R_{i,j}}}=\textstyle\sum_{j=1}^s\sum_{i=1}^m A_{i,j}Q_{j,\,i}.$\par\hspace{0pt}
\;\;$\hText{\Mt[\BigPar]{S,v\rightarrow w}\Mt[\BigPar]{T,u\rightarrow v}=\Mt[\BigPar]{ST,u\rightarrow w}=\\
\hfill\Mt[\BigPar]{\mA,R\rightarrow Q}\Mt{T,R}=\Mt[\BigBigPar]{\!\mA\Par{T},Q}=}{\small\begin{pmatrix}
A_{1,1} &\hspace{-6pt} \cdots &\hspace{-6pt} A_{1,s} &\hspace{-6pt} \cdots &\hspace{-6pt} A_{1,m}\\[-2pt]
\vdots &\hspace{-6pt} \ddots &\hspace{-6pt} \vdots &\hspace{-6pt} \ddots &\hspace{-6pt} \vdots\\[-4pt]
A_{s,1} &\hspace{-6pt} \cdots &\hspace{-6pt} A_{s,s} &\hspace{-6pt} \cdots &\hspace{-6pt} A_{s,m}\\[-2pt]
\vdots  &\hspace{-6pt} \ddots &\hspace{-6pt} \vdots  &\hspace{-6pt} \ddots &\hspace{-6pt} \vdots\\[-4pt]
0 &\hspace{-6pt} \cdots &\hspace{-6pt} 0&\hspace{-6pt} \cdots &\hspace{-6pt} 0
\end{pmatrix}}\hText{$
又 $\Mt{T,R}=\Mt[\BigPar]{T,u\rightarrow v}.\\$
If $m=p,$ let $\Mt{T,R}=I,\\$
$\Mt[\BigPar]{\mA,R\rightarrow Q}=\Mt{S,v\rightarrow w}.}$\par\quad
{\FontSmall $B_{\range\mA}=\Bra{Q_{x,y}:1\leqslant x\leqslant m,\text{ and }1\leqslant y\leqslant s},\:B_{\null\mA}=\Bra{R_{x,y}:1\leqslant x\leqslant m,\text{ and }s+1\leqslant y\leqslant p}.$}\PfEnd
%$\range\mA={}\Spn{\small\begin{Bmatrix} \underset{\;,}{Q_{1,1}}, &\hspace{-6pt} \cdots &\hspace{-6pt} ,\underset{\!,}{Q_{m,1}},\\[-2pt] \vdots &\hspace{-6pt} \ddots &\hspace{-6pt} \vdots\\[-2pt] \overset{\;,}{Q_{1,s}}, &\hspace{-6pt} \cdots &\hspace{-6pt} ,\overset{\!\!\!,}{Q_{m,s}}\end{Bmatrix}},\;\null \mA=\Spn{\small\begin{Bmatrix} \underset{\;,}{R_{1,s+1}}, &\hspace{-6pt} \cdots &\hspace{-6pt} ,\underset{\!\!,}{R_{m,s+1}},\\[-2pt] \vdots &\hspace{-6pt} \ddots &\hspace{-6pt} \vdots\\[-2pt] \overset{\;,}{R_{1,\,p}}, &\hspace{-6pt} \cdots &\hspace{-6pt} ,\overset{\!\!\!,}{R_{m,\,p}}\end{Bmatrix}}.$
%\;\;$\hText{$
%(a) $\dim\null \mA=m\times\Par{p-s};\\$
%(b) $\dim\range \mA=m\times s.}$\par\PfEnd[-10pt]\vspace{-4pt}
\SepLine

\Anchor{3D7}\Anchor{3D4e10}\ProblemBnoor{4E 10}{
	\TextA{Supp $V,W$ finide, $U$ is a subsp of $V,$ $\mE=\Bra{T\in\Lm{V,W}:T\mmid_U=0}.$ Find $\dim\mE.$}
}Define $\Phi:\Lm{V,W}\rightarrow\Lm{U,W}$ by $\Phi\Par{T}=T\mmid_U.$ By \Sbra{3.A {\NOTEFOR} Restr}, $\Phi$ is liney.\vspace{0pt}\parSol{}
$\Phi\Par{T}=0\Longleftrightarrow\forall u\in U,Tu=0\Longleftrightarrow T\in\mE.$ Thus $\null\Phi=\mE.$\parSol{}
Extend $S\in\Lm{U,W}$ to $T\in\Lm{V,W}\Rightarrow\Phi\Par{T}=S\in\range\Phi.$ Thus $\range\Phi=\Lm{U,W}.$\PfEnd\vspace{2pt}\parSol{}
\Or Let $B_U=\Par{u_1,\dots,u_m},B_V=\BigPar{u_1,\dots,u_m,\dots,u_n},B_W=\Par{w_1,\dots,w_p}.$\parSol{}
Define $E_{i,j}\in\Lm{V,W}:u_x\mapsto\delta_{i,x}w_j.$\vspace{-17pt}\parSol{}
$\forall T\in\mE,k\in\;\!\!\Bra{1,\dots,m},TE_{k,k}=0\Rightarrow{\normalsize\Spn\underbrace{\begin{Bmatrix} \underset{\;,}{E_{1,1}}, &\hspace{-6pt} \cdots &\hspace{-6pt} ,\underset{\!,}{E_{m,1}},\\[-2pt] \vdots &\hspace{-6pt} \ddots &\hspace{-6pt} \,\vdots\\[-2pt] \overset{\;,}{E_{1,\,p}}, &\hspace{-6pt} \cdots &\hspace{-6pt} ,\overset{\!\!,}{E_{m,\,p}}\end{Bmatrix}}_{=\:R}}\,\cap\,\mE=\zeroSubs$.\vspace{-32pt}\parSol{}
\hspace{-7pt}$\hText{$
	又 $C=\Spn${\small$\begin{Bmatrix} \underset{\;,}{E_{m+1,1}}, &\hspace{-6pt} \cdots &\hspace{-6pt} ,\underset{\!,}{E_{n,1}},\\[-2pt] \vdots &\hspace{-6pt} \ddots &\hspace{-6pt} \,\vdots\\[-2pt] \overset{\;,}{E_{m+1,\,p}}, &\hspace{-6pt} \cdots &\hspace{-6pt} ,\overset{\!\!,}{E_{n,\,p}}\end{Bmatrix}$}$\,\subseteq\mE.\\\;\\\;}$\vspace{-80pt}\par
\hfill Now $\Lm{V,W}=\spn R\oplus C$\Blind{.\quad}\vspace{0pt}\par
\hfill$\Rightarrow\Lm{V,W}=\spn R+\mE.$\Blind{\quad}\PfEnd\vspace{10pt}
\SepLine

\Anchor{3D'3}\ProblemB{
	\TextB{Supp $U,V,W$ finide, $S\in\Lm{U,V},\mB\in\Lm[\BigPar]{\Lm{V,W},\Lm{U,W}}:T\mapsto TS.$\vspace{1pt}}
	\TextB{Show $\dim\null\mB=\BigPar{{\dim W}}\BigPar{{\dim\null S}},\,\dim\range\mB=\BigPar{{\dim W}}\BigPar{{\dim\range S}}.$\vspace{2pt}}
}(a) $\forall T\in\Lm{V,W},TS=0\Longleftrightarrow\range S\subseteq\null T.$ Thus $\null\mB=\Bra{T\in\Lm{V,W}:T\mmid_{\range S}=0}.$\parSol{\vspace{2pt}}
(b) $\forall R\in\Lm{U,W},\null S\subseteq\null R\Longleftrightarrow\exists\,T\in\Lm{V,W},R=TS,$ by (3.B.24).\parSol{\Hb}
Thus $\range\mB=\Bra{R\in\Lm{U,W}:R\mmid_{\null S}=0}.$ \;Now by Exe (4E 10).\PfEnd\vspace{4pt}\quad
\Or Let $B_{\range S}=\Par{v_1,\dots,v_r}$ with each $u_i=Sv_i.$ Let $B_V=\Par{v_1,\dots,v_m},B_{\null S}=\Par{u_{r+1},\dots,u_n}.$\par\quad
Let $B_W=\Par{w_1,\dots,w_p}.$ Define $E_{i,j}\in\Lm{U,V}:u_x\mapsto\delta_{i,x}v_j\Rightarrow S=E_{1,1}+\dots+E_{r,r}.$\par\quad
Define $R_{i,j}\in\Lm{V,W}:v_x\mapsto\delta_{i,x}w_j.$ Let $R_{k,j}E_{i,k}=Q_{i,j}:u_x\mapsto\delta_{i,x}w_j.$\par\vspace{-20pt}\quad
$TS=\XPar{{\sum_{j=1}^p\sum_{i=1}^m A_{j,\,i}R_{i,j}}}\XPar{{\sum_{k=1}^r E_{k,k}}}=\sum_{j=1}^p\sum_{i=1}^r A_{j,\,i}Q_{i,j}\Rightarrow\Mt{TS,v}={\small\begin{pmatrix}
		A_{1,1} &\hspace{-6pt} \cdots &\hspace{-6pt} A_{1,r} &\hspace{-6pt} \cdots &\hspace{-6pt} 0\\[-2pt]
		\vdots  &\hspace{-6pt} \ddots &\hspace{-6pt} \vdots  &\hspace{-6pt} \ddots &\hspace{-6pt} \vdots\\[-4pt]
		A_{r,1} &\hspace{-6pt} \cdots &\hspace{-6pt} A_{r,r} &\hspace{-6pt} \cdots &\hspace{-6pt} 0\\[-2pt]
		\vdots  &\hspace{-6pt} \ddots &\hspace{-6pt} \vdots  &\hspace{-6pt} \ddots &\hspace{-6pt} \vdots\\[-4pt]
		A_{p,1} &\hspace{-6pt} \cdots &\hspace{-6pt} A_{p,r} &\hspace{-6pt} \cdots &\hspace{-6pt} 0\end{pmatrix}}.$\par\vspace{-20pt}\quad
%{\FontSmall $B_{\range\mB}=\Bra{Q_{x,y}:1\leqslant x\leqslant r,\text{ and }1\leqslant y\leqslant p},\:B_{\null\mB}=\Bra{R_{x,y}:r+1\leqslant x\leqslant n}.$}\PfEnd
$\range\mB=\Spn{\small\begin{Bmatrix} \underset{\;,}{Q_{1,1}}, &\hspace{-6pt} \cdots &\hspace{-6pt} ,\underset{\!\!,}{Q_{r,1}},\\[-2pt] \vdots &\hspace{-6pt} \ddots &\hspace{-6pt} \vdots\\[-2pt] \overset{\;,}{Q_{1,\,p}}, &\hspace{-6pt} \cdots &\hspace{-6pt} ,\overset{\!\!\!,}{Q_{r,\,p}}\end{Bmatrix}},\;\null\mB=\Spn{\small\begin{Bmatrix} \underset{\;,}{R_{r+1,1}}, &\hspace{-6pt} \cdots &\hspace{-6pt} ,\underset{\!\!,}{R_{n,1}},\\[-2pt] \vdots &\hspace{-6pt} \ddots &\hspace{-6pt} \vdots\\[-2pt] \overset{\;,}{R_{r+1,\,p}}, &\hspace{-6pt} \cdots &\hspace{-6pt} ,\overset{\!\!\!,}{R_{n,\,p}}\end{Bmatrix}}.$\PfEnd[-14pt]
\SepLine\pagebreak

%\ProblemB{
%	\TextA{Supp $U,V,W,X$ of dim $m,n,p,q$ respectly, and $S\in\Lm{U,V},R\in\Lm{W,X}.$}
%	\TextA{Supp $\dim\range S=s,\dim\range R=r.$ Define $\mC\in\Lm[\BigPar]{\Lm{V,W},\Lm{U,X}}:T\mapsto RTS.$}
%	\TextA{Show $\dim\null\mC=\Par{{\dim\null R}}\Par{{\dim\null S}},\,\dim\range\mC=\Par{{\dim\range R}}\Par{{\dim\range S}}.$}
%}Apply Exe ('4) to $\mB.$ Done.\par\quad
%Define $\mA\in\Lm[\BigPar]{\Lm{V,W},\Lm{V,X}}:T\mapsto RT,\;\;\mB\in\Lm[\BigPar]{\Lm{V,\range R},\Lm{U,\range R}}:T\mapsto TS.$\PfEnd
%\SepLine

\Anchor{3D'4}\ProblemB{
	\TextB{Supp $A\in\FbbP{n,n}.$ Define $T,S\in\Lm{\FbbP{n,n}}$ by $T\Par{X}=AX,\:S\Par{Y}=YA^t.$ Find $\dim\range ST.$\vspace{2pt}}
}Becs $A\mEnt{j,\,k}=\BigSbra{{\sum_{x=1}^nA_{x,j}\mEnt{x,\,j}}}\mEnt{j,\,k}=\sum_{x=1}^nA_{x,j}\mEnt{x,\,k}.$ \;Let $B_{\col A}=\Par{C_{\cdot,1},\dots,C_{\cdot,r}}.$\vspace{2pt}\parSol{}
Each $A_{\cdot,j}=R_{1,j}C_{\cdot,1}+\dots+R_{r,j}C_{\cdot,r}\Rightarrow B_{\range T}=\Bra{\mC_{\!j,k}=\sum_{x=1}^nC_{x,j}\mEnt{x,\,k}:1\leqslant j\leqslant r,\,1\leqslant k\leqslant n}.$\vspace{4pt}\parSol{}
Becs $\mC_{\!j,k}A^t=\mC_{\!j,k}\BigSbra{{\sum_{y=1}^nA^t_{k,y}\mEnt{k,\,y}}}=\sum_{x=1}^n\sum_{y=1}^nC_{x,j}A_{y,k}\mEnt{x,\,y}.$\vspace{-20pt}\parSol{}
Simlr, $B_{\range ST}=\Bra{\mX_{\!j,k}=\sum_{x=1}^n\sum_{y=1}^nC_{x,j}C_{y,k}\mEnt{x,\,y}:1\leqslant j,k\leqslant r}.\;\hMath{l}{\left|}{\right.}{\;\mX_{\!j,k}=\small\begin{pmatrix}C_{1,j}C_{1,k} &\hspace{-8pt} \cdots &\hspace{-8pt} C_{1,j}C_{n,k}\\[-4pt]
		\vdots &\hspace{-8pt} \ddots &\hspace{-8pt} \vdots\\[-4pt]
		C_{n,j}C_{1,k} &\hspace{-8pt} \cdots &\hspace{-8pt} C_{n,j}C_{n,k}\end{pmatrix}.\\[20pt]\,}$\vspace{-20pt}\parSol{}
%Now $\Par{\mX_{\!1,k},\dots,\mX_{\!r,k}}$ and  $\Par{\mX_{\!j,1},\dots,\mX_{\!j,r}}$ are liney indep lists.\parSol{}
Each $\mX_{\!j,k}=C_{1,k}\mC_{\!j,1}+\dots+C_{n,k}\mC_{\!j,n}=C_{1,j}\Par{\mC_{\!k,1}}{^t}+\dots+C_{n,j}\Par{\mC_{\!k,n}}{^t}.$\PfEnd\vspace{6pt}\par\quad
\Or By \TIPSN{3}. Define $\varphi\in\Lm[\BigPar]{\Lm{\FbbP{n,1}}}:X\mapsto AX;$ \;$\psi\in\Lm[\BigPar]{\Lm{\FbbP{n,1},\range A}}:Y\mapsto YA^t.$\par\quad
Then $\range\psi\varphi=\Bra{X\in\Lm{\FbbP{n,1},\range A}:X\mmid_{\null A^T}=0}$ of dim $\Par{{\range A}}\Par{{\range A^t}}=\Par{\rank A}{^2}.$\PfEnd
\SepLine
\ProblemN{\Anchor{3D16}{16}}{
	\TextA{Supp $V$ is finide and non0 $S\in\Lm{V}$ suth $\forall T\in\Lm{V},ST=TS$. Prove $\exists\,\lambda\in\Fbb,S=\lambda I$.}
}Let $B_{\range S}=\Par{w_1,\dots,w_m}$ with each $w_i=Sv_i.$ Extend to bses $\Par{w_1,\dots,w_n},\Par{v_1,\dots,v_n}$ of $V.$\parSol{}
{\vspace{2pt}Let {$S=E_{1,1}+\dots+E_{m,m}\Rightarrow \Mt{S,B_V}=\Mt{I,B_{\range S},B_V}.$} Note that $R_{k,1}:w_x\mapsto\delta_{k,x}v_1.$}\parSol{}
{\vspace{2pt}Then $\forall k\in\;\!\!\Bra{1,\dots,n},$ {\FontLarge$0\neq SR_{k,1}=R_{k,1}S.$} Hence $\dim\null S=0,\;\dim\range S=m=n.$}\parSol{}
\vspace{2pt}\NOTICE that {\FontLarge$G_{i,j}=R_{i,j}S=SR_{i,j}=Q_{i,j}$}. 又 For each $w_i,\exists\,!\,a_{k,\,i}\in\Fbb,w_i=a_{1,i}v_1+\dots+a_{n,i}v_n.$\parSol{}
Then fix one $i.$ Now for each $j\in\;\!\!\Bra{1,\dots,n},Q_{i,j}\Par{w_i}=w_j=a_{i,i}v_j=G_{i,j}\BigBigPar{{\sum_{k=1}^na_{k,\,i} v_k}}.$\parSol{}
Let $\lambda=a_{i,i}.$ Hence each $w_j=\lambda v_j.$ Now fix one $j,$ we have $a_{1,1}v_j=\dots=a_{n,n}v_j.$\PfEnd
\SepLine

%\BulletPointX\TipsN{5}\;\;Identify $v$ with $I$ restr to $\Span{v}$ onto $V,$ to behave as liney maps and use Exe (4E 17).
%\SepLine
%
%\BulletPointX\NoteForSmall{\tgnr{$\null A$ and $\null A^t$}}\;\;
%\TextB{}
%\TextB{}
%\TextB{}
%\TextB{}
%\TextB{}
%\SepLine
\ChEnd
\pagebreak

\ChDecl{Ch3E}{3$\cdot$E}{}

\vspace{4pt}

\Anchor{3EN3.79}\BulletPointX\NoteForSmall{[3.79], def of {\tgsc v + U}}\;\;Given $v+U,$ $v$ is already uniqly determined, as a sort of precond.\TextB{}
Even though $v+U=v'+U,$ where $v'$ is {\tgsl purer} than $v.$
%While becs $\pi$ is not inje, $\forall x\in V\XSlash U,\exists\,v_1,v_2,\cdots\in V,x=\pi\Par{v_k}.$\TextB{}
%Which is the reason we use $v+U\in V\XSlash U$ instead of $x\in V\XSlash U.$
\SepLine

\Anchor{3EN3.85}\BulletPointX\NoteForSmall{[3.85]}\;\;\TextB{$v+U=w+U\Longleftrightarrow v\in w+U,\;w\in v+U$}
\Blind{\NoteForSmall{[3.85]}\;\;}{\FontLarge$\Blind{v+U=w+U}\Longleftrightarrow v-w\in U\Longleftrightarrow \Par{v+U}\cap\Par{w+U}\neq\emptySet.$}
\SepLine

\Anchor{3EN3.7983}\BulletPointX\NoteFor{[3.79, 3.83]}\TextB{}
If $U$ is merely a subset of $V,$ then [3.85, 86] do not hold $\Rightarrow V\XSlash U$ not a vecsp.\TextB{}
If $V$ is merely a subset of a vecsp of which $U$ is a subsp, then [3,79, 86] do not hold $\Rightarrow V\XSlash U$ not a vecsp.\TextB{}
%Becs $\BigPar{\Par{v-w}+u}\in U$ or $u-u\apostrophe\in U$ needs that $U$ is closd add.\TextB{}
%And becs $\Par{v-\hat{v}}+\Par{w-\hat{w}}\in U$ and $\lambda\Par{v-\hat{v}}\in U$ asum $U$ is a subsp.\vspace{2pt}\TextB{}
\uline{If $U$ is a vecsp but not a subsp of $V,$ while $U,V$ are subsps of some vecsp, then everything's alright.}\TextB{}
Hence if $V\XSlash U$ is a vecsp, then $V,U$ are subsps of some vecsp.\TextB{}
\AComm Supp $U,V$ are subsps and $U$ is not a subsp of $V.$ Note that $V\XSlash U=\Par{V+U}\XSlash U.$\TextB{}
Supp $v+U\in V\XSlash U.$ Then $v\in V$, or possibly $v\in V+U$ as well. To avoid this ambiguity,\TextB{}
you have to specify the precond, what subsp that $v$ belongs to.\TextB{\vspace{2pt}}
\AExa Supp $U+W=V.$ Then $V\XSlash U=\Par{U+W}\XSlash U=W\XSlash U.$ Let $W\cap U=I,U_I\oplus I=U,W_I\oplus I=W.$\parExa{\IndentB}
Now $U_I\oplus W_I\oplus I=V.$ Thus $W\XSlash U=\Par{W_I\oplus I}\XSlash U=W_I\XSlash U.$\parExa{\IndentB}
$\forall w'_1,w'_2\in W_I$ suth $w'_1+U=w'_2+U\in W_I\XSlash U,\;w'_1-w'_2\in U\cap W_I=\zeroSubs\Rightarrow w'_1=w'_2.$\par\vspace{4pt}
\BulletPointX{\tgsc Trivial Cases\hspace{1pt}$:$}\;\;If $v\in U,$ then $v+U=0+U=\Bra{u:u\in U}=U.$ Now $U=0\in V\XSlash U.$\TextB{}
\Blind{{\tgsc Trivial Cases\hspace{1pt}$:$}\;\;}If $U=\zeroSubs,$ then $v+U=v+\zeroSubs=\Bra{v},\;V\XSlash U=V\XSlash\zeroSubs=\Bra{\Bra[\envFontB]{v}:v\in V}.$\TextB{}
\Blind{{\tgsc Trivial Cases\hspace{1pt}$:$}\;\;}If $U=\emptySet,$ then $v+U=v+\emptySet=\emptySet,\;V\XSlash U=V\XSlash\emptySet=\Bra{\emptySet}.$\par\vspace{2pt}
\Anchor{3ET1}\BulletPointX\TipsN{1}\,\,\,$V$ is a subsp of $U\Longleftrightarrow\forall v\in V,v+U=0+U=U\Longleftrightarrow V\XSlash U=\zeroSubs=\Bra{U}.$
\SepLine

\Anchor{3EN3.88}\BulletPointX\NoteForSmall{[3.88]}\;\;If $U,V$ are subsp of some vecsp $\mV$. Define the quot map $\pi\in\Lm{V,V\XSlash U}.$\TextB{}
Then $\pi$ is surj by def, and $\null\pi=V\cap U.$ \,Thus if $\mV$ is finide, then $\dim V=\dim V\XSlash U+\Dim\Par{V\cap U}.$\TextB{}
\Or Let $I=V\cap U,\;V_{\!I}\oplus I=V.$ Becs $V\XSlash U=V_{\!I}\XSlash U,$ iso to $V_{\!I}.$ Now $\dim V=\dim V_{\!I}+\dim I.$
\SepLine

%\Anchor{3E4e7}\ProblemBnoor[]{{4E 7}}{
	%	Let $U=\Bra{\Par{x,y,z}\in\Rbb^3:2x+3y+5z=0}.$ Supp $A\subseteq\Rbb^3$.\TextB{}
	%	Then $A$ is a trslate of $U\Longleftrightarrow\exists\,c\in\Rbb,A=\Bra{\Par{x,y,z}\in\Rbb^3:2x+3y+5z=c}.$\TextB{\vspace{-6pt}}
	%}\SepLine


\ProblemN{\Anchor{3E7}{7}}{
	\TextA{Supp $\alpha,\beta\in V,$ and $U,W$ are subsps of $V.$ Prove $\alpha+U=\beta+W\Rightarrow U=W$.}
}(a) $\alpha\in\alpha+U=\beta+W\Rightarrow\exists\,w\in W,\alpha=\beta+w\Rightarrow\alpha-\beta\in W\Rightarrow\alpha+W=\beta+W.$\parSol{}
(b) $\beta\in\beta+W=\alpha+U\Rightarrow\exists\,u\in U,\beta=\alpha+u\Rightarrow\beta-\alpha\in U\Rightarrow\alpha+U=\beta+U.$\PfEnd\parSol{\vspace{6pt}}
\Or $\pm\Par{\alpha-\beta}\in U\cap W \Rightarrow\MathLeftrightBrace{l}{U\ni u=\Par{\beta-\alpha}+w\in W\Rightarrow U\subseteq W\\W\ni w=\Par{\alpha-\beta}+u\in U\Rightarrow W\subseteq U}\Rightarrow U=W.$\PfEnd[-14pt]
\SepLine

\ProblemN{\Anchor{3E8}{8}}{
	\TextA{Supp $\emptySet\neq A\subseteq V.$ Prove $A$ is a trslate $\Longleftrightarrow\lambda v+\Par{1-\lambda}w\in A\,\,,\forall v,w\in A,\lambda\in\Fbb.$\vspace{2pt}}
}(a) Supp $A=a+U.$ Then $\lambda\Par{a+u_1}+\Par{1-\lambda}\Par{a+u_2}=a+\BigPar{\lambda\Par{u_1-u_2}+u_2}\in A.$\vspace{2pt}\parSol{}
(b) Supp $\lambda v+\Par{1-\lambda}w\in A,\forall v,w\in A,\lambda\in\Fbb.$ \;Supp \uline{$a\in A$} and let $A\apostrophe=\Bra{x-a:x\in A}.$\parSol{\Hb}
Then $0\in A\apostrophe$ and $\forall\Par{v-a},\Par{w-a}\in A\apostrophe,\lambda\in\Fbb,$ (I) $\lambda\Par{v-a}=\Sbra{\lambda v+\Par{1-\lambda}a}-a\in A\apostrophe.$\parSol{\vspace{2pt}\Hb}
(II) Becs $\lambda\Par{v-a}+\Par{1-\lambda}\Par{w-a}=\Sbra{\lambda v+\Par{1-\lambda}w}-a\in A\apostrophe.$\parSol{\Hb\HII}
Let $\lambda=\frac{\;1\;}{2}$ here and use (I) above by $\lambda=2,$ we have $\Par{v-a}+\Par{w-a}\in A\apostrophe.$\vspace{4pt}\parSol{\Hb\HII}
\Or Note that $v,a\in A\Rightarrow\lambda v+\Par{1-\lambda}a=2v-a\in A.$ Simlr $2w-a\in A.$\parSol{\Hb\HII}
Now $\Par{v-\frac{\;1\;}{2}a} + \Par{w-\frac{\;1\;}{2}a}=v+w-a\in A\Rightarrow v+w-2a=\Par{v-a}+\Par{w-a}\in A\apostrophe.$\vspace{4pt}\parSol{\Hb}
Thus $A\apostrophe=-a+A$ is a subsp of $V.$ Hence $a+A\apostrophe=a+\Bra{x-a:x\in A}=A$ is a trslate.\PfEnd
\SepLine

\ProblemN{\Anchor{3E9}{9}}{
	\TextA{Supp $A=\alpha+U$ and $B=\beta+W$ for some $\alpha,\beta\in V$ and some subsps $U,W$ of $V$.}
	\TextA{Prove $A\cap B$ is either a trslate of some subsp of $V$ or is $\emptySet$.}
}$\forall\alpha+u,\beta+w\in A\cap B\neq\emptySet,\lambda\in\Fbb,\lambda\Par{\alpha+u}+\Par{1-\lambda}\Par{\beta+w}\in A\cap B.$ By Exe (8).\PfEnd\parSol{}
\Or Let $A=\alpha+U,\;B=\beta+W.$ \;Supp $v\in\Par{\alpha+U}\cap\Par{\beta+W}\neq\emptySet.$\parSol{}
Then $v-\alpha\in U\Rightarrow v+U=\alpha+U=A,$ and simlr $v+W=\beta+W=B.$\parSol{}
We show $A\cap B=v+\Par{U\cap W}.$ Note that $v+\Par{U\cap W}\subseteq A\cap B.$\parSol{}
And $\forall\gamma=v+u=v+w\in A\cap B\Rightarrow u=w\in U\cap W\Rightarrow\gamma\in v+\Par{U\cap W}.$\PfEnd
\SepLine

\ProblemN{\Anchor{3E10}{10}}{
	\TextA{Prove the intersec of any collec of trslates of subsps is either a trslate of some subsps or $\emptySet.$}
}Supp $\Bra{A_{\alpha}}{_{\alpha\in\Gamma}}$ is a collec of trslates of subsps of $V$, where $\Gamma$ is an index set.\vspace{2pt}\parSol{}
$\forall x,y\in\bigcap_{\alpha\in\Gamma}A_{\alpha}\neq\emptySet,\lambda\in\Fbb,\lambda x+\Par{1-\lambda}y\in A_{\alpha}$ for each $\alpha.$ By Exe (8).\PfEnd\parSol{\vspace{6pt}}
\Or Let each $A_\alpha=w_\alpha+V_{\!\!\alpha}.$ Supp $x\in\bigcap_{\alpha\in\Gamma}\Par{w_\alpha+V_{\!\!\alpha}}\neq\emptySet.$\parSol{}
Then $x-w_\alpha\in V_{\!\!\alpha}\Longrightarrow x+V_{\!\!\alpha}=w_\alpha+V_{\!\!\alpha}=A_\alpha,$ for each $\alpha.$\parSol{}
We show $\bigcap_{\alpha\in\Gamma}A_\alpha=\bigcap_{\alpha\in\Gamma}\Par{x+V_{\!\!\alpha}}=x+\bigcap_{\alpha\in\Gamma}V_{\!\!\alpha}.$\parSol{}
$y\in\bigcap_{\alpha\in\Gamma}A_\alpha\Longleftrightarrow$ for each $\alpha,\;y=x+v_{\:\!\!\alpha}\in A_\alpha$\parSol{}
$\Blind{y\in\bigcap_{\alpha\in\Gamma}A_\alpha}\Longleftrightarrow$ each $v_{\alpha}=y-x\in\bigcap_{\alpha\in\Gamma}V_{\!\!\alpha}\Longleftrightarrow y\in x+\bigcap_{\alpha\in\Gamma}V_{\!\!\alpha}.$\PfEnd
\SepLine

\ProblemN{\Anchor{3E11}{11}}{
	\TextA{Supp $A=\Bra{\lambda_1 v_1+\dots+\lambda_m v_m:\sum_{i=1}^m \lambda_i=1}$, where each $v_i\in V,\lambda_i\in\Fbb.$\vspace{2pt}}
	\PrePa\TextA{Prove $A$ is a trslate of some subsp of $V$}
	\PrePb\TextA{Prove if $B$ is a trslate of some subsp of $V$ and $\Bra{v_1,\dots,v_m}\subseteq B$, then $A\subseteq B$.}
	\PrePc\TextA{Prove $A$ is a trslate of some subsp of $V$ of dim $< m.$\vspace{2pt}}
}(a) By Exe (8), $\forall u, w\in A,\lambda\in\Fbb,\lambda u+\Par{1-\lambda}w=\XPar{\lambda\sum_{i=1}^m a_i+\Par{1-\lambda}\sum_{i=1}^m b_i}v_i\in A.$\par\vspace{4pt}\quad
(b) Supp $B=v+U,$ where $v\in V$ and $U$ is a subsp of $V.$ \uline{Let each $v_k=v+u_k\in B,\exists\,!\,u_k\in U.$}\par\vspace{2pt}\quad\Hb
$\forall w\in A,\;w=\sum_{i=1}^m \lambda_i v_i=\sum_{i=1}^m \lambda_i\Par{v+u_i}=\uline{\sum_{i=1}^m \lambda_i v}+\sum_{i=1}^m \lambda_i u_i=v+\sum_{i=1}^m \lambda_i u_i\in v+U=B.$\PfEnd\vspace{6pt}\quad\Hb 
\Or Let $v=\lambda_1 v_1+\dots+\lambda_m v_m\in A$. To show $v\in B$, use induc on $m$ by $k$.\par\quad\Hb
(i) $k=1,v=\lambda_1 v_1\Rightarrow \lambda_1=1.$ 又 $v_1\in B.$ Hence $v\in B$.\par\quad\Hb\Hi
\vspace{4pt}$k=2,v=\lambda_1 v_1+\lambda_2 v_2\Rightarrow\lambda_2=1-\lambda_1.$ 又 $v_1,v_2\in B.$ By Exe (8), $v\in B$.\par\quad\Hb\Endi
(ii) $2\leqslant k<m.$ Asum $v=\lambda_1 v_1+\dots+\lambda_k v_k\in A\subseteq B.\;\Sbra{\forall\lambda_i\text{ suth }\sum_{i=1}^k\lambda_i=1}$\par\quad\Hb\Hii
\vspace{4pt}For $u=\mu_1 v_1+\dots+\mu_k v_k+\mu_{k+1} v_{k+1}\in A.$ \;Fix one $\mu_\iota\neq 1.$\par\quad\Hb\Hii
\vspace{4pt}Then \;$\sum_{i=1}^{k+1}\mu_i-\mu_\iota=1-\mu_\iota\Longrightarrow\envFontLarge\XSbra{{\envFontDefault\sum_{i=1}^{k+1}\Frac{\mu_i}{1-\mu_\iota}}}-\Frac{\mu_\iota}{1-\mu_\iota}=1.$\par\quad\Hb\Hii
\vspace{2pt}Let \,$w=\underbrace{\Frac{\mu_1}{1-\mu_\iota}v_1+\dots+\Frac{\mu_{\iota-1}}{1-\mu_\iota}v_{\iota-1}+\Frac{\mu_{\iota+1}}{1-\mu_\iota}v_{\iota+1}+\dots+\Frac{\mu_{k+1}}{1-\mu_\iota}v_{k+1}}_{k\,terms}.$\par\quad\Hb\Hii
\vspace{4pt}Let \,$\lambda_i=\Frac{\mu_i}{1-\mu_\iota}$ for $i\in\;\!\!\Bra{1,\dots,\iota-1};$ \;$\lambda_j=\Frac{\mu_{j+1}}{1-\mu_\iota}$ for $j\in\;\!\!\Bra{\iota,\dots,k}.$ \;Then,\par\quad\Hb\Hii
$\MathRightBrace{r}{\sum_{i=1}^k\lambda_i=1\Rightarrow w\in B\\ v_{\:\!\!\iota}\in B\Rightarrow u\apostrophe=\lambda w+\Par{1-\lambda}v_{\:\!\!\iota}\in B}\Rightarrow$ Let $\lambda=1-\mu_\iota$. Thus $u\apostrophe=u\in B\Rightarrow A\subseteq B.$\PfEnd\vspace{10pt}\quad
(c) If $m=1,$ then let $A=v_1+\zeroSubs$ and done. \;Now supp $m\geqslant 2.$ Fix one $k\in\;\!\!\Bra{1,\dots,m}.$\par\quad\Hc
$A\ni \lambda_1v_1+\dots+\lambda_{k-1}v_{k-1}+\BigPar{1-\lambda_1-\dots-\lambda_{k-1}-\lambda_{k+1}-\dots-\lambda_m}v_k+\lambda_{k+1}v_{k+1}+\dots+\lambda_m v_m$\par\quad\Hc
$\Blind{A}=v_k+\lambda_1\Par{v_1-v_k}+\dots+\lambda_{k-1}\Par{v_{k-1}-v_k}+\lambda_{k+1}\Par{v_{k+1}-v_k}+\dots+\lambda_m\Par{v_m-v_k}$\par\quad\Hc
$\Blind{A}\in v_k+\Span{v_1-v_k,\dots,v_m-v_k}.$\PfEnd
\SepLine\pagebreak

\ProblemN{\Anchor{3E18}{18}}{
	\TextA{Supp $T\in\Lm{V,W}$ and $U,V$ are subsps of $\mV.$ Let $\pi:V\rightarrow V\XSlash U$ be the quot map.}
	\TextA{Prove $\exists\,S\in\Lm{V\XSlash U,W},T=S\circ\pi\Longleftrightarrow U\cap V=\null\pi\subseteq\null T$.}
}Supp $\null\pi\subseteq\null T.$ By ({3.B.24}), done. \;\Or Define $S:\Par{v+U}\mapsto Tv.$\parSol{}
$\forall v_1,v_2\in V$ suth $v_1+U=v_2+U\Longleftrightarrow v_1-v_2\in U\cap V\subseteq\null T\Longleftrightarrow Tv_1=Tv_2.$\parSol{}
Thus $S$ is well-defined. Convly true as well.\PfEnd\vspace{2pt}\Anchor{3E20}
\ACoro $\Gamma:$ {\FontSmall$\Lm{V\XSlash U,W}\rightarrow\Lm{V,W}$} with $S\mapsto S\circ\pi$ is inje, $\range\Gamma=\Bra{T\in\Lm{V,W}:U\subseteq\null T}.$\par\vspace{0pt}
\AComm If $T=I_V.$ Then $S:v+U\mapsto v$ is not well-defined, unless $U\cap V=\zeroSubs\subseteq\null I_V.$\vspace{-3pt}
\SepLine

\Anchor{3EN3.889091}\BulletPointX\NoteFor{[3.88, 3.90, 3.91]}\;\;{Supp $W\oplus U=V.$ Then $V\XSlash U=W\XSlash U$ is iso to $W.$\quad\FontSmall\Sbra{\,Convly not true.\,}}\TextB{}
{Becs $\forall v\in V,\exists\,!\,u_v\in U,w_v\in W,v=u_v+w_v.$ \uline{Define $T\in\Lm{V}$ by $T\Par{v}=w_v.$}}\vspace{1pt}\TextB{}
{Hence $\null T=U,\;\range T=W,\;\range T\oplus\null T=V.$}\vspace{1pt}\TextB{}
{Then $\tilde{T}\in\Lm{V\XSlash \null T,V}$ is defined by $\tilde{T}\Par{v+U}=\tilde{T}\Par{w_v'+U}=Tw_v'=w_v.$ \FontSmall\Sbra{\tgsl See Exa below}}\vspace{1pt}\TextB{}
{Now $\pi\circ\tilde{T}=I_{V\XSlash U},\;\tilde{T}\circ\pi\mmid_W=I_{W}=T\mmid_W.$ \;Hence $\tilde{T}=\Par{\pi\mmid_W}{^{-1}}$ is iso of $V\XSlash U$ onto $W.$}\par\vspace{3pt}
%\BulletPointX\TipsN{1}\,\,\,{Supp $U$ is a subsp of $V.$ \uline{Define $S\in\Lm{V\XSlash U,V}$ by $S\Par{v+U}=v.$}}\TextB{}
%{Then $\range S$ is the {\tgsl purest} in $\Scom{V}{U}.$ Now $\null S=\zeroSubs,\;U\oplus\range S=V.$}\TextB{}
%{\uline{Let $E=S\circ\pi.$} Becs $S$ is inje and $\pi$ is surj, $\null E=\null\pi=U,\;\range E=\range S.$}\TextB{}
%{Then $\range E\oplus\null E=V.$ \;\NOTICE that $E:V\rightarrow W$ is the {\tgsl purest eraser}. Now we explain why:}\TextB{\vspace{2pt}}
\ProblemBX[]{\Example}{
	Let $V=\FbbP{2},B_U=\Par{e_1},B_W=\Par{e_2-e_1}\Rightarrow U\oplus W=V.$\TextA{}
	Although $\Par{e_2-e_1}+U=e_2+U,$ $\tilde{T}\Par{e_2+U}=T\Par{e_2}=e_2-e_1.$ Becs $e_2=e_1+\Par{e_2-e_1}\in U\oplus W.$\vspace{-3pt}\TextA{}
%{Notice that $T\Par{e_2-e_1}=\Par{e_2-e_1},$ while $\Par{e_2-e_1}+U=e_2+U,$ but}\TextE{}
%{becs $e_2=e_1+\Par{e_2-e_1},$ now still, $\tilde{T}\BigPar{\Par{e_2-e_1}+U}=e_2-e_1=Te_2.$}\TextE{}
%{In contrast, $S\BigPar{\Par{e_2-e_1}+U}=S\Par{e_2+U}=e_2,\;E\Par{e_2-e_1}=e_2.$}\TextE{}
%{And $\range E=\range S=\Span{e_2}$ is the {\tgsl purest} in $\Scom{V}{U}.$}
}\SepLine

\ProblemN{\Anchor{3E17}{17}}{
	\TextA{Supp $V\XSlash U$ is finide. Supp $W$ is finide and $V=U+W.$ Show $\dim W\geqslant\dim V\XSlash U$.}
}Let $Y\oplus\Par{U\cap W}=W.$ Then by \Sbra{1.C \TIPSN{3}}, $V=U\oplus Y.$ Note that $V\XSlash U$ and $Y$ are iso.\PfEnd\parSol{\vspace{2pt}}
\Or Let $B_W=\Par{w_1,\dots,w_n}.$ Then $V=U+\Span{w_1,\dots,w_n}.$\parSol{}
$\forall v\in V,\exists\,u\in U,\;v=u+\Par{a_1w_1+\dots+a_nw_n}\Rightarrow v+U=\Par{a_1w_1+\dots+a_nw_n}+U.$\PfEnd\vspace{2pt}
\ANote If $\dim W=\dim V\XSlash U.$ Then $B_{V\XSlash U}=\Par{w_1+U,\dots,w_n+U}.$ Supp $v=\sum_{i=1}^na_iw_i\in U\cap W$\parNot
$\Rightarrow v+U=0=\sum_{i=1}^na_i\Par{w_i+U}\Rightarrow$ each $a_i=0.$ Thus $V=U\oplus W.$
\SepLine

\ProblemN{\Anchor{3E12}{12}}{
	\TextA{Supp $U$ is a subsp of $V.$ Prove is $V$ is iso to $U\times\Par{V\XSlash U}$.}
}\par\quad
\!\Sbra[3pt]{{\tgsl Req $V\XSlash U$ Finide}} \;Let $B_{V\XSlash U}=\BigPar{v_1+U,\dots,v_n+U}.$\par\quad
Now $\forall v\in V,\exists\,!\,a_i\in\Fbb,v+U=\sum_{i=1}^n a_iv_i+U\Rightarrow v-\sum_{i=1}^na_iv_i\in U\Rightarrow\exists\,!\,u\in U,v=\sum_{i=1}^n a_i v_i+u.$\vspace{2pt}\par\quad
Thus define \;$\varphi\in\Lm[\BigPar]{V,U\times\Par{V\XSlash U}}$\quad\;\;\;\,\hspace{0pt}and \;$\psi\in\Lm[\BigPar]{U\times\Par{V\XSlash U},V}$\par\vspace{1pt}\quad
\Blind{Thus defi}by \;$\varphi\Par{v}=\BigPar{u,\;\sum_{i=1}^na_iv_i+U},$ and \;$\psi\Par{u,\;v+U}=\sum_{i=1}^na_iv_i+u.$ \quad Then $\psi=\varphi^{-1}.$\PfEnd\vspace{8pt}\quad
\Or Let $W\oplus U=V.$ Define $Tv=u_v,Sv=w_v\Rightarrow\tilde{T}\in\Lm{V\XSlash W,U},\tilde{S}\in\Lm{V\XSlash U,W}$ are iso.\par\quad
Define $\psi\Par{u,v+U}=u+\tilde{S}\Par{v+U}=u+w_v.$ Define $\varphi\Par{v}=\BigPar{\tilde{T}\Par{v},v+U}.$\par\quad
\!\!\!$\MathRightBrace{l}{\Par{\psi\circ\varphi}\Par{u_v+w_v}=\psi\Par{u_v,w_v+U}=u_v+w_v\\\Par{\varphi\circ\psi}\Par{u,v+U}=\varphi\Par{u+w_v}=\Par{u,w_v+U}}\Rightarrow\psi=\varphi^{-1}.$ \; \OR Becs $\psi$ or $\varphi$ is inje and surj.\PfEnd\vspace{2pt}
%\Or Define $S\in\Lm{V\XSlash U,V}$ by $S\Par{v+U}=v.$\vspace{2pt}\par\quad
%By {\NOTEFOR} [3.88, 90, 91], $\range S\oplus U=V.$ Thus $\forall v\in V,\exists\,!\,u\in U,w\in\range S,\;v=u+w.$\vspace{2pt}\par\quad
%Define $T\in\Lm[\BigPar]{U\times\Par{V\XSlash U},V}$ by $T\Par{u,v+U}=u+S\Par{v+U}=u+w=v.$ Then $T$ is surj.\vspace{2pt}\par\quad
%And $T\Par{u,v+U}=u+S\Par{v+U}=0\Longrightarrow\pi\BigPar{T\Par{u,v+U}}=v+U=0,$ and $u=-S\Par{v+U}=0.$\vspace{4pt}\par\quad
%\Or Define $R\in\Lm[\BigPar]{V,U\times\Par{V\XSlash U}}$ by $R\Par{v}=\BigPar{u,\Par{w+U}}.$ Now $R\circ T=I_{U\times\SmallPar{V\XSlash U}},\;T\circ R=I_V.$\PfEnd
\SepLine

\ProblemN{\Anchor{3E13}{13}}{
	\TextA{Prove $B_{V\XSlash U}=\BigPar{v_1+U,\dots,v_m+U},B_U=\Par{u_1,\dots,u_n}\Rightarrow B_V=\BigPar{v_1,\dots,v_m,u_1,\dots,u_n}.$\vspace{4pt}}
}$\forall v\in V,\exists\,!\,a_i\in\Fbb,v+U=\sum_{i=1}^m a_i v_i+U\Rightarrow\exists\,!\,b_i\in\Fbb,v-\sum_{i=1}^m a_i v_i=\sum_{i=1}^n b_i u_i\in U$\parSol{\vspace{2pt}}
$\Rightarrow \forall v\in V,\exists\,!\,a_i,b_j\in\Fbb,v=\sum_{i=1}^m a_i v_i+\sum_{j=1}^n b_j u_j.$\PfEnd\parSol{\vspace{6pt}}
\Or $\sum_{i=1}^m a_i v_i+\sum_{i=1}^n b_i u_i=0\Rightarrow\sum_{i=1}^m a_i\Par{v_i+U}=0\Rightarrow$ each $a_i=0\Rightarrow$ each $b_i=0.$\PfEnd\parSol{\vspace{6pt}}
\Or Note that $B=\Par{v_1,\dots,v_m}$ is liney indep, and $\Sbra{\Span{v_1,\dots,v_m}+U}\subseteq V.$\parSol{}
$v\in\spn B\cap U\Longleftrightarrow v+U=\sum_{i=1}^ma_i\Par{v_i+U}=0+U\Longleftrightarrow v=0.$ \,Hence $\spn B\cap U=\zeroSubs.$\parSol{}
Becs $\Dim\Sbra{\Span{v_1,\dots,v_m}\oplus U}=m+n=\dim V.$ \;Now by (2.B.8).\PfEnd
\SepLine

\Anchor{3E4e14}\ProblemBnoor{{4E 14}}{
	\TextB{Supp $V=U\oplus W,\;B_W=\Par{w_1,\dots,w_m}.$ Prove $B_{V\XSlash U}=\BigPar{w_1+U,\dots,w_m+U}.$\vspace{4pt}}
}$\forall v\in V,\exists\,!\,u\in U,w\in W,v=u+w.$ 又 $\exists\,!\,c_i\in\Fbb,w=\sum_{i=1}^m c_i w_i\Rightarrow v=\sum_{i=1}^m c_i w_i+u$.\parSol{}
Hence $\forall v+U\in V\XSlash U,\exists\,!\,c_i\in\Fbb,v+U=\sum_{i=1}^m c_i w_i+U.$\PfEnd\vspace{2pt}\parSol{}
\Or Becs $\pi\mmid_W:W\rightarrow W\XSlash U$ is inv, and $V\XSlash U=W\XSlash U.$\PfEnd
\SepLine

%\BulletPointX\NoteForSmall{Exe (13) and (4E 14)}\;\;Let $U\oplus W=V.$ Define $S\Par{w+U}=w.$ \;\Sbra{ See also \TIPSN{1}. }\TextB{}
%(a) Let $B_W=\Par{w_1,\dots,w_m}\Rightarrow B_{V\XSlash U}=\BigPar{w_1+U,\dots,w_m+U}.$ Then $S\Par{w_k+U}$ might not equal $w_k.$\TextB{}
%(b) Let $B_{V\XSlash U}=\BigPar{w_1+U,\dots,w_m+U},$ then let $B_W=\Par{w_1,\dots,w_m}.$ Now each $S\Par{w_k+U}=w_k.$
\def\Pure{{\textup{\tgnr Pure}}\,}
%\par\vspace{2pt} \BulletPointX\NewNotation\;$\Pure V\XSlash U=W\Longleftrightarrow V=U\oplus W,\;W=\range S.$ {\FontSmall\tgsl The uniqnes of $\Pure V\XSlash U$ follows from $\range S.$}
%\SepLine

%\Anchor{3E4e8}\ProblemBnoor{{4E 8}}{
%		\TextB{Supp $T\in\Lm{V,W},w\in\range T.$ Prove $\Bra{v\in V:Tv=w}=u+\null T.$}
%	}{\def\mK{\mathcal{K}_{\!w}}Let $\mK=\Bra{v\in V:Tv=w}.$ \Sbra{ Not a vecsp.\hspace{1pt}} \;Supp $u\in\mK.$ Then $u+\null T\subseteq\mK.$\parSol{}
%	And $\forall u\apostrophe\in\mK,\;u\apostrophe-u\in\null T\Rightarrow u\apostrophe\in u+\null T.$ Now $\mK\subseteq u+\null T.$\PfEnd}
%\SepLine

%\ProblemN{\Anchor{3E15}{15}}{
%	\TextA{Supp $\varphi\in\Lm[\Par]{V,\Fbb}\nonzero.$ Prove $\dim V\XSlash \Par{\null\varphi}=1$.\vspace{2pt}}
%}By [3.91] (d), $\dim\range\varphi=1=\dim V\XSlash \Par{\null \varphi}.$\parSol{}
%\Or By ({3.B.29}), $\exists\,u,\;\Span{u}\oplus\null\varphi=V.$ Then $B_{V\XSlash \null\varphi}=\Par{u+\null\varphi}.$\PfEnd\vspace{-2pt}
%\SepLine

\ProblemN{\Anchor{3E16}{16}}{
	\TextA{Supp $\dim V\XSlash U=1$. Prove $\exists\,\varphi\in\Lm{V,\Fbb},\null\varphi=U.$\vspace{2pt}}
}Supp $V_{\!0}\oplus U=V.$ Then $V_{\!0}$ is iso to $V\XSlash U.$ Define $\varphi\in\Lm{V,\Fbb}$ by $\varphi\Par{av_0+u}=a.$\PfEnd\parSol{\vspace{4pt}}
\Or Let $B_{V\XSlash U}=\Par{w+U}.$ Then $\forall v\in V,\exists\,!\,a\in\Fbb,v+U=aw+U.$\parSol{}
Define $\varphi\in\Lm{V\XSlash U,\Fbb}$ by $\varphi\Par{aw+U}=a.$ Then $\Null\Par{\varphi\circ\pi}=U.$\PfEnd
\SepLine

\Anchor{3E'1}\ProblemB{
	\TextB{Supp $U,W$ are subsps of $\mV$, and $X,Y$ are subsps of $\mathcal{W}.$}
	\TextB{Supp $U,X$ are iso, $W,Y$ are iso. Prove or give a countexa\hspace{1pt}$:$ $U\XSlash W$ and $X\XSlash Y$ are iso.}
}A countexa: Let $\mV=\mathcal{W}=\FbbP{2}.$ Let $U=X=Y=\Span{e_1},W=\Span{e_2}.$\parSol{}
Then $\dim U\XSlash W=\dim U-\Dim\Par{U\cap W}=1\neq 0=\dim X-\Dim\Par{X\cap Y}=\dim X\XSlash Y.$\PfEnd\parSol{}
\Or Let $\mV=U=W=\FbbP{\infty}=X,\,Y=\Bra{\Par{0,x_1,x_2,\cdots}}.$ Then $U\XSlash W=\zeroSubs,$ while $\dim X\XSlash Y=1.$\PfEnd
\SepLine

\Anchor{3ET2}\ProblemBX{\TipsN{2}}{
	\TextB{Supp $U,W$ are vecsps, $I=U\cap W.$ Prove $V=U+W\Longleftrightarrow V\XSlash I=U\XSlash I\oplus W\XSlash I.$}
}(a) Supp $V=U+W.$ Then $\forall v+I\in V\XSlash I,\exists\,\Par{u_v,w_v}\in U\times W,\;v+I=\Par{u_v+w_v}+I.$\parSol{\Ha}
Note that $U\XSlash I,W\XSlash I\subseteq V\XSlash I.$ Thus $V\XSlash I=U\XSlash I+W\XSlash I.$\parSol{\Ha}
$\forall u+I=w+I\in\Par{U\XSlash I}\,${\Large$\cap$}$\,\Par{W\XSlash I},\;$\uline{$u-w\in I=U\cap W$}\parSol{\Ha}
\uline{$\Rightarrow\exists\,w'\in I,u=w+w'\in U\cap W$}${}\Rightarrow u+I=0+I=w+I.$ Thus $\Par{U\XSlash I}\,${\Large$\cap$}$\,\Par{W\XSlash I}=\zeroSubs.$\parSol{\vspace{3pt}}
(b) Supp $V\XSlash I=U\XSlash I\oplus W\XSlash I.$ Then $\forall v\in V,v+I=\Par{u+I}+\Par{w+I}$\parSol{\Hb}
$\Rightarrow v-u-w\in I=U\cap W\Rightarrow\exists\,x\in U\cap W,v=u+w+x\in U+W.$\PfEnd
\SepLine

\Anchor{3ET4}\ProblemB{
	Supp $T\in\Lm{V,W},$ and $U,V$ are subsps of some vecsp, and $X,W$ are subsps of some vecsp.\TextB{\vspace{2pt}}
	\TextB{Define $\LmQxx{T}:V\XSlash U\rightarrow W\XSlash X$ by $\LmQxx{T}\Par{v+U}=Tv+X.$\vspace{2pt}}
	\PrePa\TextB{Prove \FontNorm $\LmQxx{T}$ is well-defined $\Longleftrightarrow\BigPar{\range T\mmid_{U\,\cap\,V}}\XSlash\Par{X\cap W}=\zeroSubs${${}\Longleftrightarrow\range T\mmid_{U\,\cap\,V}$ is a subsp of $X\cap W$}.\vspace{2pt}}
	Supp $\LmQxx{T}$ is well-defined, and thus is liney. \;Define $\pi_U\in\Lm{V,V\XSlash U},\pi_X\in\Lm{W,W\XSlash X}.$\TextB{}
	Then $\LmQxx{T}\circ\pi_U=\pi_X\circ T.$ \;Define $\LmQxx{T}[]\in\Lm{V,W\XSlash X}$ by $\LmQxx{T}[]\Par{v}=Tv+X.$\TextB{\vspace{2pt}}
	\PrePb\TextB{$\range \LmQxx{T}={}${$\Range\BigPar{\LmQxx{T}\circ\pi_U}=\Range\Par{\pi_X\circ T}$}${}=\Par{\range T}\XSlash X.$\vspace{3pt}}
	\PrePc\TextB{Prove $\LmQxx{T}$ is surj $\Longleftrightarrow W=\range T+X\cap W.$\vspace{4pt}}
	\PrePd\TextB{Show $\null\LmQxx{T}=\BigPar{\null\LmQxx{T}[]}\XSlash U.$\;\;{\tgnr\large(e)} $\LmQxx{T}$ is inje $\Longleftrightarrow\null\LmQxx{T}[]\subseteq U.$\vspace{4pt}}
}(a) For $v,w\in V.$ If $v+U=w+U\Longleftrightarrow v-w\in U\Rightarrow Tv-Tw\in X\cap W\Longleftrightarrow Tv+X=Tw+X.$\parSol{\Ha}
Then $\forall u\in V\cap U,Tu\in X\Rightarrow\range T\mmid_{U\,\cap\,V}\subseteq X\cap W.$ \,Convly true as well.\vspace{2pt}\parSol{}
(c) Supp $\LmQxx{T}$ is surj. $\forall w\in W,w+X\in W\XSlash X\Rightarrow\exists\,v+U\in V\XSlash U,\,Tv+X=w+X$\parSol{\Hc}
$\Rightarrow w-Tv\in X\cap W\Rightarrow w\in\range T+X\cap W.$ \,Hence $W\subseteq\range T+X\cap W.$\vspace{2pt}\parSol{\Hc}
Convly, $W=\range T+X\cap W\Rightarrow\Par{\range T}\XSlash{X}=\Par{\range T+X\cap W}\XSlash{X}=W\XSlash{X}.$\vspace{4pt}\parSol{}
(d) $v+U\in\null\LmQxx{T}\Longleftrightarrow Tv\in X\Longleftrightarrow v\in\null\LmQxx{T}[]\Longleftrightarrow v+U\in\BigPar{\null\LmQxx{T}[]}\XSlash U.$\PfEnd\vspace{4pt}
\BulletPointX\AComm Supp $T\in\Lm{V}.$ Define $T\XSlash U\in\Lm{V\XSlash U}$ by $T\XSlash U=\LmQxx{T}[U][U].$ Then\TextB{}
\!\!\;(a) $T\XSlash U$ well-defined $\Longleftrightarrow U\cap V$ invard $T.$\;\;(b) $\range T\XSlash U=\Range\Par{\pi\circ T}=\Par{\range T}\XSlash U.$\TextB{}
\!\!\;(c) $T\XSlash U$ surj $\Longleftrightarrow V=\range T+U\cap V.$\;\;(d) $\null T\XSlash U=\BigPar{\null\LmQxx{T}[][U]}\XSlash U.$\;\;(e) $T\XSlash U$ inje $\Longleftrightarrow\null\LmQxx{T}[][U]\subseteq U.$\par
\SepLine

\Anchor{5A33}\ProblemBnoor{{5.A.33}}{
	\TextB{Supp $T\in\Lm{V}.$ Prove $T\XSlash{\range T}=0.$\FontNorm\hfill\tgnr By (b) or (d) above, immed.}
}$v+\range T\in V\XSlash \range T\Rightarrow v+\range T\in\Null\Par{T\XSlash{\range T}}.$ Thus $T\XSlash{\range T}=0.$\PfEnd
\SepLine

\Anchor{5A34}\ProblemBnoor{{5.A.34}}{
	\TextB{Supp $T\in\Lm{V}.$ Prove $T\XSlash{\null T}$ is inje $\Longleftrightarrow{\null T}${\LARGE${}\cap{}$}${\range T}=\zeroSubs.$}
}\NOTICE that $\BigPar{T\XSlash{\null T}}\Par{u+\null T}=Tu+\null T=0\Longleftrightarrow Tu\in{\null T}${\Large${}\cap{}$}${\range T}.$\parSol{}
Now $T\XSlash{\null T}$ is inje $\Longleftrightarrow u+\null T=0\Longleftrightarrow Tu=0 \Longleftrightarrow{\null T}${\Large${}\cap{}$}${\range T}=\zeroSubs.$\PfEnd
\SepLine

\Anchor{3ET3}\ProblemBX{\TipsN{3}}{
	\TextA{Supp $U,W$ are subsps of $V$ and $X$ is a subsp of $U\cap W.$}
	\TextA{Prove $U\XSlash W$ and $\BigPar{U\hspace{-1pt}\Slash X}\XSlash\BigPar{W\hspace{-1.5pt}\Slash X}$ are iso.}
}Let $U_X\oplus X=U,W_X\oplus X=W.$ Becs $U\XSlash W=U_X\XSlash W,$ and $U\XSlash X=U_X\XSlash X.$\vspace{1pt}\par\quad
Define $T\in\Lm[\XPar]{\BigPar{U_X\hspace{-1pt}\Slash X}\XSlash\BigPar{W\hspace{-1.5pt}\Slash X},U_X\XSlash W}$ by $T\BigPar{\Par{u_x+X}+W\XSlash X}=u_x+W.$\par\quad
$\forall u_1,u_2\in U_X$ suth $\Par{u_1+X}+W\XSlash X=\Par{u_2+X}+W\XSlash X\Rightarrow u_1-u_2+X\in W\XSlash X$\par\quad
$\Rightarrow u_1-u_2\in X+W$ 又 $u_1,u_2\in U_X\Rightarrow u_1-u_2\in W\Rightarrow u_1+W=u_2+W.$ \;Now $T$ is well-defined.\par\quad
Inje: $\forall u_x\in U_X$ suth $u_x+W=0\Rightarrow u_x\in W_X\Rightarrow\Par{u_x+X}\in W_X\XSlash X.$\par\quad
Surj: $\forall u_x\in U_X,u_x+W=T\BigPar{\Par{u_x+X}+W\XSlash X}.$ \;Hence $T$ is iso.\PfEnd\vspace{6pt}\quad
\Or Define $S\in\Lm{U_X\XSlash X,U_X}$ by $S\Par{u_x+X}=u_x.$
Becs $\forall u_1+X=u_2+X\in U_X\XSlash X,$\par\quad
$u_1-u_2\in X$ 又 $u_1,u_2\in U_X\Rightarrow u_1=u_2.$ Now $S$ well-defined, and $\LmQxx{S}[\SmallPar{W\XSlash X}][W][\Big/]=T$ defined above.\par\quad
Becs $\range S\mmid_{W\XSlash X\,\cap\,U_X\XSlash X}\subseteq W,$ and $U_X=\range S\Rightarrow U_X\subseteq\range S+W.$ \;Well-defined. Surj.\par\quad
For $u_x\in U_X,\;u_x+W=0\Longleftrightarrow u_x\in U_X\cap W\Longleftrightarrow u_x+X\in\Par{U_X\cap W}\XSlash X=\null\LmQxx{S}[][W].$ \;Inje.\PfEnd
\SepLine
\ChEnd

\vfill\ChDecl{Ch3F}{3$\cdot$F}{}

\vspace{4pt}

\ProblemN{\Anchor{3F4}{4}}{
	\TextA{Supp $U$ is a subsp of $V\neq U.$ Prove $U^0\neq\zeroSubs.$\vspace{1pt}}
}Let $X\oplus U=V\Rightarrow X\neq\zeroSubs.$ Supp $s\in X\nonzero.$ Let $Y\oplus\Span{s}=X.$\parSol{}
Define $\varphi\in V\apostrophe$ by $\varphi\Par{u+\lambda s+y}=\lambda.$ Hence $\varphi\neq 0$ and $\varphi\Par{u}=0$ for all $u\in U.$\PfEnd\parSol{\vspace{4pt}}
\Or \Sbra[3pt]{{\tgsl Req $V$ Finide}} \;By [3.106], $\dim U^0=\dim V-\dim U>0.$\parSol{}
\Blind{\Or }\Or Let $B_U=\Par{u_1,\dots,u_m},B_V=\Par{u_1,\dots,u_m,v_1\dots,v_n}$ with $n\geqslant 1.$\parSol{}
\Blind{\Or}Let $B_{V\apostrophe}=\Par{\psi_1,\dots,\psi_m,\varphi_1,\dots,\varphi_n}.$ Then each $\varphi\in\Span{\varphi_1,\dots,\varphi_n}$ will do.\PfEnd\vspace{4pt}\Anchor{3F19}
\ACoro \ProblemN[]{19}{
	\TextA{\tgnr $U^0=\zeroSubs=V^0\Longleftrightarrow U=V$.\vspace{0pt}}
}\par
\Anchor{3FN3.108}\AComm {\tgsc\large Another proof of \tgbfx[3.108]}: $T$ is surj $\Longleftrightarrow T\apostrophe$ is inje.\parCom
(a) Supp $T\apostrophe$ is inje. \NOTICE that $\psi\neq 0\Longleftrightarrow T\apostrophe\Par{\psi}\neq 0\Longleftrightarrow\psi\not\in\Par{\range T}{^0}.$\parCom
(b) $T$ is surj $\Rightarrow\Par{\range T}{^0}=\zeroSubs=\null T\apostrophe.$\PfEnd
\SepLine\pagebreak

\Anchor{3FN3.102}\Anchor{3FN18}\Anchor{3F18}\BulletPointX\NoteForSmall{[3.102] and Exe (18)}\;\;For $U=\emptySet,\;U^0=\Bra{\varphi\in V\apostrophe:U\subseteq\null\varphi}=V\apostrophe.$ While $\zeroSubs{^0_V}=V\apostrophe.$\TextB{}
Not a ctradic to Exe (21) becs $\emptySet$ is not a subsp. Now $U^0=V\apostrophe$ can be true with $U=\emptySet\neq\zeroSubs.$\vspace{-2pt}
\SepLine

\Anchor{3FT1}\BulletPointX\TipsN{1}\,\,\,Supp $\varphi_1,\dots,\varphi_m\in V\apostrophe.$ Denote $\Sbra{\null\varphi_a\cap\cdots\cap\null\varphi_b}$ by \,$\bigcap_a^b\null\varphi_I.$\TextB{}
\IndentTipsN{1}Supp $\Omega$ is a subsp of $V\apostrophe.$ Denote $\Bra{v\in V:\varphi\Par{v}=0,\forall\varphi\in\Omega}$ by $C^0\,\Omega.$\TextB{\vspace{1pt}}
(a) $\Omega$ is infinide. By def, $\bigcap_{\in\,\Omega}\null\varphi=C^0\,\Omega.$\TextB{}
(b) $\Omega=\Span{\varphi_1,\dots,\varphi_m}.$ Becs $v\in\bigcap_1^m\null\varphi_I\Longleftrightarrow\forall\varphi=\sum_{i=1}^na_i\varphi_i\in\Omega,\varphi\Par{v}=0\Longleftrightarrow v\in C^0\,\Omega.$
\SepLine

\ProblemN{\Anchor{3F25}{25}}{
	\TextA{Supp $U$ is a subsp of $V$. Explain why $U=C^0U^0.$}
}Asum $v\in C^0U^0$ while $v\in V\Backslash U.$ Then let $\Span{v}\oplus U\oplus X=V.$\parSol{}
$\exists\,\varphi\in V\apostrophe,\null\varphi=U\oplus X\Rightarrow\varphi\in U^0.$ 
又 $\varphi\Par{v}=0\Rightarrow 0\neq v\in\null\varphi\cap\Span{v}.$ Ctradic.\PfEnd\vspace{2pt}
\Anchor{3FNP}\AComm $X\subseteq W=\Bra{v\in V:\varphi\Par{v}=0,\forall\varphi\in X^0},$ the {\tgsc promotion} of the subset $X$ of $V.$
\SepLine

\Anchor{3F'1}\ProblemB{
	\TextB{Supp $U,W$ are subsps of $V.$ Prove the promotion of $U\cup W$ is $U+W.$}
}$\Par{U\cup W}{^0}=\Bra{\varphi\in V\apostrophe:\varphi\Par{u}=\varphi\Par{w}=\varphi\Par{u+w}=0,\forall u\in U,w\in W}=\Par{U+W}{^0}.$\PfEnd
\SepLine

%\Anchor{3F'2}\ProblemB[]{
%	Supp $X=\Bra{x_1,\dots,x_m}\subsetneq V.$ Then $X^0=\BigBra{{\varphi\in V\apostrophe:\text{each }\varphi\Par{x_k}=0}}=\Span{x_1,\dots,x_m}{^0}.$\TextB{}
%}\AComm The promotion of every finite subset $X$ of $V$ is the smallest subsp of $V$ containing $X.$
%\SepLine

%\ProblemN{\Anchor{3F20}{20}}{
%	\TextA{Supp $U,W$ are subsets of $V$. Prove $U\subseteq W\Rightarrow W^0\subseteq U^0.$}
%}$\forall\varphi\in W^0,u\in U\subseteq W,\varphi\Par{u}=0\Rightarrow\varphi\in U^0.$ Thus $W^0\subseteq U^0.$\PfEnd
%\SepLine

\ProblemN{\Anchor{3F21}{21}}{
	\TextA{Supp $U,W$ are subsps of $V$. Prove $W^0\subseteq U^0\Rightarrow U\subseteq W.$}
}{\FontSmall$\varphi\in W^0\Longleftrightarrow{}$\uline{$\null\varphi\supseteq W\Rightarrow\null\varphi\supseteq U$}${}\Longleftrightarrow\varphi\in U^0.$} {\FontSmall \uline{Wrong} becs generally $\nexists\,\varphi\in V\apostrophe$ with $\null\varphi=W.$}\vspace{2pt}\parSol{}
$v\in U\Rightarrow\forall\varphi\in W^0\subseteq U^0,\,\varphi\Par{v}=0\Rightarrow v\in W,$ by Exe (25).\PfEnd\vspace{4pt}
\AComm (1) If $U$ is merely a subset and $W$ is a subsp. Promote $U$ as $X,$ let $W=Y.$\parCom
\Blind{(1)} Then $Y^0=W^0\subseteq U^0=X^0\Rightarrow Y=W\supseteq X\supseteq U.$ Still true.\parCom
(2) If $W$ is merely a subset and $U$ is a subsp. Promote $W$ as $Y,$ let $U=X.$ For exa,\parCom
\Blind{(2)} Let $W=\Bra{\Par{1,0},\Par{0,1}}\not\supseteq U=\Bra{\Par{x,0}\in\Rbb^2}.$ Then $Y=\Rbb^2\supseteq X=U,\;Y^0=\zeroSubs\subseteq X^0.$
\SepLine

\ProblemN{\Anchor{3F22}{22}}{
	\TextA{Supp $U$ and $W$ are subsps of $V$. Prove $\BigPar{U+W}{^0}=U^0\cap W^0$.}
}\vspace{-14pt}\parSol{}
\!\!\!$\hText{$
	(a) $\varphi\in\BigPar{U+W}{^0}\Rightarrow\forall u\in U,w\in W,\\$
	\Ha$\varphi\Par{u}=\varphi\Par{w}=0\Rightarrow\varphi\in U^0\cap W^0.}\;\;\hMath{l}{\left|\;}{\;\right.}{U\subseteq U+W\Rightarrow\BigPar{U+W}{^0}\subseteq U^0\\W\subseteq U+W\Rightarrow\BigPar{U+W}{^0}\subseteq W^0}$\vspace{4pt}\parSol{}
(b) $\varphi\in U^0\cap W^0\subseteq V\apostrophe\Rightarrow\forall u\in U,w\in W,\varphi\Par{u+w}=0\Rightarrow\varphi\in \BigPar{U+W}{^0}.$\PfEnd
\SepLine

\ProblemN{\Anchor{3F37}{37}}{
	\TextA{Supp $U$ is a subsp of $V$ and $\pi$ is the quot map. Thus $\pi\apostrophe\in\Lm[\BigPar]{\Par{V\XSlash U}\apostrophe,V\apostrophe}$.\vspace{2pt}}
	\PrePa\TextA{Show $\pi\apostrophe$ is inje\hspace{1pt}$:${\tgnr\FontNorm\;\;Becs $\pi$ is surj. Use [3.108].}\vspace{2pt}}
	\PrePb\TextA{Show $\range \pi\apostrophe=U^0$\hspace{1pt}$:${\tgnr\FontNorm\;\;By {\tgnr[3.109](b)}, $\range\pi\apostrophe=\BigPar{\null\pi}{^0}=U^0.$}\vspace{2pt}}
	\PrePc\TextA{Conclude that $\pi\apostrophe$ is iso from $\Par{V\XSlash U}\apostrophe$ onto $U^0$\hspace{1pt}$:${\tgnr\FontNorm\;\;Immed.}}
}(a) \Or $\pi\apostrophe\Par{\varphi}=0\Longleftrightarrow\forall v\in V\,\BigPar{\,\forall v+U\in V\,},\varphi\BigPar{\pi\Par{v}}=\varphi\Par{v+U}=0\Longleftrightarrow \varphi=0.$\parSol{}
(b) \Or $\psi\in\range\pi\apostrophe\Longleftrightarrow\exists\,\varphi\in \Par{V\XSlash U}\apostrophe,\psi=\varphi\,\circ\,\pi\Longleftrightarrow\null\psi\supseteq U\Longleftrightarrow \psi\in U^0.$\PfEnd
\SepLine

\Anchor{3FN3.106}\ProblemB{
	\TextB{Supp $U$ is a subsp of $V.$ Prove $\Par{V\XSlash U}\apostrophe$ is iso to $U^0.$ \hfill\Sbra[3pt]{{\large\tgsc Another proof of \tgbfx[3.106]}}}
}Define $\xi:U^0\rightarrow\Par{V\XSlash U}\apostrophe$ by $\xi\Par{\varphi}=\tilde{\varphi},$ where $\tilde{\varphi}\in\Par{V\XSlash U}\apostrophe$ is defined by $\tilde{\varphi}\Par{v+U}=\varphi\Par{v}.$\vspace{1pt}\parSol{}
Inje: $\xi\Par{\varphi}=0=\tilde{\varphi}\Rightarrow\forall v\in V\,\BigPar{\,\forall v+U\in V\XSlash U\,},\tilde{\varphi}\Par{v+U}=\varphi\Par{v}=0\Rightarrow\varphi=0.$\parSol{}
Surj: $\varPhi\in\Par{V\XSlash U}\apostrophe\Rightarrow\forall u\in U,\varPhi\Par{u+U}=\varPhi\Par{0+U}=0\Rightarrow U\subseteq\Null\Par{\varPhi\circ\pi}\Rightarrow\xi\Par{\varPhi\circ\pi}=\varPhi.$\vspace{2pt}\parSol{}
\Or Define $\nu:\Par{V\XSlash U}\apostrophe\rightarrow U^0$ by $\nu\Par{\varPhi}=\varPhi\circ\pi.$ Now $\nu\circ\xi=I_{U^0},\;\xi\circ\nu=I_{\SmallPar[1pt]{V\XSlash U}\apostrophe}\Rightarrow\xi=\nu^{-1}.$\PfEnd\vspace{-2pt}
\SepLine

%\ProblemN{\Anchor{3F5}{5}}{
%	\TextA{Prove $\BigPar{V_{\!1}\times\dots\times V_{\!m}}\apostrophe$ and $\dualVn{1}\times\dots\times\dualVm$ are iso.\hfill\FontNorm\Sbra[3pt]{{\normalsize\tgnr Using notats in (3.E.2).}}}
%}Define $\chi:\Par{V_{\!1}\times\dots\times V_{\!m}}\apostrophe\rightarrow \dualVn{1}\times\dots\times \dualVm$ by $\chi\Par{T}=\Par{T\circ R_1,\dots,T\circ R_m}=\Par{R\apostrophe_1\Par{T},\dots,R\apostrophe_m\Par{T}}.$\parSol{}
%Define the $\chi^{-1}$ by $\chi^{-1}\Par{T_{\!1},\dots,T_{\!m}}=T_{\!1}S_1+\dots+T_{\!m}S_m=S\apostrophe_1\Par{T_{\!1}}+\dots+S\apostrophe_m\Par{T_{\!m}}.$\PfEnd
%\SepLine
\pagebreak

\ProblemN{\Anchor{3F23}{23}}{
	\TextA{Supp $U$ and $W$ are subsps of $V$. Prove $\BigPar{U\cap W}{^0}=U^0+W^0$.}
}\vspace{3pt}\par\quad
\!\!\!$\hText{$
	(a) $\varphi=\psi+\beta\in U^0+W^0\Rightarrow\forall v\in U\cap W,\\$
	\Ha $\varphi\Par{v}=\Par{\psi+\beta}\Par{v}=0\Rightarrow\varphi\in\BigPar{U\cap W}{^0}.
}\;\;\hMath{l}{\left|\;}{\right.}{$
	\!\Or\;$U\cap W\subseteq U\Rightarrow\BigPar{U\cap W}{^0}\supseteq U^0\\$
	\!\Blind{\Or\;}$U\cap W\subseteq W\Rightarrow\BigPar{U\cap W}{^0}\supseteq W^0}$\vspace{6pt}\par\quad
(b) \!\Sbra[3pt]{{\tgsl Req Finide}} \;By Exe (22), $\Dim\BigPar{U^0+W^0}=\dim V-\Dim\Par{U\cap W}.$\PfEnd\vspace{4pt}\quad\Hb
%\dim U^0+\dim W^0-\Dim\BigPar{U^0\cap W^0}$\par\vspace{0pt}\quad\Hb
%$=2\dim V-\dim U-\dim W-\BigPar{\!\dim V-\Dim\Par{U+W}}=
\Or Let $I=U\cap W.$ We show $\Par{U\cap W}{^0}\subseteq U^0+W^0.$\par\quad\Hb
Define $\chi\in\Lm{V\XSlash I,V\XSlash U\times V\XSlash W}$ by $\chi:v+I\mapsto\Par{v+U,v+W}.$\par\quad\Hb
Well-defined: $v_1+I=v_2+I\in V\XSlash I\Longleftrightarrow v_1-v_2\in I$\par\quad\Hb
\Blind{Well-defined:} $\Longleftrightarrow v_1-v_2\in U$ and $v_1-v_2\in W\Rightarrow\Par{v_1+U,v_1+W}=\Par{v_2+U,v_2+W}.$\vspace{2pt}\par\quad\Hb
Inje: $\Par{v+U,v+W}=0\Longleftrightarrow v\in U\cap W=I\Longleftrightarrow v+I=0.$\par\quad\Hb
Surj: $\forall v\in V$ suth $\Par{v+U,v+W}\in V\XSlash U\times V\XSlash W,$ becs $\emptySet\neq\Par{v+U}\cap\Par{v+W}=v+I\in V\XSlash I.$\par\quad\Hb
Thus $\chi\apostrophe\in\Lm[\BigPar]{\Par{{V\XSlash U}\times{V\XSlash W}}\apostrophe,\Par{V\XSlash I}\apostrophe}$ is iso. Now we find an iso of $U^0\times W^0$ onto $\Par{U\cap W}{^0}.$\par\quad\Hb
By (3.E.4), supp $\xi:\Par{V\XSlash U}\apostrophe\times\Par{V\XSlash W}\apostrophe\rightarrow\Par{{V\XSlash U}\times{V\XSlash W}}\apostrophe$ is iso.\par\quad\Hb
By (c) in Exe (37), supp $\Lambda_1:U^0\times W^0\rightarrow\Par{V\XSlash U}\apostrophe\times\Par{V\XSlash W}\apostrophe$ and $\Lambda_2:\Par{V\XSlash I}\apostrophe\rightarrow\Par{U\cap W}{^0}$ are isos.\par\quad\Hb
Hence $\Par{\Lambda_2\circ\chi\apostrophe\circ\xi\circ\Lambda_1}:U^0\times W^0\rightarrow\Par{U\cap W}{^0}$ is iso. \;Now we see how it works:\par\quad\Hb
$\forall\Par{\varphi_U,\varphi_W}\in U^0\times W^0,\;\null\pi_U\subseteq\null\varphi_U\Rightarrow\exists\,\psi_U\in\Par{V\XSlash U}\apostrophe,\;\psi_U\circ\pi_U=\varphi_U,$ simlr for $\varphi_W,$\par\quad\Hb
thus $\Lambda_1:\Par{\varphi_U,\varphi_W}\mapsto\Par{\psi_U,\psi_W}.$ Then $\xi:\Par{\psi_U,\psi_W}\mapsto\Par{\psi_US_U+\psi_WS_W},$ \Sbra{ See notats in (3.E.2). }\vspace{2pt}\par\quad\Hb
Now $\Par{\psi_US_U+\psi_WS_W}\mathop{\longmapsto}\limits^{\chi\apostrophe}\Par{\psi_US_U+\psi_WS_W}\circ\chi\mathop{\longmapsto}\limits^{\Lambda_2}\Par{\psi_US_U+\psi_WS_W}\circ\chi\circ\pi_I,$\par\quad\Hb
which sends $v$ to $\psi_U\Par{v+U}+\psi_W\Par{v+W}=\Par{\varphi_U+\varphi_W}\Par{v},$ which is $\varphi_U+\varphi_W.$\par\quad\Hb
Thus $\Par{\Lambda_2\circ\chi\apostrophe\circ\xi\circ\Lambda_1}$ is the surj $\Lambda:U^0\times W^0\rightarrow U^0+W^0$ defined in [3.77].\PfEnd\vspace{3pt}
\AExa Not true if $U$ or $W$ is merely a subset. Let $V=\FbbP{2},U=\Span{e_1},W=\Bra{\Par{1,1},\Par{0,1}}.$
\SepLine

\ProblemBX[]{\ACoro}{
	\TextB{$V=U\oplus W\Longleftrightarrow V\apostrophe=U^0\oplus W^0.$\vspace{-4pt}}
}\SepLine

\Anchor{3F'2}\ProblemB{
	\TextB{Supp $V=U\oplus W$. Define $\iota:V\rightarrow U$ by $\iota\Par{u+w}=u.$ Thus $\iota\apostrophe\in\Lm{U\apostrophe,V\apostrophe}.$\vspace{2pt}}
	\PrePa\TextB{Show $\null \iota\apostrophe=\zeroSubs$\hspace{1pt}$:${\tgnr\FontNorm\;\;$\null\iota\apostrophe=\Par{\range\iota}{^0_U}=U_U^0=\zeroSubs.$\;\Or $\iota\apostrophe\Par{\psi}=\psi\circ\iota=0\Longleftrightarrow U\subseteq\null\psi.$}\vspace{2pt}}
	\PrePb\TextB{Prove $\range\iota\apostrophe=W_V^0$\hspace{1pt}$:${\tgnr\FontNorm\;\;$\range\iota\apostrophe=\BigPar{\null\iota}{^0_V}=W_V^0.$ \:\:Now $\tilde{\iota\apostrophe}$ is iso from $U\apostrophe\XSlash{\envFontDefault\zeroSubs}$ onto $W^0.$}\vspace{0pt}}
}(b) \Or Note that $W=\null{\iota}\subseteq\Null\Par{\psi\circ\iota}.$ Then $\psi\circ\iota\in W^0\Rightarrow\range\iota\apostrophe\in W^0.$\parSol{\Hb}
\Blind{\Or}Supp $\varphi\in W^0.$ Becs $\null\iota=W\subseteq\null\varphi.$ By \Sbra{3.B \TIPSN{3}}, $\varphi=\varphi\circ\iota=\iota\apostrophe\Par{\varphi}.$\PfEnd
\SepLine

\Anchor{3F'3}\ProblemB{
	\TextB{Supp $V=U\oplus W.$ Prove $U^0=\Bra{\varphi\in V\apostrophe:\varphi=\varphi\circ\iota},$ \FontNorm where $\iota\in\Lm{V,W}:u_v+w_v\rightarrow w_v$.}
}$\varphi\in U^0\Longleftrightarrow U\subseteq\null\varphi\Longleftrightarrow\varphi=\varphi\circ\iota,$ by \Sbra{3.B \TIPSN{3}}.\PfEnd\vspace{3pt}
\Anchor{3FNdual}\ANote The notat $W_V\upapostrophe=\Bra{\varphi\in V\apostrophe:\varphi=\varphi\circ\iota}=U^0$ is not well-defined \Sbra{without a bss}.\parNot
Simply becs $W$ is not uniq. A bss of $V\apostrophe$ as precond would fix this. See {\NOTEFOR} Exe (31)\par\vspace{2pt}
\AExa Let $B_V=\Par{e_1,e_2}.$ Let $B_U=\Par{e_1},B_X=\Par{e_2-e_1},B_Y=\Par{e_2}.$\parExa
Then $\iota_X:ae_1+b\Par{e_2-e_1}\mapsto b\Par{e_2-e_1},\;\;\iota_Y:ae_1+be_2\mapsto be_2.$ Now by notat asum, $X_V\upapostrophe=Y_V\upapostrophe=U^0.$\parExa
Everything seems alright until you notice the following:\parExa
(1) For $V=U\oplus X,$ let $B_{U_V'}=\Par{\varphi}$ with $\varphi:e_1\mapsto 1,\;e_2-e_1\mapsto 0\Rightarrow e_2\mapsto 1.$ \hfill Now $X^0=U_V\upapostrophe.$\parExa
(2) For $V=U\oplus Y,$ let $B_{U_V'}=\Par{\psi}$ with $\psi:e_1\mapsto 1,e_2\mapsto 0.$ \hfill Now $Y^0=U_V\upapostrophe.$\parExa
Thus $X=Y,$ ctradic. But what if let $B_{V\apostrophe}=\Par{\beta_1,\beta_2}$ and thus fix '$B_{U_V'}=\Par{c_1\beta_1+c_2\beta_2}$'? \par\vspace{2pt}
\AComm {\tgsl\normalsize Supp $U$ is a subsp of $V.$ Then finding the corres subsp in $V\apostrophe$ req another 'half' $W\in\Scom{V}{U}$ to be uniq,}\vspace{-3pt}\parCom
{\tgsl\normalsize while finding the corres subsp of $V$ for a subsp of $V\apostrophe$ must have the another 'half' asumed as precond.}
\SepLine\pagebreak

\ProblemN{\Anchor{3F31}{31}}{
	\TextA{Supp $V$ is finide and $B_{V\apostrophe}=\Par{\varphi_1,\dots,\varphi_n}.$ Show $\exists\,!\,B_V$ whose dual bss is the $B_{V\apostrophe}$.}
}For each $k\in\;\!\!\Bra{1,\dots,n},$ let $\Gamma_{\!k}=\Bra{1,\dots,n}\Backslash[\Big]\Bra{k}.$ Let each $U_k=\bigcap_{j\,\in\,\Gamma_{\!k}}\null\varphi_j.$\parSol{}
By Exe (4E 23), $V\apostrophe=\Span{\varphi_1,\dots,\varphi_n}=\BigPar{\null\varphi_1\cap\cdots\cap\null\varphi_n}{^0}\Rightarrow U_k\cap\null\varphi_k=\zeroSubs.$\parSol{}
Thus $\forall x_k\in U_k\nonzero,\;x_k\not\in\null\varphi_k$ while $x_k\in\null\varphi_j$ for all $j\in\Gamma_{\!k}.$\parSol{}
Fix one $x_k$ and let $v_k=\Sbra{\varphi_k\Par{x_k}}{^{-1}}x_k\Rightarrow\varphi_k\Par{v_k}=1,\,\varphi_j\Par{v_k}=0$ for all $j\neq k.$\parSol{}
Simply for each $v_k,$ \,$\varphi_j\Par{v_k}=\delta_{j,k}$ for all $j\Longleftrightarrow$ for each $\varphi_j,$ $\varphi_j\Par{v_k}=\delta_{j,k}$ for all $k.$\parSol{}
又 $a_1v_1+\dots+a_nv_n=0\Rightarrow$ each $\varphi_k\Par{0}=a_k.$\vspace{2pt}\parSol{}
Now we prove the uniqnes part. Supp the dual bss of $B_V'=\Par{u_1,\dots,u_n}$ is the $B_{V\apostrophe}.$\parSol{}
For each $k,$ we have $\varphi_j\Par{v_k}=\varphi_j\Par{u_k}$ for all $k\Rightarrow v_k-u_k\in\bigcap_{j=1}^n\null\varphi_j=\zeroSubs.$\PfEnd
\SepLine

\Anchor{3FN31}\BulletPointX\NoteForSmall{Exe (31)}\;\;Supp $V$ is finide, and $\Omega$ is a subsp of $V\apostrophe$ with $B_{\Omega}=\Par{\varphi_1,\dots,\varphi_m}.$\TextB{}
The '$W$' is not clear when we are to find one suth $W_V\upapostrophe=\Omega,$ becs the another 'half' is undefined.\TextB{}
Extend to $B_{V\apostrophe}=\Par{\varphi_1,\dots,\varphi_n}.$ By Exe (31), $\exists\,!\,$corres $B_V=\Par{v_1,\dots,v_n}.$ Let $B_U=\Par{v_{m+1},\dots,v_n}.$\TextB{}
Let $B_W=\Par{v_1,\dots,v_m}.$ \,Thus $W_V\upapostrophe=\Omega.$ Now $W$ is well-defined with $B_V$ as precond.
\SepLine

\Anchor{3FN1}\BulletPointX\NoteForSmall{Exe (1)}\;\;Every liney functional is either surj or is a zero map.\TextB{}
Which means, for $\varphi\in V\apostrophe,$ $\varphi=0\Longleftrightarrow\dim\Span{\varphi}=0\Longleftrightarrow\dim\range\varphi=0.$\TextB{}
And $\varphi\neq 0\Longleftrightarrow\dim\Span{\varphi}=1\Longleftrightarrow\dim\range\varphi=1.$ Thus $\dim\Span{\varphi}=\dim\range\varphi.$\vspace{-2pt}
\SepLine

\Anchor{3F4e23}\ProblemBnoor{{4E 23}}{
	\TextB{Supp $V$ is finide, $\Omega=\Span{\varphi_1,\dots,\varphi_m}\subseteq V\apostrophe.$ Prove $\Omega=\BigPar{\null\varphi_1\cap\cdots\cap\null\varphi_m}{^0}.$}
}Becs each $\Span{\varphi_k}\subseteq\Par{\null\varphi_k}{^0}.$ By {\NOTEFOR} Exe (1) and Exe (23), Immed.\PfEnd\parSol{\vspace{4pt}}
\Or Reduce to $B_{\Omega}=\Par{\beta_1,\dots,\beta_p}.$ We show $\Omega=\Par{\null\beta_1\cap\cdots\cap\null\beta_p}{^0},$ then done by \TIPSN{2}.\parSol{}
Let $B_{V\apostrophe}=\Par{\beta_1,\dots,\beta_p,\gamma_1,\dots,\gamma_q}.$ By Exe (31), let $B_V=\Par{v_1,\dots,v_p,u_1,\dots,u_q}.$\parSol{}
Define each $\Gamma_{\!k}=\Bra{1,\dots,p}\Backslash[\Big]\Bra{k}.$ Then $\null\beta_k=\Span[\Bra]{v_j}{_{j\,\in\,\Gamma_{\!k}}}\oplus\Span{u_1,\dots,u_q}.$\parSol{}
Now $\Par{\null\beta_1\cap\cdots\cap\null\beta_p}=\Span{u_1,\dots,u_q}.$ Simlr to (4E 2.C.16).\parSol{}
Supp $\varphi=\sum_{i=1}^pa_i\beta_i+\sum_{j=1}^qb_j\gamma_j\in\Span{u_1,\dots,u_q}{^0}.$ Then each $\varphi\Par{u_k}=0=b_k$\parSol{}
Thus $\Span{u_1,\dots,u_q}{^0}\subseteq\Span{\beta_1,\dots,\beta_p}=\Omega.$\PfEnd
\SepLine

\Anchor{3FT2}\ProblemBX{\TipsN{2}}{
	\TextA{Supp each $\varphi_i,\beta_j\in\Lm{V,W}.$ Supp $\Span{\varphi_1,\dots,\varphi_m}=\Span{\beta_1,\dots,\beta_n}.$}
	\TextA{Prove $\null\varphi_1\cap\dots\cap\null\varphi_m=\null\beta_1\cap\dots\cap\null\beta_n.$\vspace{0pt}}
}Becs each $\beta_k\in\Span{\varphi_1,\dots,\varphi_m}.$\parSol{}
$\forall v\in\bigcap_1^m\null\varphi_I,\beta_k\Par{v}=0.$ Thus $\bigcap_1^m\null\varphi_I\subseteq\bigcap_1^n\null\beta_I.$ \;Rev the roles and done.\PfEnd\vspace{4pt}\parSol{}
\Or Supp $\Par{\varphi_1,\dots,\varphi_j}$ is a bss of $\Span{\varphi_1,\dots,\varphi_m}.$ Let $N_{k}\oplus\bigcap_1^{j}\null\varphi_I=\null\varphi_k.$\vspace{2pt}\parSol{}
Now $\bigcap_1^j\null\varphi_I\cap\BigPar{\null\varphi_k}=\bigcap_1^j\null\varphi_I.$ Thus $\bigcap_1^m\null\varphi_I=\bigcap_1^j\null\varphi_I.$\vspace{2pt}\parSol{}
又 $\beta_k\in\Span{\varphi_1,\dots,\varphi_j}.$ Let $M_k\oplus\bigcap_1^{j}\null\varphi_I=\null\beta_k.$ Simlr, $\bigcap_1^n\null\beta_I=\bigcap_1^j\null\varphi_I.$\PfEnd
\SepLine

\ProblemN[]{\Anchor{3F26}{26}}{
	Supp $V$ is finide, $\Omega$ is a subsp of $V\apostrophe.$ Then get a $B_{\Omega}$ and by {\TIPSN{1}} and Exe (4E 23), $\Omega=\BigPar{C^0\,\Omega}{^0}.$\TextB{}
}\AExa Immed, $\Omega\subseteq\Par{C^0\,\Omega}{^0}.$ Now we give a countexa for $\Omega\supseteq\Par{C^0\,\Omega}{^0}.$\parExa
Let $V=\Bra{\Par{x_1,x_2,\cdots}\in\FbbP{\infty}:x_k\neq 0\text{ for only finily many }k}.$ Then $V\apostrophe=\Par{\FbbP{\infty}}\apostrophe.$\parExa
Let $\Omega=\Bra[\envFontA]{\varphi\in\Span{\varphi_{\alpha_1},\dots,\varphi_{\alpha_m}}:\exists\:m,\alpha_k\in\Nbp}\subsetneq V\apostrophe.$ Then $C^0\,\Omega=\zeroSubs\Rightarrow\Par{C^0\,\Omega}{^0}=V\apostrophe.$\par\vspace{2pt}
\ACoro Supp $V$ is finide. For every subsp $\Omega$ of $V\apostrophe,\;\exists\,!\,$subsp $U$ of $V$ suth $\Omega=U^0.$\vspace{-2pt}\par
\SepLine\pagebreak

\Anchor{3F'4}\ProblemB{
	\TextB{Supp $\Span{\varphi_1,\dots,\varphi_m}\subseteq V\apostrophe.$ Let each $U_k\oplus\null\varphi_k=V.$}
	\TextB{Prove or give a countexa\hspace{1pt}$:$ $\Par{U_1+\dots+U_m}\oplus\BigPar{\null\varphi_1\cap\cdots\cap\null\varphi_m}=V.$}
}Let $V=\Rbb^2.$ Define $\varphi_1=\varphi_2:\Par{x,y}\mapsto x.$ Let $B_{U_1}=\Par{e_1},B_{U_2}=\Par{e_1+e_2}\Rightarrow U_1+U_2=V.$\parSol{}
\Or Let $B_{V\apostrophe}=\Par{\varphi_1,\varphi_2}$ be corres to the std bss. Let $B_{U_1}=B_{U_2}=\Par{e_1+e_2}\Rightarrow U_1+U_2\subsetneq V.$\PfEnd
\SepLine

\Anchor{3FT3}\ProblemBX[]{\TipsN{3}}{
	Let $B_{U^0}=\Par{\varphi_1,\dots,\varphi_m},B_{V\apostrophe}=\Par{\varphi_1,\dots,\varphi_n}\Rightarrow B_V=\Par{v_1,\dots,v_n}.$\TextA{}
	We show (a) $B_U=\Par{v_{m+1},\dots,v_n};$ \;(b) $U=\null\varphi_1\cap\cdots\cap\null\varphi_m.$\TextA{}
	(a) Becs $\Span{v_{m+1},\dots,v_n}{^0}=\Span{\varphi_1,\dots,\varphi_m}=U^0.$ Now by Exe (20, 21).\TextA{}
	\Ha\Or Becs by (b), $U=\bigcap_1^m\null\varphi_I=\Span{v_{m+1},\dots,v_n}.$\vspace{2pt}\TextA{}
	(b) Each $\null\varphi_k=\Span[\Bra]{B_V\Backslash\Bra[\envFontB]{v_k}}\Rightarrow\bigcap_1^m\null\varphi_I=\Span{v_{m+1},\dots,v_n}.$ Now by (a).\TextA{}
	\Hb\Or Becs $\Span{\varphi_1,\dots,\varphi_m}=U^0=\BigPar{\null\varphi_1\cap\cdots\cap\null\varphi_m}{^0}.$ Now by Exe (20, 21).\PfEnd\TextA{\vspace{-3pt}}
}\SepLine

\ProblemN{\Anchor{3F24}{24}}{
	\TextA{Prove, using the pattern of [3.104], that $\dim U+\dim U^0=\dim V$.}
}By \TIPSN{3}. \;\Or Let $B_U=\Par{u_1,\dots,u_m},B_V=\Par{u_1,\dots,u_m,v_1,\dots,v_n},B_{V\apostrophe}=\Par{\psi_1,\dots,\psi_m,\varphi_1,\dots,\varphi_n}.$\parSol{}
Supp $\psi=\sum_{i=1}^ma_i\psi_i+\sum_{j=1}^nb_j\varphi_j\in U^0\Rightarrow$ each $\psi\Par{u_k}=a_k=0.$ Thus $U^0\subseteq\Span{\varphi_1,\dots,\varphi_n}.$\PfEnd
\SepLine

\Anchor{3F28}\Anchor{3F29}\ProblemB[]{
	\TextB{Supp $T\in\Lm{V,W},$ each $\varphi_k\in V\apostrophe,$ and each $\psi_k\in W\apostrophe.$}
	\ProblemN[]{}{}
	\ProblemN[]{28}{
		\TextA{\;$\null T\apostrophe=\Span{\psi_1,\dots,\psi_m}\Longleftrightarrow\range T=\Par{\null\psi_1}\cap\cdots\cap\Par{\null\psi_m}.$}}
	\ProblemN[]{29}{
		\TextA{\;$\range T\apostrophe=\Span{\varphi_1,\dots,\varphi_m}\Longleftrightarrow\null T=\Par{\null\varphi_1}\cap\cdots\cap\Par{\null\varphi_m}.$}}
	\TextB{\vspace{-4pt}}
}%$\Par{\range T}{^0}=\null T\apostrophe=\Span{\psi_1,\dots,\psi_m}=\BigPar{{\null\psi_1}\cap\cdots\cap{\null\psi_m}}{^0}.$\parSol{}
%$\Par{\null T}{^0}=\range T\apostrophe=\Span{\varphi_1,\dots,\varphi_m}=\BigPar{{\null\varphi_1}\cap\cdots\cap{\null\varphi_m}}{^0}.$\PfEnd
\SepLine

\ProblemN{\Anchor{3F34}{34}}{
	\TextA{Define $\Lambda:V\rightarrow \FbbP{V\apostrophe}$ by $\Lambda v=\overline{v},$ and $\overline{v}:V\apostrophe\rightarrow\Fbb$ by $\overline{v}\Par{\varphi}=\varphi\Par{v}.$\vspace{2pt}}
	\PrePa\TextA{Show $\overline{v}\in V\apostrophe\apostrophe$ and $\Lambda\in\Lm{V,V\apostrophe\apostrophe}.$\vspace{2pt}}
	\PrePb\TextA{Show if $T\in\Lm{V}$, then $T\apostrophe\apostrophe\circ\Lambda=\Lambda\circ T$, where $T\apostrophe\apostrophe=\Par{T\apostrophe}\apostrophe$.\vspace{2pt}}
	\PrePc\TextA{Show if $V$ is finide, then {\tgsc $\Lambda$ is iso from $V$ onto $V\apostrophe\apostrophe$}.}
%	\TextA{\normalsize Supp $V$ is finide. Then $V$ and $V\apostrophe$ are iso, and finding iso from $V$ onto $V\apostrophe$ generally req choosing\vspace{-3pt}}
%	\TextA{\normalsize a bss of $V$. In contrast, the iso $\Lambda$ from $V$ onto $V\apostrophe\apostrophe$ does not req a choice of bss and thus is considered more natural.}
}(a) $\overline{v}\Par{\varphi+\lambda\psi}=\Par{\varphi+\lambda\psi}\Par{v}=\varphi\Par{v}+\lambda\psi\Par{v}=\overline{v}\Par{\varphi}+\lambda\overline{v}\Par{\psi}.$\parSol{\Ha}
$\overline{v+\lambda w}\Par{\varphi}=\varphi\Par{v+\lambda w}=\varphi\Par{v}+\lambda\varphi\Par{w}=\overline{v}\Par{\varphi}+\lambda\overline{w}\Par{\varphi}.$\vspace{2pt}\parSol{}
(b) $\BigPar{T\apostrophe\apostrophe\overline{v}}\Par{\varphi}=\BigPar{\overline{v}\circ{T\apostrophe}}\Par{\varphi}=\overline{v}\BigPar{T\apostrophe\Par{\varphi}}=\BigPar{T\apostrophe\Par{\varphi}}\Par{v}=\Par{\varphi\circ T}\Par{v}=\varphi\Par{Tv}=\overline{Tv}\Par{\varphi}.$\vspace{2pt}\parSol{}
(c) $\overline{v}=0\Rightarrow\forall\varphi\in V\apostrophe,\overline{v}\Par{\varphi}=\varphi\Par{v}=0\Rightarrow v=0.$ Inje. Now becs $V$ finide.\PfEnd
%\vspace{2pt}
%\AComm Supp $\Phi\in V\apostrophe\apostrophe$ and $\Phi\neq 0.$ Then $\exists\,\varphi\in V\apostrophe,\:\Phi\Par{\varphi}=1\Rightarrow\null\Phi\oplus\Span{\varphi}=V\apostrophe.$\parCom
%And $\varphi\neq 0\Rightarrow\exists\,v\in V,\:\varphi\Par{v}=1,\null\varphi\oplus\Span{v}=V.$ Becs $\Lambda$ is surj.\parCom
%Now $\exists\,x\in V,\forall\psi=c\varphi+\rho\in V\apostrophe,\psi\Par{x}=\overline{x}\Par{\psi}=\Phi\Par{\psi}=c.$
\SepLine

\ProblemN{\Anchor{3F36}{36}}{
	\TextA{Supp $U$ is a subsp of $V$. Define $i:U\rightarrow V$  by $i\Par{u}=u$. Thus $i\apostrophe\in\Lm{V\apostrophe,U\apostrophe}.$\vspace{2pt}}
	\PrePa\TextA{Show $\nullp i\apostrophe=U^0$\hspace{1pt}$:${\tgnr\FontNorm\;\;$\nullp i\apostrophe=\Par{\rangep i}{^0}=U^0\Leftarrow\rangep i=U$.}\vspace{2pt}}
	\PrePb\TextA{Prove $\rangep i\apostrophe=U\apostrophe$\hspace{1pt}$:${\tgnr\FontNorm\;\;$\rangep i\apostrophe=\Par{\nullp i}{_U^0}={\zeroSubs}{_U^0}=U\apostrophe$.}\vspace{2pt}}
	\PrePc\TextA{Prove $\tilde{i\apostrophe}$ is iso from $V\apostrophe\XSlash U^0$ onto $U\apostrophe$\hspace{1pt}$:${\tgnr\FontNorm\;\;Immed.}\vspace{1pt}}
}(a) \Or $\forall\varphi\in V\apostrophe,i\apostrophe\Par{\varphi}=\varphi\circ i=\varphi\mmid_U.$ Thus $i\apostrophe\Par{\varphi}=0\Longleftrightarrow\forall u\in U,\varphi\Par{u}=0\Longleftrightarrow\varphi\in U^0.$\parSol{}
(b) \Or Supp $\psi\in U\apostrophe.$ By (3.A.11), $\exists\,\varphi\in V\apostrophe,\varphi\mmid_U=\psi.$ Then $i\apostrophe\Par{\varphi}=\psi.$\PfEnd
\SepLine

\Anchor{3FN3.109b}\ProblemB{
	\TextB{Supp $T\in\Lm{V,W}.$ Prove $\range T\apostrophe\supseteq\BigPar{\null T}{^0}.$\hfill\Sbra[3pt]{{\tgsc\large Another proof of \tgnr\tgbfx[3.109](b)}}\vspace{1pt}}
}Let $V=U\oplus\null T.$ Let $R=\Par{T\mmid_U}{^{-1}}\Big|{_{\range T}}.$ Define $\iota\in\Lm{V,U}$ by $\iota\Par{u+w}=u.$\vspace{1pt}\parSol{}
$\forall\varPhi\in\BigPar{\null T}{^0},$ let $\psi=\varPhi\circ R,$ then $T\apostrophe\Par{\psi}=\psi\circ T=\varPhi\circ\BigPar{R\circ T\mmid_V}=\varPhi\circ\iota=\varPhi\in\range T\apostrophe.$\PfEnd\vspace{2pt}\Anchor{3F4e17}
\ACoro [3.108] and [3.110] hold without the hypo of finide. Now $T$ inv $\Longleftrightarrow T\apostrophe$ inv.
\SepLine\pagebreak

%\Anchor{3F12}\ProblemN[]{12}{
%	\TextA{{\large\tgnr Note that $I_{V}\upapostrophe,I_{V\apostrophe}:V\apostrophe\rightarrow V\apostrophe.$} \,For $\varphi\in V\apostrophe,\:I_{V\apostrophe}\Par{\varphi}=\varphi=\varphi\circ I_V=I_V\upapostrophe\Par{\varphi}.$ Thus $I_{V\apostrophe}=I_V\upapostrophe.$}
%}\SepLine

\ProblemN{\Anchor{3F15}{15}}{
	\TextA{Supp $T\in\Lm{V,W}$. Prove $T\apostrophe=0\Rightarrow T=0.$ \hfill\tgnr\FontNorm\ACoro If $V,W$ finide, then $\Gamma:T\mapsto T\apostrophe$ is iso.}
}Supp $T\apostrophe=0.$ Then $\null T\apostrophe=\zeroSubs=\Par{\range T}{^0}.$\PfEnd\parSol{}
\Or By Exe (25), $\range T=\Bra{w\in W:\varphi\Par{w}=0,\forall\varphi\in\Par{\range T}{^0}=\null T\apostrophe=W\apostrophe}=\zeroSubs.$
%Asum $w\neq 0$ suth $\forall\varphi\in W\apostrophe,\varphi\Par{w}=0.$ Let $U\oplus\Span{w}=W.$ \parSol{}
%Define $\psi\in W\apostrophe$ by $\psi\Par{u+\lambda w}=\lambda\Rightarrow\psi\Par{w}\neq 0.$ Ctradic. Now $\range T=\zeroSubs.$
\PfEnd
\SepLine

\BulletPointX{\FontSmall Let $B_V=\Par{v_1,\dots,v_n},B_{V\apostrophe}=\Par{\varphi_1,\dots,\varphi_n},B_W=\Par{w_1,\dots,w_m},B_{W\apostrophe}=\Par{\psi_1,\dots,\psi_m}.$}\par
\Anchor{3FT4}\BulletPointX\TipsN{4}\,\,\,Define $\Phi\in\Lm{V\apostrophe,V}:\varphi_k\mapsto v_k;\;\Psi\in\Lm{W,W\apostrophe}:w_j\mapsto\psi_j.$\TextB{}
\IndentTipsN{4}Define $T\in\Lm{V,W}$ suth $\Mt{T,B_V,B_W}=A.$ Let $S=\Phi T\apostrophe\Psi\Rightarrow\Mt{S,B_W,B_V}=A^t.$\vspace{2pt}\par
\Anchor{3FT5}\BulletPointX\TipsN{5}\,\,\,Define each $E_{j,k}\in\Lm{V,W}:v_x\mapsto\delta_{j,x}w_k,$ and each $\reflectbox{\textit{E}}{_{k,j}}\in\Lm{W\apostrophe,V\apostrophe}:\psi_x\mapsto\delta_{k,x}\varphi_j.$\TextB{}
\IndentTipsN{5}Note that each $E_{j,k}\!\!\upapostrophe\,\,\Par{\psi_x}=\psi_x\circ E_{j,k}=\delta_{k,x}\varphi_j=\reflectbox{\textit{E}}{_{k,j}}\Par{\psi_x}\Rightarrow E_{j,k}\!\!\upapostrophe\,\,=\reflectbox{\textit{E}}{_{k,j}}.$\vspace{1pt}\TextB{}
\IndentTipsN{5}$\Lm{V,W}\ni \sum_{j=1}^n\sum_{k=1}^mA_{k,j}E_{j,k}\Longleftrightarrow\sum_{j=1}^n\sum_{k=1}^mA_{k,j}\reflectbox{\textit{E}}{_{k,j}}\in\Lm{W\apostrophe,V\apostrophe}.$ Uniqly by Exe (16).%\vspace{3pt}\Anchor{5E5}\TextB{}
%\ACoro $ST=TS\Longleftrightarrow S\apostrophe T\apostrophe=T\apostrophe S\apostrophe.$ \;By Exe (16). \Or Becs $AC=CA\Longleftrightarrow A^tC^t=C^tA^t.$
\SepLine

%\Anchor{3F4e8}\ProblemBnoor{{4E 8}}{
%	\TextB{Describe the relation of $B_V=\Par{v_1,\dots,v_{n}}$ and the corres $B_{V\apostrophe}=\Par{\varphi_1,\dots,\varphi_{n}}$ using isos.}
%}Define $\Gamma:V\rightarrow\FbbP{n}$ by $\Gamma\Par{v}=\BigPar{\varphi_1\Par{v},\dots,\varphi_{n}\Par{v}},$ and
%$\Gamma^{-1}\Par{a_1,\dots,a_n}=a_1v_1+\dots+a_nv_n.$\PfEnd\parSol{}
%\SepLine

\ProblemN{\Anchor{3F4e24}\Anchor{3F6}{6}}{
	\TextA{Define \,$\Gamma: V\apostrophe\rightarrow\FbbP{m}\,$ by $\,\Gamma\Par{\varphi}=\BigPar{\varphi\Par{v_1},\dots,\varphi\Par{v_m}},$ where $v_1,\dots,v_m\in V$.\vspace{2pt}}
	\TextA{Show {\tgnr\large(a)} $\Span{v_1,\dots,v_m}=V\Longleftrightarrow\Gamma$ inje. \, {\tgnr\large(b)} $\Par{v_1,\dots,v_m}$ liney indep $\Longleftrightarrow\Gamma$ surj.}
}Let $\Par{e_1,\dots,e_m}$ be the std bss of $\FbbP{m}.$\par\quad
(a) Becs $\Gamma\Par{\varphi}=0\Longleftrightarrow\varphi\Par{v_1}=\dots=\varphi\Par{v_m}=0\Longleftrightarrow\null\varphi=\Span{v_1,\dots,v_m}.$ Immed.\vspace{1pt}\par\quad
(b) Supp $\Gamma$ is surj. Let each $e_k=\Gamma\Par{\varphi_k}\Rightarrow\varphi_k\Par{v_j}=\delta_{j,k}.$ Now $a_1v_1+\dots+a_mv_m=0\Rightarrow$ each $a_k=\varphi_k\Par{0}.$\vspace{1pt}\par\quad\Hb
Supp $\Par{v_1,\dots,v_m}$ is liney indep. Let $U=\Span{v_1,\dots,v_m},B_{U\apostrophe}=\Par{\psi_1,\dots,\psi_m}.$ Let $W\oplus U=V.$\par\quad\Hb
Define $\iota:u_v+w_v\mapsto u_v.$ Each $\psi_k\circ\iota=\varphi_k\in V\apostrophe\Rightarrow\varphi_k\Par{v_j}=\psi_k\Par{v_j}=\delta_{j,k}\Rightarrow$ each $e_k=\Gamma\Par{\varphi_k}.$\PfEnd\vspace{4pt}\quad
\Or Let $\Par{\psi_1,\dots,\psi_m}$ be dual bss of the std bss of $\FbbP{m}.$ Define an iso $\Psi:\FbbP{m}\rightarrow\Par{\FbbP{m}}\apostrophe$ by $\Psi\Par{e_k}=\psi_k.$\par\quad
Define $T\in\Lm{\FbbP{m},V}$ by $Te_k=v_k.$ Now $T\Par{x_1,\dots,x_m}=x_1v_1+\dots+x_mv_m.$\par\quad
$\forall\varphi\in V\apostrophe,k\in\;\!\!\Bra{1,\dots,m},\Sbra{T\apostrophe\Par{\varphi}}\Par{e_k}=\varphi\Par{Te_k}=\varphi\Par{v_k}=\Sbra{\varphi\Par{v_1}\psi_1+\dots+\varphi\Par{v_m}\psi_m}\Par{e_k}$\par\quad
Now $T\apostrophe\Par{\varphi}=\varphi\Par{v_1}\psi_1+\dots+\varphi\Par{v_m}\psi_m=\Psi\BigPar{\Gamma\Par{\varphi}}.$ Hence $T\apostrophe=\Psi\circ\Gamma.$\par\quad
By (3.B.3),
(a) $\range T=\Span{v_1,\dots,v_m}=V\Longleftrightarrow T\apostrophe$ inje $\Longleftrightarrow\Gamma$ inje.\par\quad
\Blind{By (3.B.3),} (b) $\Par{v_1,\dots,v_m}$ is liney indep $\Longleftrightarrow T$ is inje $\Longleftrightarrow T\apostrophe$ surj $\Longleftrightarrow\Gamma$ surj.\PfEnd
\SepLine

\Anchor{3F4e25}\ProblemBnoor{{4E 25}}{
	\TextB{Define \,$\Gamma: V\rightarrow\FbbP{m}\,$ by $\,\Gamma\Par{v}=\BigPar{\varphi_1\Par{v},\dots,\varphi_m\Par{v}},$ where $\varphi_1,\dots,\varphi_m\in V\apostrophe.$\vspace{2pt}}
	\TextB{Show {\tgnr\large(c)} $\Span{\varphi_1,\dots,\varphi_m}=V\apostrophe\Longleftrightarrow \Gamma$ inje. \, {\tgnr\large(d)} $\Par{\varphi_1,\dots,\varphi_m}$ liney indep $\Longleftrightarrow \Gamma$ surj.}
}Let $\Par{e_1,\dots,e_m}$ be the std bss of $\FbbP{m}.$\par\quad
(c) Becs $\Gamma\Par{v}=0\Longleftrightarrow\varphi_1\Par{v}=\dots=\varphi_m\Par{v}=0\Longleftrightarrow v\in\Par{\null \varphi_1}\cap\dots\cap\Par{\null\varphi_m}.$\par\quad\Hc
By Exe (4E 23), $\Span{\varphi_1,\dots,\varphi_m}=V\apostrophe\Longleftrightarrow\null\Gamma=\Par{\null \varphi_1}\cap\dots\cap\Par{\null\varphi_m}=\zeroSubs.$\par\quad
(d) Supp $\Par{\varphi_1,\dots,\varphi_m}$ is liney indep. \Sbra[3pt]{{\tgsl Req Finide}} \;Extend to $B_{V\apostrophe}=\Par{\varphi_1,\dots,\varphi_n}.$\par\quad\Hb
Then by Exe (31), $B_V=\Par{v_1,\dots,v_n}$ and each $\varphi_k\Par{v_j}=\delta_{j,k}\Rightarrow$ each $e_k=\Gamma\Par{\varphi_k}.$\par\quad\Hd
Convly, let each $v_k$ be suth $e_k=\Gamma\Par{v_k}=\BigPar{\varphi_1\Par{v_k},\dots,\varphi_m\Par{v_k}}.$ If $a_1\varphi_1+\dots+a_m\varphi_m=0.$ Immed.\par\quad\Hd
\Or Let $U=\Span{v_1,\dots,v_m}.$ Then $B_{U\apostrophe}=\BigPar{\varphi_1\Big|{_U},\dots,\varphi_m\Big|{_U}}\Rightarrow\Par{\varphi_1,\dots,\varphi_m}$ liney indep.\PfEnd\vspace{4pt}\quad
\Or Let $\Par{\psi_1,\dots,\psi_m}$ be dual bss of the std bss of $\FbbP{m}.$ Define an iso $\Psi:\FbbP{m}\rightarrow\Par{\FbbP{m}}\apostrophe$ by $\Psi\Par{e_k}=\psi_k.$\par\quad
$\forall\Par{x_1,\dots,x_m}\in\FbbP{m},\Gamma\apostrophe\BigPar{\Psi\Par{x_1,\dots,x_m}}=x_1\varphi_1+\dots+x_m\varphi_m.$ Define $\Phi=\Gamma\apostrophe\circ\Psi.$ Thus by (3.B.3),\par\quad
(c) $\Gamma$ inje $\Longleftrightarrow\Gamma\apostrophe$ surj $\Longleftrightarrow\Phi$ surj $\Longleftrightarrow\Par{\varphi_1,\dots,\varphi_m}$ spanning $V\apostrophe.$ \;Simlr for (d).\PfEnd
\SepLine

\ProblemN{\Anchor{3F9}{9}}{
	\TextA{Show $\forall\psi\in V\apostrophe,\psi=\psi\Par{v_1}\;\!\varphi_1+\dots+\psi\Par{v_n}\;\!\varphi_n,$ \FontNorm where $B_V=\Par{v_1,\dots,v_n},B_{V\apostrophe}=\Par{\varphi_1,\cdots,\varphi_n}.$}
}$\psi\Par{v}=a_1\psi\Par{v_1}+\dots+a_n\psi\Par{v_n}=\psi\Par{v_1}\;\!\varphi_1\Par{v}+\dots+\psi\Par{v_n}\;\!\varphi_n\Par{v}.$\PfEnd
\SepLine

%\Anchor{3F13}\ProblemN[]{13}{
%	\TextA{Define $T:\Rbb^3\!\rightarrow\!\Rbb^2$ by $T\Par{x,y,z}=\Par{4x+5y+6z,7x+8y+9z}$.\vspace{3pt}}
%	\TextA{Let $\Par{\varphi_1,\varphi_2},\Par{\psi_1,\psi_2,\psi_3}$ denote the dual bss of std bss of $\Rbb^2$ and $\Rbb^3$.\vspace{4pt}}
%	\PrePa\TextA{Describe the liney functionals $T\apostrophe\Par{\varphi_1},T\apostrophe\Par{\varphi_2}.$}
%	\Blind{\PrePb}\TextA{{\FontNorm\tgnr For any $\Par{x,y,z}\in\Rbb^3$, $\BigPar{T\apostrophe\Par{\varphi_1}}\Par{x,y,z}=4x+5y+6z, \BigPar{T\apostrophe\Par{\varphi_2}}\Par{x,y,z}=7x+8y+9z$.}\vspace{6pt}}
%	\PrePb\TextA{Write $T\apostrophe\Par{\varphi_1}$ and $T\apostrophe\Par{\varphi_2}$ as liney combinas of $\psi_1,\psi_2,\psi_3$.}
%	\Blind{\PrePb}\TextA{{\FontNorm\tgnr$T\apostrophe\Par{\varphi_1}=4\psi_1+5\psi_2+6\psi_3,\,\,T\apostrophe\Par{\varphi_2}=7\psi_1+8\psi_2+9\psi_3.$}\vspace{6pt}}
%	\PrePc\TextA{What is $\null T\apostrophe$? What is $\range T\apostrophe$?}
%}\TextA{\vspace{4pt}}
%\Hc $T\Par{x,y,z}=0\Longleftrightarrow\MathLeftBrace{l}{4x+5y+6z=0\\7x+8y+9z=0}\Longleftrightarrow\hMath{l}{\left\{}{\;\right|}{x=z,\\y=-2z.}\,\hText{$Thus $\null T=\Span{e_1-2e_2+e_3},\\$where $\Par{e_1,e_2,e_3}$ is std bss of $\Rbb^3.}$\vspace{4pt}\TextA{}
%\Hc Let $\Par{e_1-2e_2+e_3,-2e_2,e_3}$ be a bss, with corres dual bss $\Par{\varepsilon_1,\varepsilon_2,\varepsilon_3}$.\vspace{1.5pt}\TextA{}
%\Hc Thus $\Span{e_1-2e_2+e_3}=\null T\Rightarrow \Span{e_1-2e_2+e_3}{^0}=\Span{\varepsilon_2,\varepsilon_3}=\range T\apostrophe.$\vspace{1.5pt}\TextA{}
%\Hc Note that $\varepsilon_k=\varepsilon_k\Par{e_1}\psi_1+\varepsilon_k\Par{e_2}\psi_2+\varepsilon_k\Par{e_3}\psi_3.$\vspace{1.5pt}\TextA{}
%\Hc And $\MathLeftMid{l}{\varepsilon_2\Par{e_2}=-\frac{\;1\;}{2},\varepsilon_2\Par{e_1}=\varepsilon_2\Par{e_1-2e_2+e_3}+\varepsilon_2\Par{2e_2}-\varepsilon_2\Par{e_3}=1,\\[1.5pt]\varepsilon_3\Par{e_2}=0,\varepsilon_3\Par{e_3}=\varepsilon_3\Par{e_1-2e_2+e_3}+\varepsilon_3\Par{2e_2}-\varepsilon_3\Par{e_3}=-1.}$\vspace{1.5pt}\TextA{}
%\Hc Hence $\varepsilon_2=\psi_1-\frac{\;1\;}{2}\psi_2,\;\varepsilon_3=-\psi_1+\psi_3.$ Now $\range T\apostrophe=\Span{\psi_1-\frac{\;1\;}{2}\psi_2,\:-\psi_1+\psi_3}.$\vspace{3pt}\TextA{}
%\Hc\Or $\range T\apostrophe=\Span[\BigPar]{T\apostrophe\Par{\varphi_1},\,T\apostrophe\Par{\varphi_2}}=\Span{4\psi_1+5\psi_2+6\psi_3,\:7\psi_1+8\psi_2+9\psi_3}.$\vspace{6pt}\TextA{}
%\Hc Supp $T\apostrophe\Par{x\varphi_1+y\varphi_2}=\Par{4x+7y}\psi_1+\Par{5x+8y}\psi_2+\Par{6x+9y}\psi_3=0.$\TextA{}
%\Hc Then $x+y=4x+7y=x=y=0.$ Hence $\null T\apostrophe=\zeroSubs.$\vspace{3pt}\TextA{}
%\Hc\Or $\null T=\Span{e_1-2e_2+e_3}\Rightarrow V=\Span{{-2e_2,\:e_3}}\oplus\null T.$\vspace{1pt}\TextA{}
%\Hc$\Rightarrow \range T=\Bra{Tx:x\in\Span{{-2e_2,\:e_3}}}=\Span[\BigPar]{T\Par{{-2e_2}},\,T\Par{e_3}}$\vspace{1pt}\TextA{}
%\Hc$=\Span{-10f_1-16f_2,\:6f_1+9f_2}=\Span{f_1,\,f_2}=\Rbb^2.$ Now $\null T\apostrophe=\Par{\range T}{^0}=\zeroSubs.$\vspace{4pt}\TextA{}
%\Hc\Or For any $A,B\in\Rbb,$ asum $\Par{x,y,z}$ is suth $A=4x+5y+6z,\,B=7x+8y+9z.$\TextA{}
%\Hc By computing $x=z+4\big/3\Par{B-A},\;y=-2z+\Par{7A-4B}\XSlash 3,\;z=z.$ \;\; {\tgsl\FontSmall An exa for (4E 3.E.8).}\TextA{}
%\Hc Hence $\Par{x,y,z}$ exis $\Rightarrow\Par{A,B}\in\range T.$ \,Now $T$ surj $\Rightarrow T\apostrophe$ inje.\PfEnd
%\SepLine

%\Anchor{3F14}\ProblemN[]{14}{
%		\TextA{Define $T:\PoRi\rightarrow\PoRi$ by $\Par{Tp}\Par{x}=x^2 p\Par{x}+p\apostrophe\apostrophe\Par{x}$ for each $x\in\Rbb.$\vspace{4pt}}
%		(a) \TextA{Supp $\varphi\in\PoRi\apostrophe$ is defined by $\varphi\Par{p}=p\apostrophe\Par{4}$. Describe $T\apostrophe\Par{\varphi}\in\PoRi\apostrophe$.\vspace{4pt}}
%		\Ha\TextA{\FontNorm$\BigPar{T\apostrophe\Par{\varphi}}\Par{p}=\Sbra{x^2p\Par{x}+p\apostrophe\apostrophe\Par{x}}\apostrophe\Par{4}=\Sbra{2xp\Par{x}+x^2p\apostrophe\Par{x}+p\apostrophe\apostrophe\apostrophe\Par{x}}\Par{4}=8p\Par{4}+16p\apostrophe\Par{4}+p\apostrophe\apostrophe\apostrophe\Par{4}$.\vspace{8pt}}
%		(b) \TextA{Supp $\varphi\in\PoRi\apostrophe$ is defined by $\varphi\Par{p}=\int_0^1 p\Par{x}\d x$. Evaluate $\BigPar{T\apostrophe\Par{\varphi}}\Par{x^3}$.\vspace{4pt}}
%		\Hb\TextA{\FontNorm $\BigPar{T\apostrophe\Par{\varphi}}\Par{x^3}=\int_0^1\Par{x^5+6x}\d x=\int_0^1\XPar{\frac{1}{6}x^6+3x^2}\apostrophe\d x=\frac{19}{6}.$\PfEnd}
%	}\SepLine

\ChEnd\pagebreak

\ChDecl{}{\Largebfx{I Will Never Forget{\,}——{\,}In 3.E, naive but fantastical:}}

\vspace{4pt}

\BulletPointX\TipsN{1}\,\,\,{Supp $U$ is a subsp of $V.$ \uline{Define $S\in\Lm{V\XSlash U,V}$ by $S\Par{v+U}=v.$} \quad{\tgsc Badly defined.}}\TextB{}
{Then $\range S$ is the {\tgsl purest} in $\Scom{V}{U}.$ Now $\null S=\zeroSubs,\;U\oplus\range S=V.$}\TextB{}
{\uline{Let $E=S\circ\pi.$} Becs $S$ is inje and $\pi$ is surj, $\null E=\null\pi=U,\;\range E=\range S.$}\TextB{}
{Then $\range E\oplus\null E=V.$ \;\NOTICE that $E:V\rightarrow W$ is the {\tgsl purest eraser}. Now we explain why:}\TextB{\vspace{2pt}}
\AExa {Let $V=\Fbb^2,B_U=\Par{e_1},B_W=\Par{e_2-e_1}\Rightarrow U\oplus W=V.$}\parExa{\IndentB}
{Notice that $T\Par{e_2-e_1}=\Par{e_2-e_1},$ while $\Par{e_2-e_1}+U=e_2+U,$ but}\parExa{\IndentB}
{becs $e_2=e_1+\Par{e_2-e_1},$ now still, $\tilde{T}\BigPar{\Par{e_2-e_1}+U}=e_2-e_1=Te_2.$}\parExa{\IndentB}
{In contrast, $S\BigPar{\Par{e_2-e_1}+U}=S\Par{e_2+U}=e_2,\;E\Par{e_2-e_1}=e_2.$}\parExa{\IndentB}
{And $\range E=\range S=\Span{e_2}$ is the {\tgsl purest} in $\Scom{V}{U}.$}
\SepLine

\BulletPointX\NoteForSmall{Exe (13) and (4E 14)}\;\;Let $U\oplus W=V.$ Define $S\Par{w+U}=w.$ \;\Sbra{ See also \TIPSN{1}. }\TextB{}
(a) Let $B_W=\Par{w_1,\dots,w_m}\Rightarrow B_{V\XSlash U}=\BigPar{w_1+U,\dots,w_m+U}.$ Then $S\Par{w_k+U}$ might not equal $w_k.$\TextB{}
(b) Let $B_{V\XSlash U}=\BigPar{w_1+U,\dots,w_m+U},$ then let $B_W=\Par{w_1,\dots,w_m}.$ Now each $S\Par{w_k+U}=w_k.$
\def\Pure{{\textup{\tgnr Pure}}\,}
\par\vspace{2pt} \BulletPointX\NewNotation\;$\Pure V\XSlash U=W\Longleftrightarrow V=U\oplus W,\;W=\range S.$ {\FontSmall\tgsl The uniqnes of $\Pure V\XSlash U$ follows from $\range S.$}
\SepLine

\ProblemB{
	\TipsN{3}\,\,\,\TextB{Supp $I$ is a subsp of $U.$ Supp $U$ is a subsp of $V.$}
	\IndentTipsN{3}\TextB{Let $V=S_V I\oplus I=S_V U\oplus U.$ \;Let $U=S_U I\oplus I.$ \;Then $V=S_V U\oplus S_U I\oplus I.$\vspace{2pt}}
	\IndentTipsN{3}\TextB{Supp $S_V I=\Pure V\!\!\:\XSlash I,$ simlr for $S_V U,S_U I.$ \;Prove $S_V I=S_V U\oplus S_U I.$}
}$\forall v_i\in S_V I,\,v_i=v_u+u,\exists\,!\,v_u\in S_V U,u\in U\Rightarrow\exists\,!\,u_i\in S_U I,i\in I,v_i=v_u+u_i+i.$\parSol{}
又 $v_i\in\Pure V\!\!\:\XSlash I.$ Hence $i=0,$ and $v_i\in S_V U\oplus S_U I.$ Now becs $S_V U,S_U I\subseteq S_V I.$\PfEnd
\SepLine

\ProblemB{
	\TipsN{4}\,\,\,\TextB{Supp $T\in\Lm{V,W}$ is inv, $V=M\oplus N,\:\range T\mmid_M=X,\range T\mmid_N=Y.$}
	\IndentTipsN{4}\TextB{Then $W=\range T=X\oplus Y.$ Show $\Pure V\!\!\;\XSlash M=N\Longleftrightarrow\Pure W\XSlash X=Y.$}
}Define $S_1:V\XSlash M\rightarrow V$ and $S_2:W\XSlash X\rightarrow W$ by $S_1:v+M\mapsto v$ and $S_2:w+X\mapsto w.$\par\quad
Now $v=v_m+v_n\Longleftrightarrow Tv=Tv_m+Tv_n.$ We show $S_1\Par{v+M}=v_n\Longleftrightarrow S_2\Par{Tv+X}=Tv_n.$\par\quad
(a) Supp $S_2\Par{Tv_n+X}=Tv_n'.$ Then $Tv_n-Tv_n'\in X\Longleftrightarrow v_n-v_n'=v_m'\in M,\,v=v_n'+\Par{v_m'+v_m}.$\par\quad
(b) Supp $S_1\Par{v_n+M}=v_n'.$ Then $v_n-v_n'\in M\Longleftrightarrow Tv_n-Tv_n'=Tv_m'\in X,\,Tv=Tv_n'+\Par{Tv_m'+Tv_m}.$\vspace{2pt}\par\quad
Hence by def, $\Pure V\!\!\;\XSlash M=\range S_1=N\Longleftrightarrow\Pure W\XSlash X=\range S_2=Y.$\PfEnd
\SepLine

\ProblemN{3.F.23}{
	\TextA{Supp $U$ and $W$ are subsps of $V$. Prove $\BigPar{U\cap W}{^0}\subseteq U^0+W^0$.}
}Let $I=U\cap W.$ \,Using \Sbra{3.E \TIPSN{3}}. Supp $\varphi\in I^0.$\vspace{-10pt}\parSol{}
\!$\hText{$Now $S_V I=S_V U\oplus S_U I=S_V W\oplus S_W I.\\$Let $\Span{x}=\Pure V\!\!\:\XSlash\null\varphi.$ If $x=0$ then done. Supp $0\neq x\in S_V I.\\\;\\\;}\hfill\hText{$\includegraphics[width=80pt]{diagram3F-1}$\\\;\\\;}$\vspace{-42pt}\parSol{}
Now $\exists\,!\,\Par{u_v,i_u,w_v,i_w}\in S_V U\times S_U I\times S_V W\times S_W I,\,x=u_v+i_u=w_v+i_w.$\parSol{}
Define $\alpha\in U^0,\beta\in W^0$ by $\alpha:u_v\mapsto\varphi\Par{u_v},u\mapsto 0,$ and $\beta:i_u\mapsto\varphi\Par{i_u},i\mapsto 0,$\parSol{}
for all $u\in\Pure V\!\!\:\XSlash\Span{u_v}$ and $i\in\Pure V\!\!\:\XSlash\Span{i_u}.$ \OR Define $\rho\in W^0,\gamma\in U^0,$ simlr.\parSol{}
Then $\varphi=\alpha+\beta=\rho+\gamma\in U^0+W^0.$\PfEnd
\SepLine
\ChEnd\pagebreak

\ChDecl{}{\Largebfx{Exes about Sequences and Number Theory before Chapter 4}}

\vspace{4pt}

\Anchor{2A16}\ProblemBnoor{2.A.16}{
	\TextA{Prove the vecsp $U$ of all continuous functions in $\Rbb^{[0,1]}$ is infinide.}
}By \Sbra{3.A {\NOTEFOR} $\FbbP{S}$}, immed.\PfEnd\vspace{2pt}\parSol{}
\Or Fix one $m\in\Nbp$ and $p\Par{x}=a_0+a_1x+\dots+a_mx^m$ for $x\in\;\!\!\Interval{[}{]}{0,1}.$\parSol{}
Then $p$ has infily many roots and hence each coeff is zero, othws $\deg p\geqslant 0,$ ctradic [4.12].\parSol{}
Thus $\Par{1,x,\dots,x^m}$ is liney indep in $\Rbb^{[0,1]}.$ Simlr to [2.16], $U$ is infinide.\PfEnd\vspace{8pt}\parSol{}
\Or Note that\; $\Frac{\;1\;}{1}>\Frac{\;1\;}{2}>\dots>\Frac{\;1\;}{m},\,\,\,\forall m\in\Nbp.$ Supp\; $f_m=\MathLeftBrace{l}{
	\!x-{}${\Large$\frac{\;1\;}{m}$}$,\;\;x\in\;\!\!\Interval{(}{]}{${\Large$\frac{\;1\;}{m}$}$,\,1\,}\\
	\!0,\hfill x\in\;\!\!\Sbra{\,0,\,${\Large$\frac{\;1\;}{m}$}$\,}
}$\vspace{2pt}\parSol{}
Then\; $f_1\XPar[0pt]{${\Large$\frac{\;1\;}{m}$}$}=\dots=f_m\XPar[0pt]{${\Large$\frac{\;1\;}{m}$}$}=0\neq\,f_{m+1}\XPar[0pt]{${\Large$\frac{\;1\;}{m}$}$}.$ 
\;Hence $f_{m+1}\not\in\Span{\:\!f_1,\dots,f_m}.$ By (2.A.14).\PfEnd
\SepLine

\Anchor{3F35}\ProblemBnoor{3.F.35}{
	\TextA{Prove $\BigPar{\PoFi}\apostrophe$ is iso to $\FbbP{\infty}.$}
}Define $\theta\in\Lm[\Sbra]{\BigPar{\PoFi}\apostrophe,\FbbP{\infty}}$ by $\theta\Par{\varphi}=\BigPar{\varphi\Par{1},\varphi\Par{z},\cdots,\varphi\Par{z^m},\cdots}.$\parSol{}
\NOTICE that $\forall p\in\PoRi,\exists\,!\,c_i\in\Fbb,m=\deg p,\;p\Par{z}=c_0+c_1z+\dots+c_{m}z^{m}\in\PoF{m}.$\vspace{1pt}\parSol{}
Inje: $\theta\Par{\varphi}=0\Rightarrow\forall p\in\PoFi{},\varphi\Par{p}=c_0\varphi\Par{1}+c_1\varphi\Par{z}+\dots+c_m\varphi\Par{z^m}=0.$\vspace{1pt}\parSol{}
Surj: Supp $x=\Par{x_0,x_1,\cdots}\in\FbbP{\infty}.$ Define $\psi_x\Par{p}=x_0c_0+\dots+x_mc_m\Rightarrow$ each $\psi_x\BigPar{z^k}=x_k.$\parSol{}
\Blind{Surj:} $\forall p,q\in\PoFi{},$ supp $\deg p=m\geqslant n=\deg q,$ \Sbra{{\tgsl which is why we do not write $\Par{p+\lambda q}.$}}\parSol{}
\Blind{Surj:} $\psi_x\Par{\lambda p+\mu q}=\sum_{j=0}^nx_j\Par{\lambda a_j+\mu b_j}+\sum_{k=1}^{m-n}x_{n+k}\lambda a_{n+k}=\lambda\psi_x\Par{p}+\mu\psi_x\Par{q}.$\PfEnd\vspace{4pt}
\AComm $\PoFi,\FbbP{\infty}$ not iso $\Longrightarrow\PoFi,\BigPar{\PoFi}\apostrophe$ not iso. But $\PoFi$ is iso to $\FbbP{\Nbb},$ {\tgsl see the '$U$' in (3.E.14).}
\SepLine

\Anchor{3E14}\ProblemBnoor{3.E.14}{
	\TextA{Supp $U=\Bra{\Par{x_1,x_2,\cdots}\in\FbbP{\infty}:x_k\neq 0\text{\;for\;only\;finily\;many}\;k}.$\quad\FontSmall Denote it by $\FbbP{\Nbb}.$\vspace{2pt}}
	\PrePa\TextA{Show $U$ is a subsp of $\FbbP{\infty}$. {\FontNorm\Sbra{Do it in your mind}} \;\; {\tgnr\large(b)} Prove $\FbbP{\infty}\XSlash[-3pt]U$ is infinide.}
}{\tgsl\FontSmall For ease of nota, denote the $p^\text{th}$ term of $u=\Par{x_1,\cdots,x_p,\cdots}\in\FbbP{\infty}$ by $u\Sbra{p}$.}\vspace{0pt}\par\quad
For each $r\in\Nbp,$ let $\;e_r\Sbra{k}={}${\FontSmall$\hMath[0pt]{l}{\left\{\hspace{-2pt}}{\right|}{1\,,\;\Par{k-1}\equiv 0\,\BigPar{\text{mod}\,r}\\0\,,\;\text{othws}}$} \;simply \,$e_r=\BigPar{1,\underbrace{0,\cdots,0}_{\SmallPar{r-1}},1,\underbrace{0,\cdots,0}_{\SmallPar{r-1}},1,\cdots}.$\vspace{0pt}\par\quad
For $m\in\Nbp.$ Let $a_1\Par{e_1+U}+\dots+a_m\Par{e_m+U}=0+U\Rightarrow\exists\,u\in U,a_1 e_1+\dots+a_m e_m=u$.\vspace{0pt}\par\quad
Supp $u=\Par{x_1,\cdots,x_L,{0,\cdots}},$ where $L$ is the largest suth $u\Sbra{L}\neq 0.$\vspace{1pt}\par\quad
Let $s\in\Nbp$ be suth $h=s\cdot m!+1> L,$ \,and \,$e_1\Sbra{h}=\cdots=e_m\Sbra{h}=1.$\vspace{1pt}\par\quad
\NOTICE that for any $p,r\in\;\!\!\Bra{1,\dots,m},$ \;$e_r\Sbra{s\cdot m!+1+p}=e_r\Sbra{p+1}=1\Longleftrightarrow p\equiv 0\,\Par{\,\text{mod}\,r\,}\Longleftrightarrow r\,\Big|\,p.$\par\vspace{1pt}\quad
Let \,$1=p_1\leqslant\cdots\leqslant p_{\tau\TinyPar{p}}=p$\, be the disti factors of $p.$ Moreover, $r\,\Big|\,p\Longleftrightarrow r=p_k$ for some $k.$\par\vspace{1pt}\quad
Now $u\Sbra{h+p}=0=\sum_{r=1}^m a_r e_r\Sbra{p+1}=\sum_{k=1}^{\tau\TinyPar{p}}a_{p_k}.$\par\vspace{1pt}\quad
Let $q=p_{\tau\TinyPar{p}-1}$. Then $\tau\Par{q}=\tau\Par{p}-1,$ and each $q_k=p_k.$ Again, $\sum_{r=1}^m a_r e_r\Sbra{h+q}=0=\sum_{k=1}^{\tau\TinyPar{p}-1}a_{p_k}.$\par\vspace{1pt}\quad
Thus $a_{p_{\tau\TinyPar{p}}}=a_p=0$ for all $p\in\;\!\!\Bra{1,\dots,m}\Rightarrow\Par{e_1,\dots,e_m}$ is liney indep in $\FbbP{\infty}.$\PfEnd\vspace{12pt}\quad
%\vspace{1pt}\par\quad So is $\Par{e_1+U,\dots,e_m+U}$ in $\FbbP{\infty}\XSlash[-2.5pt]U.$ Becs $m$ is arb. By (2.A.14).
\Or For each $r\in\Nbp,$ let $\;e_r\Sbra{p}={}${\FontSmall$\hMath[0pt]{l}{\left\{\hspace{-2pt}}{\right|}{1\,,\text{if }2^r\,\Big|\,p\\0\,,\text{othws}}$}$\hText{$
	Simlr, let $m\in\Nbp$ and $a_1\Par{e_1+U}+\dots+a_m\Par{e_m+U}=0\\$
	$\Rightarrow a_1e_1+\dots+a_me_m=u\in U.$
	$}$\vspace{3pt}\par\quad
Supp $L$ is the largest suth $u\Sbra{L}\neq 0.$ And $l$ is suth $2^{ml}> L.$ \;Then for each $k\in\;\!\!\Bra{1,\dots,m},$\vspace{2pt}\par\quad
$u\Sbra{2^{ml}+2^k}=0=\sum_{r=1}^m a_re_r\Sbra{2^k}=a_1+\dots+a_k.$ \,Thus each $a_k=0.$ Simlr.\PfEnd
\SepLine
\ChEnd
\pagebreak

\ChDecl{}{\Largebfx{Exes about Polys before Chapter 4}}

\vspace{4pt}

\Anchor{1C9}\ProblemBnoor{1.C.9}{
	\TextA{A function $f:\Rbb\rightarrow\Rbb$ is called periodic if $\exists\,p\in\Nbp,\;f\Par{x}=f\Par{x+p}$ for all $x\in\Rbb.$}
	\TextA{Is the set of periodic functions $\Rbb\rightarrow\Rbb$ a subsp of $\Rbb^\Rbb$ ? Explain.}
}Denote the set by $S$.\par\quad
Supp $h\Par{x}=\cos x+\sin\!\sqrt{2}x\in S$, since $\cos x,\sin\!\sqrt{2}x\in S$.\par\quad
Asum $\exists\,p\in\Nbp$ suth $h\Par{x}=h\Par{x+p},\forall x\in\Rbb.$ Let $x=0\Rightarrow h\Par{0}=h\Par{\pm p}=1$.\par\quad
Thus $1=\cos p+\sin\!\sqrt{2}p=\cos p-\sin\!\sqrt{2}p$\par\quad
$\Rightarrow\sin\!\sqrt{2}p=0,\,\,\cos p=1\Rightarrow p=2k\pi,k\in\Zbb$, while $p=\Frac{m\pi}{\sqrt{2}},m\in\Zbb$.\par\vspace{-2pt}\quad
Hence $2k=\Frac{m}{\sqrt{2}}\Rightarrow \sqrt{2}=\Frac{m}{2k}\in\Qbb$. Ctradic!\PfEnd\vspace{10pt}\par\quad
\Or Becs $\cos x+\sin\!\sqrt{2}x=\cos\!\Par{x+p}+\sin\!\BigBigPar{\!\sqrt{2}x+\sqrt{2}p}.$ By diff twice,\par\vspace{2pt}\quad
\Blind{\Or Becs} $\cos x+2\sin\!\sqrt{2}x=\cos\!\Par{x+p}+2\envFontLarge\sin\!\BigPar{\!\sqrt{2}x+\sqrt{2}p}.$\par\vspace{6pt}\quad
\!\!\!$\MathRightBrace{r}{
	\sin\!\sqrt{2}x=\sin\!{\BigBigPar{\!\sqrt{2}x+\sqrt{2}p}}\vspace{2pt}\\ 
	\cos x=\cos\!\Par{x+p}}\Rightarrow$ Let $x=0,$\;\,$ p=\Frac{m\pi}{\sqrt{2}}=2k\pi.$\; Ctradic.\PfEnd\vspace{4pt}\par
\SepLine

\Anchor{1C24}\ProblemBnoor{1.C.24}{
	\TextA{Let $V_{\!E}=\Bra{\,f\in\Rbb^{\Rbb}:\text{f is even}},V_{\!O}=\Bra{\,f\in\Rbb^{\Rbb}:\text{f is odd}}.$ Show $V_{\!E}\oplus V_{\!O}=\Rbb^{\Rbb}.$\vspace{4pt}}
}(a) {$V_{\!E}\cap V_{\!O}=\Bra{\,f\in\Rbb^{\Rbb}:f\Par{x}=f\Par{{-x}}=-f\Par{{-x}}}=\zeroSubs.$}\parSol{\vspace{8pt}}
(b) $\hMath{l}{\left|}{\right\}}{$
	Let \;$f_e\Par{x}=\Frac{\;1\;}{2}\XSbra{g\Par{x}+g\Par{{-x}}}\Longrightarrow f_e\in V_{\!E}\vspace{4pt}\\$
	Let \;$f_o\Par{x}=\Frac{\;1\;}{2}\XSbra{g\Par{x}-g\Par{{-x}}}\Longrightarrow f_o\in V_{\!O}
}\Rightarrow\forall g\in\Rbb^\Rbb,\;g\Par{x}=f_e\Par{x}+f_o\Par{x}.$\PfEnd\vspace{4pt}
\SepLine

\Anchor{2C7}\ProblemBnoor{2.C.7}{
	(a) \TextA{Let $U=\Bra{p\in\PoF{4}:p\Par{2}=p\Par{5}=p\Par{6}}$. Find a bss of $U$.}
	(b) \TextA{Extend the bss in {\tgnr(a)} to a bss of $\PoF{4}$, and find a $W$ suth $\PoF{4}=U\oplus W$.}
}Using (2.C.10).
%Supp $p\Par{z}=az^4+bz^3+cz^2+dz+e$ suth $p\Par{2}=p\Par{5}=p\Par{6}$.\vspace{4pt}\par\quad
%Then $\hMath[0pt]{r}{\left|}{\right\}}{
	%	p\Par{2}=16a+8b+4c+2d+e\,\;\;\Par{\text{I}}\;\;\\
	%	p\Par{5}=625a+125b+25c+5d+e\;\;\Par{\text{II}}\,\;\\
	%	p\Par{6}=1296a+216b+36c+6d+e\;\,\Par{\text{III}}
	%}\Rightarrow\MathLeftBrace{l}{
	%	\Par{\text{II}}\;-\;\Par{\text{I}}=0\\
	%	\Par{\text{III}}-\Par{\text{II}}=0\\
	%	\Par{\text{III}}-\;\Par{\text{I}}=0\\
	%}$\vspace{4pt}\par\quad
%{\tgsl You don't have to compute to know that the dimension of the set of solutions is 3.}
\par\quad
\NOTICE that $\not\exists\,p\in\PoFi$ of deg $1$ and $2,$ while $p\in U.$ Thus $\dim U\leqslant \dim\PoF{4}-2=3.$\par\vspace{2pt}\quad
(a) Consider $B=\BigBigPar{1,\Par{z-2}\Par{z-5}\Par{z-6},z\Par{z-2}\Par{z-5}\Par{z-6}}.$\par\quad\Ha
Let $a_0+a_3\Par{z-2}\Par{z-5}\Par{z-6}+a_4z\Par{z-2}\Par{z-5}\Par{z-6}=0\Rightarrow a_0=a_3=a_4=0.$\par\quad\Ha
Thus the list $B$ is liney indep in $U.$ Now $\dim U\geqslant 3\Rightarrow \dim U=3.$ Thus $B_U=B.$\par\vspace{2pt}\quad
(b) Extend to a bss of $\PoF{4}$ as $\BigBigPar{1,z,z^2,\Par{z-2}\Par{z-5}\Par{z-6},z\Par{z-2}\Par{z-5}\Par{z-6}}.$\par\quad\Hb
Let $W=\Span{z,z^2}=\Bra{az+bz^2:a,b\in\Fbb}$, so that $\PoF{4}=U\oplus W.$\PfEnd
\SepLine

\Anchor{2CN10}\BulletPointX\NoteForSmall{(2.C.10)} \,\,\,For each nonC $p\in\Span{1,z,\dots,z^m},\;\exists$ smallest $m\in\Nbp,$ which is $\deg p.$\TextB{}
(a) If $p_0,p_1,\dots,p_m$ are suth all $a_{k,k}\neq 0,$ and\TextB{}
\Hb $p_0=a_{0,0},\,$ each $p_k=a_{0,k}+a_{1,k}z+\dots+a_{k,k}z^k.$\TextB{\vspace{-18pt}}
\Ha Then the upper-trig $\Mt[\BigBigPar]{I,\Par{p_0,p_1,\dots,p_m},\Par{1,z,\dots,z^m}}={}${\small$\begin{pmatrix}
		a_{0,0} & a_{0,1} & \cdots & a_{0,m}\\
		0       & a_{1,1} & \cdots & a_{1,m}\\
		\vdots  & \vdots  & \ddots & \vdots\\
		0       & 0       & \cdots & a_{m,m}
	\end{pmatrix}$}.\TextB{\vspace{-8pt}}
(b) If $p_0,p_1,\dots,p_m$ are suth all $a_{k,k}\neq 0,$ and\TextB{}
\Hb $p_0=a_{0,0}+\dots+a_{m,0}x^m,\,$ each $p_k=a_{k,k}x^k+\dots+a_{m,k}x^m.$\TextB{\vspace{-18pt}}
\Hb Then the lower-trig \;$\Mt[\BigBigPar]{I,\Par{p_0,p_1,\dots,p_m},\Par{1,z,\dots,z^m}}={}${\small$\begin{pmatrix}
		a_{0,0} & 0       & \cdots & 0\\
		a_{1,0} & a_{1,1} & \cdots & 0\\
		\vdots  & \vdots  & \ddots & \vdots\\
		a_{m,0} & a_{m,1} & \cdots & a_{m,m}
	\end{pmatrix}$}.\TextB{\vspace{-12pt}}
\AComm Define $\xi_k\Par{p}$ by the coeff of $z^k$ in $p\in\PoF{m}.$\parCom\IndentB{}
Then $\Mt[\BigBigPar]{\xi_k,\Par{1,z,\dots,z^m},\Par{1}}=\mEnt{1,\,k}\in\FbbP{1,m+1}.$\vspace{-2pt}
\SepLine\pagebreak

\Anchor{2C10}\ProblemBnoor{2.C.10}{
	\TextA{Supp $m\in\Nbp,\;p_0,p_1,\dots,p_m\in\PoFi$ are suth each $\deg p_k=k.$}
	\TextA{Prove $\Par{p_0,p_1,\dots,p_m}$ is a bss of $\PoF{m}$.}
}{Using induc on $m$.}\par\quad
(i) {$k=1.$ \;$\deg p_0=0;\;\deg p_1=1\Rightarrow\Span[\BigPar]{p_0,p_1}=\Span[\BigPar]{1,x}.$}\par\vspace{2pt}\quad\Endi
(ii) {$1\leqslant k\leqslant m-1.$ \;Asum $\Span[\BigPar]{p_0,p_1,\dots,p_k}=\Span[\BigPar]{1,x,\dots,x^k}.$}\par\quad\Hii
{Then $\Span[\BigPar]{p_0,p_1,\dots,p_k,p_{k+1}}\subseteq\Span[\BigPar]{1,x,\dots,x^k,x^{k+1}}$.}\par\vspace{2pt}\quad\Hii
{又 $\deg p_{k+1}=k+1,\;\;p_{k+1}\Par{x}=a_{k+1}x^{k+1}+r_{k+1}\Par{x};\;\;a_{k+1}\neq 0,\;\;\deg r_{k+1}\leqslant k.$}
\par\vspace{2pt}\quad\Hii
{$\Rightarrow x^{k+1}=\Frac{1}{a_{k+1}}\XPar{p_{k+1}\Par{x}-r_{k+1}\Par{x}}\in\Span[\BigPar]{1,x,\dots,x^k,p_{k+1}}=\Span[\BigPar]{p_0,p_1,\dots,p_k,p_{k+1}}$.}\par\vspace{2pt}\quad\Hii
{$\therefore\,\,x^{k+1}\in\Span[\BigPar]{p_0,p_1,\dots,p_k,p_{k+1}}\Rightarrow\Span[\BigPar]{1,x,\dots,x^k,x^{k+1}}\subseteq\Span[\BigPar]{p_0,p_1,\dots,p_k,p_{k+1}}$.}\par\vspace{2pt}\quad
{Thus $\PoF{m}=\Span[\BigPar]{1,x,\dots,x^m}=\Span[\BigPar]{p_0,p_1,\dots,p_m}.$}\FontNorm\PfEnd\vspace{8pt}\quad
\Or By comparing coeffs. {Denote the coeff of $x^k$ in $p\in\PoFi$ by $\xi_k\Par{p}.$}\par\quad
{Supp $L=a_m p_m\Par{x}+\dots+a_1 p_1\Par{x}+a_0p_0\Par{x}=0\cdot x^m+\dots+0\cdot x+0\cdot 1=R,\forall x\in\Fbb.$}\par\quad
{We show $a_m=\dots=a_0=0$ via the following process. So that $\Par{p_0,p_1,\dots,p_m}$ is liney indep.}\vspace{2pt}\par\quad
{\tgbfx Step 1.} {For $k=m,$ \;$\xi_{m}\Par{L}=a_{m}\xi_{m}\Par{p_m}=\xi_{m}\Par{R}=0$ 又 $\deg p_m=m,\;\xi_{m}\Par{p_m}\neq 0\Rightarrow a_m=0.$}\par\quad
\Blind{{\tgbfx Step 1.}} {Now $L=a_{m-1}p_{m-1}\Par{x}+\dots+a_0p_0\Par{x}.$}\vspace{2pt}\par\quad
{\tgbfx Step k.} {For $0\leqslant k\leqslant m,$ we have $a_m=\dots=a_{k+1}=0.$}\par\quad
\Blind{{\tgbfx Step k.}} {Now $\xi_{k}\Par{L}=a_{k}\xi_{k}\Par{p_k}=\xi_{k}\Par{R}=0$ 又 $\deg p_k=k,\;\xi_{k}\Par{p_k}\neq 0\Rightarrow a_k=0.$}\par\quad
\Blind{{\tgbfx Step k.}} {Now if $k=0,$ then done. Othws, we have $L=a_{k-1}p_{k-1}\Par{x}+\dots+a_0p_0\Par{x}.$}\PfEnd
\SepLine

\Anchor{2CT10}\ProblemBX{\Tips}{
	\TextA{Supp $m\in\Nbp,\;p_0,p_1,\dots,p_m\in\PoF{m}$ are suth the lowest term of each $p_k$ is of deg $k.$}
	\TextA{Prove $\Par{p_0,p_1,\dots,p_m}$ is a bss of $\PoF{m}.$}
}{Using induc on $m.$\par}\quad
{Let each $p_k$ be defined by $p_k\Par{x}=a_{k,k} x^k+\dots+a_{m,k} x^m,$ where $a_{k,k}\neq 0.$\par}\quad
(i) {$k=1.$ \;$p_m\Par{x}=a_{m,m}x^m;\;p_{m-1}\Par{x}=a_{m-1,m-1}x^{m-1}+a_{m,m-1}x^m\Longrightarrow\Span[\BigPar]{x^m,x^{m-1}}=\Span[\BigPar]{p_m,p_{m-1}}.$\par}\vspace{2pt}\quad\Endi
(ii) {$1\leqslant k\leqslant m-1.$ \;Asum $\Span[\BigPar]{x^m,\dots,x^{m-k}}=\Span[\BigPar]{p_m,\dots,p_{m-k}}.$}\par\quad\Hii
{Then $\Span[\BigPar]{p_m,\dots,p_{m-\SmallPar{k+1}}}\subseteq\Span[\BigPar]{x^m,\dots,x^{m-\SmallPar{k+1}}}.$\par}\quad\Hii
{又 $p_{m-\SmallPar{k+1}}$ has the form $a_{m-\SmallPar{k+1},m-\SmallPar{k+1}}x^{m-\SmallPar{k+1}}+r_{m-\SmallPar{k+1}}\Par{x};\;$\par}\quad\Hii
{\Blind{又} where the lowest term of $r_{m-\SmallPar{k+1}}\in\PoF{m}$ is of deg $\Par{m-k}.$\par}\vspace{24pt}\quad\Hii
{\Blind{$\Rightarrow x^{m-\SmallPar{k+1}}=\Frac{}{a_{m-\SmallPar{k+1},m-\SmallPar{k+1}}}\XPar{p_{m-\SmallPar{k+1}}\Par{x}-r_{m-\SmallPar{k+1}}\Par{x}}$}${}=\Span[\BigPar]{p_m,\dots,p_{m-k},p_{m-\SmallPar{k+1}}}.$\par}\vspace{-50pt}\quad\Hii
{$\Rightarrow x^{m-\SmallPar{k+1}}=\Frac{1}{a_{m-\SmallPar{k+1},m-\SmallPar{k+1}}}\XPar{p_{m-\SmallPar{k+1}}\Par{x}-r_{m-\SmallPar{k+1}}\Par{x}}\in\Span[\BigPar]{x^m,\dots,x^{m-k},p_{m-\SmallPar{k+1}}}$\par}\vspace{4pt}\quad\Hii
{$\therefore\;x^{m-\SmallPar{k+1}}\in\Span[\BigPar]{p_m,\dots,p_{m-k},p_{m-\SmallPar{k+1}}}$\par}\vspace{3pt}\quad\Hii
{\Blind{$\therefore\;$}$\Rightarrow\Span[\BigPar]{x^m,\dots,x^{m-k},x^{m-\SmallPar{k+1}}}\subseteq\Span[\BigPar]{p_m,\dots,p_{m-k},p_{m-\SmallPar{k+1}}}.$\par}\vspace{3pt}\quad
{Thus $\PoF{m}=\Span[\BigPar]{x^m,\dots,x,1}=\Span[\BigPar]{p_m,\dots,p_1,p_0}.$}\PfEnd\vspace{8pt}\quad
\Or By comparing coeffs. {Denote the coeff of $x^k$ in $p\in\PoFi$ by $\xi_k\Par{p}.$}\par\quad
{Supp $L=a_m p_m\Par{x}+\dots+a_1 p_1\Par{x}+a_0p_0\Par{x}=0\cdot x^m+\dots+0\cdot x+0\cdot 1=R,\forall x\in\Fbb.$}\par\quad
{We show $a_m=\dots=a_0=0$ via the following process. So that $\Par{p_0,p_1,\dots,p_m}$ is liney indep.}\vspace{2pt}\par\quad
{\tgbfx Step 1.} {For $k=0,$ \;$\xi_{0}\Par{L}=a_{0}\xi_{0}\Par{p_0}=\xi_{0}\Par{R}=0$ 又 $\deg p_0=0,\;\xi_{0}\Par{p_0}\neq 0\Rightarrow a_0=0.$}\par\quad
\Blind{{\tgbfx Step 1.}} {Now $L=a_1p_1\Par{x}+\dots+a_{m}p_{m}\Par{x}.$}\vspace{2pt}\par\quad
{\tgbfx Step k.} {For $0\leqslant k\leqslant m,$ we have $a_{k-1}=\dots=a_0=0.$}\par\quad
\Blind{{\tgbfx Step k.}} {Now $\xi_{k}\Par{L}=a_{k}\xi_{k}\Par{p_k}=\xi_{k}\Par{R}=0$ 又 $\deg p_k=k,\;\xi_{k}\Par{p_k}\neq 0\Rightarrow a_k=0.$}\par\quad
\Blind{{\tgbfx Step k.}} {Now if $k=m,$ then done. Othws, we have $L=a_{k+1}p_{k+1}\Par{x}+\dots+a_mp_m\Par{x}.$}\PfEnd
\SepLine

\Anchor{2AN2.11}\BulletPointX\NoteForSmall{[2.11]} {\tgsc Good definition for a general term always aviods undefined behaviours.}\TextB{}
If $\deg p=0,$ then $p\Par{z}=a_0\neq 0,$ but \uline{not literally $a_0z^0,$} by which if $p$ is defined, then it comes to $0^0.$\TextB{}
To make it clear, we \uline{specify that {\tgsl in} $\PoFi,$ $a_0z^0=a_0,$ where $z^0$ appears just for notat conveni.}\TextB{}
Becs by def, the term $a_0z^0$ in a poly only represents the const term of the poly, which is $a_0.$\TextB{}
For conveni, we asum $z^0=1$ in formula deduction and poly def. Absolutely without $0^0.$
\SepLine

\Anchor{2C4e10}\ProblemBnoor{4E 2.C.10}{
	\TextA{Supp $m$ is a positive integer. For $0\leqslant k\leqslant m$, let $p_k\Par{x}=x^k\Par{1-x}{^{m-k}}$.}
	\TextA{Show $\Par{p_0,\dots,p_m}$ is a bss of $\PoF{m}$.\vspace{0pt}}
	%\TextA{{\large The bss in this exe leads to what are called Bernstein polys. You can do a web search to learn how}}
	%\TextA{{\large Bernstein polys are used to approximate continuous functions on $[0, 1]$.}}
}{\tgsl We may see $p_0=1$ and $p_m\Par{x}=x^m,$ from the expansion below, by the {\NOTEFOR} [2.11] above.}\vspace{4pt}\par\quad
Note that each $p_k\Par{x}={\sum_{j=0}^{m-k}\mathC_{m-k}^j\Par{{-1}}{^{j}}\cdot x^{j+k}\cdot 1^j}=\underset{\text{of deg k}}{\uline{\Par{{-1}}{^0}\cdot x^k\cdot 1^0}}+\underset{\text{of deg m; denote it by }q_k\SmallPar{x}}{\uline{\sum_{j=1}^{m-k}\mathC_{m-k}^j\Par{{-1}}{^{j}}\cdot x^{j+k}\cdot 1^j}}.$\vspace{-12pt}\par\quad
And, each $q_k\in\Span{x^{k+1},\dots,x^m}.$ Using {\TIPS} above.\PfEnd\vspace{6pt}\quad
\Or Simlr to the {\TIPS} above. We will recurly prove each $x^{m-k}\in\Span{p_m,\dots,p_{m-k}}.$\par\quad
(i) $k=1.$ \;$p_m\Par{x}=x^m\in\Span{p_m};$ \;\; $p_{m-1}\Par{x}=x^{m-1}-x^m\Rightarrow x^{m-1}\in\Span{p_{m-1},p_m}.$\vspace{2pt}\par\quad\Endi
(ii) $k\in\;\!\!\Bra{1,\dots,m-1}.$ \;Supp for each $j\in\;\!\!\Bra{0,\dots,k},$ we have $x^{m-j}\in\Span[\BigPar]{p_{m-j},\dots,p_m},\exists\,!\,a_m\in\Fbb.$\vspace{2pt}\par\quad\Hii
Note that $x^{m-\SmallPar{k+1}}=p_{m-\SmallPar{k+1}}\Par{x}+\sum_{j=1}^{k+1}\mathC_{k+1}^j\Par{{-1}}{^{j+1}}x^{m-\SmallPar{k+1}+j}\in\Span[\BigPar]{p_{m-\SmallPar{k+1}},x^{m-k},\dots,x^{m}}.$\vspace{2pt}\par\quad\Hii
Thus $x^{m-\SmallPar{k+1}}\in\Span[\BigPar]{p_{m-\SmallPar{k+1}},p_{m-k},\dots,p_m}.$\PfEnd\vspace{10pt}\quad
\Or For any $m,k\in\Nbp$ suth $k\leqslant m.$ Define $p_{k,m}$ by $p_{k,m}\Par{x}=x^k\Par{1-x}{^{m-k}}.$\par\quad
Define the stmt $S\Par{m}:\Par{p_{0,m},\dots,p_{m,m}}$ is liney indep \BigPar{ and therefore is a bss }.\par\quad
We use induc on to show $S\Par{m}$ holds for all $m\in\Nbp.$\vspace{2pt}\par\quad
(i) $m=0.$ \;$p_{0,0}=1,$ and $ap_{0,0}=0\Rightarrow a=0.$\par\quad\Hi
$m=1.$ \;Let $a_0\Par{1-x}+a_1x=0,\forall x\in\Fbb.$ \;Then take $x=1,x=0\Rightarrow a_1=a_0=0.$\par\vspace{4pt}\quad
%$m=2.$ \;Let $a_0\Par{1-x}{^2}+a_1\Par{1-x}x+a_2x^2=0,\forall x\in\Fbb.$ \;Then $\small\MathLeftBrace{l}{\!\!x=0\Rightarrow a_0+a_1=0;\\\!\!x=1\Rightarrow a_2=0;\\\!\!x=2\Rightarrow a_0+2a_1=0.}$\par\vspace{0pt}\quad\Endi
(ii) $1\leqslant m.$ \;Asum $S\Par{m}$ and $S\Par{m-1}$ holds. Now we show $S\Par{m+1}$ holds.\vspace{2pt}\par\quad\Hii
Supp $\sum_{k=0}^{m+1}a_kp_{k,m+1}\Par{x}=\sum_{k=0}^{m+1}a_k\Sbra{x^k\Par{1-x}{^{m+1-k}}}=0,\forall x\in\Fbb.$\vspace{6pt}\par\quad\Hii
\envFontLarge{Now $a_0\Par{1-x}{^{m+1}}+\sum_{k=1}^{m}a_kx^k\Par{1-x}{^{m+1-k}}+a_{m+1}x^{m+1}=0,\forall x\in\Fbb.$}\par\vspace{2pt}\quad\Hii
While $x=0\Rightarrow a_0=0;$ \;and $x=1\Rightarrow a_{m+1}=0.$\par\vspace{2pt}\quad\Hii
Then $0=\sum_{k=1}^{m}a_kx^k\Par{1-x}{^{m+1-k}}$\par\vspace{4pt}\quad\Hii
\Blind{Then $0$}${}=x\Par{1-x}\sum_{k=1}^{m}a_kx^{k-1}\Par{1-x}{^{m-k}},$ {\normalsize note that} {\small\envFontSmall[\small]$ m-k=\Par{m-1}-\Par{k-1}$}\par\vspace{4pt}\quad\Hii
\Blind{Then $0$}${}=x\Par{1-x}\textstyle\sum_{k=0}^{m-1}a_{k+1}x^{k}\Par{1-x}{^{m-1-k}}=x\Par{1-x}\sum_{k=0}^{m-1}a_{k+1}p_{k,m-1}\Par{x}.$\par\vspace{6pt}\quad\Hii
\FontNorm{\vspace{4pt}Hence \;$\sum_{k=0}^{m-1}a_{k+1}p_{k,m-1}\Par{x}=0,\forall x\in\Fbb\Backslash{\def\envFont{\envFontB}\Bra{\,0,1}}.$ Which has infily many zeros.}\par\quad\Hii
Moreover, \;$\sum_{k=0}^{m-1}a_{k+1}p_{k,m-1}\Par{x}=0.$ By asum, $a_1=\dots=a_{m-1}=a_{m}=0.$\par\quad\Hii
Thus $\Par{p_{0,m+1},\dots,p_{m+1,m+1}}$ is liney indep and $S\Par{m+1}$ holds.\PfEnd
\SepLine

%\Anchor{3D4e20}\ProblemBnoor{4E 3.D.20}{
%	\TextA{Supp $q\in\PoRi.$ Prove $\exists\,p\in\PoRi,q\Par{x} = \Par{x^2 + x}p\apostrophe\apostrophe\Par{x} + 2xp\apostrophe\Par{x} + p\Par{3}.$\vspace{1pt}}
%}Note that $\Deg\Sbra{\Par{x^2 + x}p\apostrophe\apostrophe\Par{x} + 2xp\apostrophe\Par{x} + p\Par{3}}=\deg p.$\parSol{}
%Define $T_{\!n}\in\Lm[\BigPar]{\PoR{n}}$ by $T_{\!n}\Par{p}=\Par{x^2 + x}p\apostrophe\apostrophe\Par{x} + 2xp\apostrophe\Par{x} + p\Par{3}.$\parSol{}
%And note that $T_{\!n}\Par{p}=0\Rightarrow\deg T_{\!n}\Par{p}=-\infty=\deg p\Rightarrow p=0.$ Thus $T_{\!n}$ is inv.\parSol{}
%$\forall q\in\PoRi,$ if $q=0,$ let $n=0;$ if $q\neq 0,$ let $n=\deg q,$ we have $q\in\PoR{n}.$\parSol{}
%Now $\exists\,p\in\PoR{n}, q\Par{x}=T_{\!n}\Par{p}=\Par{x^2 + x}p\apostrophe\apostrophe\Par{x} + 2xp\apostrophe\Par{x} + p\Par{3}$ for all $x\in\Rbb.$\PfEnd
%\SepLine
%\pagebreak

\Anchor{3D19}\ProblemBnoor{3.D.19}{
	\TextA{Supp $T\in\Lm[\BigPar]{\PoRi}$ is inje. And $\deg Tp\leqslant\deg p$ for every non0 $p\in\PoRi$.}
	\PrePa\TextA{Prove $T$ is surj. \quad {\large\tgnr(b)} Prove for every non0 $p$, $\deg Tp=\deg p$.}
}(a) $T$ is inje $\Longleftrightarrow\forall n\in\Nbp,T\mmid_{\PoR[\SmallPar]{n}}\in\Lm[\BigPar]{\PoR{n}}$ is inje, so is inv $\Longleftrightarrow T$ is surj.\vspace{2pt}\parSol{}
(b) (i) $\deg p=-\infty\geqslant\deg Tp\Longleftrightarrow p=0=Tp.$ \;And $\deg p=0\geqslant\deg Tp\Longleftrightarrow p=C\neq 0.$\parSol{\vspace{1pt}\Endi\Hb}
(ii) Asum $\forall s\in\PoR{n},\deg s=\deg Ts.$ We show $\forall p\in\PoR{n+1},\deg Tp=\deg p$ by ctradic.\vspace{1pt}\parSol{\Hii\Hb}
Supp $\exists\,r\in\PoR{n+1},\;\deg Tr\leqslant n<n+1=\deg r.$ \,By (a), $\exists\,s\in\PoR{n},\;T\Par{s}=\Par{Tr}.$\parSol{\Hii\Hb}
又 $T$ is inje $\Rightarrow s=r.$ While $\deg s=\deg Ts=\deg Tr<\deg r.$ Ctradic.\PfEnd
\SepLine\pagebreak

\Anchor{3B26}\ProblemBnoor{3.B.26}{
	\TextA{Supp $D\in\Lm[\BigPar]{\PoRi}$ and $\forall p,\Deg\BigPar{Dp}=\Par{\!\deg p}-1.$ \;Prove $D\in\PoRi$ is surj.\vspace{2pt}}
}\!\Sbra{ $D$ might not be $D:p\mapsto p\apostrophe.$\hspace{1pt}} \;The proof here is too informal to be valid:\parSol{}
Becs $\Span{Dx,Dx^2,Dx^3,\cdots}\subseteq\range D,$ and $\deg Dx^n=n-1.$\parSol{}
又 By (2.C.10), $\Span{Dx,Dx^2,Dx^3,\cdots}=\Span{1,x,x^2,\cdots}=\PoRi.$\par\vspace{4pt}\quad
\uline{Let $D\Par{C}=0,Dx^k=p_k$} of deg $\Par{k-1},$ for all $C\in\PoR{0}$ and each $k\in\Nbp.$ \NOTICE that $\Rbb\neq\PoR{0}.$\par\quad
Becs $B_{\PoR[\SmallPar]{m}}=\Par{p_1,\dots,p_m,p_{m+1}}.$ And for all $p\in\PoRi,\exists\,!\,m=\deg p\in\Nbp.$\par\quad
So that $\exists\,!\,a_i\in\Rbb,p=\sum_{i=1}^{m+1}a_ip_i\Rightarrow\exists\,q=\sum_{i=1}^{m+1}a_{i}x^{i},Dq=p.$\PfEnd\vspace{6pt}\quad
{\Or We will \uline{recurly define a seq of polys $\Par{p_k}{_{k=0}^\infty}$ where $Dp_0=1,Dp_k=x^k$} for each $k\in\Nbp.$}\par\vspace{2pt}\quad
{\FontSmall So that $\forall p=\sum_{k=0}^{\deg p}a_k x^k\in\PoRi,Dq=p,\exists\,q={\sum_{k=0}^{\deg p}a_k p_k}.$}\par\vspace{4pt}\quad
(i) {Becs $\deg Dx=\Par{\!\deg x}-1=0,Dx=C\in\nonzeroFbb.$ Let $p_0=C^{-1}x\Rightarrow Dp_0=C^{-1}Dx=1.$}\vspace{2pt}\par\quad\Endi
(ii) {Supp we have defined $Dp_0=1,Dp_k=x^k$ for each $k\in\;\!\!\Bra{1,\dots,n}.$ Becs $\deg D\Par{x^{n+2}}=n+1.$}\vspace{2pt}\par\quad\Hii
{Let {\;$D\Par{x^{n+2}}=a_{n+1}x^{n+1}+a_n x^n+\dots+a_1 x+a_0,$} with $a_{n+1}\neq 0.$}\vspace{2pt}\par\quad\Hii
{Then {\;$a_{n+1}^{-1}D\BigPar{x^{n+2}}=x^{n+1}+a_{n+1}^{-1}\BigPar{a_n Dp_n+\dots+a_1 Dp_1 +a_0 Dp_0}$}}\vspace{2pt}\par\quad\Hii
{$\Rightarrow x^{n+1}=D\XSbra{\uline{a_{n+1}^{-1}\Par{x^{n+2}-a_n p_n-\dots-a_1 p_1-a_0 p_0}}}.$ Thus defining $p_{n+1},$ so that $Dp_{n+1}=x^{n+1}.$}\PfEnd
\SepLine

\Anchor{3E'2}\ProblemB{
	\TextB{Supp $V=\Rbb^\Rbb$ with a subsp $U=\Bra{\hspace{3pt}f\in\Rbb^\Rbb:f\Par{x_1}=\cdots=f\Par{x_m}=0},$ where each $x_k\in\Rbb.$\vspace{2pt}}
	\TextB{Prove if $W\in\Scom{V}{U},$ then $\dim W=m.$\hfill{\large\tgsc Hint$:$ }{\FontNorm Find an iso from $V\XSlash U$ onto $\Rbb^m.$}}
}Define $T\in\Lm{V\XSlash U,\Rbb^m}$ by $T\Par{\,f+U}=\BigPar{\,f\Par{x_1},\dots,\,f\Par{x_m}}.$\parSol{}
$\forall\,f+U=g+U\in V\XSlash U,\;f-g\in U\Rightarrow f\Par{x_k}=g\Par{x_k}.$ Well-defined.\parSol{\vspace{0pt}}
Inje: Each $\,f\Par{x_k}=0\Rightarrow f+U=0.$ \; Surj: Immed.\PfEnd
\SepLine

\Anchor{3F7}\ProblemBnoor{3.F.7}{
	\TextA{Show the dual bss of $\Par{1, x,\dots,x^m}$ of $\PoR{m}$ is $\Par{\varphi_0,\varphi_1,\dots,\varphi_m}$, where {\FontNorm $\varphi_k\Par{p}=\Frac{p^{\SmallPar{k}}\Par{0}}{k!}$.}}
%	\TextB{{\FontNorm Here $p{^{\SmallPar{k}}}$ denotes the $k^{th}$ deri of $p$, with the understanding that the $0^{th}$ deri of $p$ is $p$.}\vspace{-2pt}}
}{\tgsl\normalsize The uniqnes of dual bss is guaranteed by [3.5].}\vspace{2pt}\parSol{}
For $j,k\in\Nbb,\:\Par{x^{j}}{^{\SmallPar{k}}}=\left\{\begin{array}{l}j\Par{j-1}\cdots\Par{j-k+1}\cdot x^{\SmallPar{j-k}}\,,\quad j \geqslant k.\\j\Par{j-1}\cdots\Par{j-j+1}=j!\hfill j=k.\\0,\hfill j \leqslant k.\end{array}\right|\hspace{4pt}\Rightarrow\Par{x^{j}}{^{\SmallPar{k}}}\Par{0}=\MathLeftBrace{l}{0\,\,\,,\,\hspace{5.2pt}j\neq k. \\ k!\,\,,\hspace{6pt}j=k.}$\PfEnd\vspace{8pt}\Anchor{3F8}
%{\Large$\envFontSmall[\footnotesize]\frac{p^{\SmallPar[0pt]{k}}\Par[0pt]{0}}{k!}$}
\AExa By [2.C.10], $B_m=\BigBigPar{1,7x-5,\dots,\Par{7x-5}{^m}}$ is a bss of $\PoR{m}$. Let each $\varphi_k={}${\Large\envFontSmall[\footnotesize]$\frac{p^{\SmallPar{k}}\Par{5\text{\large/}\hspace{0.5pt}7}}{7\hspace{1pt}\cdot\,k!}$}.\par\vspace{2pt}
\SepLine
\ChEnd\pagebreak

\ChDecl{Ch4O}{4}{}

\vspace{3pt}

\Anchor{4OT1}\BulletPointX\TipsN{1}\,\,\,{\tgsl Supp $p\in\PoF{n}$ has at least $n+1$ disti zeros. Then by the ctrapos of [4.12], $\deg p<0\Rightarrow p=0.$}\TextB{}
{\IndentTipsN{1}}\Or We show if $p\in\PoFi{}$ has at least $m$ disti zeros, then either $p=0$ or $\deg p\geqslant m.$\TextB{}
{\IndentTipsN{1}}Supp $p\neq0.$ Becs $\exists\,!\,\alpha_i\geqslant 1,q\in\PoFi,\:p\Par{z}=\Sbra{\Par{z-\lambda_1}{^{\alpha_1}}\cdots\Par{z-\lambda_m}{^{\alpha_m}}}q\Par{z}.$\PfEnd\vspace{2pt}\Anchor{4ON4.7}\TextB{}
%\BulletPointX\AComm {\tgsl\NOTICE that by [4.17], some term of the poly factoriz might not be in the form $\Par{x-\lambda_k}{^{\alpha_k}}.$}\vspace{-2pt}
\!\Sbra[3pt]{{\tgsc\large Another proof of \tgnr\tgbfx[4.7]}} \;If a poly had two different sets of coeffs,\TextB{}
then
subtracting the two exprs would give a poly with some non0 coeffs but infily many zeros.
\SepLine

\Anchor{4ON4.8}\BulletPointX\NoteFor{[4.8]} {\tgsl div algo for polys} \hfill[{\tgsc Another proof}]\TextB{\vspace{-11pt}}
Supp $\deg p\geqslant \deg s$. Then $\FontLarge\BigPar{\FontNorm\underbrace{1, z,\dots, z^{\deg s-1}}_{\text{of\,len}\,\deg s},\overbrace{s,zs,\cdots,z^{\deg p-\deg s}s}^{\text{of\,len}\,\SmallPar{\deg p\,-\,\deg s\,+\,1}}\FontLarge}$ is a bss of $\PoF{\deg p}.$\TextB{\vspace{-7pt}}
Becs $q\in\PoFi,\exists\,!\,a_i,b_j\in\Fbb,$\TextB{}
$q=a_0+a_1 z+\dots+a_{\deg s-1}z^{\deg s-1}+ b_0 s+b_1 zs +\dots+ b_{\deg p-\deg s}z^{\deg p-\deg s}s$\TextB{}
$\Blind{q}=\underbrace{a_0+a_1 z+\dots+a_{\deg s-1}z^{\deg s-1}}_{r}+s\underbrace{\XPar{b_0+b_1 z +\dots+ b_{\deg p-\deg s}z^{\deg p-\deg s}}}_{q}.$ Note that $r,q$ are uniq.\PfEnd[-16pt]
\SepLine

\Anchor{4ON4.11}\BulletPointX\NoteFor{[4.11]}\;\;{\tgsl each zero of a poly corres to a deg-one factor;}\hfill[{\tgsc Another proof}]\TextB{\vspace{2pt}}
First supp $p\Par{\lambda}=0.$ Write $p\Par{z}=a_0+a_1 z+\dots+a_m z^m,\exists\,!\,a_0,a_1,\dots,a_m\in\Fbb$ for all $z\in\Fbb.$\vspace{2pt}\TextB{}
Then $p\Par{z}=p\Par{z}-p\Par{\lambda}=a_1\Par{z-\lambda}+\dots+a_m\Par{z^m-\lambda^m}$ for all $z\in\Fbb.$\vspace{2pt}\TextB{}
Hence $\forall k\in\;\!\!\Bra{1,\dots,m},z^k-\lambda^k=\Par{z-\lambda}\Par{ z^{k-1}\lambda^0+ z^{k-2}\lambda^1+\dots+z^{k-\SmallPar{j+1}}\lambda^j+\dots+z\lambda^{k-2}+z^0\lambda^{k-1}}.$\vspace{4pt}\TextB{}
Thus $p\Par{z}=\sum_{j=1}^m a_j \Par{z-\lambda}\sum_{i=1}^k \lambda^{i-1}z^{k-i}=\Par{z-\lambda}\sum_{j=1}^m a_j\sum_{i=1}^k \lambda^{i-1}z^{k-i}=\Par{z-\lambda}q\Par{z}.$\PfEnd
\SepLine

%\BulletPointX\NoteFor{[4.13]}\;\;{\tgsl Every nonC poly with complex coeffs has a zero in $\Cbb$.} \vspace{2pt}\hfill[{\tgsc Another proof}]\TextB{}
%For any $w\in\Cbb,k\in\Nbp,$ by polar coordinates, $\exists\,r\geqslant 0,\theta\in\Rbb,r\Par{\cos\theta + \i \sin\theta} = w$.\vspace{2pt}\TextB{}
%By De Moivre' theorem, $w^k=\Sbra{r\Par{\cos\theta +\i\sin \theta}}{^k}=r^k\Par{\cos k\theta + \i \sin k\theta}$.\vspace{2pt}\TextB{}
%Hence $\BigPar{r^{1/k}\Par{\cos\frac{\;\theta\;}{k} +\i\sin\frac{\;\theta\;}{k}}}{^k}=w$. Thus every complex number has a {\tgsl $k^{th}$ root}.\vspace{6pt}\TextB{}
%Supp a nonC $p\in\PoCi$ with highest-order non0 term $c_m z_m.$\vspace{3pt}\TextB{}
%Then $\aMid{p\Par{z}}\rightarrow\infty$ as $\aMid{z}\rightarrow\infty$ \XPar{ becs $\Frac{\aMid{p\Par{z}}}{\aMid{z_m}}\rightarrow\aMid{c_m}$ as $\aMid{z}\rightarrow\infty$ }.\vspace{3pt}\TextB{}
%\vspace{3pt}Thus the continuous function $z\rightarrow\aMid{p\Par{z}}$ has a global min at some point $\zeta\in\Cbb.$\TextB{}
%\vspace{3pt}To show $p\Par{\zeta} = 0$, asum $p\Par{\zeta}\neq 0$. Define $q\in\PoCi$ by $q\Par{z}=\Frac{p\Par{z+\zeta}}{p\Par{\zeta}}.$\TextB{}
%\vspace{3pt}The function $z\rightarrow\aMid{q\Par{z}}$ has a global min value of $1$ at $z = 0$.\TextB{}
%\vspace{3pt}Write $q\Par{z} = 1 + a_k z^k + \dots + a_m z^m$, where $k\in\Nbp$ is the smallest suth $a_k\neq 0$.\TextB{}
%\vspace{3pt}Let $\beta\in\Cbb$ be suth $\beta^k=-\Frac{1}{a_k}$.\TextB{}
%\vspace{4pt}There is a const $c > 1$ so that if
%$t\in \Par{0, 1}$, then $\aMid{q\Par{t\beta}}\leqslant\aMid{1 + a_k t^k\beta^k}+t^{k+1}c = 1 - t^k \Par{1 - tc}$.\TextB{}
%Now letting $t=1/\Par{2c}$, we get $\aMid{q\Par{t\beta}}<1$. Ctradic. Hence $p\Par{\zeta} = 0$, as desired.\PfEnd
%\SepLine

\Anchor{4O4e2}\ProblemBnoor{4E 2}{
	\TextB{Prove if $w,z\in\Cbb,$ then $\aXMid{\aMid{w}-\aMid{z}}\leqslant\aMid{w-z}.$}
}$\aMidsq{w-z}=\Par{w-z}\Par{\overline{w}-\overline{z}}=\aMidsq{w}+\aMidsq{z}-2Re\Par{w\overline{z}}\geqslant\aMidsq{w}+\aMidsq{z}-2\aMid{w\overline{z}}=\aXMidsq{\aMid{w}-\aMid{z}}.$\vspace{2pt}\parSol{}
\Or $\aMid{w}=\aMid{w-z+z}\leqslant\aMid{w-z}+\aMid{z}\Rightarrow \aMid{w}-\aMid{z}\leqslant\aMid{w-z}.$\parSol{}
\Blind{\Or}$\aMid{z}=\aMid{z-w+w}\leqslant\aMid{z-w}+\aMid{w}\Rightarrow \aMid{z}-\aMid{w}\leqslant\aMid{w-z}.$\PfEnd
\SepLine

\ProblemN{\Anchor{4O5}{5}}{
	\TextA{Supp $m\in\Nbb,$ and $z_1,\dots,z_{m+1}$ are disti in $\Fbb,$ and $w_1,\dots,w_{m+1}\in\Fbb.$}
	\TextA{Prove $\exists\,!\,p\in\PoF{m},p\Par{z_k}=w_k$ for each $k\in\;\!\!\Bra{1,\dots, m+1}.$}
}\par\quad
Define $T\in\Lm[\BigPar]{\PoF{m},\FbbP{m+1}}$ by $Tq=\BigPar{q\Par{z_1},\dots,q\Par{z_m},q\Par{z_{m+1}}}.$\vspace{1pt}\par\quad
Becs $Tq=0\Rightarrow\Par{m+1}$ disti zeros for $q$ of deg no more than $m\Rightarrow q=0.$ \;Now $T$ iso.\PfEnd\vspace{4pt}\quad
\Or Let $p_1=1,\;p_k\Par{z}=\prod_{i=1}^{k-1}\Par{z-z_i}=\Par{z-z_1}\cdots\Par{z-z_{k-1}}$ for each $k\in\;\!\!\Bra{2,\dots,m+1}.$\vspace{1pt}\par\quad
By (2.C.10), $B_p=\Par{p_1,\dots,p_{m+1}}$ is a bss of $\PoF{m}.$ Let $B_e=\Par{e_1,\dots,e_{m+1}}$ be the std bss of $\FbbP{m+1}.$\vspace{2pt}\par\quad
Now $Tp_1=\Par{1,\dots,1},\;Tp_k=\XPar{\prod_{i=1}^{k-1}\Par{z_1-z_i},\dots,\prod_{i=1}^{k-1}\Par{z_j-z_i},\dots,\prod_{i=1}^{k-1}\Par{z_{m+1}-z_i}};$\vspace{3pt}\par\quad
{\normalsize$\begin{pmatrix}
		1 &\hspace{-6pt} 0 &\hspace{-6pt} 0 &\hspace{-6pt} \cdots &\hspace{-6pt} 0\\[-2pt]
		1 &\hspace{-6pt} A_{2,2} &\hspace{-6pt} 0 &\hspace{-6pt} \cdots &\hspace{-6pt} 0\\[-2pt]
		1 &\hspace{-6pt} A_{3,2} &\hspace{-6pt} A_{3,3} &\hspace{-6pt} \cdots &\hspace{-6pt} 0\\[-2pt]
		\vdots &\hspace{-6pt} \vdots &\hspace{-6pt} \vdots &\hspace{-6pt} \ddots &\hspace{-6pt} \vdots\\[-2pt]
		1 &\hspace{-6pt} A_{m+1,2} &\hspace{-6pt} A_{m+1,3} &\hspace{-6pt} \cdots &\hspace{-6pt} A_{m+1,m+1}
	\end{pmatrix}$}${}=\Mt[\BigPar]{T,B_p,B_e}.$ Where $A_{j,k}=\prod_{i=1}^{k-1}\Par{z_j-z_i}\neq 0$ for all $j> k-1\geqslant 1.$\vspace{-74pt}\par\quad
\hfill And $\prod_{i=1}^{k-1}\Par{z_j-z_i}=0 \Longleftrightarrow j\leqslant k-1,$ becs $z_1,\dots,z_{m+1}$ are disti.\Blind{\quad\qquad}\vspace{30pt}\par\quad
\hfill Now the rows of $\Mt{T}$ liney indep. By (4E 3.C.17) \OR (3.F.32).\Blind{\quad}\PfEnd\vspace{3pt}
\SepLine

%\Anchor{4O4}\ProblemN[]{{4}}{
	%	For $m,n\in\Nbp$ with $m\leqslant n,\;\lambda_1,\dots,\lambda_m\in\Fbb,$ let $p\Par{z}=\Par{z-\lambda_1}{^{n-\SmallPar{m-1}}}\Par{z-\lambda_2}\cdots\Par{z-\lambda_m}\Rightarrow\deg p=n.$\TextA{}
	%}\SepLine

%\ProblemN{\Anchor{4O2}{2}}{
	%	\TextA{Supp $m\in\Nbp$. Is the set $U=\zeroSubs\cup\Bra{p\in\PoFi:\deg p = m}$ a subsp of $\PoFi$?}
	%}$x^m,x^m+x^{m-1}\in U$ \,\,but\,\, $\Deg\Sbra{\Par{x^m+x^{m-1}}-\Par{x^m}}\neq m\Rightarrow \Par{x^m+x^{m-1}}-\Par{x^m}\not\in U.$\PfEnd
%\SepLine
%
%\ProblemN{\Anchor{4O3}{3}}{
	%	\TextA{Supp $m\in\Nbp$. Is the set $U=\zeroSubs\cup\Bra{p\in\PoFi:2\;\mmid\,\deg p}$ a subsp of $\PoFi$?}
	%}$x^2,x^2+x\in U$ \,\,but\,\, $\Deg\Sbra{\Par{x^2+x}-\Par{x^2}}$ is odd and hence $\Par{x^2+x}-\Par{x^2}\not\in U.$\PfEnd
%\SepLine

\Anchor{4OT2}\ProblemBX{\TipsN{2}}{
	\TextB{Supp non0 $p,q\in\PoFi$ are multi of each other. Prove $p=cq$ for a $c\neq0.$}
}Let $p=rq,q=sp\Rightarrow p=rsp\Rightarrow r\Par{z}s\Par{z}=1$ for all $z$ with $p\Par{z}\neq 0,$ while such $z$ is fini.\parSol{}
Thus $\Par{rs}\Par{z}=1$ for infily many $z,$ so for all $z.$ Now $\deg rs=1=\deg r+\deg s.$\PfEnd
\SepLine

\ProblemN{\Anchor{4O6}{6}}{
	\TextA{Supp non0 $p\in\PoC{m}$ has deg $m.$ Prove}
	\TextA{[P] $p$ has $m$ disti zeros $\Longleftrightarrow p$ and its deri $p\apostrophe$ have no common zeros. [Q]}
}(a) Supp $p$ of deg $m$ has $m$ disti zeros. By [4.14], $p\Par{z}=c\Par{z-\lambda_1}\cdots\Par{z-\lambda_m}.$\vspace{1pt}\parSol{\Ha}
If $m=0,$ then $p=c\neq 0\Rightarrow p$ has no zeros, and $p\apostrophe=0,$ done.\parSol{\Ha}
If $m=1,$ then $p\Par{z}=c\Par{z-\lambda_1},$ and $p\apostrophe=c$ has no zeros, done.\vspace{1pt}\parSol{\Ha}
For each $j\in\;\!\!\Bra{1,\dots,m}$, let $q_j\Par{z-\lambda_j}=p\Par{z}\Rightarrow q_j\Par{\lambda_j}\neq 0.$\vspace{1pt}\parSol{\Ha}
Now $p\apostrophe\Par{z}=\Par{z-\lambda_j}q_j\apostrophe\Par{z}+q_j\Par{z}\Rightarrow p\apostrophe\Par{\lambda_j}=q_j\Par{\lambda_j}\neq 0.$\vspace{4pt}\parSol{\Ha}
\Or ${}^{\neg}Q\Rightarrow{}{^\neg}P:$ \;Supp $p\Par{z}=\Par{z-\lambda}q\Par{z},\;p\apostrophe\Par{z}=\Par{z-\lambda}r\Par{z}.$\parSol{\Ha}
Becs $p\apostrophe\Par{z}=\Par{z-\lambda}q\apostrophe\Par{z}+q\Par{z}\Rightarrow p\apostrophe\Par{\lambda}=q\Par{\lambda}=0\Rightarrow q\Par{z}=\Par{z-\lambda}s\Par{z}.$\parSol{\Ha}
Now $p\Par{z}=\Par{z-\lambda}{^2}s\Par{z}.$ Hence $p$ has strictly less than $m$ disti zeros.\vspace{4pt}\parSol{}
(b) ${}^{\neg}P\Rightarrow{}{^\neg}Q:$ \;Becs $0\neq p\in\PoC{m}.$ Supp all disti zeros are $\lambda_1,\dots,\lambda_M,$ with $M<m.$\parSol{\Hb}
By Pigeon Hole Principle, $\Par{z-\lambda_k}{^{2}}q\Par{z}=p\Par{z}$ for some $\lambda_k\Rightarrow p\apostrophe\Par{\lambda_k}=0=p\Par{\lambda_k}.$\PfEnd\vspace{4pt}
\ANote If $\Fbb=\Rbb.$ Then replace ``$m$ disti zeros'' with ``$m$ disti zeros in $\Cbb$'' and the result still holds.
\SepLine

%\ProblemN{\Anchor{4O7}{7}}{
%	\TextA{Prove every $p\in\PoRi$ of odd deg has a zero.\hfill\tgnr\FontNorm Using [4.17], $\deg p=2M+m\Rightarrow m$ is odd.}
%}\Or Write $p\Par{x}=x^m\envFontSmall\XPar{${\FontLarge$\frac{a_0}{x^m}$}${}+{}${\FontLarge$\frac{a_1}{x^{m-1}}$}${}+\dots+{}${\FontLarge$\frac{a_{m-1}}{x}$}${}+a_m\envFontSmall}\Rightarrow p$ continuous on $\Interval{(}{]}{{-\infty,0}}\cup\Interval{[}{)}{0,\infty}.$\parSol{\vspace{3pt}}
%Let $\delta=\aMid{a_m}{^{-1}}a_m.$ \:\:Becs $\lim\limits_{x\rightarrow -\infty}p\Par{x}=-\delta\infty\,;\;\;\lim\limits_{x\rightarrow \infty}p\Par{x}=\delta\infty\Rightarrow p$ has at least one real zero.\PfEnd
%\SepLine
\vspace{2pt}
\ProblemN{\Anchor{4O8}{8}}{
	\TextA{Supp $p\in\PoRi.$ Let $\Par{Tp}\Par{x}={}${\FontNorm$\MathLeftBrace{l}{${\Large\envFontSmall[\footnotesize]\def\SmallPar{\Par}$\frac{p\SmallPar{x}\,-\,p\SmallPar{3}}{x\,-\,3}$}$\;$ if $x\neq 3,\\[4pt]p\apostrophe\Par{3}$\hfill if $x=3.}$} \;Show $Tp\in\PoRi.$\vspace{-10pt}}
}\par\quad
For $x\neq 3,T\Par{x^k}={}${\Large$\frac{x^k\,-\,3^k}{x\,-\,3}$}${}=\sum_{i=1}^k 3^{i-1}x^{k-i}.$ \;Still true for $x=3.$\par\quad
Each $T\Par{a_0+a_1x+\dots+a_mx^m}=a_1+\dots+a_k\sum_{i=1}^k3^{i-1}x^{k-i}+\dots+a_m\sum_{i=1}^m 3^{i-1}x^{m-i}\in\PoRi{}.$\PfEnd\vspace{6pt}\quad
\Or \NOTICE that \uline{$\exists\,!\,q\in\PoRi,$}$\:p\Par{x}-p\Par{3}=\Par{x-3}q\Par{x}.$ For $x\neq 3,\;q\Par{x}={}${\Large\envFontSmall[\footnotesize]\def\SmallPar{\Par}$\frac{p\SmallPar{x}\,-\,p\SmallPar{3}}{x\,-\,3}$}.\par\quad
$p\apostrophe\Par{x}=\BigPar{p\Par{x}-p\Par{3}}\apostrophe=q\Par{x}+\Par{x-3}q\apostrophe\Par{x}.$ For $x=3,\;p\apostrophe\Par{3}=q\Par{3}.$ Now $Tp=q.$\PfEnd
%Let each $q_k\Par{x}\Par{x-3}=p_k\Par{x}-p_k\Par{3}\Rightarrow Tp_k=q_k,$ for $k=1,2.$\par\quad
%Then $\Par{p_1+\lambda p_2}\Par{x}-\Par{p_1+\lambda p_2}\Par{3}=\Par{x-3}\Par{q_1+\lambda q_2}\Par{x}.$ \,By the uniqnes of $q_1+\lambda q_2.$\PfEnd
\SepLine

%\ProblemN{\Anchor{4O9}{9}}{
%	\TextA{Supp $p\in\PoCi$. Define $q:\Cbb\rightarrow\Cbb\,$ by $q\Par{z} = p\Par{z}\overline{p\Par{\overline z}}$. Prove $q\in\PoRi$.\vspace{1pt}}
%}By [4.5], $\overline z{^n}=\overline{z^n}.$ \;For any $f\Par{z}=a_nz^n+\dots+a_1z+a_0,\;\overline{f\Par{\overline{z}}}=\overline{a_n}z^n+\dots+\overline{a_1}z+\overline{a_0}.$\parSol{}
%Becs $q\Par{z}=p\Par{z}\overline{p\Par{\overline z}}=\overline{p\Par{\overline z}}p\Par{z}=\overline{q\Par{\overline z}}.$ \;Each $c_k=\overline{c_k}\Rightarrow c_k\in\Rbb.$\PfEnd
%\vspace{6pt}\parSol{}
%\Or Becs $q\Par{z}=p\Par{z}\overline{p\Par{\overline z}}=\sum_{k=0}^{2n}\XPar{{\sum_{i+j=k}c_i\overline{c_j}}}\,z^k.$ For each $k\in\;\!\!\Bra{0,\dots,2n},$\vspace{0pt}\parSol{}
%$\overline{\sum_{i+j=k}c_i\overline{c_j}}=\sum_{i+j=k}\overline{c_i\overline{c_j}}=\sum_{i+j=k}{c_j}\overline{c_i}=\sum_{i+j=k}c_i\overline{c_j}\in\Rbb.$\PfEnd
%\SepLine

%\ProblemN{\Anchor{4O10}{10}}{
%	\TextA{Supp disti $x_0,x_1,\dots,x_m\in\Rbb,$ and $p\in\PoC{m}$ suth each $p\Par{x_k}\in\Rbb.$ Prove $p\in\PoRi$.}
%}By {\TIPSN{1}} and Exe (5), $\exists\,!\,q\in\PoR{m}$ suth $q\Par{x_k}=p\Par{x_k}.$ Hence $p=q.$\PfEnd
%\vspace{11pt}\quad
%\Or Define $q\Par{x}=\sum_{j=0}^m{}${\Large\envFontSmall[\footnotesize]\def\SmallPar{\Par}$\frac{\SmallPar{x\,-\,x_0}\SmallPar{x\,-\,x_1}\,\cdots\,\SmallPar{x\,-\,x_{j-1}}\SmallPar{x\,-\,x_{j+1}}\,\cdots\,\SmallPar{x\,-\,x_m}}{\SmallPar{x_j\,-\,x_0}\SmallPar{x_j\,-\,x_1}\,\cdots\,\SmallPar{x_j\,-\,x_{j-1}}\SmallPar{x_j\,-\,x_{j+1}}\,\cdots\,\SmallPar{x_j\,-\,x_m}}$}${\,}p\Par{x_j}.$\par\vspace{4pt}\quad
%又 Each $x_j,\,p\Par{x_j}\in\Rbb\Rightarrow q\in\PoR{m}.$ \;Becs each $q\Par{x_k}=1\cdot p\Par{x_k}\Rightarrow\Par{q-p}\Par{x_k}=0.$\par\quad
%$\Par{q-p}$ has $\Par{m+1}$ zeros. By {\TIPSN{1}}, $q-p=0\Rightarrow p=q\in\PoRi.$\PfEnd
%\SepLine

\ProblemN{\Anchor{4O11}{11}}{
	\TextA{Supp $p\in\PoFi$ with $\deg p=m\in\Nbb.$ Let $U=\Bra{pq:q\in\PoFi}$. Find a bss of $\PoFi\XSlash U.$}
	%	\TextA{{\tgnr\large(a)} Show $\dim\PoFi\XSlash U = \deg p$; \; {\tgnr\large(b)} Find a bss of $\PoFi\XSlash U$.}
}If $\deg p=0,$ then $U=\PoFi,\PoFi\XSlash U=\Bra{0+U},$ with the uniq bss $\Par{}.$ \;Supp $\deg p\geqslant 1.$\par\vspace{1pt}\quad
Becs $\forall s\in\PoFi,\exists\,!\,r\in\PoF{m-1},q\in\PoFi\Rightarrow\exists\,!\,pq\in U,\;s=\Par{p}q+\Par{r}\Rightarrow{}$\uline{$\PoFi=U\oplus\PoF{m-1}$}.\PfEnd
\SepLine

\Anchor{4OL1}\ProblemN{L1}{
	\TextB{Prove $\forall p,q\in\PoFi,k\in\Nbp,\Par{pq}{^{\SmallPar{k}}}=\mathC_{k}^k p{^{\SmallPar{k}}}q{^{\SmallPar{0}}}+\dots+\mathC_{k}^j p{^{\SmallPar{j}}}q{^{\SmallPar{k-j}}}+\dots+\mathC_{k}^0 p{^{\SmallPar{0}}}q{^{\SmallPar{k}}}.$\vspace{2pt}}
}We use induc on $k\in\Nbp.$ \;(i) $k=1.$ $\Par{pq}{^{\SmallPar{1}}}=\Par{pq}\apostrophe=C_{1}^1 p{^{\SmallPar{1}}}q{^{\SmallPar{0}}}+C_{1}^0 p{^{\SmallPar{0}}}q{^{\SmallPar{1}}}.$ \;(ii) $k\geqslant 2.$\vspace{2pt}\par\quad
Asum for $\Par{pq}{^{\SmallPar{k-1}}}=\mathC_{k-1}^{k-1} p{^{\SmallPar{k-1}}}q{^{\SmallPar{0}}}+\dots+\mathC_{k-1}^j p{^{\SmallPar{j}}}q{^{\SmallPar{k-1-j}}}+\dots+\mathC_{k-1}^0 p{^{\SmallPar{0}}}q{^{\SmallPar{k-1}}}.$\vspace{4pt}\par\quad
Now $\Par{pq}{^{\SmallPar{k}}}=\BigPar{\Par{pq}{^{\SmallPar{k-1}}}}\apostrophe=\XPar{{\sum_{j=0}^{k-1}\mathC_{k-1}^{j} p{^{\SmallPar{j}}}q{^{\SmallPar{k-j-1}}}}}\apostrophe=\sum_{j=0}^{k-1}\XSbra{\mathC_{k-1}^{j} \XPar{p{^{\SmallPar{j+1}}}q{^{\SmallPar{k-j-1}}}+p{^{\SmallPar{j}}}q{^{\SmallPar{k-j}}}}}.$\vspace{4pt}\par\quad
\Blind{Now} $\Blind{\Par{pq}{^{\SmallPar{k}}}\!}=\XSbra{\mathC_{k-1}^{0}\XPar{ \uwave{p{^{\SmallPar{1}}}q{^{\SmallPar{k-1}}}}+\boxed{p{^{\SmallPar{0}}}q{^{\SmallPar{k}}}}}}+\XSbra{\mathC_{k-1}^1\XPar{p{^{\SmallPar{2}}}q{^{\SmallPar{k-2}}}+\uwave{p{^{\SmallPar{1}}}q{^{\SmallPar{k-1}}}}}}$\vspace{4pt}\par\quad
\Blind{Now} $\Blind{\Par{pq}{^{\SmallPar{k}}}=}+\dots+\XSbra{\mathC_{k-1}^{j-2}\XPar{ \uline{p{^{\SmallPar{j-1}}}q{^{\SmallPar{k-j+1}}}}+p{^{\SmallPar{j-2}}}q{^{\SmallPar{k-j+2}}}}}+\XSbra{\mathC_{k-1}^{j-1}\XPar{ \uuline{p{^{\SmallPar{j}}}q{^{\SmallPar{k-j}}}}+\uline{p{^{\SmallPar{j-1}}}q{^{\SmallPar{k-j+1}}}}}}$\vspace{4pt}\par\quad
\Blind{Now} $\Blind{\Par{pq}{^{\SmallPar{k}}}=}+\XSbra{\mathC_{k-1}^{j}\XPar{ \uwave{p{^{\SmallPar{j+1}}}q{^{\SmallPar{k-j-1}}}}+\uuline{ p{^{\SmallPar{j}}}q{^{\SmallPar{k-j}}}}}}+\XSbra{\mathC_{k-1}^{j+1}\XPar{p{^{\SmallPar{j+2}}}q{^{\SmallPar{k-j-2}}}+\uwave{p{^{\SmallPar{j+1}}}q{^{\SmallPar{k-j-1}}}}}}$\vspace{4pt}\par\quad
\Blind{Now} $\Blind{\Par{pq}{^{\SmallPar{k}}}=}+\dots+\XSbra{\mathC_{k-1}^{k-2}\XPar{ \uline{p{^{\SmallPar{k-1}}}q{^{\SmallPar{1}}}}+p{^{\SmallPar{k-2}}}q{^{\SmallPar{2}}}}}+\XSbra{\mathC_{k-1}^{k-1}\XPar{ \boxed{p{^{\SmallPar{k}}}q{^{\SmallPar{0}}}}+\uline{p{^{\SmallPar{k-1}}}q{^{\SmallPar{1}}}}}}.$\vspace{4pt}\par\quad
Hence $\Par{pq}{^{\SmallPar{k}}}=\mathC_{k}^0{p{^{\SmallPar{0}}}q{^{\SmallPar{k}}}}+\dots+\XSbra{\mathC_{k-1}^{j}+\mathC_{k-1}^{j-1}}\BigPar{p{^{\SmallPar{j}}}q{^{\SmallPar{k-j}}}}+\dots+\mathC_{k}^{k}{p{^{\SmallPar{k}}}q{^{\SmallPar{0}}}}.$\PfEnd
\SepLine

\Anchor{4OL2}\ProblemN{L2}{
	\TextB{Supp $\alpha\in\Nbp$ suth $p\Par{z}=\Par{z-\lambda}{^\alpha}q\Par{z}.$ Prove ${p{^{\SmallPar{\alpha-1}}}}\Par{\lambda}=0.$}
}$\Sbra{\Par{z-\lambda}{^\alpha}q\Par{z}}{^{\SmallPar{\alpha-1}}}=\sum_{j=1}^{\alpha-1}\mathC_{\alpha-1}^j\Sbra{\Par{z-\lambda}{^\alpha}}{^{\SmallPar{j}}}\:\!\Sbra{q\Par{z}}{^{\SmallPar{\alpha-1-j}}}.$\parSol{}
Note that $\Sbra{\Par{z-\lambda}{^\alpha}}{^{\SmallPar{j}}}=\alpha\Par{\alpha-1}\cdots\Par{\alpha-j+1}\cdot\Par{z-\lambda}{^{\SmallPar{\alpha-j}}}.$\PfEnd
\SepLine

\Anchor{4O4e13}\ProblemBnoor{4E 13}{
	\TextA{Supp nonC $p,q\in\PoCi$ have no common zeros. Let $m = \deg p$, $n = \deg q$.\vspace{1pt}}
	\TextB{Define $T:\PoC{n-1}\times\PoC{m-1}\rightarrow\PoC{m+n-1}$ by $T\Par{r, s} = rp + sq.$ Prove $T$ is inje.\vspace{3pt}}
	\ACoro \TextB{$\exists\,!\,r\in\PoC{n-1},s\in\PoC{m-1}$ suth $rp + sq = 1.$}
}Immed, $T$ is liney. \;Supp $T\Par{r,s}=rp+sq=0.$\par\quad
Then $rp=-sq.$ Becs $p,q$ are \uline{coprime} $\Rightarrow p\:\mmid\:s,$ while $\deg s\leqslant m-1\Rightarrow s=0\Rightarrow r=0.$\PfEnd\vspace{6pt}\quad
\Or Let $\lambda_1,\dots,\lambda_M$ and $\mu_1,\dots,\mu_N$ be the disti zeros of $p$ and $q$ respectly. \NOTICE that $M\leqslant m,N\leqslant n.$\vspace{2pt}\par\quad
By the ctrapos of [4.13], $M=0\Longleftrightarrow m=0\Rightarrow s=0\Longleftrightarrow r=0 \Leftarrow n=0\Longleftrightarrow N=0.$\vspace{2pt}\par\quad
Now supp $M,N\geqslant 1.$ We show $s=0.$ \,{\FontSmall Simlr for $r=0.$ \Or $s=0\Rightarrow r=0.$}\vspace{2pt}\par\quad
Write $p\Par{z}=a\Par{z-\lambda_1}{^{\alpha_1}}\cdots\Par{z-\lambda_M}{^{\alpha_M}}.$ \BigPar{ $\exists\,!\,\alpha_j\geqslant 1,a\in\Fbb.$ } Let $\max\!\Bra{\alpha_1,\dots,\alpha_M}=A=\alpha_L.$\vspace{2pt}\par\quad
For each $D\in\;\!\!\Bra{0,1,\dots,A-1},$ let $I_{>D}=\Bra{{I_{D,1}},\dots,{I_{D,\,J_D}}}$ be suth each $\alpha\Sbra{I_{D,j}}=\alpha_{I_{D,j}}\geqslant D+1.$\vspace{2pt}\par\quad
Now $\Bra{L}=I_{>A-1}\subseteq \cdots\subseteq I_{>0}=\Bra{1,\dots,M}.$ Becs $rp+sq=0\Rightarrow\Par{rp+sq}{^{\SmallPar{k}}}=0$ for all $k\in\Nbp.$\vspace{2pt}\par\quad
We use induc on $D$ to show $s{^{\SmallPar{D}}}\BigPar{\lambda\Sbra{I_{D,j}}}=0$ for each $D\in\;\!\!\Bra{0,\dots,A-1}.$\vspace{2pt}\par\quad
\NOTICE that $p{^{\SmallPar{D}}}\BigPar{\lambda\Sbra{I_{D,j}}}=0$ for each $D\in\;\!\!\Bra{0,\dots,A-1}$ and each $I_{D,j}\in I_{>D}.$\hfill(L2)\vspace{2pt}\par\quad
(i) $D=0.$ Each $\Par{rp+sq}\BigPar{\lambda\Sbra{I_{0,j}}}=\Par{sq}\BigPar{\lambda\Sbra{I_{0,j}}}=s\BigPar{\lambda\Sbra{I_{0,j}}}=0.$ Where $q\BigPar{\lambda\Sbra{I_{0,j}}}\neq0.$\vspace{2pt}\par\quad\Hi
$D=1.$ Each $\Par{r\apostrophe p+rp\apostrophe}\BigPar{\lambda\Sbra{I_{1,j}}}+\Par{s\apostrophe q+sq\apostrophe}\BigPar{\lambda\Sbra{I_{1,j}}}=\Par{s\apostrophe q}\BigPar{\lambda\Sbra{I_{1,j}}}=s\apostrophe\BigPar{\lambda\Sbra{I_{1,j}}}=0.$\vspace{1pt}\par\quad\Hi
\Blind{$D=1.$} Where $p\apostrophe\BigPar{\lambda\Sbra{I_{1,j}}}=0,$ and each $I_{1,j}\subseteq I_{0,j}\Rightarrow s\BigPar{\lambda\Sbra{I_{1,j}}}=0.$\vspace{4pt}\par\quad\Endi
(ii) $2\leqslant D\leqslant A-1.$ Asum $\;s{^{\SmallPar{d}}}\BigPar{\lambda\Sbra{I_{d,j}}}=0$ for each $d\in\;\!\!\Bra{0,1,\dots,D-1}$ and each $\lambda\Sbra{I_{d,j}}\in I_{>d}.$\vspace{2pt}\par\quad\Hii
Each $\Sbra{rp+sq}{^{\SmallPar{D}}}\BigPar{\lambda\Sbra{I_{D,j}}}=\Sbra{\mathC_{D}^D r{^{\SmallPar{D}}}p{^{\SmallPar{0}}}+\dots+\mathC_{D}^d r{^{\SmallPar{d}}}p{^{\SmallPar{D-d}}}+\dots+\mathC_{D}^0 r{^{\SmallPar{0}}}p{^{\SmallPar{D}}}}\BigPar{\lambda\Sbra{I_{D,j}}}$\hfill(L1)\vspace{4pt}\par\quad\Hii
\Blind{Each} $\Blind{\Sbra{rp+sq}{^{\SmallPar{D}}}\BigPar{\lambda\Sbra{I_{D,j}}}=}\;\!+\Sbra{\mathC_{D}^D s{^{\SmallPar{D}}}q{^{\SmallPar{0}}}+\dots+\mathC_{D}^d s{^{\SmallPar{d}}}q{^{\SmallPar{D-d}}}+\dots+\mathC_{D}^0 s{^{\SmallPar{0}}}q{^{\SmallPar{D}}}}\BigPar{\lambda\Sbra{I_{D,j}}}$\vspace{4pt}\par\quad\Hii
\Blind{Each} $\Blind{\Sbra{rp+sq}{^{\SmallPar{D}}}\BigPar{\lambda\Sbra{I_{D,j}}}}=\Sbra{\mathC_{D}^D s{^{\SmallPar{D}}}q{^{\SmallPar{0}}}}\BigPar{\lambda\Sbra{I_{D,j}}}.$\; Where each $\lambda\Sbra{I_{D,j}}\in I_{>D}\subseteq I_{D-1,\alpha}.$\vspace{4pt}\par\quad\Hii
Hence $s{^{\SmallPar{D}}}\BigPar{\lambda\Sbra{I_{D,j}}}=0.$ The asum holds for all $D\in\;\!\!\Bra{0,\dots,A-1}.$\vspace{6pt}\par\quad
\NOTICE that $\forall k=\Bra{0,\dots,A-2},s{^{\SmallPar{k}}}$ and $s{^{\SmallPar{k+1}}}$ have zeros $\BigBra{{\lambda\Sbra{I_{k+1,1}},\dots,\lambda\Sbra{I_{k+1,\,J_{k+1}}}}}$ in common.\vspace{2pt}\par\quad
Now $\forall D\in\;\!\!\Bra{1,\dots,A-1},s=s{^{\SmallPar{0}}},\dots,s{^{\SmallPar{D}}}$ have zeros $\BigBra{{\lambda\Sbra{I_{D,1}},\dots,\lambda\Sbra{I_{D,\,J_D}}}}$ in common.\vspace{2pt}\par\quad
Thus $s\Par{z}$ is divisible by $\BigPar{z-\lambda\Sbra{I_{D,1}}}{^{\alpha\def\envFont{\scriptsize}\Sbra{I_{D,1}}}}\cdots\BigPar{z-\lambda\Sbra{I_{D,\,J_D}}}{^{\alpha\def\envFont{\scriptsize}\Sbra{I_{D,\,J_D}}}},$ for each $D\in\;\!\!\Bra{0,\dots,A-1}.$\vspace{2pt}\par\quad
Hence $s\Par{z}=\Sbra{\Par{z-\lambda_1}{^{\alpha_1}}\cdots\Par{z-\lambda_M}{^{\alpha_M}}}\,s_0\Par{z},$ while $\deg s<m=\alpha_1+\dots+\alpha_M.$ Now by {\TIPSN{1}}.\PfEnd
\SepLine
\ChEnd

\textsl{\normalsize 凭借我的经验,我认为,好的自学教材,除了提供足够的一级知识外,还能通过各种方式,将二级、三级知识顺理成章地经过学科思维的浓缩喻于习题或课文中。在LADR的熏陶下,我渐渐认为,自学教材带来的长期收益更重要——所谓素养一类的隐形东西,无论堆砌多少知识记忆都难以学到;外在的选拔,表面上看都是知识竞赛,但真正有含金量的选拔,往往十二分地注重隐性能力;实际的工作表现也是如此;客观上看这确实是在当今公共信息过剩的时代下人与人拉开差距的核心原因之一,也是我相信最能仅通过自身努力耕耘获得长期稳定回报的地方。现在看看速学速成应付选拔竞争的选择有多愚蠢吧:考不上放弃吧,因为几乎没有习得那些隐性能力,就确实是除了知识和解题技巧之外啥也没得到,这些知识中实用的那些内容怎么着都能学到,不具有不可替代性,实际工作更需要隐形的素养;再考再战吧,就得辛苦刷题,总归不如在学的过程中把\!“和习题的挣扎”\!当作练习对学科思维的启发最好。考上了吧又要和更\!“拔尖创新人才”\!竞争隐形的能力,一样难以优胜,只不过这个情况下可以做一个更\!“优越”\!的平庸之人罢了,除了短期速成而来的外在\!“纪念品”\!之外再也没有什么学习成果可长期变现——和质量至上、\!不怕耽误时间进度的学习者相比又能有什么优越之处呢?}\par\vspace{4pt}

{\small
此章核心内容3/4e差距过大。4e将第2章线性相关性引理和多项式结合,更自然地引出原来3e的8.C节的极小多项式,并前置了相关习题,让定理和习题更加富有动机和系统性。这份笔记主要面向3e纸质书的读者,所以题号和定理索引都采用3e(除4e新增章节)。为了严密性,我决定将3e第8章提前到第5章后,对应到4e只有第8章前三节。\par\vspace{14pt}
}\pagebreak

\ChDecl{Ch5A}{5.A}{\qquad{\small\textbf{注意\,:}\;这里将5.B节多项式作用于算子部分与5.C节的本征空间的定义前置.}}

\vspace{4pt}

%\Anchor{5AN5.6}\BulletPointX\NoteForSmall{[5.6]}\;\;If $V$ is infinide. Then (a) $\Longleftrightarrow$ (b) $\Rightarrow$ (d), while (b) $\notRightarrow$(c), and (b) $\notRightarrow$(d).\par
%\BulletPointX\AComm $\lambda$ not an eigval of $T\Longleftrightarrow{T-\lambda I}$ inje $\Longleftrightarrow$ inv, if finide.
%\SepLine

\Anchor{5AT1}\ProblemBX{\TipsN{1}}{
	\TextA{Supp $V=U\oplus W$ and $U,W$ invard $T\in\Lm{V}.$ Prove $\null T\mmid_U\oplus\null T\mmid_W=\null T.$}
}$\forall v=u+w\in\null T, Tv=Tu+Tw=0\Rightarrow Tu,Tw=0\Rightarrow v\in\null T\mmid_U\oplus\null T\mmid_W.$\PfEnd
\ACoro $E\Par{\lambda,T}=E\Par{\lambda,T\mmid_U}\oplus E\Par{\lambda,T\mmid_W}.$ Replace $T$ with $T-\lambda I,$ immed.
\SepLine

\Anchor{5ANE2.3}\ProblemBX[]{\NoteForSmall{Exe (2, 3)}}{
	$ST=TS\Rightarrow p\Par{S}\;\!q\Par{T}=q\Par{T}\;\!p\Par{S}.$ And $\nullp q\Par{T},\rangep q\Par{T}$ invard $p\Par{S}.$\TextB{\vspace{-3pt}}
}\SepLine

\Anchor{5E1}\ProblemBnoor{5.E.1}{
	\TextA{Give $S,T\in\FbbP{4}$ suth $ST=TS$ while $\exists$ invarspd $S$ but not $T,$ invarspd $T$ but not $S$.}
}Define $S:\Par{x,y,z,w}\mapsto\Par{y,x,0,0}$ and $T:\Par{x,y,z,w}\mapsto\Par{0,0,w,z}\Rightarrow TS=ST=0.$\parSol{}
Thus ${e_1,e_2}$ are eigvecs of $T$ but not of $S,$ and ${e_3,e_4}$ are eigvecs of $S$ but not of $T.$
\SepLine

%\ProblemN{\Anchor{5A6}{6}}{
%	\TextA{Supp $U$ is invarsp of non0 $V$ under any $T\in\Lm{V}.$ Show $U=V$ or $\zeroSubs$.}
%}We show the ctrapos: {\tgsl Supp $U\neq\zeroSubs$ and $U\neq V.$ Prove $\exists\,T\in\Lm{V},\,U$ is not invard $T$.}\parSol{}
%Let $W\oplus U=V.$ Define $T\in\Lm{V}$ by $T\Par{u+w}=w.$\PfEnd
%\SepLine

\ProblemN{\Anchor{5A10}{10}}{
	\TextA{Define $T\in\Lm{\FbbP{n}}$ by $T\Par{x_1,x_2,x_3,\dots,x_n}=\BigPar{x_1,2 x_2,3 x_3,\dots,n x_n}.$}
	\TextA{Find all invarsps of $V$ under $T.$\hfill\FontNorm\tgnr Let $\Par{e_1,\dots,e_n}$ be the std bss of $\FbbP{n}.$ Let each $E_k=\Span{e_k}.$}
}The eigvals are $\Bra{1,\dots,n}$ of len $\dim\FbbP{n}.$ The set of all eigvecs is $\Par{E_1\cup\cdots\cup E_n}\nonzero[\Big].$\parSol{}
Supp $U$ is invarsp. Then $u=\Par{x_1,x_2,\dots,x_n}\in U\Rightarrow Tu=\Par{x_1,2x_2,\dots,x_n}\in U.$\parSol{}
And $Tu-u=\BigPar{0,x_2,2x_3,\dots,\Par{n-1}x_n}\in U\Rightarrow\cdots\Rightarrow\Par{0,\dots,0,x_n}\Rightarrow$ each $x_ke_k\in U.$\parSol{}
Get a $B_U$ and pick/count all possible non0 $x_k.$ \,Forming $\Span{e_{k_1},\dots,e_{k_m}}=U.$\PfEnd\vspace{2pt}
\AComm The result (b) holds generally where $\exists\,B_V$ consists of eigvecs of $T.$
\SepLine

\ProblemB[]{
	\TextB{Supp $T\in\Lm{V},\lambda_1,\dots,\lambda_m$ are the disti eigvals corres $v_1,\dots,v_m,$ and $U$ invarspd $T.$}
}
\Anchor{5AT2}\ProblemBX{\TipsN{2}}{
	\TextA{Supp $v_1+\dots+v_m\in U.$ Prove each $v_k\in U.$\vspace{-2pt}}
}Consider the stmt $P\Par{k}:$ if $v_1+\dots+v_k\in U,$ then each $v_j\in U.$\parSol{}
(i) $v_1\in U.$ $P\Par{1}$ holds. (ii) For $2\leqslant k\leqslant m.$ Asum $P\Par{k-1}$ holds. Supp $v=v_1+\dots+v_k\in U.$\parSol{}
Then $Tv=\lambda_1v_1+\dots+\lambda_k v_k\in U\Longrightarrow Tv-\lambda_k v=\Par{\lambda_1-\lambda_k}v_1+\dots+\Par{\lambda_{k-1}-\lambda_k}v_{k-1}\in U.$\parSol{}
For each $j\in\;\!\!\Bra{1,\dots,k-1},\lambda_j-\lambda_{k}\neq 0\Rightarrow\Par{\lambda_j-\lambda_k}v_j=v_j'$ is an eigvec of $T$ corres $\lambda_j.$\parSol{}
By asum, each $v_j'\in U.$ Thus $v_1,\dots,v_{k-1}\in U.$ So that $v_k=v-v_1-\dots-v_{k-1}\in U.$\PfEnd
\SepLine[0pt][\Blind{\BulletPointX} ]

\Anchor{5AT3}\Anchor{5C4e16}\ProblemBX{\TipsN{3}}{
	\TextA{Supp $V=E\Par{\lambda_1,T}\oplus\cdots\oplus E\Par{\lambda_m,T}.$ Prove $U=E\Par{\lambda_1,T\mmid_U}\oplus\cdots\oplus E\Par{\lambda_m,T\mmid_U}.$}
}Becs $\forall u\in U,\exists\,!\,v_j\in E\Par{\lambda_j,T},v=v_1+\dots+v_m.$ By \TIPSN{2}, each $v_j\in U.$\PfEnd%%\vspace{2pt}\parSol{}
%----------\Or \Sbra[3pt]{{\tgsl Req Finide}} \;Using induc on $\dim U=N.$ (i) $N=1.$ Immed. (ii) $N>1.$ Asum it holds for smaller $U.$\parSol{}
\SepLine

\Anchor{5AT4}\BulletPointX\TipsN{4}\,\,\,Supp $T\in\Lm{\Rbb^2}$ is the countclockws rotat by $\theta\in\Rbb.$ Define $\mC\Par{a,b}=a+\i b.$\TextB{}
Becs $\Par{{\cos\theta+\i\sin\theta}}\Par{a+\i b}=r\BigPar{{\cos\!\Par{\alpha+\theta}+\i\sin\!\Par{\alpha+\theta}}}.$\vspace{-6pt}\TextB{}
Hence $T\Par{a,b}=\BigPar{a\cos\theta-b\sin\theta,\,a\sin\theta+b\cos\theta}.$ \:Now $\Mt{T}={}${\normalsize$\begin{pmatrix}\cos\theta & -\sin\theta\\[-4pt]\sin\theta & \Blind{-}\cos\theta\end{pmatrix}$}.\vspace{4pt}
\SepLine

%\ProblemN{\Anchor{5A8}{8}}{
	%	\TextA{Define $T\in\Lm{\FbbP{2}}$ by $T\Par{w, z} = \Par{z, w}$. Find all eigvals and eigvecs.}
	%}Supp $\lambda$ is an eigval with an eigvec $\Par{w,z}.$ Then $z=\lambda w$ \,{\small and}\, $ w=\lambda z.$\parSol{}
%Thus $z=\lambda^2 z\Rightarrow \lambda^2=1,$ ignoring the possibility of $z=0$ \Par{ $z=0\Longleftrightarrow w=0$ }.\parSol{}
%Hence $\lambda_1=-1$ and $\lambda_2=1$ are all the eigvals of $T.$ And $T\Par{z,z}=\Par{z,z},T\Par{z,-z}=\Par{{-z,z}}.$\parSol{}
%又 $\dim\FbbP{2}=2.$ Thus the set of all eigvecs is $\Bra{\Par{z,z},\Par{z,-z}:z\neq 0}.$\PfEnd
%\SepLine
%
%\ProblemN{\Anchor{5A9}{9}}{
	%	\TextA{Define $T\in\Lm{\FbbP{3}}$ by $T\Par{z_1,z_2,z_3}=\Par{2z_2, 0,5z_3}$. Find all eigvals and eigvecs.}
	%}Supp $\lambda$ is an eigval with an eigvec $\Par{z_1,z_2,z_3}.$\parSol{}
%Then $\Par{2z_2, 0,5z_3}=\lambda\Par{z_1,z_2,z_3}.$ We discuss in two cases:\parSol{}
%For $\lambda=0,$\, $z_2=z_3=0$ and $z_1$ can be arb \Par{ $z_1\neq 0$ }.\parSol{}
%For $\lambda\neq 0,$\, $z_2=0=z_1,$ and $z_3$ can be arb \Par{ $z_3\neq 0$ }, then $\lambda=5$.\parSol{}
%The set of all eigvecs is $\Bra{\Par{0,0,w},\Par{w,0,0}:w\neq 0}.$\PfEnd
%\SepLine

%\ProblemN{\Anchor{5A18}{18}}{
%	\TextA{Define $T\in\Lm{\FbbP{\infty}}$ by $T\Par{z_1,z_2,\cdots}=\BigPar{0, z_1,z_2,\cdots}.$ Show $T$ has no eigvals.}
%}Supp $z_k\neq 0$ and $T\Par{z_1,z_2,\cdots}=\Par{\lambda z_1,\lambda z_2,\cdots}=\Par{0,z_1,z_2,\cdots}.$ Thus $\lambda z_1=0,\,\lambda z_k=z_{k-1}.$\parSol{}
%($-$) $\lambda=0\Rightarrow\lambda z_2=z_1=0=\dots=z_k.$ \, ($=$) $\lambda\neq 0\Rightarrow z_1=0\Rightarrow z_2=\dots=z_k=0.$\PfEnd
%\SepLine

\ProblemN{\Anchor{5A19}{19}}{
	\TextA{Supp $n\in\Nbp$. Define $T\in\Lm{\FbbP{n}}$ by $T\Par{x_1,\dots,x_n}=\BigPar{x_1+\dots+x_n,\cdots,x_1+\dots+x_n}.$}
	\TextA{Find all eigvals and eigvecs of $T.$\vspace{-2pt}}
}Supp $x_k\neq 0$ and $T\Par{x_1,\dots,x_n}=\Par{\lambda x_1,\dots,\lambda x_n}=\BigPar{x_1+\dots+x_n,\dots,x_1+\dots+x_n}.$\parSol{}
Then (I) $\lambda=0\Rightarrow x_1+\dots+x_n=0.$ \,If $n>1,$ then $\lambda=0$ is eigval; othws not, becs $T=I.$\parSol{}
\Blind{Then }(II) $\lambda\neq 0\Rightarrow x_1=\dots=x_n\Rightarrow\lambda x_k=n x_k.$ Now $n$ is eigval.\PfEnd\vspace{2pt}\parSol{}
To show no other eigvals. \;Note that $\range T=\Bra{\Par{x,\dots,x}\in\FbbP{n}}$ of dim $1.$ By Exe (29).\PfEnd\parSol{}
\Or Supp $n>1,$ and non0 $x\neq0$ with $Tx=\lambda x.$ Becs $\exists\,!\,\alpha\in\range T,\lambda x=\alpha.$\parSol{}
Now $x$ corres $\lambda$ and $\alpha$ corres $n$ are liney dep. By the ctrapos of [5.10], $\lambda=n.$\PfEnd
\SepLine

\ProblemN{\Anchor{5A20}{20}}{
	\TextA{Define $S\in\Lm{\FbbP{\infty}}$ by $S\Par{z_1,z_2,z_3,\cdots} = \Par{z_2,z_3,\cdots}.$}
	\TextA{Show every elem of $\Fbb$ is an eigval of $S$, and find all eigvecs of $S$.\vspace{-2pt}}
}Supp $z_k\neq 0$ and $S\Par{z_1,z_2,\dots}=\Par{\lambda z_1,\lambda z_2,\dots}=\Par{z_2,z_3,\dots}.$ Then each $\lambda z_k=z_{k+1}.$\parSol{}
(I) $\lambda=0\Rightarrow$ each $z_k=\dots=z_2=\lambda z_1=0.$ \;Thus $E\Par{0,S}=\Span{e_1}.$\parSol{}
(II) $\lambda\neq 0\Rightarrow\lambda^k z_1=\lambda^{k-1} z_2=\dots=\lambda z_k=z_{k+1}.$ \;Thus $E\Par{\lambda,S}=\Span[\Sbra]{\Par{1,\lambda^1,\cdots,\lambda^k,\cdots}}.$\PfEnd
\SepLine


%\ProblemN{\Anchor{5A12}{12}}{
%	\TextA{Define $T\in\Lm[\BigPar]{\PoR{n}}$ by $\Par{Tp}\Par{x} = xp\apostrophe\Par{x}$ for all $x\in\Rbb.$ Find all eigvals and eigvecs.}
%}Supp $p\neq 0$ and $\Par{Tp}\Par{x}=xp\apostrophe\Par{x}=\lambda p\Par{x}.$ Define an iso $S\Par{a_0,a_1,\dots,a_n}=a_0+a_1x+\dots+a_nx^n.$\parSol{}
%Let $p=S\Par{a_0,a_1,\dots,a_n}\Rightarrow xp\apostrophe\Par{x}=S\Par{a_1,2a_2,\dots,na_n}=\Par{\lambda a_0,\lambda a_1,\dots,\lambda a_n}.$\parSol{}
%Now $S^{-1}TS:\Par{x_0,x_1,\dots,x_n}\mapsto\Par{0x_0,1x_1,2x_2,\dots,nx_n}.$ Simlr to Exe (10).\PfEnd
%\SepLine

\ProblemB[]{
	\TextB{Supp $V$ is finide, $T\in\Lm{V},\lambda\in\Fbb.$}
}
\ProblemN{\Anchor{5A13}{13}}{
	 \TextB{Prove $\exists\,\alpha\in\Fbb,\aMid{\alpha-\lambda}<\frac{1}{\;1000\;}$ suth $\Par{T-\alpha I}$ is inv.}
}Let each $\aMid{\alpha_k-\lambda}=\frac{1}{\;1000\,+\,k\;},$ where $k\in\;\!\!\Bra{1,\dots,\uline{\dim V+1}}.$ \,Then $\exists\,\alpha_k$ not an eigval.\PfEnd
\SepLine[0pt][\Blind{\BulletPointX} ]

\Anchor{5A4e11}\ProblemBnoor{4E 11}{
	\TextB{Prove $\exists\,\delta > 0$ suth $\Par{T-\alpha I}$ is inv for all $\alpha\in\Fbb$ suth $0 < \aMid{\alpha-\lambda} < \delta$.}
}If $T$ has no eigvals, then $\Par{T-\alpha I}$ is inje for all $\alpha\in\Fbb,$ done.\parSol{}
Supp $\lambda_1,\dots,\lambda_m$ are all the disti eigvals of $T$ unequal to $\lambda.$\parSol{}
Let $\delta>0$ be suth, for each eigval $\lambda_k,$ $\lambda_k\not\in\Par{\lambda-\delta,\lambda}\cup \Par{\lambda,\lambda+\delta}.$\parSol{}
So that for all $\alpha\in\Fbb$ suth $0<\aMid{\alpha-\lambda}<\delta,\Par{T-\alpha I}$ is inv.\PfEnd\vspace{4pt}\parSol{}
\Or Let $\delta=\min\!\Bra{\aMid{\lambda-\lambda_k}:k\in\envFontSmall\Bra{1,\dots,m}\envFontDefault,\lambda_k\neq\lambda}.$\parSol{}
Then $\delta>0$ and each $\lambda_k\neq\alpha$ \Sbra{ $\Longleftrightarrow\Par{T-\alpha I}$ is inv } for all $\alpha\in\Fbb$ suth $0<\aMid{\alpha-\lambda}<\delta.$\PfEnd
\SepLine

%\Anchor{5A4e15}
\Anchor{5A'1}\ProblemB{
	\TextB{Supp $T\in\Lm{V}.$ \,Give a countexa\hspace{1pt}$:$ $\lambda$ is eigval of $T$ $\Longleftrightarrow$ of $T\apostrophe.$}
}Let $T$ be the forwd shift optor on $V=\FbbP{\infty}.$ No eigvals for $T,$ by Exe (18).\parSol{}
Define $\psi\in V\apostrophe$ by $\psi\Par{x_1,x_2,\cdots}=x_1.$ Then $\Sbra{T\apostrophe\Par{\psi}}\Par{x_1,x_2,\cdots}=\psi\Par{0,x_1,x_2,\cdots}=0.$\PfEnd
%\Sbra[3pt]{{\tgsl Req Finide; For [5.6]}} \;${T-\lambda I_V}$ not inv $\Longleftrightarrow\Par{T-\lambda I_V}\apostrophe=T\apostrophe-\lambda I_{V\apostrophe}$ not inv.\PfEnd\vspace{2pt}\parSol{}
%(a) Supp $\lambda$ is eigval with $v.$ Let $U$ be invar with $U\oplus\Span{v}=V,$ by Exe (4E 39).\parSol{\Ha}
%Define $\psi\in V\apostrophe$ by $\psi\Par{cv+u}=c.$ Then $\Sbra{T\apostrophe\Par{\psi}}\Par{cv+u}=\psi\BigPar{c\lambda v+Tu}=\lambda c=\lambda\psi\Par{cv+u}.$\vspace{2pt}\parSol{}
\SepLine

%\ProblemN{\Anchor{5A11}{11}}{
%	\TextA{Define $T:\PoRi\rightarrow\PoRi$ by $Tp = p\apostrophe$. Find all eigvals and eigvecs.}
%}For $0\neq p\in\PoRi,\deg p\apostrophe<\deg p.$ And $\deg 0=-\infty.$ Supp $p\apostrophe=\lambda p.$\parSol{}
%Asum $\lambda\neq 0.$ Then $\deg\lambda p=\deg p\apostrophe<\deg\lambda p,$ ctradic. Thus $\lambda=0.$\parSol{}
%Therefore $\deg\lambda p=-\infty=\deg p\apostrophe\Rightarrow p\in\PoR{0}.$\PfEnd
%\SepLine

\ProblemN{\Anchor{5A15}{15}}{
	\TextA{Supp $R,S,T\in\Lm{V},$ and $S$ is inv.}
	\PrePa\TextA{Prove $T$ and $S^{-1}TS$ have the same eigvals.}
	\PrePb\TextA{Describe the relationship between eigvecs of $T$ and eigvecs of $S^{-1}TS.$}
	\PrePc\TextA{Supp $R$ and $T$ have the same eigvals. Then $\exists$ inv $P\in\Lm{V},\,R=P^{-1}TP.$}
	\PrePd\TextA{Supp $R,T$ have the same eigvecs. Then $RT=TR.$}
	\Anchor{5A23}\PrePe\TextA{Prove $RT$ and $TR$ have the same eigvals.\vspace{2pt}}
}(a) $\lambda$ is an eigval of $T$ with an eigvec $v\Rightarrow S^{-1}TS\Par{\uline{S^{-1}v}}=S^{-1}Tv=S^{-1}\Par{\lambda v}=\uline{\lambda S^{-1}v}.$\parSol{\Ha}
$\lambda$ is an eigval of $S^{-1}TS$ with an eigvec $v\Rightarrow S\Par{S^{-1}TS}v=T\uline{Sv}=\uline{\lambda Sv}.$\vspace{2pt}\parSol{\Ha}
\Or Note that $S\Par{S^{-1}TS}S^{-1}=T.$ Every eigval of $S^{-1}TS$ is an eigval of $S\Par{S^{-1}TS}S^{-1}=T.$\vspace{2pt}\parSol{\Ha}
\Or $Tv=\lambda v\Longleftrightarrow{TS}{u}=\lambda Su\Longleftrightarrow\Par{S^{-1}TS}{u}=\lambda u.$ Where $v=Su.$\parSol{\Ha}
\Blind{\Or}$\Par{S^{-1}TS}{u}=\lambda u\Longleftrightarrow{S^{-1}T}{v}=\lambda S^{-1}v\Longleftrightarrow Tv=\lambda v.$ Where $u=S^{-1}v.$\vspace{4pt}\parSol{}
(b) $E\Par{\lambda,T}=\Bra{Su:u\in E\Par{\lambda,S^{-1}TS}};\;E\Par{\lambda,S^{-1}TS}=\Bra{S^{-1}v:v\in E\Par{\lambda,T}}.$\vspace{3pt}\parSol{}
(e) \!\Sbra{{\tgsl False if infinide. See Exe (18, 20).}} Supp $v\neq 0$ and $RTv=\lambda v\Rightarrow T\Par{RTv}=\lambda Tv=TR\Par{Tv}.$\parSol{\He}
If $Tv=0$, then $T$ not inje, so are $TR,RT.$ Othws, $\lambda$ is eigval of $TR.$ \;Rev the roles.\PfEnd
\SepLine

%\ProblemN{\Anchor{5A17}{17}}{
%	\TextA{Give an exa of an optor on $\Rbb^4$ that has no real eigvals.}
%}\par\quad
%Let $\Par{e_1,e_2,e_3,e_4}$ be the std bss of $\Rbb^4.$\vspace{-6pt}\par\quad
%Define $T\in\Lm{\Rbb^4}$ by $\Mt[\BigPar]{T,\Par{e_1,e_2,e_3,e_4}}={}${\normalsize$\begin{pmatrix}
%		1 & 1 & 1 & 1\\
%		-1 & 1 & -1 & -1\\
%		3 & 8 & 11 & 5\\
%		3 & -8 & -11 & 5
%	\end{pmatrix}.$}\vspace{-12pt}\par\quad
%Supp $\lambda$ is an eigval of $T$ with an eigvec $\Par{x,y,z,w}.$ Then we get \;{\normalsize$\MathLeftBrace{l}{
%		\Par{1-\lambda}x+y+z+w=0,\\
%		-x+\Par{1-\lambda}y-z-w=0,\\
%		3x+8y+\Par{11-\lambda}z+5w=0,\\
%		3x-8y-11 z+\Par{5-\lambda}w=0.}$}\vspace{-8pt}\par\quad
%This set of liney equations has no solutions.\par\quad
%\Sbra{ You can type it on \url{https://zh.numberempire.com/equationsolver.php} to check. }\large\par\vspace{6pt}\quad
%\Or Define $T\in\Lm{\Rbb^4}$ by $T\Par{x_1,x_2,x_3,x_4}=\Par{{-x_2,x_1,-x_4,x_3}}.$\par\quad
%Supp $\lambda$ is an eigval of $T$ with an eigvec $\Par{x,y,z,w}.$\vspace{-3pt}\par\quad
%Then $T\Par{x,y,z,w}=\Par{\lambda x,\lambda y,\lambda z,\lambda w}=\Par{{-y,x,-w,z}}$ { $\Longrightarrow\MathLeftBrace{l}{
%		-y=\lambda x,x=\lambda y\Longrightarrow-xy=\lambda^2 xy\\
%		-w=\lambda z,z=\lambda w\Longrightarrow-zw=\lambda^2 zw}$}\vspace{-3pt}\par\quad
%If $xy\neq 0$ or $zw\neq 0,$ then $\lambda^2=-1,$ we fail.\par\quad
%Othws, $xy=0\Rightarrow x=y=0,$ for if $x\neq 0,$ then $\lambda=0\Rightarrow x=0,$ ctradic.\par\quad
%Simlr, $y=z=w=0.$ Then we fail. Thus $T$ has no eigvals.\PfEnd
%\SepLine

\Anchor{5AT5}\ProblemBX{\TipsN{5}}{
	\TextA{Supp $S,T\in\Lm{V},p\in\PoFi$. Prove $Sp\Par{TS}=p\Par{ST}S.$}
}We prove each $S\Par{TS}{^m}=\Par{ST}{^m} S$ by induc. (i) $m=0,1.$ Immed.\parSol{}
(ii) $m>1.$ $S\Par{TS}{^{m-1}}=\Par{ST}{^{m-1}} S\Rightarrow S\Par{TS}{^{m}}=S\Par{TS}{^{m-1}}\Par{TS}=\Par{ST}{^{m-1}}\Par{ST}S=\Par{ST}{^{m}}S.$\PfEnd\vspace{2pt}
\AComm If $S$ is inv. Then $p\Par{TS} = S^{-1} p\Par{ST}S,\;p\Par{ST} = Sp\Par{TS}S^{-1}.$\par\Anchor{5BI5}\Anchor{5A4e40}
\ACoro Becs $S$ is inv, $T\in\Lm{V}$ is arb $\Longleftrightarrow ST=R\in\Lm{V}$ is arb. Hence $p\Par{S^{-1}RS}=S^{-1}p\Par{R}S.$
\SepLine

%\Anchor{5A4e17}\Anchor{5A16}\ProblemB{
%	\TextB{Supp $\Fbb=\Rbb,$ $T\in\Lm{V}$.}
%	\Sbra{{\normalsize\Onumber{16} \;\OR [9.16]\,}} \TextB{$\lambda\in\Cbb.$ Prove $\lambda$ is eigval of $T_{\!\Cbb}$ $\Longleftrightarrow$ $\overline \lambda$ is eigval of $T_{\!\Cbb}.$\vspace{1pt}}
%}Supp $T_{\!\Cbb}\Par{v+\i u}=Tv+\i Tu=\lambda\Par{v+\i u}.$\vspace{2pt}\parSol{}
%Becs $\overline{T_{\!\Cbb}\Par{v+\i u}}=\overline{Tv+\i Tu}=Tv-\i Tu=T_{\!\Cbb}\Par{v-\i u}=T_{\!\Cbb}\Par{\overline{v+\i u}}.$\vspace{2pt}\parSol{}
%And $\overline{\lambda\Par{v+\i u}}=\overline\lambda v-\i\overline\lambda u=\overline\lambda\Par{v-\i u}=\overline\lambda\Par{\overline{v+\i u}}.$\PfEnd\vspace{4pt}\parSol{}
%\Or Supp $\lambda=a+\i b$ is eigval of $T_{\!\Cbb}$ with $v+\i u.$\parSol{}
%Becs $T_{\!\Cbb}\Par{v+\i u}=\lambda\Par{v+\i u}=\Par{\uline{av-bu}}+\i\Par{\uwave{{au+bv}_{\,\!}}}=\uline{Tv}+\i\uwave{Tu_{\,\!}}.$\vspace{-4pt}\parSol{}
%Now $T_{\!\Cbb}\Par{\overline{v+\i u}}=Tv-\i Tu=\Par{av-bu}-\i\Par{au+bv}=\Par{a-\i b}\Par{v-\i u}=\overline\lambda\Par{\overline{v+\i u}}.$\PfEnd
%\SepLine

%\ProblemN{\Anchor{5A21}{21}}{
%	\TextA{Supp $T\in\Lm{V}$ is inv.\;\FontNorm Then $0$ is not eigval of $T$ or $T^{-1}.$}
%	\PrePa\TextA{Supp $\lambda\in\Fbb$ with $\lambda\neq 0$. Prove $\lambda$ is eigval of $T$ $\Longleftrightarrow\lambda^{-1}$ is eigval of $T^{-1}$.}
%	\PrePb\TextA{Prove $T,T^{-1}$ have the same eigvecs.}
%}$Tv=\lambda v\Longleftrightarrow v=\lambda T^{-1}v\Longleftrightarrow \lambda^{-1}v=T^{-1}v.$ Where $v\neq 0.$\PfEnd
%\SepLine

%\ProblemN{\Anchor{5A22}{22}}{
	%	\TextA{Supp $T\in\Lm{V}$ and $\exists$ non0 vecs $u,w$ in $V$ suth $Tu = 3w,\;Tw = 3u$.}
	%	\TextA{Prove $3$ or $-3$ is an eigval of $T$.}
	%}$T\Par{u+w}=3\Par{u+w},\;T\Par{u-w}=3\Par{w-u}=-3\Par{u-w}.$ Note that $u-w\neq 0$ or $u+w\neq 0.$\parSol{}
%\Or $T\Par{Tu}=9u\Rightarrow T^2-9=\Par{T-3I}\Par{T+3I}$ is not injective $\Rightarrow 3$ or $-3$ is an eigval.\PfEnd
%\SepLine

%\ProblemN{\Anchor{5A23}{23}}{
%	\TextA{Supp $V$ is finide, and $S, T\in\Lm{V}$. Prove $ST$ and $TS$\, have the same eigvals.}
%}\!\Sbra{{\tgsl False if infinide. See Exe (18, 20).}} Supp $v\neq 0$ and $STv=\lambda v\Rightarrow T\Par{STv}=\lambda Tv=TS\Par{Tv}.$\parSol{}
%If $Tv=0$, then $T$ not inje, so are $TS,ST.$ Othws, $\lambda$ is eigval of $TS.$ \,Rev the roles in asum.\PfEnd
%\SepLine
\pagebreak

\Anchor{5A4e37}\ProblemBnoor{4E 37}{
	\TextA{Supp $V$ is finide, $T\in\Lm{V}$. Define $\mA\in\Lm[\BigPar]{\Lm{V}}$ by $\mA\Par{S} = TS.$}
	\TextA{Prove the set of eigvals of\;$T$ equals the set of eigvals of $\mA$.\hfill\tgnr\FontNorm Not true if $\mA:S\mapsto ST.$}
}(a) For $v\neq 0$ and $Tv=\lambda v,$ let $v_1=v\Rightarrow B_V=\Par{v_1,\dots,v_n}.$\parSol{\Ha}
Define $S\in\Lm{V}:v_j\mapsto v,$ \OR $v_j\mapsto\delta_{1,j}v_1.$ Then each $\Par{T-\lambda I}Sv_j=0.$\vspace{2pt}\parSol{}
(b) Supp $S\neq 0$ and $TS=\lambda S.$ Then $\exists\,v\in V\Backslash\null S.$ Let $u=Sv\Rightarrow Tu=TSv=\lambda Sv=\lambda u.$\parSol{\Hb}
\Or $TS-\lambda S=\Par{T-\lambda I}S=0\Rightarrow\zeroSubs\neq\range S\subseteq\Null\Par{T-\lambda I}\Rightarrow\Par{T-\lambda I}$ not inje.\PfEnd
\SepLine

%\ProblemN{\Anchor{5A25}{25}}{
%	\TextA{Supp $u,w,$ and $u+w$ are eigvecs of $T\in\Lm{V}.$ Prove the eigvals are the same.}
%}Supp $\lambda_1,\lambda_2,\lambda_0$ are eigvals of $T$ with eigvecs to $u,w,u+w$ respectly.\parSol{}
%Then $T\Par{u+w}=\lambda_0\Par{u+w}=Tu+Tw=\lambda_1 u+\lambda_2 w\Rightarrow \Par{\lambda_0-\lambda_1}u=\Par{\lambda_2-\lambda_0}w.$\parSol{}
%If $\Par{u,w}$ is liney dep, then let $w=cu,$ therefore $\lambda_2 cu=Tw=cTu=\lambda_1 cu\Rightarrow\lambda_2=\lambda_1.$\parSol{}
%Othws, $\Par{u,w}$ is liney indep. Then $\lambda_0-\lambda_1=\lambda_2-\lambda_0=0\,\Rightarrow\lambda_1=\lambda_2=\lambda_0.$\PfEnd\parSol{}
%\Or Asum $\lambda_1\neq \lambda_2.$ Then $\Par{u,w}$ is liney indep. Thus $\lambda_0-\lambda_1=\lambda_0-\lambda_2.$ Ctradic.\PfEnd
%\SepLine

%\ProblemN{\Anchor{5A26}{26}}{
%	\TextA{Supp $T\in\Lm{V}$ is suth $\forall v\in V,\exists\,!\,\lambda_v\in\Fbb,Tv=\lambda_v v.$ Prove $T=\lambda I.$\hfill\tgnr\FontNorm By Exe (25).}
%}Supp $V$ non0. Becs $\forall v\in V,\exists\,!\,\lambda_v\in\Fbb,Tv=\lambda_v v.$ For any disti non0 $v,w\in V,$\parSol{}
%$T\Par{v+w}=\lambda_{v+w}\Par{v+w}=Tv+Tw=\lambda_v v+\lambda_w w\Rightarrow\Par{\lambda_{v+w}-\lambda_v}v=\Par{\lambda_w-\lambda_{v+w}}w.$\PfEnd
%\SepLine

\Anchor{5A27}\Anchor{5A28}\ProblemN{27, 28}{
	\TextA{Supp $\dim V>1,\,k\in\;\!\!\Bra{1,\dots,\dim V-1}$.}
	\TextA{Supp every subsp of dim $k$ is invard a $T\in\Lm{V}.$ \,Prove $T=\lambda I.$}
}We prove the ctrapos. Supp $\exists\,v\in V\nonzero$ not eigvec.\parSol{}
Then $\Par{v,Tv}$ liney indep $\Rightarrow B_V=\Par{v,Tv,u_1,\dots,u_n}.$ Let $U=\Span{v,u_1,\dots,u_{k-1}}.$\PfEnd\vspace{3pt}\parSol{}
\Or Supp $v=v_1\in V\nonzero\Rightarrow B_V=\Par{v_1,\dots,v_n}.$ Let $Tv_1=c_1 v_1+\dots+c_n v_n.$\parSol{}
Let $B_U=\Par{v_1,v_{\:\!\!\alpha_1},\dots,v_{\:\!\!\alpha_{k-1}}}.$ Becs every such $U$ invar. Now $Tv_1\in U\Rightarrow Tv_1=c_1v_1.$\parSol{}
By Exe (26), done. \BigSbra{For $0\neq c_j\in\;\!\!\Bra{c_2,\dots,c_n},$ let $B_W=\Par{v_1,v_{\beta_1},\dots,v_{\beta_{k-1}}}$ with each $\beta_i\neq j.$}\PfEnd%\vspace{4pt}\quad
%\Or For each $k\in\;\!\!\Bra{1,\dots,\dim V-1},$ define $P\Par{k}:$ \Largesl{if} each subsp $U$ with $\dim U=k$ invar, then $T=\lambda I.$\par\quad
%(i) $k=1.$ $P\Par{1}$ holds, by Exe (26).\par\quad\Endi
%(ii) $1\leqslant k\leqslant\dim V-2.$ Asum $P\Par{k}$ holds, and each subsp $U$ with $\dim U=k+1$ invar.\par\quad\Hii
%Supp $\dim U=k,$ and $v,w\not\in U$ liney indep. Then $U\oplus\Span{v},U\oplus\Span{w}$ invar.\par\quad\Hii
%Supp $u\in U\subseteq U\oplus\Span{v},U\oplus\Span{w}.$ Let $Tu=u_1+bv=u_2+cw\in U.$\par\quad\Hii
%Then $bv-cw=u_2-u_1\in U\cap\Span{v,w}\Rightarrow bv,cw\in U\Rightarrow b=c=0\Rightarrow Tu\in U.$\par\quad\Hii
%Becs $P\Par{k}$ holds, we conclude that $T=\lambda I.$ Thus $P\Par{k+1}$ holds.\PfEnd
\SepLine

\ProblemN{\Anchor{5A29}{29}}{
	\TextA{Supp $T\in\Lm{V},\,\range T$ is finide. Prove $T$ has at most $1 + \dim\range T$ disti eigvals.}
}Becs $\range T$ finide $\Rightarrow$ not too many. Let $\lambda_1,\dots,\lambda_m$ be the disti eigvals of $T$ with corres $v_1,\dots,v_m.$\parSol{}
Then $\Par{v_1,\dots,v_m}$ liney indep $\Rightarrow\Par{\lambda_1 v_1,\dots,\lambda_m v_m}$ liney indep, if each $\lambda_k\neq 0.$ \;Othws,\parSol{}
$\exists\,!\,\lambda_k=0.$ Now $\Bra{\lambda_jv_j:j\neq k}$ liney indep. Thus $m-1\leqslant\dim\range T.$\PfEnd
\SepLine

%\ProblemN{\Anchor{5A30}{30}}{
	%	\TextA{Supp $T\in\Lm{\Rbb^3}$ and $-4,5,\sqrt{7}$ are eigvals. Prove $\exists\,x,Tx - 9x = \Par{{-4, 5, \sqrt{7}}}$.}
	%}$T$ has $\dim\Rbb^3$ eigvals not including $9\Rightarrow\Par{T-9I}$ is inv. $x=\Par{T-9I}{^{-1}}\Par{{-4,5,\sqrt{7}}}.$\PfEnd
%\SepLine

%\ProblemN{\Anchor{5A31}{31}}{
	%	\TextA{Supp $V$ is finide, and $v_1,\dots,v_m \in V$. Prove}
	%	\TextA{$\Par{v_1,\dots,v_m}$ is liney indep $\Longleftrightarrow v_1,\dots,v_m$ are eigvecs of some $T$ corres to disti eigvals.}
	%}Supp $\Par{v_1,\dots,v_m}$ is liney indep. Let $B_V=\Par{v_1,\dots,v_m,\dots,v_n}.$\parSol{}
%Define $T\in\Lm{V}$ by $Tv_k=k\cdot v_k$ for each $k\in\;\!\!\Bra{1,\dots,m,\dots,n}.$ Convly by [5.10].\PfEnd\vspace{-4pt}
%\SepLine

%\Anchor{5A32}\Anchor{5A4e36}\ProblemB{
%	\TextB{Supp $\lambda_1,\dots,\lambda_n\in\Rbb$ are disti.}
%	(a)\ProblemN[]{32}{
%		\TextB{Prove $\Par{e^{\lambda_1x},\dots,e^{\lambda_nx}}$ is liney indep in $\Rbb^{\Rbb}$.}}
%	\TextB{\vspace{-2pt}}
%	(b)\ProblemNnoor[]{\hspace{-7pt}}{4E 36}{
%		\TextB{Show $\BigPar{\!\cos{\lambda_1 x},\dots,\cos{\lambda_n x}}$ is liney indep in $\Rbb^{\Rbb}.$\vspace{-3pt}}}
%	\TextB{}\vspace{2pt}
%}(a) Let $V=\Span{e^{\lambda_1x},\dots,e^{\lambda_nx}}.$ Define $D\in\Lm{V}$ by $Df=f\apostrophe.$\parSol{\Ha}
%Then becs each $\lambda_k e^{\lambda_k x}=D\Par{e^{\lambda_k x}}.$ Now $\lambda_1,\dots,\lambda_n$ are disti eigvals of $D.$ By [5.10].\PfEnd\vspace{2pt}\parSol{}
%(b) %Let $V=\Span[\BigPar]{\!\cos{\lambda_1 x},\dots,\cos{\lambda_n x}}.$ Define $Df=f\apostrophe.$\parSol{\Hb}
%Define $V,D$ simlr. Becs $D\BigPar{\!\cos{\lambda_k x}}=-\lambda_k\sin{\lambda_k x}.$ 又 $D\BigPar{\!\sin{\lambda_k x}}=\lambda_k\cos{\lambda_k x}.$\parSol{\Hb}
%Thus $D^2\BigPar{\!\cos{\lambda_k x}}=-\lambda_k^2\cos{\lambda_k x}.$ Now $-\lambda_1^2,\dots,-\lambda_n^2$ are disti eigvals of $D^2.$ Simlr.\PfEnd
%\SepLine

\ProblemN{\Anchor{5A35}{35}}{
	\TextA{Supp $V$ is finide, $T\in\Lm{V}$, and $U$ is invard $T$. Show $\lambda$ is eigval of $T\XSlash U\Rightarrow$ of $T$.}
}\par\quad
Supp $v+U\neq 0$ and $Tv+U=\lambda v+U\Rightarrow\Par{T-\lambda I}v=u\in U.$ \,If $u=0,$ done. Othws, two cases.\par\quad
If $\Par{T-\lambda I}\Big|{_U}$ inje $\Rightarrow$ surj. Then $\Par{T-\lambda I}v=u=\Par{T-\lambda I}\Big|{_U}\Par{w},\exists\,w\in U\Rightarrow T\Par{v+w}=\lambda\Par{v+w}.$\par\quad
If $\Par{T-\lambda I}\Big|{_U}=T\mmid_U-\lambda I_U$ not inje. Then $\lambda$ is eigval of $T\mmid_U\Rightarrow$ of $T.$\PfEnd\vspace{4pt}\quad
\Or Let $B_U=\Par{u_1,\dots,u_m}\Rightarrow\BigPar{Tv-\lambda v,\,Tu_1-\lambda u_1,\cdots,Tu_m-\lambda u_m}$ of len $\Par{m+1}$ liney dep in $U.$\par\quad
So that $a_0\Par{T-\lambda I}v+a_1\Par{T-\lambda I}u_1+\dots+a_m\Par{T-\lambda I}u_m=0,\exists\,a_k\neq 0.$\par\quad
Then $Tw=\lambda w,$ where $w=a_0v+a_1u_1+\dots+a_mu_m\neq 0\Leftarrow w\not\in U\Leftarrow v\not\in U.$\PfEnd\vspace{4pt}\Anchor{5A36}
\AExa Let $V=\FbbP{\Nbb},U=\Bra{\,x\in\FbbP{\Nbb}:x_1=0},T\in\Lm{V}:e_k\mapsto e_{k+1}.$ Then $\Par{T\XSlash U}\Par{e_1+U}=e_2+U=0.$
\SepLine

\Anchor{5A4e39}\ProblemBnoor{4E 39}{
	\TextB{Supp $T\in\Lm{V},$ $V$ is finide. Prove $\exists$ eigval of $T$ $\Longleftrightarrow\exists$ invarsp $U$ of dim $\dim V-1$.}
}Supp $\lambda$ is eigval with $v.$ Becs $\dim E\Par{\lambda,T}\geqslant 1\Longleftrightarrow\dim\Range\Par{T-\lambda I}\leqslant\dim V-1=N.$\parSol{}
Let $B_{\Range\SmallPar{T\,-\,\lambda I}}=\Par{w_1,\dots,w_m}.$ Extend to a liney indep $\Par{w_1,\dots,w_m,\dots,w_N}=B_U.$\parSol{}
Now $U$ invard $\Par{T-\lambda I}\Rightarrow$ invard $T.$ \;Convly, becs $\dim V\XSlash U=1.$ By (3.A.7), Exe (35).\PfEnd
\SepLine

\Anchor{5A4e16}\ProblemBnoor{4E 16}{
	\TextB{Supp $B_V=\Par{v_1,\dots,v_n},T\in\Lm{V},$ and $\lambda$ is eigval.}
	\TextB{Let $A_M$ be the max of all ent of $A=\Mt[\BigPar]{T,B_V}.$ Prove $\aMid{\lambda}\leqslant A_M\cdot\dim V.$}
}Supp $\lambda$ is eigval with $v.$ Let $v=c_1v_1+\dots+c_nv_n.$\vspace{1pt}\par\quad
Becs $\lambda c_1v_1+\dots+\lambda c_nv_n=c_1Tv_1+\dots+c_nTv_n=\sum_{k=1}^nc_k\BigSbra{{\sum_{j=1}^n A_{j,k}v_j}}=\sum_{j=1}^n\BigSbra{{\sum_{k=1}^nc_kA_{j,k}}}v_j.$\vspace{2pt}\par\quad
Thus $\lambda c_j=\sum_{k=1}^n c_kA_{j,k}\Rightarrow$ each $\aMid{\lambda}\aMid{c_j}=\sum_{k=1}^n\aMid{c_k}\aXMid{A_{j,k}}.$ Let $\aMid{c_M}=\max\!\Bra{\aMid{c_1},\dots,\aMid{c_n}}.$\vspace{2pt}\par\quad
Becs $v\neq 0\Rightarrow\aMid{c_M}\neq 0.$ Now $\aMid{\lambda}\aMid{c_M}=\sum_{k=1}^n\aMid{c_k}\aXMid{A_{M,k}}\Rightarrow\aMid{\lambda}\leqslant\sum_{k=1}^n\aMid{A_{M,k}}\leqslant nM.$\PfEnd
\SepLine\pagebreak

\ProblemN{\Anchor{5A24}{24}}{
	\TextA{Supp $A\in\FbbP{n,n}.$ Define $T\in\Lm{\FbbP{n,1}}$ by $Tx = Ax$. Prove $\lambda$ is eigval of $T$ if\hspace{2pt}$:$}
	\PrePa\TextA{the sum of the ent in each row of $A$ equals $\lambda.$ \, {\tgnr\large(b)} each col of $A.$}
}Supp $x\neq0$ and $Ax={\BigPar{A_{j,1}x_1+\dots+A_{j,n}x_n}{_{j=1}^n}}=\alpha\Par{x_j}{_{j=1}^n}=\alpha x.$\par\quad
(a) Supp $A_{R,1}+\dots+A_{R,n}=\lambda.$ Let $x_1=\dots=x_n.$ Immed.\vspace{2pt}\par\quad
(b) Supp $A_{1,C}+\dots+A_{n,C}=\lambda.$ Note that $\BigSbra{{\sum_{R=1}^n A_{R,\cdot}}}x=\sum_{k=1}^n\BigPar{{A_{1,k}+\dots+A_{n,k}}}x_k.$\par\quad\Hb
Each $\Par{Ax}{_{R,1}}=\lambda\Par{x}{_{R,1}}.$ \,Thus for $x$ with $\sum_{k=1}^nx_k\neq 0,$ $\lambda$ is the corres eigval.\PfEnd\vspace{3pt}\quad\Hb
\Or Becs $\Par{T-\lambda I}x=\BigPar{\Par{A_{j,1}x_1+\dots+A_{j,n}x_n}-\lambda x_j}{_{j=1}^n}=\Par{y_j}{_{j=1}^n}.$\vspace{1pt}\par\quad\Hb
Now $y_1+\dots+y_n=\sum_{k=1}^n x_k\BigSbra{{\sum_{j=1}^n A_{j,k}}}-\lambda\sum_{j=1}^n x_j=0.$ Thus $\Par{T-\lambda I}$ not surj.\PfEnd\vspace{5pt}\quad\Hb
\Or Let $\Par{e_1,\dots,e_n}$ be the std bss of $\FbbP{n,1}.$ Define $\psi\in\BigPar{\FbbP{n,1}}\apostrophe$ with each $\psi\Par{e_k}=1.$\vspace{0pt}\par\quad\Hb
Becs $Ae_k=A_{\cdot,k}=\sum_{j=1}^nA_{j,k}e_j\Rightarrow\psi\Sbra{\Par{T-\lambda I}{e_k}}=\psi\BigBigPar{{\sum_{j=1}^n A_{j,k}e_j}-\lambda e_k}=\sum_{j=1}^n A_{j,k}-\lambda=0.$\PfEnd\vspace{5pt}\quad\Hb
\Or Define $S\in\Lm{\FbbP{n,1}}$ by $Sx=A^tx.$ \;By (a), \Sbra{3.F \TIPSN{4}}, and Exe (15, 4E 15),\par\quad\Hb
the sum of the ent in each row of $A^t$ equals $\lambda\Rightarrow \lambda$ is eigval of $S=\Phi^{-1}T\apostrophe\Phi,$ so of $T\apostrophe,$ of $T.$\PfEnd
\SepLine

\Anchor{5A'2}\ProblemB{
	\TextB{Supp $A\in\FbbP{n,n}.$ Define $T\in\Lm{\FbbP{1,n}}$ by $Tx = xA.$ Prove $\lambda$ is eigval of $T$ if\hspace{2pt}$:$}
	\PrePa\TextB{the sum of the ent in each col of $A$ equals $\lambda.$ \, {\tgnr\large(b)} each row of $A.$}
}Supp $x\neq 0$ and $xA=\BigPar{x_1A_{1,k}+\dots+x_nA_{n,k}}{_{k=1}^n}=\alpha\Par{x_k}{_{k=1}^n}=\alpha x.$\par\quad
(a) Supp $A_{1,C}+\dots+A_{n,C}=1.$ Let $x_1=\dots=x_n.$ Immed.\vspace{2pt}\par\quad
(b) Supp $A_{R,1}+\dots+A_{R,n}=\lambda.$ Note that $\sum_{C=1}^nxA_{\cdot,C}=\sum_{j=1}^n\BigPar{A_{j,1}+\dots+A_{j,n}}x_j.$\vspace{0pt}\par\quad\Hb
Each $\Par{xA}{_{1,C}}=\lambda\Par{x}{_{1,C}}.$ \,Thus for $x$ suth $\sum_{k=1}^nx_k\neq 0,$ $\lambda$ is the corres eigval.\PfEnd\vspace{3pt}\quad\Hb
\Or Becs $\Par{T-\lambda I}x=\BigPar{\Par{x_1 A_{1,k}+\dots+x_nA_{n,k}}-\lambda x_k}{_{k=1}^n}=\Par{y_k}{_{k=1}^n}.$\vspace{1pt}\par\quad\Hb
Now $y_1+\dots+y_n=\sum_{j=1}^n x_j\BigSbra{{\sum_{k=1}^n A_{j,k}}}-\lambda\sum_{k=1}^n x_k=0.$\PfEnd\vspace{5pt}\quad\Hb
\Or Simlr. Becs $e_jA=A_{j,\cdot}=\sum_{k=1}^n A_{j,k}e_k\Rightarrow\psi\Sbra{\Par{T-\lambda I}{e_j}}={\sum_{k=1}^n A_{j,k}}-\lambda=0.$\PfEnd\vspace{4pt}\quad\Hb
\Or Define $S\in\Lm{\FbbP{1,n}}$ by $Sx=xA^t\Rightarrow S=\Phi{^{-1}}T\apostrophe\Phi.$ \;Simlr and by \Sbra{3.D \TIPSN{3}}.
%Let $\Par{\varphi_1,\dots,\varphi_n}$ be the dual bss. Define $\Phi$ by $\Phi\Par{e_k}=\varphi_k.$\par\quad\Hb
%Becs $\Sbra{T\apostrophe\Par{\varphi_k}}\Par{e_j}=\varphi_k\BigBigPar{{\sum_{i=1}^n A_{j,\,i}e_i}}=A_{j,k}.$ By (3.F.9), $T\apostrophe\Par{\varphi_k}=\sum_{j=1}^n A_{j,k}\varphi_j.$\vspace{1pt}\par\quad\Hb
%Now $\BigPar{\Phi^{-1}T\apostrophe\Phi}{e_k}=\BigPar{\Phi^{-1}T\apostrophe}{\varphi_k}=\Phi^{-1}\BigBigPar[0pt]{\sum_{j=1}^n A_{j,k}\varphi_j}=\sum_{j=1}^n A_{j,k}e_j=e_k A^t=Se_k.$ \,Simlr.
\PfEnd
\SepLine

%\Anchor{5C16}\ProblemBnoor{5.C.16}{
%	\TextB{Let $\Bra{F_n}$ be the Fibonacci Seq. Define $T\in\Lm{\Rbb^2}:\Par{x,y}\mapsto\Par{y,x+y}.$}
%	(a) \TextB{Find all eigvals and eigvecs. \;{\tgnr\large(b)} Show $T^n\Par{0,1}=\Par{F_n,F_{n+1}}$ and find the formula.}
%	%{	{and find the general formula of $F_n.$}
%		%and each $F_n={}${\LARGE$\frac{1}{\sqrt{5}\;}$}$\,\bigg[\bigg(\!${\LARGE$\frac{\:1\,+\,\sqrt{5}\:}{2}$}$\!\bigg)\begin{array}{l}\hspace{-5pt}{}_n\\\\\vspace{-18pt}\end{array}\hspace{-8pt}-\bigg(\!${\LARGE$\frac{\:1\,-\,\sqrt{5}\:}{2}$}$\!\bigg)\begin{array}{l}\hspace{-5pt}{}_n\\\\\vspace{-18pt}\end{array}\hspace{-5pt}\bigg].$\vspace{-4pt}}
%	%	\PrePd\TextA{Show each $F_n$ is the integer that closest to {\LARGE$\frac{1}{\sqrt{5}\;}$}$\,\bigg(\!${\LARGE$\frac{\:1\,+\,\sqrt{5}\:}{2}$}$\!\bigg)\begin{array}{l}\hspace{-5pt}{}_n\\\\\vspace{-18pt}\end{array}\hspace{-8pt}.$\vspace{-8pt}}
%}(a) {\FontSmall Supp $\lambda\Par{x,y}=\Par{y,x+y}$ with $x$ or $y$ non0. Note that $x=0\Longleftrightarrow y=0,$ and $0$ is not eigval.}\vspace{1pt}\parSol{\Ha}
%{\FontSmall Then $\lambda_1={}${\large$\frac{\:1\,+\,\sqrt{5}\:}{2}$}, $v_1=\Par{1,\frac{\:1\,+\,\sqrt{5}\:}{2}};$
%	\,and $\lambda_2={}${\large$\frac{\:1\,-\,\sqrt{5}\:}{2}$}, $v_2=\Par{1,\frac{\:1\,-\,\sqrt{5}\:}{2}}.$ Becs $\dim\Rbb^2=2.$}\vspace{2pt}\parSol{}
%(b) {\FontSmall $T\Par{0,1}=\Par{F_1,F_2}.$ Asum $T^k\Par{0,1}=\Par{F_k,F_{k+1}}.$ Then $T^{k+1}\Par{0,1}=\Par{F_{k+1},\,F_{k}+F_{k+1}}.$}\vspace{-6pt}\parSol{\Hb}
%{\FontSmall $T^n\Par{0,1}=T^n\XSbra{${\large$\frac{1}{\sqrt{5}\;}$}$\BigPar{v_1-v_2}}={}${\large$\frac{1}{\sqrt{5}\;}$}$\def\envFont{\Large}\XSbra{\bigg(\!${\large$\frac{\:1\,+\,\sqrt{5}\:}{2}$}$\!\bigg)\begin{array}{l}\hspace{-7pt}{}_n\\\\\vspace{-8pt}\end{array}\hspace{-8pt}v_1-\bigg(\!${\large$\frac{\:1\,-\,\sqrt{5}\:}{2}$}$\!\bigg)\begin{array}{l}\hspace{-7pt}{}_n\\\\\vspace{-8pt}\end{array}\hspace{-8pt}v_2\def\envFont{\Large}}.$ Take the first slot.}\PfEnd[-28pt]\vspace{4pt}
%\SepLine
\ChEnd


\vfill\ChDecl{Ch5BI}{5.B}{\quad\normalsize$\hText{$
	(I)覆盖4e的本节全部、\!3e前半部分。(II)覆盖3e本节后半部分「上三角矩阵」、\!4e 5.C节。$\\$
	{\textbf{注意\,:}\;4e的5.B节和3e的8.C节、\!9.A节许多结论和习题有交集。5.B(II)的题号使用4e 5.C节.}$}$
}
\vspace{8pt}

%\Anchor{5BIN5.17}\ProblemB[]{
%	\NoteForSmall{[5.17]}\;\;\TextB{By [8.20], $\nullp p\Par{T}$ and $\rangep p\Par{T}$ are invard $T$.\vspace{-4pt}}
%}\SepLine

%\ProblemN{\Anchor{5BI1}{I.1}}{
%	\TextA{Supp $T\in\Lm{V}$ and $T^n=0.$ Prove $\Par{I-T}$ is inv and $\Par{I-T}{^{-1}}=I+T+\dots+T^{n-1}.$}
%}Becs $p\Par{z}=1-x^n=\Par{1-x}\Par{1+x+\dots+x^{n-1}}.$ Consider $p\Par{T}=I,$ by [5.20].\PfEnd
%\SepLine

%\ProblemN{\Anchor{5BI2}{2}}{
	%	\TextA{Supp $T\in\Lm{V}$ and $\Par{T - 2I}\Par{T - 3I}\Par{T - 4I} = 0.$ Prove the eigvals are $2,3,4.$}
	%}\par\quad
%Supp $v$ is an eigvec corres to $\lambda.$ Then for any $p\in\PoFi,p\Par{T}v=p\Par{\lambda}v.$\par\quad
%Hence $0=\Par{T - 2I}\Par{T - 3I}\Par{T - 4I}v=\Par{\lambda-2}\Par{\lambda-3}\Par{\lambda-4}v$ while $v\neq 0\Rightarrow\lambda = 2,3$ or $4.$\PfEnd
%\SepLine

%\Anchor{5BI7}\ProblemNnoor{7}{See \hLk{5A22}{5.A.22}}{
	%	\TextA{Supp $T\in\Lm{V}$. Prove $9$ is an eigval of $T^2$ $\Longleftrightarrow$ $3$ or $-3$ is an eigval of $T$.}
	%}\par\quad
%(a) Supp $\lambda$ is an eigval of $T$ with an eigvec $v.$\par\quad\Ha
%Then $\Par{T-3I}\Par{T+3I}v=\Par{\lambda-3}\Par{\lambda+3}v=0\Rightarrow\lambda=\pm 3.$\par\quad
%(b) Supp $3$ or $-3$ is an eigval of $T$ with an eigvec $v.$ Then $Tv=\pm 3v\Rightarrow T^2 v=T\Par{Tv}=9v$\PfEnd\vspace{4pt}\quad
%\Or $9$ is an eigval of $T^2\Longleftrightarrow\Par{T^2-9I}=\Par{T-3I}\Par{T+3I}$ is not inje $\Longleftrightarrow\pm 3$ is an eigval.\PfEnd
%\SepLine

%\ProblemN{\Anchor{5BI3}{3}}{
	%	\TextA{Supp $T\in\Lm{V},T^2=I,$ and $-1$ is not eigval of $T.$ Prove $T=I.$}
	%}$T^2-I=\Par{T+I}\Par{T-I}$ is not inje, 又 $-1$ is not an eigval of $T.$\PfEnd\vspace{4pt}\parSol{}
%\Or Note that $\forall v\in V,\,v=\Par{I-T}v\Big/2+\Par{I+T}v\Big/2.$ 又 $I-T^2=\Par{I\pm T}\Par{I\mp T}=0.$\parSol{}
%Then $\Range\Par{I\mp T}\subseteq\Null\Par{I\pm T}\Rightarrow V=\Null\Par{I-T}+\Null\Par{I+T}.$\parSol{}
%又 $-1$ is not eigval of $T\Longleftrightarrow\Par{I+T}$ inje $\Longleftrightarrow\Null\Par{I+T}=\zeroSubs\supseteq\Range\Par{I-T}.$\PfEnd
%\SepLine\pagebreak

%\Anchor{5A4e32}\ProblemB{
%	\TextB{Supp $T\in\Lm{V}$ has no eigvals and $T^4 = I$. Prove $T^2=-I$.}
%}Becs $T^4-I=\Par{T^2-I}\Par{T^2+I}=0$ not inje, so is $\Par{T^2-I}$ or $\Par{T^2+I},$ while $T$ has no eigvals.\parSol{}
%$\Par{T-I},\Par{I+T}$ inje, so is $\Par{T^2-I}\Rightarrow\forall v\in V,\,0=\Par{T^2-I}\Par{T^2+I}v\Longleftrightarrow 0=\Par{T^2+I}v.$\PfEnd\vspace{4pt}\parSol{}
%\Or Note that $\forall v\in V,\,v=\Par{I-T^2}v\Big/2+\Par{I+T^2}v\Big/2.$ 又 $I-T^4=\Par{I\pm T^2}\Par{I\mp T^2}.$\parSol{}
%Then $\Range\Par{I\mp T^2}\subseteq\Null\Par{I\pm T^2}\Rightarrow V=\Null\Par{I-T^2}+\Null\Par{I-T^2}.$\parSol{}
%又 $T$ has no eigvals $\Longleftrightarrow \Par{I-T^2}$ inje $\Longleftrightarrow\Null\Par{I-T^2}=\zeroSubs\supseteq\Range\Par{I+T^2}.$\PfEnd
%\SepLine

%\Anchor{5BI8}\Anchor{5A4e31}\ProblemN{I.8}{
%	\TextA{Give an exa of $T\in\Lm{\Rbb^2}$ suth $T^4=-I$.}
%}Define $\i^n\in\Lm{\Rbb^2}$ by $\i^n\Par{x,y}=\XPar{\REAL\BigPar{\i^n x+\i^{n+1} y},\,\IMAGINARY\BigPar{\i^n x+\i^{n+1} y}}.$\vspace{0pt}\par\quad
%$\displaystyle T^4+I=\BigPar{T^2+\i I}\BigPar{T^2-\i I}=\BigPar{T+\i^{1/2}I}\BigPar{T-\i^{1/2}I}\BigBigPar{T-\Par{{-\i}}{^{1/2}}I}\BigBigPar{T+\Par{{-\i}}{^{1/2}}I}.$\vspace{2pt}\par\quad
%Note that $\i^{1/2}=\frac{\sqrt{2}}{2}+\i\frac{\sqrt{2}}{2},\;\Par[1pt]{{-\i}}{^{1/2}}=\i^{3/2}=-\frac{\sqrt{2}}{2}+\i\frac{\sqrt{2}}{2}.$ \;Hence $T=\pm\Par[1pt]{{\pm\i}}{^{1/2}}I.$\vspace{0pt}\PfEnd\quad
%\Or Becs $\Mt{T^4}={}$\small$\begin{pmatrix}
%	\Blind{-}\cos\SmallPar{\!-\!\pi} & \sin\SmallPar{\!-\!\pi}\\
%	-\sin\SmallPar{\!-\!\pi} & \cos\SmallPar{\!-\!\pi}
%\end{pmatrix}$\large. \;Using {\small$\begin{pmatrix}\Blind{-}\cos\alpha & \sin\alpha\\-\sin\alpha & \cos\alpha\end{pmatrix}$}$\begin{array}{l}\hspace{-5pt}{}_n\\\\\vspace{-6pt}\end{array}\hspace{-8pt}={}${\small$\begin{pmatrix}\Blind{-}\cos n\alpha & \sin n\alpha\\-\sin n\alpha & \cos n\alpha\end{pmatrix}$}.\vspace{-4pt}\PfEnd%\par\vspace{-6pt}\quad
%%\Blind{\Or}We define $T\in\Lm{\Rbb^2}$ suth $\Mt{T}={}${\small$\begin{pmatrix}
%		%\Blind{-}\cos\SmallPar{\!-\!{\pi}/{4}} & \sin\SmallPar{\!-\!{\pi}/{4}}\\
%		%-\sin\SmallPar{\!-\!{\pi}/{4}} & \cos\SmallPar{\!-\!{\pi}/{4}}\end{pmatrix}$}.\PfEnd\vspace{2pt}
%\SepLine

%\Anchor{5BI4e3}\ProblemBnoor{4E 3}{
	%	\TextB{Find the min poly of $T:\Par{x_1,\dots,x_n}\mapsto\Par{x_1+\dots+x_n,\cdots,x_1+\dots+x_n}.$}
	%}If $n=1$ then $T=I\in\Lm{\Fbb}$ with $\Par{z-1}$ as the min. Supp $n>1.$ Then $T\neq I.$\parSol{}
%Now each $Te_k=e_1+\dots+e_n;\;T^2 e_k=n \Par{e_1+\dots+e_n}=nTe_k\Rightarrow T^2-nT=T\Par{T-n}=0.$\PfEnd
%\SepLine

\Anchor{5BI9}\ProblemN{I.9}{
	\TextA{Supp $V$ finide, $T\in\Lm{V},$ and non0 $v\in V.$ Let $p\in\PoFi$ be non0 of smallest deg}
	\TextA{with $p\Par{T}v=0.$ \,Show every zero of $p$ is eigval of $T.$\hfill\tgnr\FontNorm By div algo, $p$ div the min.}
}\Or Let $p\Par{z}=\Par{z-\lambda}q\Par{z}\Rightarrow p\Par{T}v=0=\Par{T-\lambda I}q\Par{T}v\Rightarrow T\BigPar{q\Par{T}v}=\lambda q\Par{T}v.$\PfEnd
\SepLine

\Anchor{8C15}\Anchor{5BII4e7}\Anchor{5BIT1}\ProblemN{\BulletPointX I.\TipsN{1}}{
	\TextA{Supp $V$ is finide, $T\in\Lm{V},$ and $v \in V$.}
	\PrePa\TextA{Prove $\exists\,!$ monic $p_v$ of smallest deg suth $p_v\Par{T}v = 0$.}
	\PrePb\TextA{Prove $p_v$ is the min $q$ of $T\mmid_{\nullp p_v\SmallPar{T}}.$\hfill\tgnr\FontNorm So that the min of $T$ is a multi of $p_v.$}
}(a) {\Existns} \;If $v=0,$ then let $p_v\Par{z}=1.$ Supp $v\neq0.$ Then $\Par{v,Tv,\dots,T^{\dim V}v}$ liney dep.\parSol{\Ha}
\Blind{\Existns} \;$\exists$ smallest $m$ suth $-T^mv=c_0v+c_1Tv+\dots+c_{m-1}T^{m-1}v.$ Thus define $p_v.$\vspace{2pt}\parSol{\Ha}
\Blind{\Existns} \;\Or Let \uline{$U=\Span{v,Tv,\dots,T^{m-1}v}$} of dim $m$ invard $T.$ Let $p_v$ be the min of $T\mmid_U.$\vspace{4pt}\parSol{\Ha}
{\Uniqnes} \;Supp $q_v$ is monic of smallest deg \Sbra{$={}\deg p_v$} and $q_v\Par{T}v=0.$\parSol{\Ha}
\Blind{\Uniqnes} \;Then $\Par{p_v-q_v}\Par{T}v=0,$ while $\deg p_v=m=\deg q_v\Rightarrow\deg\Par{p_v-q_v}<m.$\vspace{2pt}\parSol{}
(b) Becs $p_v\Par{T\mmid_{\nullp p_v\SmallPar{T}}}=0\Rightarrow p_v$ is multi of $q.$ 又 $q\Par{T}v=0\Rightarrow q=p_v,$ by the min of $\deg p_v.$\PfEnd
\SepLine

%\Anchor{5BI10}\ProblemN{I.10}{
%	\TextA{Supp $T\in\Lm{V},\lambda$ is eigval of $T$ with $v.$ Prove if $p\in\PoFi,$ then $p\Par{T}v=p\Par{\lambda}v.$}
%}Define $p\Par{z}=a_0+a_1 z+\dots+a_m z^m.$ Becs for each $k\in\Nbp,T^k v=\lambda^k v,$ and $T^0v=v.$\parSol{}
%Now $p\Par{T}v=a_0v+a_1\lambda v+\dots+a_m \lambda^m v=p\Par{\lambda}v.$\PfEnd\vspace{2pt}
%\ACoro $p\Par{T}v=\BigSbra[0pt]{c\Par{T-\lambda_1I}{^{\alpha_1}}\cdots\Par{T-\lambda_mI}{^{\alpha_m}}}v=\BigSbra[0pt]{c\Par{\lambda-\lambda_1}{^{\alpha_1}}\cdots\Par{\lambda-\lambda_m}{^{\alpha_m}}}v.$
%\SepLine
\pagebreak

\ProblemN{\Anchor{5BI11}{11}}{
	\TextA{Supp $\Fbb = \Cbb,\,T\in\Lm{V},$ nonC $p\in\PoFi.$}
	\TextA{Prove $\alpha$ is eigval of $p\Par{T}$ $\Longleftrightarrow\alpha = p\Par{\lambda}$ for some eigval $\lambda$ of $T$.}
}Supp $p\Par{T}-\alpha I$ not inje. Let $p\Par{z}-\alpha=c\Par{z-\lambda_1}\cdots\Par{z-\lambda_m},$ with $c\neq 0,$ becs $p$ nonC.\parSol{}
Then $\exists\,\Par{T-\lambda_j I}$ not inje. Now $p\Par{\lambda_j}-\alpha=0.$ \,Convly true immed.\PfEnd
\SepLine

%\Anchor{5BI12}\ProblemN{12}{
%	\TextA{Give a $T\in\Lm{\Rbb^2}$ that shows the result above does not hold if $\Cbb$ is replaced with $\Rbb$.}
%}Define $T\Par{w,z}=\Par{{-z,w}},$ with the min $z^2+1.$ \,Let $p\Par{z}=z^2\Rightarrow p\Par{T}=T^2$ has eigval $-1.$\PfEnd
%\SepLine

%\BulletPointX\NoteFor{[5.21]} \TextB{\large Every optor on a finide non0 complex vecsp has an eigval.}
%Supp $V$ is a finide non0 complex vecsp, $\dim V=n$ and $T\in\Lm{V}$.\TextB{}
%Supp non0 $v\in V.$ Becs $\Par{v,Tv,T^2 v,\dots,T^n v}$ is liney dep.\TextB{}
%Let $a_0 I+a_1 T+\dots+a_n T^n=0.$ Then some $a_j\neq 0.$\TextB{}
%{\tgsl Thus $\exists$ nonC $p$ of smallest deg with $p\Par{T}v=0$.}\TextB{}
%Becs $\exists\,\lambda \in \Cbb$ suth $p\Par{\lambda} = 0\Rightarrow\exists\,q \in\PoCi,p\Par{z} = \Par{z-\lambda}q\Par{z},\forall z \in\Cbb.$\TextB{}
%Thus $0 = p\Par{T}v = \Par{T -\lambda I}\BigPar{q\Par{T}v}$. By the min of $\deg p$ and $\deg q<\deg p$, $q\Par{T}v \neq 0$.\TextB{}
%Then $\Par{T-\lambda I}$ is not inje. Thus $\lambda$ is an eigval of $T$ with eigvec $q\Par{T}v$.\par
%\BulletPointX\Example\,\,\, \TextB{\large an optor on a complex vecsp with no eigvals}
%Define $T\in\Lm[\BigPar]{\PoCi}$ by $\Par{Tp}\Par{z} = zp\Par{z}$.\TextB{}
%Supp $p\in\PoCi$ is a non0 poly. Then $\deg Tp=\deg p+1,$ and thus $Tp\neq\lambda p,\,\forall\lambda\in\Cbb.$\TextB{}
%Hence $T$ has no eigvals.\par
%\SepLine

\ProblemB[]{
	\TextB{Supp non0 $v\in V.$ Prove [5.21] using the given map below,\hfill\FontNorm and also [4E 5.22], in Exe (I.17).}
}
\Anchor{5BI16}\ProblemN[]{I.16}{
	\TextA{Define $S:\PoC{\dim V}\rightarrow V$ by $S\Par{p}=p\Par{T}v.$ \,\FontNorm Then $S$ not inje $\Rightarrow\exists$ non0 $p\in\null S$.}
}
\ProblemN[]{\Anchor{5BI17}{I.17}}{
	\TextA{Define $S:\PoC{\dim V^2}\rightarrow\Lm{V}$ by $S\Par{p}=p\Par{T}.$ \,\FontNorm Then $S$ not inje $\Rightarrow\exists$ non0 $p\in\null S$.\vspace{-3pt}}
}%Let $p\Par{z}=c\Par{z-\lambda_1}\cdots\Par{z-\lambda_m}\Rightarrow\Par{T-\lambda_1 I}\cdots\Par{T-\lambda_m I}$ not inje.\PfEnd
\SepLine

%\Anchor{5BI18}\Anchor{5BI4e15}\ProblemNnoor{I.18}{4E I.15}{
%	\TextA{Supp $\Fbb=\Cbb,$ $V$ finide and non0, $T\in\Lm{V}$.}
%	\TextA{Define $f:\Cbb\rightarrow\Nbb$ by $f\Par{\lambda} = \dim \Range\Par{T-\lambda I}$. Prove $f$ is not continuous.}
%}Let $\lambda_0$ be eigval of $T.$ Then $\Par{T-\lambda_0 I}$ is not surj. Hence $\dim\Range\Par{T-\lambda_0 I}<\dim V.$\parSol{}
%Becs $T$ has finily many eigvals. $\exists$ seq $\Bra{\lambda_n}$ with each $\lambda_n$ not eigval of $T,$ suth $\lim\limits_{n\rightarrow\infty}\lambda_n=\lambda_0$\parSol{}
%Becs each $f\Par{\lambda_n}=\dim\Range\Par{T-\lambda_n I}=\dim V\neq f\Par{\lambda_0}\Rightarrow f\Par{\lambda_0}\neq\lim\limits_{n\rightarrow\infty}\,f\Par{\lambda_n}.$\PfEnd
%\SepLine

%\Anchor{5A4e43}\ProblemN{\Anchor{5BI19}{I.19}}{
%	\TextA{Supp $V$ is finide, $\dim V>1,T\in\Lm{V}.$ Prove $\Bra{\:\!p\Par{T}:p\in\PoFi}\neq\Lm{V}.$}
%}If $\forall S\in\Lm{V},\exists\,p\in\PoFi,S=p\Par{T}.$ Then by [5.20], $\forall S_1,S_2\in\Lm{V},S_1S_2=S_2S_1.$\parSol{}
%Note that $\dim V\geqslant 2.$ By (3.A.14) \OR (3.D.16 \OR 4E 3.A.11).\PfEnd
%\SepLine

%\Anchor{5BIN4e5.22}\ProblemBX{\NoteForSmall{[4E 5.22]}}{
%	\TextB{Supp $V$ finide, $T\in\Lm{V}.$}
%	\TextB{Prove $\exists\,!$ monic $p\in\PoF{\dim V}$ of smallest deg, suth $p\Par{T}=0.$}
%}Using induc on $\dim V.$ (i) $\dim V=0.$ Let $p=1\Rightarrow p\Par{T}=T=0.$\par\quad
%(ii) Asum for each $U$ of smaller dim,
%{\tgsl $\forall S\in\Lm{U},\exists$ monic $s\in\PoF{\dim U}$, suth $s\Par{S}=0.$}\par\quad\Hii
%Let $u\in V\nonzero.$ Then $\BigPar{Iu,Tu,\dots,T^{\dim V}u}$ liney dep\par\quad\Hii
%$\Rightarrow\exists$ smallest $m\in\Nbp$ suth $c_0Iu+c_1Tu+\dots+c_{m-1}T^{m-1}u+T^mu=0.$ Thus define $q\in\PoF{m}.$\par\quad\Hii
%\NOTICE that $q\Par{T}\Par{T^ku}=0\Rightarrow\Span{Iu,Tu,\dots,T^{m-1}u}\subseteq\nullp q\Par{T}.$\par\quad\Hii
%Hence $\dim\nullp q\Par{T}\geqslant m\Longleftrightarrow\dim\rangep q\Par{T}\leqslant\dim V-m.$\par\quad\Hii
%Becs $\rangep q\Par{T}$ invard $T,$ $S=T\Big|{_{\rangep q\SmallPar{T}}}\in\Lm[\BigPar]{\rangep q\Par{T}}.$ Now by asum,\par\quad\Hii
%$\exists$ monic $s\in\PoF{\dim V-m}$, suth $s\Par{S}=0\Rightarrow\Par{sq}\Par{T}=0.$\quad\Sbra{{\tgsl The remaining part is obvious.}}\Blind{\;\;}\PfEnd
%\SepLine

\Anchor{5BI4e7}\ProblemBnoor{4E I.7}{
	\TextB{Supp $S, T\in\Lm{V}$ and $p,q$ are mins of $ST,TS$ respectly. Prove $S$ or $T$ is inv $\Rightarrow$ $p=q.$}
}$S$ inv $\Rightarrow p\Par{TS}=S^{-1}p\Par{ST}S=0$ and $q\Par{ST}=Sq\Par{TS}S^{-1}=0\Rightarrow p=q.$ Rev the roles.\PfEnd
\SepLine

\Anchor{5BI4e21}\ProblemBnoor{4E I.21}{
	\TextB{Supp $V$ finide, $T\in\Lm{V}$. Prove the min $p$ has deg at most $1 + \dim \range T$.}
}Let $q$ be the min of $T\mmid_{\range T}.$ Then $q\Par{T}Tv=0\Rightarrow zq\Par{z}$ of deg $\leqslant1+\dim\range T$ is multi of $p.$\PfEnd
\SepLine

%\Anchor{5BI4e25}\Anchor{5BI4e26}\ProblemBnoor{4E I.25,26}{
%	\TextB{Supp $V$ is finide, $U$ invarspd $T\in\Lm{V},$ with the min $p.$}
%	\TextB{Supp $r$ the min of $T\mmid_U$, and $s$ of $T\XSlash U.$ Then $p$ is a multi of $r$ and of $s,$ and $rs$ is multi of $p.$}
%	\TextB{For $q\in\PoFi{},$ define $Z_q$ as the set of zeros of $q.$ Then $Z_p$ is the set of eigvals of $T.$}
%	\TextB{Simlr for $Z_r,Z_s.$ Prove $Z_r\cup Z_s\supseteq Z_p.$}
%}By (b), $Z_p\subseteq Z_r\cup Z_s.$ Let $ar=p$ and $bs=p.$ Then for $r\Par{\lambda}=0$ or $s\Par{\lambda}=0,$\parSol{}
%\!\Sbra{which is equiv to $\lambda\in Z_r\cup Z_s,$} then $p\Par{\lambda}=0\Longleftrightarrow\lambda\in Z_p.$\PfEnd
%\SepLine

\Anchor{5BI4e28}\ProblemBnoor{4E I.28}{
	\TextB{Supp $V$ is finide and $T\in \Lm{V}$. Prove the min $p$ of $T\apostrophe$ equals the min $q$ of $T$.}
}$\forall\varphi\in V\apostrophe,\,p\Par{T\apostrophe}\Par{\varphi}=\varphi\circ p\Par{T}=0\Rightarrow\range p\Par{T}\subseteq C^0 V\apostrophe.$ Thus $p\Par{T}=0.$ 又 $\varphi\circ q\Par{T}=0.$\PfEnd\vspace{2pt}\parSol{}
\Or By (3.F.15), for any $s\in\PoFi,\:s\Par{T\apostrophe}=s\Par{T}\apostrophe=0\Longleftrightarrow s\Par{T}=0.$ Simlr.\PfEnd
\SepLine

\Anchor{5BI4e16}\Anchor{8C18}\ProblemB{
	\TextB{Define $T\in\Lm{\FbbP{n}}:\Par{x_1,\dots,x_n}\mapsto\BigPar{{-a_0\,x_n,\;x_1-a_1\hspace{1pt}x_n,\,\cdots,\;x_{n-1}-a_{n-1}\hspace{1pt}x_n}}.$\vspace{2pt}}
	\TextB{Show the min $p$ of $T$ is $q\Par{z}=a_0+a_1z+\dots+a_{n-1}z^{n-1}+z^n.$\vspace{2pt}}
}Becs $Te_1=e_2,\,T^2e_1=e_3,\cdots,\,T^{n-1}e_1=e_n,\,T^ne_1=T^{n-k}e_{k+1}=Te_n=-\Par{a_0\hspace{1pt}e_1+\dots+a_{n-1}\hspace{1pt}e_n}.$\parSol{}
Let $-T^n=c_0I+c_1T+\dots+c_{n-1}T^{n-1}\Rightarrow$ each $c_k=a_k.$ Becs $n=\dim V.$ No smaller deg.\PfEnd
\SepLine

\Anchor{5BI4e8}\ProblemBnoor{4E I.8}{
	\TextB{Find the min $p$ of $T\in\Lm{\Rbb^2},$ the countclockws rotat optor by $\theta\in\Rbb^+$.\vspace{1pt}}
}If $\theta=2k\pi,$ then $p\Par{z}=z-1.$ If $\theta=\pi+2k\pi,$ then $p\Par{z}=z+1.$\vspace{-28pt}\parSol{}
\hfill\includegraphics[width=4.4cm,height=3.2cm,scale=0.22]{diagram5BI-1.png}\vspace{-68pt}\parSol{}
Othws, let $\Span{v,Tv}=\Rbb^2.$ Let $L=x^2+y^2,$ where $v=\Par{x,y}.$\parSol{}
Supp $p\Par{z}=z^2+bz+c.$ Let $P=L\cos\theta\Rightarrow L\big/2P=1\big/\Par{2\cos\theta}.$\vspace{2pt}\parSol{}
Then $Tv=\BigPar{L\big/2P}\Par{T^2 v+v}\Rightarrow T=\BigPar{L\big/2P}\Par{T^2+I}.$\parSol{}
Hence $p\Par{T}=T^2-2\cos\theta\, T+I=0.$\PfEnd\vspace{2pt}\parSol{}
\Or Let $\Par{e_1,e_2}$ be the std bss. Becs $Te_1=\cos\theta\;e_1+\sin\theta\;e_2,\;T^2e_1=\cos2\theta\;e_1+\sin2\theta\;e_2.$\vspace{0pt}\parSol{}
$ce_1+bTe_1=-T^2 e_1\Longleftrightarrow{}${\small$\begin{pmatrix}1 &\hspace{-6pt} \cos\theta\\[-4pt]0 &\hspace{-6pt} \sin\theta\end{pmatrix}\begin{pmatrix}c \\[-4pt] b\end{pmatrix}$}${}={}${\small$\begin{pmatrix}-\cos2\theta\\[-4pt]-\sin2\theta\end{pmatrix}$}. Now $\det=\sin\theta\neq 0,\:c=1,\,b=-2\cos\theta.$\PfEnd%\vspace{10pt}\parSol{}
%\Or $\Mt[\BigPar]{T,\Par{e_1,e_2}}={}${\small$\begin{pmatrix}\Blind{-}\cos\theta & \sin\theta\\-\sin\theta & \cos\theta
%\end{pmatrix}$}. By (4E 11), $\Par{z\pm 1}$ or $\Par{z^2-2\cos\theta\,z+1}$ is the min.\PfEnd
\SepLine
		
\Anchor{5BI4e11}\ProblemBnoor{4E I.11}{
	\TextA{Supp $V$ is 2\hspace{1pt}-\hspace{1pt}dim, $T\in\Lm{V}$ with the min $p$, and $\Mt[\BigPar]{T,\Par{v,w}}={}${\large$\begin{pmatrix} a &\hspace{-4pt} c\\[-4pt] b &\hspace{-4pt} d\end{pmatrix}.$}\vspace{-6pt}}
	\PrePa\TextA{Show $q\Par{z}=\Par{z-a}\Par{z-d}-bc$ is a multi of $p.$}
	\PrePb\TextA{Show if $b=c=0$ and $a=d,$ then $p\Par{z}=z-a;$\;othws $p=q.$\vspace{1pt}}
}(a) $Tv=av+bw\Rightarrow\Par{T-aI}v=bw\Rightarrow\Par{T-dI}\Par{T-aI}v=bTw-bdw=bcv.$\parSol{\Ha}
$Tw=cv+dw\Rightarrow \Par{T-dI}w=cv\Rightarrow\Par{T-aI}\Par{T-dI}w=cTv-acv=bcw.$\vspace{0pt}\parSol{}
(b) {\vspace{-2pt}\FontSmall If $b=c=0$ and $a=d.$ Then $\Mt{T}=a\Mt{I}\Rightarrow T=aI.$ Othws, we show $T\not\in\Span{I},$}\parSol{\Hb}
{\vspace{-2pt}\FontSmall so that $\deg p=\dim V.$ Let (1) $a=d,$ (2) $b=0,$ (3) $c=0.$ Then (1), (2) and (3) cannot be all true.}\parSol{\Hb}
{\vspace{-2pt}\FontSmall(I) Asum (1) is true, with (2) or (3) not true. Then $Tv=av+bw,$ or $Tw=cv+aw\not\in\Span{w}.$}\parSol{\Hb}
{\vspace{-2pt}\FontSmall(II) Asum (2) or (3) are true, with (1) not true. Then $Tv=av+bw,$ or $Tw=cv+dw.$}\PfEnd
\SepLine

\Anchor{5BI4e29}\ProblemBnoor{4E I.29}{
	\TextB{Supp $V$ is finide, $\dim V=n\geqslant 2$, and $T\in\Lm{V}.$ Show $T$ has a $2$\hspace{1pt}-\hspace{1pt}dim invarsp.}
}See [9.8] for a graceful proof. \;\Or Let each $V_{\!k}$ be an arb vecsp of dim $k$ with an arb $T_{\!k}\in\Lm{V_{\!k}}.$\par\quad
Define the stmt $P\Par{k}:$ every optor on a $V_{\!k}$ has invarsp of dim $2.$ (i) $k=2.$ Immed.\par\quad
(ii) $k\geqslant 2.$ Asum $P\Par{k}$ holds. Let $p$ be the min of $T_{\!k+1}=T.$ Note that $V_{\!k+1}$ non0 $\Rightarrow p$ nonC, $\deg p\geqslant 1.$\par\quad
(a) If $p\Par{z}=\Par{z-\lambda}q\Par{z},$ then by (4E 5.A.39), $\exists\,U$ invarspd $T$ of dim $k.$\par\quad\Ha
By asum, the optor $T\mmid_U$ on a $k$\hspace{1pt}-\hspace{1pt}dim vecsp has invarsp of dim $2$, so has $T.$\vspace{2pt}\par\quad
(b) Othws, $T_{\!k+1}$ has no eigvals $\Rightarrow p$ of deg $\geqslant 1$ has no zeros, thus $\Fbb=\Rbb,$ and $\deg p$ is even.\par\quad\Hb
Let $p\Par{z}=\Par{z^2+b_1z+c_1}\cdots\Par{z^2+b_mz+c_m}\Rightarrow\exists\,\Par{T^2+b_jT+c_j}$ not inje\par\quad\Hb
$\Rightarrow\exists\,v\neq 0,\Par{T^2+b_jT+c_j}v=0\Rightarrow T^2v\in\Span{v,Tv},$ invard $T,$ while $\dim\Span{v,Tv}=2.$\PfEnd
\SepLine

\Anchor{5BIN4e5.33}\ProblemBX{\NoteForSmall{[4E 5.33]}}{
	\TextA{Supp $\Fbb=\Rbb,$ $V$ is finide, $T\in\Lm{V},$ and $b^2<4c$ for $b,c\in\Fbb.$}
	\TextB{Prove $\dim\Null\BigPar{T^2+bT+cI}{^j}$ is even for each $j\in\Nbp.$}
}Using induc on $j.$ \,(i) Immed. \,(ii) $j>1.$ Asum it holds for $j-1.$\parSol{}
Replace $V$ with $\Null\Par{T^2+bT+cI}{^{j}}$ and $T$ with $T$ restr to $\Null\Par{T^2+bT+cI}{^{j}}.$\parSol{}
Then $\Par{T^2+bT+cI}{^j}=0\Rightarrow \Par{z^2+bz+c}{^j}$ is a multi of the min of $T\Rightarrow$ no eigvecs for $T.$\parSol{}
Let $U$ be invarspd $T$ and has the largest even dim of all such invarsp. If $V=U,$ done. Othws,\parSol{}
for $w\in V\Backslash U\Rightarrow W=\Par{w,Tw}$ invard $T$ of dim $2\Rightarrow U+W$ of dim $\Par{{\dim U+2}}$ invard $T.$\PfEnd\vspace{2pt}\parSol{}
\Or Let $q\Par{z}=z^2+bz+c.$ Note that the min of $T$ restr to each $\nullp q\Par{T}{^j}$ has no real zeros.\parSol{}
If some $\dim\nullp q\Par{T}{^j}$ is odd. Then $T$ restr to $\nullp q\Par{T}{^j}$ must have a real eigval, ctradic.\PfEnd
\SepLine

\ProblemB[]{
	\TextA{Supp $V$ finide, $T\in\Lm{V}$ with the min $p.$}
}
\Anchor{5BI4e13}\ProblemBnoor{4E I.13}{
\TextA{Prove $\forall q\in\PoFi,$ $\exists\,!\,r\in\PoF{\deg p-1},\,q\Par{T}=r\Par{T}.$}
}Becs $p\neq 0.$ By the div algo, immed. \Sbra{$r=0$ if $q=p.$} \;\;\Or By Exe (4E I.19).\PfEnd\vspace{2pt}\parSol{}
\Or Let $\deg p=m.$ Becs $T^{m}\in\Span{I,T,\dots,T^{m-1}}.$ For $\deg q<m,$ the repres of $q\Par{T}$ is uniq.\parSol{}
If $\deg q\geqslant m.$ For each $k\in\Nbb,\exists\,!\,b_{j,k}\in\Fbb,\,T^{m+k}=b_{0,k}I+b_{1,k}T+\dots+b_{m-1,k}T^{m-1}.$\PfEnd
\SepLine[0pt][\Blind{\BulletPointX} ]

\Anchor{5BI4e19}\ProblemBnoor{4E I.19}{
	\TextA{Let $\mE =\Bra{q\Par{T}: q \in \PoFi},$ a subsp of $\Lm{V}.$ Prove $\dim\mE=\deg p.$}
}Becs $\mE=\Span[\BigPar]{I,T,\dots,T^{\dim\Lm[\SmallPar]{V}\,-\,1}}=\Span[\BigPar]{I,T,\dots,T^{\deg p\,-\,1}},$ by Exe (4E I.13). Immed.\PfEnd\vspace{2pt}\parSol{}
\Or Define $\Phi\in\Lm[\BigPar]{\PoFi,\Lm{V}}$ by $\Phi\Par{q}=q\Par{T}\Rightarrow\range\Phi=\mE.$\parSol{}
Becs $\Phi\Par{q}=q\Par{T}=0\Longleftrightarrow q$ is a multi of the min $p\Longleftrightarrow q\in\;\!\!\Bra{ps:s\in\PoFi}=\null\Phi.$\parSol{}
Now by (4.11), $\dim\PoFi\XSlash\null\Phi=\deg p.$ By [3.91](d).\PfEnd
\SepLine[0pt][\Blind{\BulletPointX} ]

\Anchor{8C11}\ProblemBnoor{8.C.11}{
	\TextA{Supp $T\in\Lm{V}$ is inv. Prove $\exists\,q\in\PoFi,T^{-1}=q\Par{T}.$}
}Becs the const term of $p$ is non0. Let $I=a_1T+\dots+a_mT^m\Rightarrow T^{-1}=a_1I+a_2T+\dots+a_mT^{m-1}.$\PfEnd
\SepLine[0pt][\Blind{\BulletPointX} ]

\Anchor{5BI4e14}\ProblemBnoor{4E I.14}{
	\TextA{Supp $p\Par{z}=a_0 + a_1 z + \dots + a_{m-1} z^{m-1}+z^m,$ and $a_0\neq 0.$\vspace{1.5pt}}
	\TextA{Give a repres of $s,$ the min of $T^{-1}.$\hfill\FontNorm\tgnr $s\Par{z}=z^m\,p\Par{0}{^{-1}}\,p\Par{z^{-1}},\:z\neq0.$\vspace{3pt}}
}Define $q\Par{z}=z^m+{}${\Large$\frac{\:a_1\:}{a_0}$}$\,z^{m-1}+\dots+{}${\Large$\frac{\:a_{m-1}\:}{a_0}$}$\,z+{}${\Large$\frac{\:1\:}{a_0}$}${}\Rightarrow q\Par{T^{-1}}=T^{-m}p\Par{T}=0.$\vspace{2pt}\parSol{}
Now $\deg s\leqslant\deg q=\deg p.$ \;Revly, $\deg q=\deg p\leqslant\deg s.$\PfEnd\vspace{4pt}\parSol{}
\Or Becs each $T^{-k}\not\in\Span[\BigPar]{I,T^{-1},\dots,T^{-\SmallPar{k-1}}}$ for $k\in\;\!\!\Bra{1,\dots,m-1}.$ Done.\parSol{}
For if not, supp $T^{-k}=b_0 I+b_1 T^{-1}+\dots+b_{k-1}T^{k-1}.$ Note that $T$ inv $\Rightarrow\exists\,b_j\neq 0.$\parSol{}
Now $T^k\Par{T^{-k}}=I=b_0 T^k+b_1 T^{k-1}+\dots+b_{k-1}T\Rightarrow T^j\in\Span{I,T,\dots,T^{k-1}}.$\PfEnd
\SepLine[0pt][\Blind{\BulletPointX} ]

\Anchor{5BI4e17}\ProblemBnoor{{4E I.17}}{
	\TextA{Show the min $s$ of $\Par{T-\lambda I}$ is $q\Par{z}=p\Par{z+\lambda}.$}
}Becs $\deg q=\deg p,$ and $q\Par{T-\lambda I}=p\Par{T}=0\Rightarrow q$ a multi of $s.$\parSol{}
Now the deg of min $p$ of $T$ is no less than the deg of min $s$ of $\Par{T-\lambda I}.$\parSol{}
Revly, the deg of min $s$ of $S=T-\lambda I$ \,is no less than the deg of min $p$ of $\Par{S+\lambda I}.$\PfEnd\parSol{}
\Or Define $r\Par{z}=s\Par{z-\lambda}\Rightarrow r\Par{T}=0\Rightarrow\deg r=\deg s\geqslant\deg p.$\PfEnd\vspace{3pt}\parSol{}
\Or Becs  $T^k\in\Span{I,T,\dots,T^{k-1}}=\Span[\BigBigPar]{I,\Par{T-\lambda I},\dots,\Par{T-\lambda I}{^{k-1}}}\ni\Par{T-\lambda I}{^k}.$\PfEnd
\SepLine[0pt][\Blind{\BulletPointX} ]

\Anchor{5BI4e18}\ProblemBnoor{{4E I.18}}{
	\TextA{Supp $\deg p=m,$ and $\lambda\neq 0.$ Show the min $s$ of $\lambda T$ is $q\Par{z}=\lambda^m p\BigPar{{z}\big/{\lambda}}.$}
}Becs $\deg q=\deg p,$ and $q\Par{\lambda T}=\lambda^m p\Par{T}=0\Rightarrow q$ is multi $s.$\parSol{}
Now the deg of min $p$ of $T$ is no less than the deg of min $s$ of $\lambda T.$\parSol{}
Revly, the deg of min $s$ of $S=\lambda T$ is no less than the deg of min $p$ of $\lambda^{-1}S.$\PfEnd\parSol{}
\Or Define $r\Par{z}=s\Par{\lambda z}\Rightarrow r\Par{T}=0\Rightarrow\deg r=\deg s\geqslant\deg p.$\PfEnd\vspace{3pt}\parSol{}
\Or Becs $\Par{\lambda T}{^k}\in\Span[\BigBigPar]{\lambda I,\lambda T,\dots,\Par{\lambda T}{^{k-1}}}=\Span{I,T\dots,T^{k-1}}\ni T^k.$\PfEnd
\SepLine[0pt][\Blind{\BulletPointX} ]

\Anchor{5BI4e10}\Anchor{5BI4e23}\ProblemBnoor{4E I.10,23}{
	\TextA{Supp $\deg p=m,$ and non0 $v\in V.$ Let each $U_k=\Span[\BigPar]{v,Tv,\dots,T^kv}.$\vspace{1pt}}
	\TextA{Prove $\exists\,j\in\;\!\!\Bra{1,\dots,m},\;U_{j-1}=U_n$ for all $n\geqslant j-1.$}
}Supp $j$ is the smallest suth $T^j v=a_0v+a_1 Tv+\dots+a_{j-1}T^{j-1}v\in U_{j-1}\Rightarrow j\leqslant m.$\parSol{}
Then $U_{j-1}$ is invard $T,$ so is each $U_n=\Span{v,Tv,\dots,T^{j-1} v,\dots,T^n v}.$\PfEnd
\SepLine

\ChDecl{Ch5BII}{}{}
%\Anchor{5BII4e2}\ProblemN{II.2}{
%	\TextA{Supp $A,B\in\FbbP{n,n}$ are up-trig with each $A_{p,\,p}=\alpha_p,B_{p,\,p}=\beta_p.$\hfill\tgnr\FontNorm $A_{j,k}=B_{j,k}=0$ for $j>k.$}
%	\TextA{Show $AB$ up-trig with $\alpha_1 \beta_1 , \dots , \alpha_n \beta_n$ on the diag.}
%}Each $\Par{AB}{_{p,\,p}}=A_{p,1}B_{1,\,p}+\dots+A_{p,\,p-1}B_{p-1,\,p}+A_{p,\,p}B_{p,\,p}+A_{p,\,p+1}B_{p+1,\,p}+\dots+A_{p,n}B_{n,\,p}=A_{p,\,p}B_{p,\,p}.$\PfEnd
%\SepLine

%\Anchor{5BII4e3}\ProblemN{II.3}{
%	\TextA{Supp $T$ inv and up-trig wrto $B_V=\Par{v_1,\dots,v_n}.$ Show $T^{-1}$ is up-trig wrto $B_V.$}
%}Each $\Span{v_1,\dots,v_k}$ invard $T\Rightarrow$ invard $T^{-1}=q\Par{T},$ by (8.C.11).\PfEnd
%\vspace{2pt}\parSol{}
%\Or Let each $Tv_k=u_k+\lambda_kv_k,$ where $u_k\in\Span{v_1,\dots,v_{k-1}}.$\parSol{}
%We use induc on $k$ to show each $\Span{v_1,\dots,v_k}$ invard $T^{-1}.$ (i) Immed. (ii) $2\leqslant k\leqslant n-1.$\parSol{}
%Asum $\Span{v_1,\dots,v_{k-1}}$ invard $T^{-1}.$ Then for each $Tv_k=u_k+\lambda_kv_k,$\parSol{}
%$T^{-1}v_k=\lambda_k^{-1}v_k-\lambda_k^{-1}T^{-1}u_k\in\Span{v_1,\dots,v_k},$ invard $T;$ and $\lambda_{k}^{-1}$ is the $k^\text{th}$ ent on diag.\PfEnd
%\SepLine

%\Anchor{5BII14}\Anchor{5BII4e4}\ProblemNor{4}{3E 5.B.14}{
%	\TextA{Give an inv $T$ and a $B_V$ suth each $\Mt{T}{_{k,k}}=0.$}
%}Define $T\in\Lm{\Rbb^2}:\Par{x,y}\mapsto\Par{y,x}.$
%\SepLine
%
%\Anchor{5BII15}\Anchor{5BII4e5}\ProblemNor{5}{3E 5.B.15}{
%	\TextA{Give a non-inv $T$ and a $B_V$ suth each $\Mt{T}{_{k,k}}\neq 0.$}
%}Define $T\in\Lm{\FbbP{2}}:\Par{z,w}\mapsto\Par{z+w,z+w}.$
%\SepLine

%\Anchor{5BII20}\Anchor{5BII4e6}\ProblemNor[]{6}{3E 5.B.20}{
%	\TextA{Supp $\Fbb=\Cbb,$ $V$ is finide, $T\in\Lm{V},$ and $k\in\;\!\!\Bra{1,\dots,\dim V}.$}
%	\TextA{Prove $V$ has a $k$\hspace{1pt}-\hspace{1pt}dim invarspd $T.$\hfill\FontNorm\tgnr By [5.27] and [5.26], immed.\vspace{-3pt}}
%}\SepLine

\Anchor{5BII4e8}\ProblemN{II.8}{
	\TextA{Supp $V$ is finide, and $v\in V$ is non0 suth $q\Par{T}v=0,$ where $q\Par{z}=z^2+2z+2.$}
	\PrePa\TextA{Supp $\Fbb=\Rbb.$ Prove $\nexists\,B_V$ suth $\Mt{T}$ up-trig.}
	\PrePb\TextA{Supp $\Fbb=\Cbb,$ and $\exists\,B_V$ suth $A=\Mt{T}$ up-trig. Prove $-1+\i$ or $-1-\i$ on diag.}
}Define $p_v$ as in \Sbra{I \TIPSN{1}}. Note that $v\neq0\Rightarrow\deg p_v\neq0.$ 又 $q\Par{T\mmid_{\nullp p_v\SmallPar{v}}}=0.$\parSol{}
Now $q$ of deg $2$ is a multi of the min of $T\mmid_{\nullp p_v\SmallPar{v}},$ which is $p_v,$ of which the min of $T$ is a multi.\parSol{}
(a) Note that $q$ has no $1$\hspace{1pt}-\hspace{1pt}deg factors $\Rightarrow\deg p_v\geqslant 2.$ By [4E 5.44].\parSol{}
(b) $q\Par{z}=\Par{z+1+\i}\Par{z+1-\i}\Rightarrow -1-\i$ or $-1+\i$ zero of $p_v\Rightarrow$ is eigval $\Rightarrow$ on diag.\PfEnd
\SepLine

%\Anchor{10A5}\Anchor{5BII4e9}\ProblemN{9}{
%	\TextA{Supp $B\in\CbbP{n,n}.$ Prove $\exists$ inv $A\in\CbbP{n,n}$ suth $A^{-1} BA$ is up-trig.}
%}Define $T\in\CbbP{n}$ with $B=\Mt[\BigPar]{T,\Par{e_1,\dots,e_n}}.$ Let $C=\Mt[\BigPar]{T,\Par{f_1,\dots,f_n}}$ be up-trig.\parSol{}
%Let $A=\Mt[\BigPar]{I,\,f\rightarrow e}.$ Then $C=A^{-1}BA.$\PfEnd
%\SepLine

%\Anchor{5BII4e10}\ProblemN{II.10}{
%	\TextA{Supp $B_V=\Par{v_1,\dots,v_n},A=\Mt{T,B_V}.$ Show the following are equiv\hspace{1pt}$:$}
%	\PrePa\TextA{$A$ is low-trig. \:{\tgnr\large(b)} Each $Tv_k\in\Span{v_k,\dots,v_n}.$ \:{\tgnr\large(c)} Each $\Span{v_k,\dots,v_n}$ invard $T.$}
%}By def, (a) and (b) are equiv, and (c) $\Rightarrow$ (b). Now supp (b) holds. For any $k\in\;\!\!\Bra{1,\dots,n}.$\parSol{}
%$Tv_k\in\Span{v_k,\dots,v_n},\,Tv_{k+1}\in\Span{v_{k+1},\dots,v_n},\cdots,\,Tv_{n}\in\Span{v_n}.$ Thus (c) holds.\PfEnd
%\SepLine

\Anchor{5BIIT1}\ProblemN[]{\BulletPointX II.\TipsN{1}}{
	Supp $B_V=\Par{v_1,\dots,v_n},B_{V\apostrophe}=\Par{\varphi_1,\dots,\varphi_n},T\in\Lm{V},A=\Mt{T,B_V}.$\TextA{}
	(a) $A$ up-trig $\Longleftrightarrow T=\sum_{k=1}^n\sum_{j=1}^kA_{j,k}E_{k,j}\Longleftrightarrow T\apostrophe=\sum_{k=1}^n\sum_{j=1}^kA^t_{k,j}\reflectbox{\textit{E}}{_{j,k}}\Longleftrightarrow A^t$ low-trig.\TextA{}
	(b) $A$ low-trig $\Longleftrightarrow T=\sum_{k=1}^n\sum_{j=1}^kA_{k,j}E_{j,k}\Longleftrightarrow T\apostrophe=\sum_{k=1}^n\sum_{j=1}^kA^t_{j,k}\reflectbox{\textit{E}}{_{k,j}}\Longleftrightarrow A^t$ up-trig.\TextA{\vspace{-3pt}}
}\SepLine

\Anchor{5BIIT2}\ProblemN{\BulletPointX II.\TipsN{2}}{
	\TextA{Supp $\Par{\alpha_1,\dots,\alpha_n},\Par{\beta_1,\dots,\beta_n}$ are bses of $V,$ with each $\alpha_k=\beta_{n-k+1}.$}
	\TextA{Prove $\Mt{T,\alpha\rightarrow\alpha}$ up-trig $\Longleftrightarrow\Mt{T,\beta\rightarrow\beta}$ low-trig.}
}For each $k\in\;\!\!\Bra{1,\dots,n},\:T\beta_{n-k+1}=T\alpha_k\in\Span{\alpha_1,\dots,\alpha_k}=\Span{\beta_n,\dots,\beta_{n-k+1}}.$\PfEnd\vspace{2pt}
\Anchor{5BII4e11}\Anchor{5BII4e14}\ACoro (a) Supp $\Fbb=\Cbb.$ Then $\exists\,B_V$ suth $\Mt{T,B_V}$ low-trig. \;(b) $T$ up-trig $\Longleftrightarrow T\apostrophe$ up-trig.
\SepLine

\Anchor{5BII4e12}\Anchor{5BII4e13}\ProblemN{II.12,13}{
	\TextA{Supp $V$ finide, $T\in\Lm{V}.$ Prove $T\mmid_U,T\XSlash U$ up-trig for some invarsp $U\Longleftrightarrow T$ up-trig.}
}Supp $B_U=\Par{u_1,\dots,u_p},B_{V\XSlash U}=\Par{w_1+U,\dots,w_q+U}$ suth $\Mt{T\mmid_U},\Mt{T\XSlash U}$ up-trig.\parSol{}
Then each $Tu_k\in\Span{u_1,\dots,u_k}$ and each $Tw_j+U\in\Span{w_1+U,\dots,w_j+U}.$\parSol{}
By (3.E.13), $B_V=\Par{u_1,\dots,u_p,w_1,\dots,w_q}.$ Now each $Tw_j\in\Span{u_1,\dots,u_p,w_1,\dots,w_j}.$\PfEnd\vspace{2pt}\parSol{}
\Or By (4E 5.B.25)(b) and [4E 5.44], immed.\quad Convly, by [4E 5.44], immed.\PfEnd
\SepLine
\ChEnd\pagebreak

\ChDecl{Ch5C}{5.C \& [4E] 5.D}{\qquad{\small\textbf{注意\,:}\;这一节的题号主要使用第四版5.D节.}}

\vspace{6pt}

%\Anchor{5C4e15}\ProblemN[]{15}{
%	Supp $\Fbb=\Cbb,$ $V$ is finide, $T\in\Lm{V}$ with the min $p.$ Then using Exe (4.6),\TextA{}
%	$T$ diag $\Longleftrightarrow\nexists\,\Par{z-\lambda}{^2}$ in $p\Longleftrightarrow p,p\apostrophe$ have no common zeros $\Longleftrightarrow\gcd\Par{p,p\apostrophe}=1.$\vspace{-4pt}\TextA{}
%}\SepLine

%\Anchor{5C1}\Anchor{5C4e3}\ProblemN{3}{
%	\TextA{Supp $T\in\Lm{V}$ is diag. Prove $V=\null T\oplus\range T.$}
%}Let $U=E\Par{\lambda_1,T}\oplus\cdots\oplus E\Par{\lambda_m,T},$ where each $\lambda_k\neq 0$ and $B_{E\SmallPar{\lambda_k,T}}=\Par{v_{1,k},\dots,v_{M_k,k}}.$\parSol{}
%By (3.B.12), $\range T=\Bra{Tu:u\in U}=\BigBra{{\sum_{k=1}^m\lambda_k\Par{a_{1,k}v_{1,k}+\dots+a_{M_k,k}v_{M_k,k}}:a_{j,k}\in\Fbb}}=U.$\PfEnd\vspace{2pt}
%\AExa Convly not true. Define the inv $T\in\Lm{\Rbb^2}:\Par{x,y}\mapsto\Par{{-y,x}}.$ No eigvals.
%\SepLine

\Anchor{5CL1}\ProblemN{L1}{
	\TextA{Supp $T\in\Lm{V},\alpha,\beta\in\Fbb$ and $\alpha\neq\beta.$ Prove $\Null\Par{T-\alpha I}\subseteq\Range\Par{T-\beta I}.$}
}$\forall v\in\Null\Par{T-\alpha I},Tv=\alpha v\Rightarrow\Par{T-\beta I}\Sbra{v\Big/\Par{\alpha-\beta}}=v\in\Range\Par{T-\beta I}.$\PfEnd
\SepLine

\Anchor{5C5}\Anchor{5C4e5}\ProblemN{5}{
	\TextA{Supp $\Fbb=\Cbb,$ $V$ is finide, and $T\in\Lm{V}.$}
	\TextA{Supp $V=\Null\Par{T-\lambda I}\oplus\Range\Par{T-\lambda I}$ for all $\lambda\in\Cbb.$ Prove $T$ is diag.}
}(i) $\dim V=1.$ Immed. (ii) $\dim V>1.$ Asum it holds for vecsps of smaller dim.\parSol{}
$\exists$ eigval $\lambda_0\Rightarrow U=\Range\Par{T-\lambda_0I}$ invard $T\Rightarrow U=\Null\Par{T\mmid_U-\lambda I}\oplus\Range\Par{T\mmid_U-\lambda I}.$\parSol{}
While $V=E\Par{\lambda_0,T}\oplus U\Rightarrow\dim U<\dim V.$ By asum, $T\mmid_U$ is diag wrto $B_U$ of eigvecs.\PfEnd\vspace{4pt}\par\quad
\Or Supp $T$ not diag. We show $\exists\,\lambda\in\Cbb,\,\Null\Par{T-\lambda I}\cap\Range\Par{T-\lambda I}\neq\zeroSubs.$\par\quad
Let the min of $T$ be $p\Par{z}=\Par{z-\lambda_1}{^{\alpha_1}}\cdots\Par{z-\lambda_m}{^{\alpha_m}},$ where each $\alpha_k\geqslant 1$ and $\exists\,\alpha_j>1.$\par\quad
Let $q\Par{z}\Par{z-\lambda_j}=p\Par{z}\Rightarrow 0=p\Par{T}=\Par{T-\lambda_jI}q\Par{T}\Rightarrow\rangep q\Par{T}\subseteq\Null\Par{T-\lambda_jI}.$\par\quad
Let $q\Par{z}=\Par{z-\lambda_j}s\Par{z}\Rightarrow\rangep q\Par{T}\subseteq\Range\Par{T-\lambda_jI}.$ \,Note that $q\Par{T}\neq0.$\PfEnd\vspace{6pt}\quad
\Or Let $\lambda_1,\dots,\lambda_m$ be disti eigvals. Now $V=\Null\Par{T-\lambda_kI}\oplus\Range\Par{T-\lambda_kI}$ for each $\lambda_k.$\par\quad
Asum $V=\Sbra{{\bigoplus_{i=1}^j\Null\Par{T-\lambda_iI}}}\oplus\Sbra{{\bigcap_{i=1}^j\Range\Par{T-\lambda_iI}}}$ for $j\in\;\!\!\Bra{1,\dots,m-1}.$\par\quad
Becs by (L1), $\bigcap_{i=1}^j\Range\Par{T-\lambda_iI}\supseteq\Null\Par{T-\lambda_{j+1}I},$ and by \Sbra{1.C \TIPSN{2}},\par\quad
$\bigcap_{i=1}^j\Range\Par{T-\lambda_iI}=\Null\Par{T-\lambda_{j+1}I}\oplus\Sbra{{\bigcap_{i=1}^j\Range\Par{T-\lambda_jI}\cap\Range\Par{T-\lambda_{j+1}I}}}.$\par\quad
By induc, $V=\Sbra{\Null\Par{T-\lambda_1I}\oplus\cdots\oplus\Null\Par{T-\lambda_mI}}\oplus\Sbra{\Range\Par{T-\lambda_1I}\cap\cdots\cap\Range\Par{T-\lambda_mI}}.$\par\quad
Asum $U=\bigcap_{k=1}^m\Range\Par{T-\lambda_kI}\neq\zeroSubs.$ Becs $U$ invard $T.$ Thus $\exists\,\mu=\lambda_j$ eigval of $T\mmid_U.$ Ctradic.\PfEnd
\SepLine

%\Anchor{5C7}\Anchor{5C4e2}\ProblemN{2}{
%	\TextA{Supp $T\in\Lm{V},A=\Mt{T,B_V}$ is diag. Prove $A$ has $\dim E\Par{\lambda,T}$ $\lambda$'s on diag.}
%}Given eigvecs $B_V=\Par{v_1,\dots,v_n}.$ Becs $T$ diag. Each $Tv_k=\lambda_jv_k.$ Forming $B_{E\SmallPar{\lambda_j,T}}.$ Immed.\PfEnd%%\vspace{2pt}\parSol{}
%%\Or Let $\lambda_1,\dots,\lambda_m$ be the disti non0 eigvals. Supp $d_0$ eigvecs of $B_V$ corres eigval $0\Rightarrow\dim E\Par{0,T}=d_0.$
%\SepLine

\Anchor{5C4e13}\ProblemN{13}{
	\TextA{Supp $A,B\in\FbbP{n,n}$ and $A$ is diag with {\tgsc dist} ents on diag. Prove $AB=BA\Longleftrightarrow B$ is diag.}
}\NOTICE that for any diag $C,$ each $C_{j,k}=0$ for $j\neq k.$\parSol{}
Becs (I) $A_{j,j}B_{j,k}=A_{j,1}B_{1,k}+\dots+\Sbra{A_{j,j}B_{j,k}}+\dots+A_{j,n}B_{n,k}=\Par{AB}{_{j,k}}.$\parSol{}
And (II) $B_{j,k}A_{k,k}=B_{j,1}A_{1,k}+\dots+\Sbra{B_{j,k}A_{k,k}}+\dots+B_{j,n}A_{n,k}=\Par{BA}{_{j,k}}.$\parSol{}
Supp $B$ diag. If $j=k,$ then $\Par{BA}{_{j,k}}=\Par{AB}{_{j,k}},$ othws true as well.\parSol{}
Supp $AB=BA\Rightarrow A_{j,j}B_{j,k}=A_{k,k}B_{j,k}.$ Asum $B_{j,k}\neq0$ with $j\neq k.$ Then $A_{j,j}=A_{k,k},$ ctradic.\PfEnd
\SepLine

\Anchor{5C4e14}\ProblemN{14}{
	\TextA{Supp $\Fbb=\Cbb,$ $k\in\Nbp,$ and $T\in\Lm{V}$ is inv. Prove $T^k$ diag $\Rightarrow T$ diag.}
}Let the min of $T^k$ be $p\Par{z}=\Par{z-\lambda_1}\cdots\Par{z-\lambda_m}\Rightarrow$ each $\lambda_k$ non0 and disti.\parSol{}
Becs any non0 $\lambda\in\Cbb$ has $k$ disti $k^{\text{th}}$ roots. Let $\Bra{\mu_{1,j},\dots,\mu_{k,j}}$ be the roots of $z^k=\lambda_j.$\parSol{}
For $x,y\in\;\!\!\Bra{1,\dots,n},\:x\neq y\Longleftrightarrow \mu_{p,x}^k=\lambda_x\neq\lambda_y=\mu_{q,y}^k$ for each $p,q\in\;\!\!\Bra{1,\dots,k}\Rightarrow\mu_{p,x}\neq\mu_{q,y}.$\parSol{}
Thus all $\mu$'s are dist. Let $s\Par{z}=\Par{z^k-\lambda_1}\cdots\Par{z^k-\lambda_m}=\prod_{j=1}^m\prod_{i=1}^k\Par{z-\mu_{i,j}}\Rightarrow s\Par{T}=0.$\PfEnd\vspace{3pt}
\AExa Not true if $\Fbb=\Rbb.$ Define $T\in\Lm{\Rbb^2}:\Par{x,y}\mapsto\Par{{-y,x}}.$ No eigvals.
\SepLine

\Anchor{5C'1}\ProblemB{
	\TextB{Supp $\Fbb=\Cbb,n\in\Nbb,n\geqslant2.$ Prove $T$ is diag $\Longleftrightarrow\forall p\in\PoFi,\nullp p\Par{T}=\Null\Sbra{p\Par{T}}{^n}.$}
}(a) Supp $T$ diag. Let $p\Par{z}=\Par{z-\alpha_1}\cdots\Par{z-\alpha_m}.$ We show each $\Null\Par{T-\alpha_kI}{^n}=\Null\Par{T-\alpha_kI}.$\parSol{\Ha}
Done if $T-\alpha_kI=S$ inje. Supp $S$ not inje. \NOTICE that $\null S\mmid_{\range S}=\null S\cap\range S=\zeroSubs.$\parSol{\Ha}
By (3.B.22), $\dim\null S^2=\dim\null S\Rightarrow\null S^2=\null S.$ Asum $\null S^j=\null S$ for $j\geqslant 2.$\parSol{\Ha}
Becs $\dim\Null\Par{S^jS}=\Dim\Par{\null S^j\cap\range S}+\dim\null S.$ By induc.\vspace{3pt}\parSol{}
(b) Supp $\Null\Par{T-\lambda I}=\Null\Par{T-\lambda I}{^n}$ for all $\lambda\in\Cbb.$ Let $\lambda_1,\dots,\lambda_m$ be disti eigvals of $T.$\parSol{\Hb}
Define $p\Par{z}=\Par{z-\lambda_1}\cdots\Par{z-\lambda_m}.$ Then $\Sbra{p\Par{T}}{^{\dim V}}=0\Rightarrow p\Par{T}=0\Rightarrow p$ is the min.\PfEnd\vspace{3pt}\parSol{}
\Or By (4E 8.A.3) and Exe (5), $T$ diag $\Longleftrightarrow\forall\lambda\in\Fbb,\Null\Par{T-\lambda I}=\Null\Par{T-\lambda I}{^2}.$\PfEnd
\SepLine

%\Anchor{5C4e17}\ProblemN{17}{
%	\TextA{Supp $V$ is finide. Prove $\Lm{V}$ has a bss consisting of diag optors.}
%}Let $B_V=\Par{v_1,\dots,v_n}.$ Define each $E_{j,k}\in\Lm{V}:v_x\mapsto\delta_{j,x}v_k\Rightarrow$ [5.41](c) true.\PfEnd
%\SepLine

\Anchor{5C4e18}\ProblemN{18}{
	\TextA{Supp $T\in\Lm{V}$ is diag. Prove $T\XSlash U\in\Lm{V\XSlash U}$ is diag for any $U$ invarspd $T.$}
}By \Sbra{5.A \TIPSN{2}}, $\exists\,B_U=\Par{v_1,\dots,v_m}$ consists of eigvecs of $T.$\parSol{}
Extend to eigvecs $B_V=\Par{v_1,\dots,v_m,w_1,\dots,w_p}\Rightarrow B_{V\XSlash U}=\Par{w_1+U,\dots,w_p+U}.$\parSol{}
Becs for each $w_k,\:\exists$ eigval $\lambda$ of $T,\:Tw_k=\lambda w_k\Rightarrow\BigPar{T\XSlash U}\Par{w_k+U}=\lambda w_k+U.$\PfEnd\vspace{2pt}\parSol{}
\Or Becs the min of $T$ is multi of that of $T\XSlash U.$ By [4E 5.62].\PfEnd\vspace{2pt}
\Anchor{5C4e19}\AComm In Exa [5.15]: $T\in\Lm{V}$ not diag while $T\mmid_U,T\XSlash U$ diag.\PfEnd
\SepLine

%\Anchor{5C4e20}\ProblemN{20}{
%	\TextA{Supp $V$ is finide, $T\in\Lm{V}.$ Prove $T$ diag $\Longleftrightarrow T\apostrophe$ diag.\hfill\tgnr\FontNorm By (4E 5.B.28), immed.}
%}By \Sbra{5.B(II) \TIPSN{1}}. Note that $S$ low-trig and up-trig $\Longleftrightarrow S$ diag.\PfEnd
%\SepLine

%\Anchor{5C4e22}\ProblemN{22}{
%	\TextA{Supp $V$ finide, $T\in\Lm{V},$ $A=\Mt{T,B_V}\in\FbbP{n,n}.$}
%	\TextA{Prove if each $\aXMid{A_{j,j}}>\sum_{k=1}^n\aXMid{A_{j,k}}-A_{j,j},$ then $T$ is inv.\vspace{1pt}}
%}If $0$ is eigval, then $0$ is in G disk for some $j$, now $\aXMid{0-A_{j,j}}\leqslant\sum_{k=1}^n\aXMid{A_{j,k}}-A_{j,j},$ ctradic.\PfEnd\vspace{2pt}
%\AComm If each $\aXMid{A_{k,k}}>\sum_{j=1}^n\aXMid{A_{j,k}}-A_{k,k},$ then becs [5.67] still holds by Exe (4E 23), $T$ is inv.
%\SepLine

%\Anchor{5C4e23}\ProblemN{23}{
%	\TextA{Redefine G disks suth the radius of the $k^\text{th}$ disk is the sum of the absolute vals}
%	\TextA{of the ents in {\tgsc col} $k$, excluding the diag ent. Show [4E 5.67] still holds.}
%}Simlr in [5.67] \Par{without using the result} but to $T\apostrophe.$ \Or Becs $\lambda$ is eigval of $T\Longleftrightarrow$ of $T\apostrophe.$\parSol{}
%%Supp $T\apostrophe\Par{\psi}=\lambda\psi$ with $\psi=c_1\varphi_1+\dots+c_n\varphi_n\neq0\Rightarrow\lambda\psi=\sum_{j=1}^n\BigBigPar{{\sum_{k=1}^nA^t_{j,k}c_kv_j}}=\sum_{j=1}^nc_j\lambda v_j.$\vspace{3pt}\parSol{}
%%Let $\aMid{c_j}=\max\!\Bra{\aMid{c_1},\dots,\aMid{c_n}}.$ Now $\lambda c_j=\sum_{k=1}^nA^t_{j,k}c_k\Rightarrow\aXMid{\lambda-A^t_{j,j}}\leqslant\sum_{j\neq k=1}^n\aXMid{A_{k,j}}.$\PfEnd\vspace{6pt}\parSol{}
%$\lambda\in\BigBra{z\in\Fbb:\aXMid{z-A_{j,j}}\leqslant\sum_{j\neq k=1}^n\aXMid{A^t_{j,k}}=\sum_{j\neq k=1}^n\aXMid{A_{k,j}}}$ for some $j\in\;\!\!\Bra{1,\dots,n}.$\PfEnd
%\SepLine
\ChEnd

\ChDecl{Ch5E}{5.E [4E]}{}

\vspace{4pt}

%\Anchor{5E8}\ProblemN{8}{
%	\TextA{Find a bss of $\PoFx{m}{\Rbb^2}$ suth $D_x,D_y$ up-trig in [5.72].}
%}Let $B=\Par{1,x,y,x^2,xy,y^2,\cdots,\cdots,x^m,x^{m-1}y,\cdots,xy^{m-1},y^m}$ in $\PoFx{m}{\Rbb^2}.$\parSol{}
%Supp a liney combina of $B$ is $0;$ \;$\sum_{j=0}^m\sum_{k=0}^{m-j}a_{j,k}x^jy^k=0.$\parSol{}
%Let $x=0\Rightarrow$ each $a_{0,k}=0,$ and $y=0\Rightarrow$ each $a_{k,0}=0.$ Now $\sum_{j=1}^{m-1}\sum_{k=1}^{m-1-j}a_{j,k}x^jy^k=0.$\parSol{}
%Take $\BigPar{\Par{x_1,y_1},\cdots,\Par{x_q,y_q}}$ \Sbra{where $q=1+\dots+m$} suth all $\sum_{j=1}^{m-1}\sum_{k=1}^{m-1-j}x_s^jy_t^k\,a_{j,k}=0$\parSol{}
%form a system of $q$ equations having uniq solus $\Par{0,\dots,0}.$ Thus $B$ is liney indep.\parSol{}
%Apply $D_x$ to each vec in $B\Rightarrow B_x=\BigPar{0,1,0,2x,y,0,\cdots,\cdots,mx^{m-1},\Par{m-1}x^{m-2}y,\cdots,y^{m-1},0}.$\parSol{}
%Apply $D_y$ to each vec in $B\Rightarrow B_y=\BigPar{0,0,1,0,x,2y,\cdots,\cdots,0,x^{m-1},\cdots,\Par{m-1}xy^{m-2},my^{m-1}}.$\PfEnd
%\SepLine

\Anchor{5E6}\ProblemN{6}{
	\TextA{Supp $\Fbb=\Cbb,$ $V$ is finide, and $S, T\in\Lm{V}$ commu.}
	\TextA{Prove $\exists\,\alpha,\lambda\in\Cbb$ suth $\Range\Par{S -\alpha I} + \Range\Par{T -\lambda I}\neq V$.}
}Supp $A,C\in\FbbP{n,n}$ are up-trig matrices of $S,T$ wrto a $B_V=\Par{v_1,\dots,v_n}$ suth $A,C$ commu.\parSol{}
Let $\alpha=A_{n,n},\,\lambda=C_{n,n}.$ Then $\Range\Par{S-\alpha I},\Range\Par{T-\lambda I}\subseteq\Span{v_1,\dots,v_{n-1}}.$\PfEnd
\SepLine

\Anchor{5E7}\ProblemN{7}{
	\TextA{Supp $\Fbb=\Cbb,$ and $S,T\in\Lm{V}$ commu, $S$ diag. Prove $\exists\,B_V$ suth $S$ diag and $T$ up-trig.}
}Let $\lambda_1,\dots,\lambda_m$ be disti eigvals of $S\Rightarrow V=E\Par{\lambda_1,S}\oplus\cdots\oplus E\Par{\lambda_m,S}.$\parSol{}
Becs each $E_k=E\Par{\lambda_k,S}$ invard $T.$ Let each $T\mmid_{E_k}$ be up-trig with $B_{E_k}=\Par{v_{1,k},\dots,v_{M_k,k}}.$\parSol{}
Then $S$ diag while $T$ up-trig with the same $B_V=\Par{v_{1,1},\dots,v_{M_n,n}}.$\PfEnd\vspace{3pt}\parSol{}
\Or Using induc on $n=\dim V.$ (i) $n=1.$ Immed. \:(ii) $n>1.$ Asum it holds for smaller $V.$\parSol{}
$\exists$ eigval $\lambda$ of $S,\;U=\Null\Par{S-\lambda I},W=\Range\Par{S-\lambda I}\Rightarrow V=\Null\Par{S-\lambda I}\oplus\Range\Par{S-\lambda I}.$\parSol{}
Apply the asum to $T\mmid_U,S\mmid_U$ and $T\mmid_W,S\mmid_W,$ then put $B_U,B_W$ together.\PfEnd
\SepLine

\Anchor{5E2}\ProblemN{2}{
	\TextA{Supp $\mE\subseteq\Lm{V}$ and every elem of $\mE$ diag.}
	\TextA{Prove each pair of elems of $\mE$ commu $\Rightarrow\exists\,B_V$ suth all elem of $\mE$ diag.}
}Let $\dim V=n\Rightarrow\dim\Lm{V}=n^2.$ \;Write $V=\bigoplus_{\lambda_k\,\in\,\Fbb}E\Par{\lambda_k,T}$ for each $T\in\mE.$\vspace{2pt}\par\quad
$\exists\,\Bra{T_{\!1},\dots,T_{\!m}}\subseteq\mE$ with each elem of $\mE$ in $\Span{T_{\!1},\dots,T_{\!m}}$ and $m\leqslant n^2.$\vspace{2pt}\par\quad
\NOTICE that $U_{k}=E\Par{\lambda_1,T_{\!1}}\cap\cdots\cap E\Par{\lambda_{k},T_{\!k}}=E\Par{\lambda_k,T_{\!k}\mmid_{U_{k-1}}}=\bigoplus_{\lambda_{k+1}\!\!}E\Par{\lambda_{k+1},T_{\!k+1}\mmid_{U_{k}}}.$\vspace{2pt}\par\quad
Hence $V=\bigoplus_{\lambda_1\!\!}E\Par{\lambda_1,T_{\!1}}=\bigoplus_{\lambda_1,\dots,\lambda_m\!\!}\Sbra{E\Par{\lambda_1,T_{\!1}}\cap\cdots\cap E\Par{\lambda_m,T_{\!m}}}.$ Take bss of each summand.\vspace{2pt}\par\quad
Then we form $B_V.$ For any $T\in\mE,$ $\Mt{T,B_V}=c_1\Mt{T_{\!1},B_V}+\dots+c_m\Mt{T_{\!m},B_V}.$\PfEnd
\SepLine

\Anchor{5E9}\ProblemN{9}{
	\TextA{Supp $\Fbb=\Cbb,$ $V$ finide and non0. Supp $\mE\subseteq\Lm{V}$ is suth all $S,T\in\mE$ commu.}
	\PrePa\TextA{Prove $\exists$ eigvec $v\in V$ of all elem of $\mE.$ \,{\tgnr\large(b)} $\exists\,B_V$ suth all elem of $\mE$ has up-trig matrix.}
}Simlr to Exe (2). $\exists\,\Bra{T_{\!1},\dots,T_{\!m}}\subseteq\mE.$ Let $U_0=V,U_k=E\Par{\lambda_1,T_{\!1}}\cap\cdots\cap E\Par{\lambda_k,T_{\!k}}.$\parSol{}
(a) Let $\lambda_1,\dots,\lambda_m$ be eigvals of $T_{\!1},\dots,T_{\!m}$ respectly with each $\lambda_k$ eigval of $T_{\!k}\mmid_{U_k}\Rightarrow U_k\neq 0$\parSol{\Ha}
Now for non0 $v\in U_m,\:\forall T=c_1T_{\!1}+\dots+c_mT_{\!m}\in\mE,Tv=\Par{c_1\lambda_1+\dots+c_m\lambda_m}v.$\vspace{2pt}\parSol{}
(b) Using induc on $\dim V.$ (i) Immed. \,(ii) $\dim V>1.$ Asum it holds for smaller $V.$\parSol{\Hb}
Let $v_1$ be a common eigvec of all $T_{\!k}.$ Let $W\oplus\Span{v_1}=V,P:av_1+w\mapsto w.$\parSol{\Hb}
Simlr in [4E 5.80], each pair of $\Bra{\hat{T}_{1},\dots,\hat{T}_m}$ commu. By asum, $\exists\,B_W\Rightarrow\exists\,B_V.$\parSol{\Hb}
Now each $\Mt{T_{\!k},B_V}$ up-trig $\Rightarrow\forall T\in\mE,\Mt{T}=c_1\Mt{T_{\!1}}+\dots+c_m\Mt{T_{\!m}},$ wrto $B_V.$\PfEnd
\SepLine\ChEnd
\pagebreak

\ChDecl{Ch8}{8}{\quad{\ANote {\FontSmall Supp $V$ is a non0 finide vecsp over $\Fbb.$ \;Supp $T\in\Lm{V}.$ Let $m_T$ be the min of $T.$}}}\par\vspace{-6pt}
\Blind{\ChDecl{Ch8}{8}{\quad\ANote}}{\FontSmall An Exe marked by $\blacksquare$ is still true if infinide or partially finide.}\par

\vspace{4pt}

\Anchor{Ch8A}

\Anchor{8A'1}\ProblemB{
	\TextA{Supp $T$ nilp, $U$ non0 and $U\oplus\null T=\null T^2.$ Prove $U$ is not invard $T.$}
}Let $u\in U$ and $T^2u=0\neq Tu\in\null T.$ \;If $U$ invar, then $Tu\in U\cap\null T=\zeroSubs,$ ctradic.\PfEnd
\SepLine

\Anchor{8A3}\ProblemN{A.3}{
	\TextA{Supp $T$ inv. Prove $G\Par{\lambda,T}=G\Par{\lambda^{-1},T^{-1}}$ for any non0 $\lambda\in\Fbb.$\PfEndB}
}$\Par{T-\lambda I}{^j}v=0=\sum_{i=0}^j\mathC_j^i\Par{{-\lambda}}{^{j-i}}T^iv.$ Apply $\Par{{-\lambda}}{^{-j}}T^{-j}$ to both sides. $\Par{T^{-1}-\lambda^{-1}I}{^j}v=0.$\PfEnd\vspace{2pt}\parSol{}
\Or We use induc on $j$ to show each $\Null\Par{T-\lambda I}{^j}=\Null\Par{T^{-1}-\lambda^{-1}I}{^j}.$ \,(i) Immed. (ii) $j>1.$\parSol{}
Asum true for $\Par{j-1}\Rightarrow\forall v\in\Null\Par{T-\lambda I}{^j},\Par{T-\lambda I}v\in\Null\Par{T-\lambda I}{^{j-1}}=\Null\Par{T^{-1}-\lambda^{-1}I}{^{j-1}}.$\parSol{}
Which equiv \,$\Null\Par{T^{-1}-\lambda^{-1}I}{^{j-1}}v\in\Null\Par{T-\lambda I}=\Null\Par{T^{-1}-\lambda^{-1}I},$ by (i).\PfEnd
\SepLine

\Anchor{8A5}\ProblemN{A.5}{
	\TextA{Supp $T^{n-1}v\neq 0,\;T^nv=0.$ \,Prove $\BigPar{v,Tv,\dots,T^{n-1}v}$ is liney indep.}
}$a_0v+a_1Tv+\dots+a_{n-1}T^{n-1}v=0\Rightarrow a_0T^{n-1}v=0\Rightarrow a_0=0.$ \,Simlr for $a_1,\dots,a_{n-1}.$\PfEndB
\SepLine

\Anchor{8A4e24}\Anchor{8N8.19}\ProblemBX[]{\NoteForSmall{[8.19] \OR [4E 8.18]}}{
	If $m_T\Par{z}=z^m.$ Then $\exists\,v$ suth $T^{m-1}v\neq0.$\TextB{}
	\uline{If $m=\dim V.$} Then $B_V=\Par{T^{m-1}v,\dots,Tv,v}.$ Let each $w_k=T^{m-k}v.$ Then $Tw_1=0,T\Par{w_k}=w_{k-1}.$\TextB{\vspace{-2pt}}
}\SepLine

\Anchor{8A6}\ProblemN{A.6}{
	\TextA{Supp $T$ nilp, $n=\dim V,T^{n-1}\neq0.$ Prove $\nexists\,S\in\Lm{V},\,S^k=T$ for all $k>1.$}
}Asum $\exists$ suth $S\Rightarrow S$ is nilp. Then $\null S^n=\dots=\null S^{kn}=\null T^n=V.$\parSol{}
Now $\exists\,t\in\Nbb$ with $\Par{n-t}\:\!k\in\;\!\!\Bra{n,\dots,kn}\Rightarrow\null T^{n-t}=\null S^{nk-tk}=V.$\PfEnd
\SepLine

%\Anchor{8A9}\ProblemN{9}{
%	\TextA{Supp $S,T\in\Lm{V}$ and $ST$ nilp. Prove $TS$ nilp.}
%}Supp $\Par{ST}{^k}=0.$ \NOTICE that $\Par{TS}{^{k+1}}=T\Par{ST}{^k}S=0.$\PfEnd
%\SepLine

\Anchor{8A4e4}\ProblemBnoor{4E A.4}{
	\TextA{Supp $m_T$ is a multi of $\Par{z-\lambda}{^m}$ with $m\in\Nbp.$ Prove $\dim\Null\Par{T-\lambda I}{^m}\geqslant m.$}
}Becs $\lambda$ is eigval of $T.$ We show $z^m$ is the min of $N=\Par{T-\lambda I}\Big|{_{\Null\SmallPar{T\,-\,\lambda I}{^m}}}\Rightarrow N^m=0\neq N^{m-1}.$\parSol{}
Let each $U_k\oplus\null N^{k-1}=\null N^k$ for $k\in\;\!\!\Bra{2,\dots,m}\Rightarrow U_k$ not invard $N\Rightarrow U_k$ non0.\parSol{}
Thus $\null N^0\subsetneq\null N\subsetneq\cdots\subsetneq\null N^m\Rightarrow\dim\Null\Par{T-\lambda I}{^m}=\dim\null N^m\geqslant m.$\PfEnd\vspace{3pt}\parSol{}
\Or Let $m_T\Par{z}=\Par{z-\lambda}{^m}q\Par{z}.$ We show $\zeroSubs\subsetneq\Null\Par{T-\lambda I}\subsetneq\cdots\subsetneq\Null\Par{T-\lambda I}{^m}$ by ctradic.\parSol{}
Asum $\Null\Par{T-\lambda I}{^k}=\Null\Par{T-\lambda I}{^{k+1}}$ for $k\in\;\!\!\Bra{1,\dots,m-1}.$\parSol{}
Then $\Null\Par{T-\lambda I}{^k}=\Null\Par{T-\lambda I}{^m}\Rightarrow\Par{T-\lambda I}{^m}q\Par{T}v=0=\Par{T-\lambda I}{^k}q\Par{T}v.$\PfEnd
\SepLine

\Anchor{8A4e3}\ProblemBnoor{4E A.3}{
	\TextA{Prove $V=\null T\oplus\range T\Longleftrightarrow\null T^2=\null T.$}
}(a) $\null T^2=\null T=\null T^{\dim V}\Rightarrow\dim\range T^{\dim V}=\dim\range T.$\parSol{\vspace{2pt}}
(b) $V=\null T\oplus U,U=\range T,$ 又 $\dim\null T^2=\dim\null T+\dim\null T\mmid_{\range T}.$\PfEnd\vspace{4pt}\parSol{}
\Or (a) Supp $\null T^2=\null T.$ Then $Tu\in\null T\cap\range T\Longleftrightarrow T^2u=0\Longleftrightarrow Tu=0.$\parSol{}
\Blind{\Or }(b) Supp $\null T\cap\range T=\zeroSubs.$ Then $T^2u=0\Longleftrightarrow Tu\in\null T\Longleftrightarrow Tu=0.$\PfEndB
\SepLine

\Anchor{8A17}\ProblemN{A.17}{
	\TextA{Supp $\range T^m=\range T^{m+1}.$ Show $\range T^m=\range T^{m+1}=\cdots.$}
}By Exe (A.19), $\null T^m=\null T^{m+1}=\cdots\Rightarrow\dim\range T^m=\dim\range T^{m+1}=\cdots.$\PfEnd\parSol{}
\Or Supp $w=T^{m+k}v.$ Then becs $T^mv\in\range T^{m+1},\exists\,T^{m+1}u=T^mv.$ Thus $w=T^{m+k+1}u.$\PfEndB
\SepLine

\Anchor{8A18}\ProblemN{A.18}{
	\TextA{Supp $\dim V=n.$ Show $\range T^{n}=\range T^{n+1}=\cdots.$\hfill\FontNorm\tgnr By Exe (A.19), simlr.\Blind{\quad}\PfEnd}
}Asum $\range T^{n}\supsetneq\range T^{n+1}.$ By Exe (A.17), $V=\range T^0\supsetneq\range T\supsetneq\cdots\supsetneq\range T^{n+1}.$\parSol{}
Now each $\dim\range T^{k+1}\leqslant\dim\range T^k-1\Rightarrow\dim\range T^{n+1}\leqslant\dim\range T^0-\Par{n+1}.$\PfEnd
\SepLine\pagebreak

\Anchor{8A10}\ProblemN{A.10}{
	\TextA{Supp $T$ not nilp, $n=\dim V.$ Show $V=\null T^{n-1}\oplus\range T^{n-1}.$}
}\NOTICE that $\null T^{n-1}\neq\null T^n\Rightarrow\dim\null T^n=\dim V.$ Thus $\null T^{n-1}=\null T^n.$\parSol{}
又 $V=\null T^n\oplus\range T^n,\range T^n\subseteq\range T^{n-1}\Rightarrow V=\null T^{n-1}+\range T^{n-1}.$\PfEnd\parSol{}
\Or Then $\dim\range T^{n-1}=\dim\range T^n\Rightarrow\range T^{n-1}=\range T^n.$\PfEnd\vspace{2pt}\parSol{}
\Or By Exe (4E A.3), $\null T^{2\SmallPar{n-1}}=\null T^{n-1}\Longleftrightarrow V=\null T^{n-1}\oplus\range T^{n-1}.$\PfEnd
\SepLine

\Anchor{8A4e18}\ProblemBnoor{4E A.18}{
	\TextA{Supp $T$ nilp. Prove $T^{1\,+\,\dim\range T}=0.$}
}Let $\dim V=n.$ Then $\dim\null T^{n-1}\mmid_{\range T}+\dim\null T=\dim V.$\parSol{}
Now $\null T^{n-1}\mmid_{\range T}=\range T\Rightarrow T\mmid_{\range T}\in\Lm{\range T}$ is nilp.\PfEnd\vspace{4pt}\parSol{}
\Or Let $\dim\range T=k.$ Asum $T^{k+1}\neq 0.$ Let $m$ be suth $T^m=0\neq T^{m-1}.$ Then $k+2\leqslant m.$\parSol{}
Let $v$ be suth $T^{m-1}v\neq 0=T^mv\Rightarrow\Par{v,{}$\uline{$Tv,\dots,T^{m-1}v$}$}$ liney indep $\Rightarrow k\geqslant m-1\geqslant k+1.$\PfEndB
\SepLine

\Anchor{8A4e12}\ProblemBnoor{4E A.12}{
	\TextA{Supp every $v\in V$ is a g-eigvec of $T.$ Prove $V=G\Par{\lambda,T}.$}
}Becs for any liney indep $\Par{v,w},$ $\Par{v,w,v+w}$ of g-eigvecs is liney dep; say corres $\alpha,\beta,\gamma$ repectly.\parSol{}
If $\alpha=\beta$ then done. If $\alpha=\gamma,$ then $v,v+w\in G\Par{\alpha,T}\Rightarrow w\in G\Par{\alpha,T}.$ If $\beta=\gamma,$ then simlr.\parSol{}
Thus $\alpha=\beta=\gamma.$ Any two liney indep $v,w$ corres one eigval.\PfEndB
\SepLine

\Anchor{8A4e15}\Anchor{8B5}\ProblemNnoor{B.5}{4E A.15}{
	\TextA{Prove non0 $T$ diag $\Rightarrow$ each $G\Par{\lambda,T}=E\Par{\lambda,T}.$\hfill\tgnr\FontNorm Convly true if req $\Fbb=\Cbb.$}
}$\forall w\in G\Par{\lambda_j,T},\exists\,!\,v_i\in E\Par{\lambda_i,T},w=v_1+\dots+v_m,.$\parSol{}
Note that $\Par{T-\lambda_jI}{^k}w=0=\sum_{i=1}^m\Par{\lambda_i-\lambda_j}{^k}v_i\Rightarrow w=v_j\in E\Par{\lambda_j,T}.$\PfEnd\vspace{3pt}\parSol{}
\Or By (4E B.6), immed. \;\Or Supp $G\Par{\lambda_j,T}\supsetneq E\Par{\lambda_j,T}.$ Let $w\in G\Par{\lambda_j,T}\Backslash E\Par{\lambda_j,T}$\parSol{}
Let $\Par{T-\lambda_jI}{^k}w=0\neq\Par{T-\lambda_jI}{^{k-1}}w.$ By \Sbra{5.B(I) \TIPSN{1}}, $m_T$ is a multi of $\Par{z-\lambda_j}{^k}.$ 又 $k\geqslant 2.$\PfEnd
\SepLine

\Anchor{8A4e16}\ProblemBnoor{4E A.16}{
	\TextA{Supp $S,T\in\Lm{V}$ nilp and commu. Prove $S+T,ST$ are nilp}
}By [4E 5.80], $\exists\,B_V$ suth $S,T$ up-trig \Par{with only $0$'s on diags}. By (4E 5.C.2).\PfEnd\vspace{2pt}\parSol{}
\Or Let $S^p=T^q=0.$ Becs $S,T$ commu, $\Par{ST}{^{\max\!\Bra[\scriptsize]{p,\,q}}}=0=\Par{S+T}{^{p+q}}=\sum_{i=0}^{p+q}\mathC_{p+q}^iS^iT^{p+q-i}.$\PfEndB
\SepLine

%\Anchor{8A19}\ProblemN{19}{
%	\TextA{Supp $T\in\Lm{V}.$ Prove $\null T^m=\null T^{m+1}\Longleftrightarrow\range T^m=\range T^{m+1}.$}
%}If $m<0.$ Then by (4E 5.A.33), $T^{m}$ inv $\Rightarrow T$ inv. Immed. Supp $m\in\Nbp.$\parSol{}
%{\tgsl This Exe is not true if infinide. Forwd and backwd shift optors on $\FbbP{\infty}$ will serve as countexas.}\parSol{}
%Now becs $V$ finide, $\dim\null T_{\!1}=\dim\null T_{\!2}\Longleftrightarrow\dim\range T_{\!1}=\dim\range T_{\!2}.$\PfEnd
%\SepLine

\Anchor{Ch8B}

\Anchor{8B10}\ProblemN{B.10}{
	\TextA{Supp $\Fbb=\Cbb.$ Prove $\exists$ commu $D,N\in\Lm{V},T=D+N,\,D$ diag, $N$ nilp.\vspace{2pt}}
}\ANote $D$ diag, $N$ nilp $\notRightarrow D,N$ commu. \AExa $De_1=e_1,De_2=0,Ne_1=0,Ne_2=e_1.$ %$\footnotesize\begin{pmatrix}1&\hspace{-6pt}0\\[-4pt]0&\hspace{-6pt}0\end{pmatrix},\begin{pmatrix}0&\hspace{-6pt}1\\[-4pt]0&\hspace{-6pt}0\end{pmatrix}.$
\vspace{2pt}\parSol{}
We use induc on $\dim V=n.$ (i) Immed. (ii) $n>1.$ Asum it holds for smaller $V.$\parSol{}
Becs $V=G_1\oplus U,$ where $U=G_2\oplus\cdots\oplus G_m,$ and each $G_k=G\Par{\lambda_k,T}.$\parSol{}
$\exists\,B_{G_1}$ suth $T\mmid_{G_1}=\Par{T-\lambda_1I}\Big|{_{G_1}}+\lambda_1I\mmid_{G_1}=N_1+D_1$ up-trig and $N_1,D_1$ commu.\parSol{}
$\exists$ commu $D_2,N_2\in\Lm{U},T\mmid_U=D_2+N_2,\,D_2$ diag, $N_2$ nilp; wrto some $B_U,$ by (4E 5.E.7).\parSol{}
Define $P_{\!\!\!\:1},P_{\!\!\!\:2}\in\Lm{V}$ by $P_{\!\!\!\:1}\Par{v_1+u}=v_1,P_{\!\!\!\:2}\Par{v_1+u}=u.$ Let $D=D_1P_{\!\!\!\:1}+D_2P_{\!\!\!\:2},N=N_1P_{\!\!\!\:1}+N_2P_{\!\!\!\:2}.$\parSol{}
$D+N=\Par{D_1+N_1}P_{\!\!\!\:1}+\Par{D_2+N_2}P_{\!\!\!\:2}=T,$ \, $DN=D_1N_1P_{\!\!\!\:1}+D_2N_2P_{\!\!\!\:2}=NP,$ \, $B_V=B_{G_1}\cup B_U.$\PfEnd\vspace{4pt}\parSol{}
\Or $\forall v\in V,\exists\,!\,v_k\in G_k,v=v_1+\dots+v_m.$ Define $D\in\Lm{V}:v\mapsto\Par{\lambda_1v_1+\dots+\lambda_mv_m}$\parSol{}
Then $D\mmid_{G_k}=\lambda_kI.$ Let $N=T-D\Rightarrow N\mmid_{G_k}=\Par{T-D}\Big|{_{G_k}}=\Par{T-\lambda_kI}\Big|{_{G_k}}$ is nilp $\Rightarrow N$ nilp.\parSol{}
Becs $DN=DT-D^2,ND=TD-D^2,$ 又 each $TDv_k=\lambda_kTv_k=DTv_k\Rightarrow TD=DT.$\PfEnd\vspace{4pt}\parSol{}
\Or Define $P_{\!\!\!\:j}\in\Lm{V}:w_j+u\mapsto w_j,$ where $w_j\in G_j,u\in\bigoplus_{i\neq j}G_i.$\parSol{}
Now $T=T\mmid_{G_1}P_{\!\!\!\:1}+\dots+T\mmid_{G_m}P_{\!\!\!\:m}.$ Let $N_j=T\mmid_{G_j}-\lambda_jI\Rightarrow N_1P_{\!\!\!\:1}+\dots+N_mP_{\!\!\!\:m}:v_j\mapsto N_kv_j.$\parSol{}
Where $B_V=\Par{v_1,\dots,v_n}$ are g-eigvecs and $v_j\in G_k.$ Let $D=\lambda_1P_{\!\!\!\:1}+\dots+\lambda_mP_{\!\!\!\:m}:v_j\mapsto\lambda_kv_j.$\parSol{}
Hence $T=D+N,$ and $DN=ND:v_j\mapsto\lambda_kN_kv_j.$\PfEnd
\SepLine\pagebreak

\Anchor{8B4e7}\ProblemBnoor{4E B.7}{
	\TextA{Supp $\lambda$ is an eigval of $T$ with multy $d.$ Prove $G\Par{\lambda,T}=\Null\Par{T-\lambda I}{^d}.$}
}Let $N=T-\lambda I,$ and $\null N\subsetneq\cdots\subsetneq\null N^m=\null N^{m+1}.$ Choose $B_{\null N}.$\parSol{}
Extend to $B_{\null N^2}\Rightarrow\cdots\Rightarrow B_{\null N^m},$ with each step adding at least one bss vec. Thus $m\leqslant d.$\PfEnd\vspace{3pt}\parSol{}
\Or Let $m_T\Par{z}=\Par{z-\lambda}{^m}q\Par{z}$ with $q\Par{\lambda}\neq0.$\parSol{}
Becs by (4E B.6), $G\Par{\lambda,T}=\Null\Par{T-\lambda I}{^m}.$ \,Now by (4E A.4).\PfEnd\vspace{3pt}\parSol{}
\Or Let the min of $N=\Par{T-\lambda I}\Big|{_{G\SmallPar{\:\!\lambda,\,T\:\!}}}$ be $z^m\Rightarrow$ the min of $N+\lambda I=T\mmid_{G\SmallPar{\:\!\lambda,\,T\:\!}}$ is $s\Par{z}=\Par{z-\lambda}{^m}.$\parSol{}
Becs the char of $T$ \Sbra{See [9.21] for the case $\Fbb=\Rbb$} is a multi of $m_T,$ which is a multi of $s.$\PfEnd
\SepLine

\Anchor{8B4e6}\ProblemBnoor{4E B.6}{
	\TextA{Supp $\lambda$ is an eigval of $T.$ Explain why the expo of $\Par{z-\lambda}$}
	\TextA{in the factoriz of $m_T$ is the smallest $m\in\Nbp$ suth $\Par{T-\lambda I}{^m}\Big|{_{\:\!G\SmallPar{\:\!\lambda,\,T\:\!}}}=0.$\vspace{2pt}}
	%Let $G=G\Par{\lambda,T},N=\Par{T-\lambda I}\Big|{_G},$ and $N^m=0\neq N^{m-1}\Rightarrow$ the min of $T\mmid_G$ is $s\Par{z}=\Par{z-\lambda}{^m}.$\parSol{}
	%Thus $m_T$ is a multi of $s.$ Now we show the expo of $\Par{z-\lambda}$ in $m_T$ is no more than $m.$\parSol{}
	%Let $\lambda_1=\lambda.$ Asum $m_{T_{\!\Cbb}}\Par{z}=\Par{z-\lambda_1}{^{m+k}}\Par{z-\lambda_2}{^{\alpha_2}}\cdots\Par{z-\lambda_n}{^{\alpha_n}},$ where $k\in\Nbp.$\parSol{}
	%Let $r\Par{z}=\Par{z-\lambda}{^m}\Par{z-\lambda_2}{^{\alpha_2}}\cdots\Par{z-\lambda_n}{^{\alpha_n}}\Rightarrow r\Par{T_{\!\Cbb}}=0.$ Ctradic the min of $m_{T_{\!\Cbb}}.$\PfEnd\vspace{4pt}\parSol{}
}Each $\Par{T-\alpha I}{^k}\Big|{_{\:\!G\SmallPar{\:\!\lambda,\,T\:\!}}}$ are inv for $\alpha\neq\lambda.$ \;Becs $m_T\Par{T\mmid_{\:\!G\SmallPar{\:\!\lambda,\,T\:\!}}}=0\Longleftrightarrow\Par{T-\lambda I}{^k}\Big|{_{\:\!G\SmallPar{\:\!\lambda,\,T\:\!}}}=0.$
%\PfEnd\vspace{1pt}\parSol{}
%\Or Each $\Par{T-\alpha I}{^k}\Big|{_{\:\!G\SmallPar{\:\!\lambda,\,T\:\!}}}$ are inje for $\alpha\neq\lambda.$ \;Becs $m_T\Par{T}\Big|{_{\:\!G\SmallPar{\:\!\lambda,\,T\:\!}}}=0\Longleftrightarrow\Par{T-\lambda I}{^k}\Big|{_{\:\!G\SmallPar{\:\!\lambda,\,T\:\!}}}=0.$
\PfEnd\vspace{4pt}\parSol{}
\Or Let $m_T\Par{z}=\Par{z-\lambda}{^m}q\Par{z},$ with $q\Par{\lambda}\neq0.$ We show $\Null\Par{T-\lambda I}{^m}\supseteq\Null\Par{T-\lambda I}{^{m+1}}.$\parSol{}
Supp $v\in\Null\Par{T-\lambda I}{^{m+1}}\Longleftrightarrow \Par{T-\lambda I}{^m}v\in\Null\Par{T-\lambda I}=E\Par{\lambda,T}.$\parSol{}
Then $0=m_T\Par{T}v=q\Par{T}\Sbra{\Par{T-\lambda I}{^m}v}=q\Par{\lambda}\Sbra{\Par{T-\lambda I}{^m}v}\Rightarrow v\in\Null\Par{T-\lambda I}{^m}.$\vspace{2pt}\parSol{}
Let $k$ be suth $\Null\Par{T-\lambda I}{^k}=G\Par{\lambda,T}=\Null\Par{T-\lambda I}{^m}.$ Becs $\Par{T-\lambda I}{^k}q\Par{T}=0\Rightarrow k\geqslant m.$\PfEnd\vspace{2pt}
\ANote The expo of an irreducible $\omega$ in the factoriz of $m_T$ is the smallest $m$ suth $\omega^m\Par{T}\Big|{_{\nullp\omega\SmallPar{T}}}=0.$
\SepLine

%\Anchor{8B11}\ProblemN{B.11}{
%	\TextA{Supp $T\in\Lm{V}$ up-trig wrto $B_V=\Par{v_1,\dots,v_n}.$ \FontNorm\tgnr Let $Tv_k=u_k+\lambda_kv_k,\,u_k\in\Span{v_1,\dots,v_{k-1}}.$}
%	\TextA{Prove the multi of $\lambda$ is the number of $\lambda$ on the diag.\hfill\FontNorm\tgnr See [4E 8.31].}
%}Let $\Gamma=\Bra{\alpha_1,\dots,\alpha_d}$ be suth each $\lambda_{\alpha_k}=0,$ and each $\lambda_j\neq0$ for each $k\in\Lambda=\Bra{1,\dots,n}\Backslash\Gamma.$\par\quad
%We use induc to show $\dim\range T\geqslant n-d.$ (i) Immed. (ii) $n\geqslant 2.$ Asum true for smaller $V.$\par\quad
%Let $j$ be the largest suth $\lambda_j\neq0\Rightarrow Tv_j\not\in\Span{Tv_1,\dots,Tv_{j-1}}=U_{j-1}\Rightarrow \dim U_j=\dim U_{j-1}+1.$\par\quad
%By asum on $U_{j-1},$ $\dim U_j\geqslant\Par{j-1-d'}+1,$ where $d'$ is the number of times $0$ appears in $\lambda_1,\dots,\lambda_{j-1}.$\par\quad
%又 $d-d'=n-j$ is number of $0$ in $\Bra{\lambda_{j+1},\dots,\lambda_n}\Rightarrow\dim\range T\geqslant\dim U_j\geqslant j-d'=n-d.$\par\quad
%Hence $\dim\null T\leqslant d.$ Apply to $T^n\Rightarrow\dim\null T^n\leqslant d.$ Apply to $\Par{T-\lambda I}{^n}\Rightarrow\dim G\Par{\lambda,T}\leqslant d_{\lambda}.$\PfEnd
%\SepLine

%\Anchor{8B9}\ProblemN{B.9}{
%	\TextA{Supp $A,C$ are block diag matrices, and $A_k,C_k$ are of the same size $n_k$ \,for $k\in\;\!\!\Bra{1,\dots,m}.$}
%	\TextA{Show $AC$ is block diag and the $k^{\text{th}}$ block on the diag of $AC$ is $A_kC_k.$}
%}Let $A=\Mt{S},C=\Mt{T},AC=\Mt{ST}\in\FbbP{n,n},$ where $n=n_1+\dots+n_m.$\parSol{}
%Let $B_1=\Par{e_1,\cdots,e_{n_1}},$ and $B_{k}=\Par{e_{n_1\,+\,\cdots\,+\,n_{k-1}+1},\cdots,e_{n_1\,+\,\cdots\,+\,n_k}}$ for $k\in\;\!\!\Bra{2,\dots,m}.$\vspace{2pt}\parSol{}
%Let each $U_k=\spn B_k$ invard $S,T.$ Becs $\Mt{S\mmid_{U_k},B_k}=A_k,\Mt{T\mmid_{U_k},B_k}=C_k.$\vspace{2pt}\parSol{}
%Now $\Mt[\Sbra]{\Par{ST}\Big|{_{U_k}}}=\Mt{S\mmid_{U_k}T\mmid_{U_k}}=A_kC_k.$\PfEnd
%\SepLine

\ProblemB[]{
	\TextB{Supp $\lambda_1,\dots,\lambda_m$ are the disti eigvals of $T.$}
}
\Anchor{8BT1}\ProblemN{\BulletPointX B.\TipsN{1}}{
	\TextA{Supp $\Fbb=\Cbb,U$ invarspd $T.$ Prove $U=G\Par{\lambda_1,T\mmid_U}\oplus\cdots\oplus G\Par{\lambda_m,T\mmid_U}.$}
}We use induc on $\dim U=N.$ (i) Immed. (ii) $N>1.$ Asum it holds for smaller $U.$\parSol{}
Supp $\lambda_1$ is an eigval of $T\mmid_U.$ Let $W\oplus G\Par{\lambda_1,T\mmid_U}=U,$ where $W=\Range\Par{T\mmid_U-\lambda_1I}{^N}$ invard $T\mmid_U.$\parSol{}
Note that $T\mmid_U\mmid_W=T\mmid_W.$ By asum, $W=G\Par{\lambda_2,T\mmid_W}\oplus\cdots\oplus G\Par{\lambda_m,T\mmid_W}.$\parSol{}
Now we show $G\Par{\lambda_k,T\mmid_U}\subseteq G\Par{\lambda_k,T\mmid_W}$ for each $k\in\;\!\!\Bra{2,\dots,m}.$\parSol{}
$\forall v\in G\Par{\lambda_k,T\mmid_U},\exists\,!\,u_1\in G\Par{\lambda_1,T\mmid_U},w_k\in G\Par{\lambda_k,T\mmid_W},\:v=u_1+w_2+\dots+w_m.$ By [8.13].\PfEnd\vspace{3pt}
\AComm Note that generally, $X\oplus Y\supseteq U\neq\Par{X\cap U}\oplus\Par{Y\cap U},$ and $\Par{X+U}\cap\Par{Y+U}\neq U.$\vspace{-2pt}
\SepLine

\Anchor{8BT2}\ProblemN[]{\BulletPointX B.\TipsN{2}}{
	\TextA{Supp $V=U\oplus W,$ and $U,W$ invard $T.$ Then $G\Par{\lambda,T}=G\Par{\lambda,T\mmid_U}\oplus G\Par{\lambda,T\mmid_W}.$\vspace{-3pt}}
}\SepLine

\Anchor{Ch8C}
\Anchor{8BT3}\ProblemN{\BulletPointX B.\TipsN{3}}{
	\TextA{Supp $\Fbb=\Cbb,$ and $q\Par{z}=\Par{z-\lambda_1}{^{\alpha_1}}\cdots\Par{z-\lambda_k}{^{\alpha_k}}.$}
	\TextA{Let $F_{\!j}=\Null\Par{T-\lambda_jI}{^{\alpha_j}}.$ Prove $\nullp q\Par{T}=F_{\!1}\oplus\cdots\oplus F_{\!m}.$}
}Each $\Par{T-\lambda_kI}{^{\alpha_k}}\Big|{_{\:\!G\SmallPar{\:\!\lambda_j,\,T\:\!}}}$ is inje for $k\neq j\Rightarrow\nullp q\Par{T}\Big|{_{\:\!G\SmallPar{\:\!\lambda_j,\,T\:\!}}}=F_{\!j}.$ \;\Or By \Sbra{B \TIPSN{1,4}}.\PfEndB
\SepLine

\ProblemB[]{
	\TextB{Supp $p,q\in\PoFi$ have no common zeros on $\Cbb.$}
}
\Anchor{8BT4}\ProblemN{\BulletPointX B.\TipsN{4}}{
	\PrePa\TextA{Prove $\nullp p\Par{T}\oplus\nullp q\Par{T}=\Null\Par{pq}\Par{T},\,\rangep p\Par{T}+\rangep q\Par{T}=V.$}
	\PrePb\TextA{Prove $\rangep p\Par{T}\cap\rangep q\Par{T}=\Range\Par{pq}\Par{T}.$}
}(a) By Exe (4E 4.13), $\forall v\in V,\,v=r\Par{T}\:\!p\Par{T}\:\!v+s\Par{T}\:\!q\Par{T}\:\!v.$\par\quad
(b) $v\in\rangep p\Par{T}\cap\rangep q\Par{T}\Rightarrow\exists\,u,w\in V,\,v=p\Par{T}u=q\Par{T}w=\Par{pq}\Par{T}\BigPar{r\Par{T}w+s\Par{T}u}.$\PfEndB\vspace{2pt}
\ACoro Supp $\Par{pq}\Par{T}=0.$ We show $\nullp p\Par{T}=\rangep q\Par{T}.$\parCor
Becs `$\supseteq$' holds $\Rightarrow\nullp q\Par{T}\cap\rangep p\Par{T}=\zeroSubs.$ By (5.C.3) and \Sbra{1.C \TIPSN{1}}.\PfEnd\parCor
Supp $\Par{p_1\cdots p_k}\Par{T}=0,$ and $p_1,\dots,p_k\in\PoFi$ have no common zeros on $\Cbb.$\vspace{1pt}\parCor
(c) Each $\nullp p_j\Par{T}=\Range\BigPar{{\prod_{i\neq j}^k p_i}}\Par{T}=\bigcap_{i\neq j}^k\rangep p_i\Par{T}.$ \;\Or By (d).\vspace{2pt}\parCor
(d) Each $\rangep p_j\Par{T}=\bigoplus_{i\neq j}^k\nullp p_i\Par{T}.$ Note that $V=\nullp p_j\Par{T}\oplus\Sbra{{\bigoplus_{i\neq j}^k\nullp p_i\Par{T}}}.$\PfEnd\vspace{-2pt}
\SepLine[0pt][\Blind{\BulletPointX} ]

\BulletPointX\ANote If $\Par{pq}\Par{T}=0.$ \,Let $V_{\!\Cbb}=G\Par{\lambda_1,T_{\!\Cbb}}\oplus\cdots\oplus G\Par{\lambda_m,T_{\!\Cbb}},$ and $m_T\Par{z}=\Par{z-\lambda_1}{^{k_1}}\cdots\Par{z-\lambda_m}{^{k_m}}.$\parNot{\IndentB}
Let $\mA=\Bra{\alpha_1,\dots,\alpha_{\!A}}$ be the intersec of the zeros of $p$ and the eigvals of $T.$ \,Simlr for $\mB.$\parNot{\IndentB}
Then $\mA\cap\mB=\emptySet.$ Let $p=\omega_{\alpha_1}\!\!\cdots\omega_{\alpha_{\!A}}\,p_0,\;q=\omega_{\beta_1}\!\!\cdots\omega_{\beta_B}\,q_0,$ where $\omega_{\lambda_i}\Par{z}=\Par{z-\lambda_i}{^{k_i}}.$\parNot{\IndentB}
By \Sbra{B \TIPSN{3}}, $\nullp p\Par{T_{\!\Cbb}}=G\Par{\alpha_1,T_{\!\Cbb}}\oplus\cdots\oplus G\Par{\alpha_{\!A},T_{\!\Cbb}}.$ \,Simlr for $q\Par{T_{\!\Cbb}}.$\vspace{1pt}\parNot{\IndentB}
For $\Fbb=\Rbb,$ if $\exists\,\alpha_j\not\in\Rbb,$ then $\exists\,\alpha_i=\overline{\alpha_j}.$ Simlr for $\mB.$ \,Now $V_{\!\Cbb}=\nullp p\Par{T_{\!\Cbb}}\oplus\nullp q\Par{T_{\!\Cbb}}.$
\SepLine

%\Anchor{8B'2}\ProblemB{
%	\TextA{Supp $T\in\Lm{V},U$ invarspd $T.$ Give a countexa\hspace{1pt}$:$ $\exists$ invarsp $W$ suth $V=U\oplus W.$}
%}See Exa [5.43]: $V=\CbbP{2},B_U=\Par{e_1}.$ Define $Te_1=0,Te_2=e_1\Rightarrow G\Par{0,T}=V.$
%\AComm $G\Par{\lambda_k,T\mmid_W}=G\Par{\lambda_k,T}\cap\Par{W_1\oplus\cdots\oplus W_m}=G\Par{\lambda_k,T\mmid_{W_k}}.$\par\vspace{2pt}
%\ANote This Exe is not true if $\Fbb=\Rbb.$ We give an exa of $T\in\Lm{V}$ and invarsp $U$\parNot
%suth $\nexists$ invarsp $W$ of dimension $\dim V-\dim U.$ Note that no exas exis if $\dim V<5.$\parNot
%We show no exas exis for $\dim V=5.$ Supp $\dim V=5\Rightarrow\exists\,\lambda$ an eigval of $T.$\parNot
%Supp $\dim U=2,$ becs the existns of $W$ can be shown easily if $\dim U=1,3,4$ by (4E 5.A.39).\parNot
%Supp $E\Par{\lambda,T}\subseteq U,$ for if not, pick a $v\in E\Par{\lambda,T}\Backslash U$ and let $W=\Span{v}\oplus U,$ done.\parNot
%Supp $G\Par{\lambda,T}=U$ and no more eigvals. For if not, (1) $T$ has more than two eigvals, done.\parNot
%(2) $T$ has only two disti eigvals $\lambda,\mu.$ Let $m_1=\dim G\Par{\lambda,T},m_2=\dim G\Par{\mu,T}.$\parNot
%If $\Par{m_1,m_2}=\Par{1,4}$ or $\Par{4,1},$ then becs $T$ restr to $G\Par{\lambda,T}$ or $G\Par{\mu,T}$ has an eigval\parNot
%$\Rightarrow\exists$ invarspd of dim $3.$ If $\Par{m_1,m_2}=\Par{1,1},$ then by char factoriz, ctradic. Other cases are simlr.\parNot
%(3) If $U\supseteq E\Par{\lambda,T}$ of dim $1$, then by the factoriz of the char poly of $T\mmid_U,$ $\exists$ another eigval.\parNot
%Thus $G\Par{\lambda,T}=U$ of dim $2,$ then $\exists$ another eigval by char factoriz, done.\PfEnd
%\SepLine

\Anchor{Ch8D}
%\Anchor{8D7}\Anchor{8B4e21}\ProblemNnoor{D.7}{4E B.21}{
%	\TextA{Supp monic $p,q\in\PoCi$ have the same zeros, and $q$ is a multi of $p.$}
%	\TextA{Prove $\exists\,T\in\Lm{\CbbP{\deg q}}$ suth the char of $T$ is $q$ and the min of $T$ is $p.$}
%}
%\SepLine

%\Anchor{8D3}\ProblemN{D.3}{
%	\TextA{Supp $N\in\Lm{V}$ is nilp. Prove the min of $N$ is $z^{m+1},$}
%	\TextA{where $m$ is the len of the longest consecutive string of $1$'s}
%	\TextA{that appears on the line directly above the diag in any Jordan $\Mt{N,B_V}.$}
%}Get such $m.$ Becs $N^{m+1}=0.$ 又 $N^{m}\neq0.$\PfEnd
%\SepLine

\Anchor{8N8.55}\ProblemBX[]{\NoteForSmall{[8.55]}}{
	Supp $N^m=0\neq N^{m-1}.$ Let each $\null N^k=\null N^{k-1}\oplus U_k$ for $k\in\;\!\!\Bra{2,\dots,m}.$\TextB{}
%	(1) $\null N^{m-1}\oplus U_{m-1}=\null N^m=\Par{\null N^{m-2}\oplus U_{m-2}}\oplus U_{m-1}.$\TextB{}
%	(2) $\null N^{m-j}\oplus U_{m-j}=\null N^{m-j+1}.$\, (3) $\null I\oplus U_0=\null N.$\TextB{}
	(1) Start by $B_{U_m}=\Par{v_{1,1},\dots,v_{n_1,1}}.$\, But $\Par{Nv_{1,1},\dots,Nv_{n_1,1}}$ might be liney dep. Invalid method.\TextB{}
	(2) $\Mt{N,\,B_{\null N}\cup B_{U_2}\cup\cdots\cup B_{U_m}}$ is a block up-trig matrix with the diag blocks all zero.\par\vspace{2pt}
%	(2) Let $B_{U_{m-j}}=\Par{N^{j-1}v_{1,1},\dots,N^{j-1}v_{n_1,1},\dots,Nv_{1,j-1},\dots,Nv_{n_{j-1},j-1},v_{1,j},\dots,v_{n_j,j}}.$\TextB{}
%	(3) Let $B_{U_0}=\Par{N^{m-1}v_{1,1},\dots,N^{m-1}v_{n_1,1},\dots,Nv_{1,m-1},\dots,Nv_{n_{m-1},m-1},v_{1,m},\dots,v_{n_m,m}}.$ Sum up and done.
	\Anchor{8NED6}\BulletPointX\NoteForSmall{Exe (D.6)}\;\;Let $B=\Par{N^{m_1}v_1,\dots,N^{m_1\!-k}v_1,\cdots,N^{m_n}v_n,\dots,N^{m_n\!-k}v_n},$ $0\leqslant k\leqslant\min\!\Bra{m_1,\dots,m_n}.$\TextB{}
	All liney indep in $\null N^{k+1}.$ Supp $N^{k+1}$ sends a liney combina of the Jordan $B_V$ to zero.\TextB{}
	Then all coeffs of $N^{m_1-k-i}v_k$ are zero. Thus $\null N^{k+1}\subseteq\spn B.$ Now $B$ is a bss of $\null N^{k+1}.$\TextB{\vspace{-2pt}}
}
\SepLine

\Anchor{8B4e20}\Anchor{8C20}\ProblemNnoor{C.20}{4E B.20}{
	\TextA{Supp $\Fbb=\Cbb,$ and each $V_{\!k}$ non0 invarsp of $V=V_{\!1}\oplus\cdots\oplus V_{\!m}.$}
	\TextA{Let $p_k$ be the char of $T\mmid_{V_{\!k}}.$ Prove the char of $T$ is $p_1\cdots p_m.$}
}By \Sbra{B \TIPSN{1}}, $V=G\Par{\lambda_1,T}\oplus\cdots\oplus G\Par{\lambda_n,T}\Rightarrow V_{\!k}=G\Par{\lambda_1,T\mmid_{V_{\!k}}}\oplus\cdots\oplus G\Par{\lambda_m,T\mmid_{V_{\!k}}}.$\parSol{}
By \Sbra{B \TIPSN{2}}, each $G\Par{\lambda_j,T}=G\Par{\lambda_j,T\mmid_{V_{\!1}}}\oplus\cdots\oplus G\Par{\lambda_j,T\mmid_{V_{\!m}}}.$\parSol{}
Let $d_{j,k}$ be the multy of $\lambda_j$ of $T\mmid_{V_{\!k}}.$ Then $d_{j,1}+\dots+d_{j,n}=d_j,$ the multy of $\lambda_j$ of $T.$\parSol{}
Thus each $p_k\Par{z}=\Par{z-\lambda_1}{^{d_{1,k}}}\!\cdots\Par{z-\lambda_n}{^{d_{n,k}}}.$ While the char of $T$ is $\Par{z-\lambda_1}{^{d_1}}\cdots\Par{z-\lambda_n}{^{d_n}}.$\PfEnd\vspace{2pt}\parSol{}
\Or Let $A$ be a block diag matrix of $T,$ with each $A_k=\Mt{T\mmid_{V_{\!k}}}$ up-trig. By Exe (B.11).\PfEnd
\SepLine

%\Anchor{8C4e12}\ProblemBnoor{4E C.12}{
%	\TextA{Supp $T\in\Lm{V}$ diag. Show $\Mt{T}$ diag wrto any Jordan $B_V.$}
%}Each $v_k$ of a Jordan $B_V$ is a g-eigvec; so is an eigvec, by Exe (B.5).\PfEnd\vspace{2pt}\parSol{}
%\Or Let $A$ be a Jordan block diag matrix of $T.$ By Exe (D.3) and [4E 5.62].\PfEnd
%\SepLine

\Anchor{8D8}\ProblemN{D.8}{
	\TextA{Supp $\Fbb=\Cbb.$ Prove $\nexists$ non0 invarsps $U,W$ suth $U\oplus W=V\Longleftrightarrow m_T\Par{z}=\Par{z-\lambda}{^{\dim V}}.$}
}Let $N=T-\lambda I\Rightarrow$ the min of $N$ is $z^{\dim V}.$\parSol{}
Then by Exe (D.3), the line directly above the diag of {\tgsc any} Jordan $\Mt{N}$ is all $1.$\parSol{}
Thus the only Jordan block of $\Mt{N}$ is $\Mt{N}$ itself. \;Convly true as well.\PfEnd\vspace{2pt}\quad
\Or (a) If $\exists$ two or more eigvals of $T\mmid_U$ or $T\mmid_W,$ then $m_T$ has two or more disti factors, done.\par\quad\Ha
\Blind{\Or}Now supp $\exists$ only one eigval $\lambda$ for $T\mmid_U,$ $T\mmid_W,$ and $T.$ Supp $m_T\Par{z}=\Par{z-\lambda}{^m}.$\par\quad\Ha
\Blind{\Or}Let $M=\max\!\Bra{{\dim U,\dim W}}.$ Let $S=\Par{T-\lambda I}{^M}\Rightarrow\null S\mmid_U\oplus\null S\mmid_W=\null S.$\par\quad\Ha
\Blind{\Or}Becs $G\Par{\lambda,T\mmid_U}=U,\,G\Par{\lambda,T\mmid_W}=W,\,G\Par{\lambda,T}=V\Rightarrow S=0.$ Now by Exe (4E B.6).\vspace{2pt}\par\quad\Ha
\Blind{\Or}\Or Becs $\exists$ Jordan $\Mt{T\mmid_U},\Mt{T\mmid_W}\Rightarrow$ Jordan $\Mt{T}.$ Consider $z^M$ by Exe (D.3).\vspace{4pt}\par\quad
\Blind{\Or}(b) Supp $T$ has only one eigval. Let
$m_T\Par{z}=\Par{z-\lambda}{^m}$ with $m<\dim V.$\par\quad\Hb
\Blind{\Or}Becs $\exists$ Jordan $B_V=\Par{\underbrace{v_{1,1},\cdots,v_{m_1,1}}_{\text{bss for } U},\underbrace{v_{1,2},\cdots,v_{m_2,2},\cdots,v_{1,k},\cdots,v_{m_k,k}}_{\text{bss for }W}}$ for $T.$\PfEnd\vspace{4pt}
\SepLine
\ChEnd\pagebreak

\ChDecl{Ch9A}{9.A}{\quad{\ANote {\FontSmall $V$ denotes a finide non0 vecsp over $\Fbb.$}}}

\vspace{4pt}

\Anchor{9AN910}\BulletPointX\NoteForSmall{[9.10]}\;\;Let $q\in\PoCi$ be the min of $T_{\!\Cbb}.$ Note that $A=\Mt{T_{\!\Cbb}}=\Mt{T}.$\TextB{}
Then $q\Par{A}=0=\overline{q\Par{A}}=\overline{q}\Par{A}\Rightarrow\overline{q}\Par{T_{\!\Cbb}}=q\Par{T_{\!\Cbb}}=0\Rightarrow q=\overline{q}\Rightarrow q\in\PoRi.$ 又 $q\Par{T}=0.$
\SepLine

\Anchor{9AN912}\BulletPointX\NoteForSmall{[9.12]}\;\;Another proof: \;$\overline{T_{\!\Cbb}\Par{u+\i v}}=\overline{Tu+\i Tv}=Tu-\i Tv=T_{\!\Cbb}\Par{\overline{u+\i v}}.$\TextB{}
$\overline{\Par{T_{\!\Cbb}-\lambda I}\Par{u+\i v}}=\overline{T_{\!\Cbb}\Par{u+\i v}-\lambda\Par{u+\i v}}=T_{\!\Cbb}\Par{u-\i v}-\overline{\lambda}\Par{u-\i v}=\Par{T_{\!\Cbb}-\overline{\lambda}I}\Par{\overline{u+\i v}}.$\TextB{}
We use induc on $m$ to show $\overline{\Par{T_{\!\Cbb}-\lambda I}{^m}\Par{u+\i v}}=\Par{T_{\!\Cbb}-\overline{\lambda}I}{^m}\Par{\overline{u+\i v}}.$ (i) Immed. (ii) $m>1.$\TextB{}
Asum it holds for $\Par{m-1}.$ Let $\Par{T_{\!\Cbb}-\lambda I}{^{m-1}}\Par{u+\i v}=x+\i y\Rightarrow\Par{T_{\!\Cbb}-\overline{\lambda}I}{^{m-1}}\Par{\overline{u+\i v}}=x-\i y.$\TextB{}
Then $\overline{\Par{T_{\!\Cbb}-\lambda I}{^m}\Par{u+\i v}}=\overline{\Par{T_{\!\Cbb}-\lambda I}\Par{x+\i y}}=\Par{T_{\!\Cbb}-\overline{\lambda}I}\Par{x-\i y}=\Par{T_{\!\Cbb}-\overline{\lambda}I}{^m}\Par{\overline{u+\i v}}.$\PfEnd\vspace{3pt}
\Anchor{9AN917}\BulletPointX\NoteForSmall{[9.17]}\;\;Detailed proof:\TextB{}
Let $B=\Par{u_1+\i v_1,\dots,u_m+\i v_m}$ be a bss of $G\Par{\lambda,T_{\!\Cbb}}.$ By [9.12], $\overline{B}=\Par{u_1-\i v_1,\dots,u_m-\i v_m}$ in $G\Par{\overline{\lambda},T_{\!\Cbb}}.$\TextB{}
(a) If $a_1\Par{u_1-\i v_1}+\dots+a_m\Par{u_m-\i v_m}=0.$ Conjuging, now each $\overline{a_k}=0.$ Liney indep.\TextB{}
(b) $\forall u-\i v\in G\Par{\overline{\lambda},T_{\!\Cbb}},u+\i v\in G\Par{\lambda,T_{\!\Cbb}}\Rightarrow u+\i v\in\spn B\Rightarrow u-\i v\in\spn\overline{B}.$\PfEnd
\SepLine

\Anchor{9A13}\ProblemN{13}{
	\TextA{Supp $\Fbb=\Rbb,T\in\Lm{V},$ and $b^2<4c.$ Let $q\Par{z}=z^2+bz+c=\Par{z-\lambda}\Par{z-\overline{\lambda}}.$\vspace{2pt}}
	\TextA{Prove $\dim\nullp q\Par{T}{^j}$ is even for each $j\in\Nbp.$\tgnr\hfill\FontSmall\Sbra{See also {\NOTEFOR} [4E 5.33] in (5.BI).}}
}By \Sbra{8.B \TIPSN{3}}, $\nullp q\Par{T_{\!\Cbb}}{^j}=\Null\Par{T_{\!\Cbb}-\lambda I}{^j}\oplus\Null\Par{T_{\!\Cbb}-\overline{\lambda}I}{^j}.$ \;By [9.17] and [9.4].\PfEnd\vspace{4pt}
\ANote Let $Q\Par{\lambda,T}=\nullp q\Par{T}{^{\dim V}}.$ Then by (4E 8.B.6,7) for $T_{\!\Cbb},$ by [9.10,20], and by \Sbra{8.B \TIPSN{4}},\parNot
(a) $Q\Par{\lambda,T}=\nullp q\Par{T}{^d},$ where $d=\dim G\Par{\lambda,T_{\!\Cbb}}.$\parNot
(b) The expo of $q$ in the factoriz of $m_T$ is the smallest $m\in\Nbp$ suth $q\Par{T}{^m}\Big|{_{\:\!Q\SmallPar{\:\!\lambda,\,T\:\!}}}=0.$\vspace{1pt}\parNot
(c) $m_T=p_1^{\alpha_1}\!\cdots p_m^{\alpha_m}\:q_1^{\beta_1}\!\cdots q_M^{\beta_M}\Longleftrightarrow V=\Sbra{{\bigoplus_{j=1}^mG\Par{\mu_j,T}}}\oplus\Sbra{{\bigoplus_{k=1}^MQ\Par{\lambda_k,T}}}.$\vspace{1pt}\parNot\Hc
Where each $p_j\Par{z}=z-\mu_j,\,q_k\Par{z}=z^2-2\Par{\Real\lambda_k}z+\aMidsq{\lambda_k}=z^2+b_kz+c_k.$\vspace{4pt}\parNot
Fix one $k.$ Let $q\Par{z}=q_k\Par{z}=\Par{z-\lambda}\Par{z-\overline{\lambda}},\lambda=a+b\i,G=G\Par{\lambda,T_{\!\Cbb}},\overline{G}=G\Par{\overline{\lambda},T_{\!\Cbb}}.$\vspace{1pt}\parNot
Replace $T$ with $T\mmid_{\:\!Q}.$ Let $Q=Q\Par{\lambda,T}$ of dim $\beta,$ and $Q_\Cbb=G\oplus\overline{G},$ and Jordan bss $B_J$ of $Q_\Cbb.$\vspace{2pt}\parNot
Now $\Mt{T_{\!\Cbb}}=\footnotesize\begin{pmatrix}A_1 &\hspace{-8pt} 0\\[-4pt]0 &\hspace{-8pt} A_2\end{pmatrix},\Mt{T_{\!\Cbb}-\lambda I}=\footnotesize\begin{pmatrix}\overline{R_1} &\hspace{-8pt} 0\\[-2pt]0 &\hspace{-8pt}\overline{R_2}\end{pmatrix},\,\Mt{T_{\!\Cbb}-\overline{\lambda}I}=\footnotesize\begin{pmatrix}{R_2} &\hspace{-8pt} 0\\[-4pt]0 &\hspace{-8pt} R_1\end{pmatrix}$ wrto Jordan bss.\vspace{2pt}\parNot
So then $\Mt{T_{\!\Cbb}^2+bT_{\!\Cbb}+cI}=\Mt{T_{\!\Cbb}-\lambda I}\Mt{\overline{T_{\!\Cbb}-{\lambda}I}}=\footnotesize\begin{pmatrix}{R} &\hspace{-8pt} 0\\[-4pt]0 &\hspace{-8pt} \overline{R}\end{pmatrix},$ where $R=R_1R_2.$\parNot
Where $A_1,R_1,R_2,R$ are block diag matrices, and $A_1=\Mt{T_{\!\Cbb}\mmid_{\:\!G}},\,A_2=\Mt{T_{\!\Cbb}\mmid_{\:\!\overline{G}}}=\overline{\Mt{T_{\!\Cbb}\mmid_{\:\!G}}}.$\vspace{2pt}\parNot
Each $A_{1,k}=\footnotesize\begin{pmatrix}\lambda &\hspace{-6pt} 1 &\hspace{-14pt} &\hspace{-14pt} 0\hspace{-6pt}\\[-4pt]&\hspace{-18pt} \ddots &\hspace{-20pt} \ddots &\hspace{-14pt} \hspace{-6pt}\\[-6pt]&\hspace{-20pt} &\hspace{-20pt} \ddots &\hspace{-14pt} 1\hspace{-6pt}\\[-4pt]0 &\hspace{-14pt} &\hspace{-14pt} &\hspace{-14pt} \lambda\hspace{-6pt}\end{pmatrix},R_{1,k}=\footnotesize\begin{pmatrix}0 &\hspace{-6pt} 1 &\hspace{-14pt} &\hspace{-14pt} 0\hspace{-6pt}\\[-4pt]&\hspace{-18pt} \ddots &\hspace{-20pt} \ddots &\hspace{-14pt} \hspace{-6pt}\\[-6pt]&\hspace{-20pt} &\hspace{-20pt} \ddots &\hspace{-14pt} 1\hspace{-6pt}\\[-4pt]0 &\hspace{-14pt} &\hspace{-14pt} &\hspace{-14pt} 0\hspace{-6pt}\end{pmatrix},R_{2,k}=\footnotesize\begin{pmatrix}2b\i &\hspace{-6pt} 1 &\hspace{-14pt} &\hspace{-14pt} 0\hspace{-6pt}\\[-4pt]&\hspace{-18pt} \ddots &\hspace{-20pt} \ddots &\hspace{-14pt} \hspace{-6pt}\\[-6pt]&\hspace{-20pt} &\hspace{-20pt} \ddots &\hspace{-14pt} 1\hspace{-6pt}\\[-4pt]0 &\hspace{-14pt} &\hspace{-14pt} &\hspace{-14pt} 2b\i\hspace{-6pt}\end{pmatrix},R_k=\footnotesize\begin{pmatrix}
0 &\hspace{-6pt} 2b\i &\hspace{-6pt} 1 &\hspace{-6pt} &\hspace{-6pt} 0\\[-4pt]
0 &\hspace{-6pt} &\hspace{-6pt} \ddots &\hspace{-6pt} \ddots &\hspace{-6pt} \\[-4pt]
\vdots &\hspace{-6pt} &\hspace{-6pt} &\hspace{-6pt} \ddots &\hspace{-6pt} 1\\[-4pt]
0 &\hspace{-6pt} 0 &\hspace{-6pt} &\hspace{-6pt} &\hspace{-6pt} 2b\i\\[-4pt]
0 &\hspace{-6pt} 0 &\hspace{-6pt} \cdots &\hspace{-6pt} \cdots &\hspace{-6pt} 0\end{pmatrix}.$\vspace{2pt}\parNot
Let the Jordan bss $Q_\Cbb$ for $T_{\!\Cbb}$ be $\Par{u_1+\i\,v_1,\dots,u_\beta+\i\,v_{\:\!\!\beta},u_1-\i\,v_1,\dots,u_\beta-\i\,v_{\:\!\!\beta}}.$\parNot
Now due to $\Mt{T_{\!\Cbb}},$ $T\Par{u_1\pm\i\,v_1}=\Par{a\pm\i\,b}\Par{u_1\pm\i\,v_1}=\Par{a\,u_1-b\,v_1}\pm\i\,\Par{b\,u_1+a\,v_1},$\parNot
$T\Par{u_j\pm\i\,v_j}=\Par{a\pm\i\,b}\Par{u_j\pm\i\,v_j}+\Par{u_{j-1}\pm\i\,v_{j-1}}=\Par{a\,u_j-b\,v_j+u_{j-1}}\pm\i\,\Par{b\,u_j+a\,v_j+v_{j-1}}.$\parNot
Hence $Tu_1=a\,u_1-b\,v_1,Tv_1=b\,u_1+a\,v_1,$ and $Tu_j=u_{j-1}+a\,u_j-b\,v_j,Tv_j=v_{j-1}+b\,u_j+a\,v_j.$\parNot
Let $B_Q=\Par{u_1,v_1,\dots,u_\beta,v_{\:\!\!\beta}}\Rightarrow\Mt{T,B_Q}=\footnotesize\begin{pmatrix}\mR &\hspace{-6pt} I_2 &\hspace{-14pt} &\hspace{-14pt} 0\hspace{-6pt}\\[-4pt]&\hspace{-18pt} \ddots &\hspace{-20pt} \ddots &\hspace{-14pt} \hspace{-6pt}\\[-6pt]&\hspace{-20pt} &\hspace{-20pt} \ddots &\hspace{-14pt} I_2\hspace{-6pt}\\[-3pt]0 &\hspace{-14pt} &\hspace{-14pt} &\hspace{-14pt} \mR\hspace{-6pt}\end{pmatrix},$ where $\mR=\footnotesize\begin{pmatrix}a &\hspace{-6pt} b\\[-4pt]-b &\hspace{-6pt} a\end{pmatrix}$ and $I_2=\footnotesize\begin{pmatrix}1 &\hspace{-8pt} 0\\[-4pt]0 &\hspace{-8pt} 1\end{pmatrix}.$\vspace{-10pt}\parNot
\Or $B_Q=\Par{v_1,u_1,\dots,v_{\:\!\!\beta},u_\beta}\Rightarrow\mR=\footnotesize\begin{pmatrix}a &\hspace{-6pt} -b\\[-4pt]b &\hspace{-6pt} a\end{pmatrix}.$
\SepLine

%\Anchor{9A18}\ProblemN{18}{
%	\TextA{Supp $\Fbb=\Rbb,$ $T\in\Lm{V},$ and all eigvals of $T_{\!\Cbb}$ are real.}
%	\TextA{Show {\tgnr\large(a)} $\exists\,B_V$ suth $T$ up-trig; \:{\tgnr\large(b)} $\exists\,B_V$ of g-eigvecs of $T.$}
%}(a) By [9.10] and [4E 5.44], immed. \;\Or Using induc on $\dim V.$ (i) Immed. (ii) $\dim V>1.$\parSol{\Ha}
%Asum it holds for smaller $V.$ Supp all eigvals of $T_{\!\Cbb}$ are real. Let $U=\Range\Par{T-\lambda I}.$\parSol{\Ha}
%Then all eigvals of $ T_{\!\Cbb}\mmid_{U_{\Cbb}}=\Par{T\mmid_U}{_{\Cbb}}$ are real. By asum, simlr to [5.27].\vspace{3pt}\parSol{}
%(b) By (a), (4E 8.A.11) and [4E 5.44], immed. \Or $V_{\!\Cbb}=G\Par{\lambda_1,T_{\!\Cbb}}\oplus\cdots\oplus G\Par{\lambda_m,T_{\!\Cbb}}.$\parSol{\Hb}
%Becs each $G\Par{\lambda_k,T_{\!\Cbb}}=G\Par{\lambda_k,T}{_{\Cbb}}$ and $U_{\Cbb}+W_{\Cbb}=\Par{U+W}{_{\Cbb}}.$ \;By [9.4](b).\PfEnd
%\SepLine

%\ProblemB{
	%	\TextB{Supp $\lambda$ is eigval of $P\in\Lm{V}$ and $P^2=P.$ Prove $\lambda = 0$ or $1$.}
	%}$v\neq 0,\,Pv=\lambda v=\lambda^2 v=P\Par{Pv}.$ Thus $\lambda=1$ or $0.$\PfEnd
%\SepLine
%
%\ProblemB{
	%	\TextB{Supp $V=U\oplus W$, and $U,W$ non0. Define $P\Par{u + w} = u.$ Find all eigvals and eigvecs.}
	%}Supp $u+w\neq 0$ and $P\Par{u+w}=u=\lambda u+\lambda w\Rightarrow\Par{\lambda-1}u+\lambda w=0.$\parSol{}
%Becs $\Par{\lambda-1}u=\lambda w=0.$ Now $\lambda=0\Longleftrightarrow u=0,$ and $\lambda=1\Longleftrightarrow w=0.$ Thus $Pu=u,Pw=0.$\PfEnd
%\SepLine

\ChEnd