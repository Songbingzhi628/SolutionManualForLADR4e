% Copyright (C) 2024 Songbingzhi628. This work is licensed under Creative Commons Attribution-NonCommercial-ShareAlike 4.0 International License.
% Email: 13012057210@163.com

\documentclass[a4paper, 11pt, UTF8]{article}

\usepackage{amssymb,amsmath,amsfonts}
\usepackage{ctex}
\usepackage{setspace}
\usepackage{graphicx}
\usepackage{hyperref}
\usepackage{bookmark}
\usepackage{fontspec}
\usepackage{unicode-math}
\usepackage{ulem}
\usepackage{xcolor}
\usepackage{tabularx}
\usepackage{tikz-cd}
\usepackage{xargs}
\usepackage{Headers/extarrows}
\usepackage[left=1cm,right=1cm,top=1cm,bottom=0cm]{geometry}

\hypersetup{
	bookmarks=true,
	bookmarksnumbered=false,
	bookmarksopen=true,
	bookmarksdepth=2,
	colorlinks=true,
	linkcolor=blue,
	urlcolor=cyan
}

\bookmarksetup{
	numbered=false,
	open=true
}

\parindent 0pt


%{ 关于字体设置
\setmainfont{TeXGyrePagella-Regular}
\setmathfont{TeXGyrePagellaMath-Regular}
\setmathfont[range={\mathcal,\Longleftrightarrow,\Longrightarrow,\Longleftarrow,\Rightarrow,\Leftarrow,\Leftrightarrow}]{LatinModernMath-Regular}
%\setmathfont[range={\backslash}]{TeXGyrePagella-Italic}
\setmathfont[range={\mathbb}]{TeXGyreSchola-Regular}
\setmathfont[range={\complement}]{TeXGyrePagella-Bold}
%\setmathfont[range=\mathfrak]{TeXGyrePagella-Regular}
\setmathfont[range={\tilde}]{DejaVuMathTeXGyre.ttf}

\newfontfamily{\tgnr}{TeXGyrePagella-Regular}
\newfontfamily{\tgbf}{TeXGyreSchola-Regular}
\newfontfamily{\tgbfx}{TeXGyrePagella-Bold}
\newfontfamily{\tgbfxx}{FreeSerifBold}
\newfontfamily{\tgsl}{TeXGyrePagella-Italic}
\newfontfamily{\tgsc}{TeXGyrePagella-BoldItalic}
\newfontfamily{\forbra}{MathJax_Size1-Regular.otf}
\newfontfamily{\forbrax}{MathJax_Size2-Regular.otf}
\newfontfamily{\forslash}{MathJax_Size1-Regular.otf}
\newcommand{\XSlash}[1][-1.2pt]{\hspace{#1}\text{\forslash/}}
\newcommand{\Slash}[1][\envFontA]{${#1$/$}$}

\newcommand{\Largenr}[1]{{\Large\tgnr#1}}
\newcommand{\Largesl}[1]{{\Large\tgsl#1}}
\newcommand{\Largebf}[1]{{\Large\tgbf#1}}
\newcommand{\Largebfx}[1]{{\Large\tgbfx#1}}
\newcommand{\Largebfxx}[1]{{\Large\tgbfxx#1}}

\def\envFontHuge{\def\envFontA{\huge}\def\envFont{\LARGE}\def\envFontB{\Large}}
\def\envFontLarge{\def\envFontA{\LARGE}\def\envFont{\Large}\def\envFontB{\large}}
\def\envFontDefault{\def\envFontA{\Large}\def\envFont{\large}\def\envFontB{\normalsize}}
\newcommand{\envFontSmall}[1][\small]{\def\envFontA{\Large}\def\envFont{\normalsize}\def\envFontB{#1}}
\newcommand{\TextE}[1]{\envFontLarge\Largesl{#1}\envFontDefault\par\IndentE}
\newcommand{\TextN}[1]{\envFontLarge\Largesl{#1}\envFontDefault\par\IndentN}
\newcommand{\TextB}[1]{\envFontLarge\Largesl{#1}\envFontDefault\par\IndentB}
\newcommand{\Text}[1]{\envFontLarge\Largesl{#1}\envFontDefault\par{\,\!}}
\newcommand{\TextNL}[1]{\envFontLarge\Largesl{#1}\envFontDefault\par\IndentNL}
\def\FontSmall{\normalsize\envFontSmall}
\def\FontNorm{\large\envFontDefault}
\def\FontLarge{\Large\envFontLarge}
%}

%{ 关于排版缩进的便捷指令
%  · {题目内容}\par\IndentB
%  8 {题目内容}\par\IndentN
% 19 {题目内容}\par\IndentNL
% EXAMPLE: {text}\par\IndentE
\def\IndentB{\Blind{\BulletPoint }}
\def\IndentN{\hspace{8.7pt}}
\def\IndentNL{\hspace{16pt}}
\def\IndentE{\Blind{\BulletPointX\Example \,\,}}

% 用于对齐(a)(b)(c)(d)
%(a) {text} \par\quad\Ha
%... \par\quad\Ha
%{text} \par\quad
\def\Ha{{\large\Blind{(a) }}}
\def\Hb{{\large\Blind{(b) }}}
\def\Hc{{\large\Blind{(c) }}}
\def\Hd{{\large\Blind{(d) }}}
\def\He{{\large\Blind{(e) }}}
\def\Hf{{\large\Blind{(f) }}}

% 用于对齐(i)(ii)(iii)
%(i) {text} \par\quad\Hi
%... \par\quad\Hi
%{text} \par\quad\Endi
%(ii)  {text} \par\quad\Hii
%... \par\quad\Hii
%{text} \par\quad\Endii
%(iii) {text} \par\quad\Hiii
%... \par\quad\Hiii
%{\text}\par\quad
\def\Endi{}
\def\Endii{}
\def\Hi{\Blind{(i) }}
\def\Hii{\Blind{(ii) }}
\def\Hiii{}

% 用于对齐(I)(II)(III)
\def\EndI{}
\def\EndII{}
\def\EndIII{}
\def\HI{\Blind{(I) }}
\def\HII{\Blind{(II) }}
\def\HIII{\Blind{(III) }}
%}

%{ 关于文本环境
\newcommand{\hMath}[5][-4pt]{#3\hspace{#1}\begin{array}{#2}#5\end{array}\hspace{#1}#4}
\newcommand{\MathLeftBrace}				[2]{\hMath[0pt]{#1}{\left\{}{\right.}{#2}}
\newcommand{\MathRightBrace}			[2]{\hMath[0pt]{#1}{\left.}{\right\}}{#2}}
\newcommand{\MathLeftrightBrace}		[2]{\hMath[0pt]{#1}{\left\{}{\right\}}{#2}}
\newcommand{\MathLeftMid}				[2]{\hMath[0pt]{#1}{\left|}{\right.}{#2}}
\newcommand{\MathRightMid}				[2]{\hMath[0pt]{#1}{\left.}{\right|}{#2}}
\newcommand{\MathLeftrightMid}			[2]{\hMath[0pt]{#1}{\left|}{\right|}{#2}}
\newcommand{\MathLeftrightPare}			[2]{\hMath[0pt]{#1}{\left(}{\right)}{#2}}

\newcommand{\hText}[2][-4pt]{\hMath[#1]{l}{\left.}{\right.}{#2}}
\newcommand{\Par}[2][1pt]{{\text{\forbra\envFontB(\hspace{#1}}}#2{\text{\hspace{#1}\forbra\envFontB)}}}
\newcommandx{\Mid}[3][1=1pt,2=\envFontA]{{\text{#2\mid}}\hspace{#1}{#3}\hspace{#1}{\text{#2\mid}}}
\newcommand{\cMid}[2][1pt]{\Mid[#1][\envFontLarge\envFontA]{#2}}
\newcommand{\aMid}[2][1pt]{\Mid[#1][\envFontLarge\envFontA]{#2}}
\newcommand{\aXMid}[2][1pt]{\Mid[#1][\envFontHuge\envFont]{#2}}
\newcommand{\Bra}[2][\envFont]{{\text{\forbra{#1\{}}}#2{\text{\forbra{#1\}}}}}
\newcommand{\SmallPar}[2][0pt]{{\envFontSmall[\scriptsize]\Par[#1]{#2}}}
\newcommand{\TinyPar}[2][0pt]{(\hspace{#1}#2\hspace{#1})}
\newcommand{\BigPar}[1]{{\text{\envFont\forbra(\hspace{1pt}}}#1{\text{\hspace{1pt}\envFont\forbra)}}}
\newcommand{\XPar}[1]{{\text{\envFont\forbrax(\hspace{1pt}}}#1{\text{\hspace{1pt}\envFont\forbrax)}}}
\newcommand{\Sbra}[2][1pt]{{\text{\envFont\forbra[\hspace{#1}}}#2{\text{\envFont\forbra\hspace{#1}]}}}
\newcommand{\XSbra}[2][0pt]{{\text{\envFont\forbrax[\hspace{#1}}}#2{\text{\envFont\forbrax\hspace{#1}]}}}
\newcommandx{\zeroSubs}[2][1={},2=\Bra]{{#1#2{0}}}
\def\emptySet{\varnothing}

\newcommand{\Interval}[3]{{\text{\envFont\forbra#1\hspace{1pt}}}#3{\text{\envFont\forbra\hspace{1pt}#2}}}
\newcommand{\XInterval}[3]{{\text{\envFont\forbrax#1}}#3{\text{\envFont\forbrax#2}}}

\newcommand{\ExampleX}[2][\Solution]{
	{\BulletPoint} {\Example}\,\,\,{#2}\envFontDefault\vspace{-16pt}\par
	#1
}
\newcommand{\ProblemN}[3][\Solution]{
	{\Onumber{#2}} {#3}\envFontDefault\vspace{-16pt}\par
	#1
}
\newcommandx{\ProblemNor}[5][1=\Solution,4=\Sbra]{
	{\Onumber{#2}} \;{\dbsp#4{\OR({\normalsize#3})}}{#5}\envFontDefault\vspace{-16pt}\par
	#1
}
\newcommandx{\ProblemNnoor}[5][1=\Solution,4=\Sbra]{
	{\Onumber{#2}} \;{\dbsp#4{\small#3}}{#5}\envFontDefault\vspace{-16pt}\par
	#1
}
\newcommand{\ProblemB}[2][\Solution]{
	{\BulletPoint}{#2}\envFontDefault\vspace{-16pt}\par
	#1
}
\newcommand{\ProblemBc}[2][\Solution]{
	{$\circ$}{#2}\envFontDefault\vspace{-16pt}\par
	#1
}
\newcommandx{\ProblemBnoor}[4][1=\Solution,3=\Par]{
	{\BulletPoint} \dbsp#3{{\small#2}}{#4}\envFontDefault\vspace{-16pt}\par
	#1
}
\newcommand{\ProblemBor}[3][\Solution]{
	{\BulletPoint} {\dbsp\OR({\normalsize#2})}{#3}\envFontDefault\vspace{-16pt}\par
	#1
}
\newcommand{\AlignEq}[2]{
	\vspace{-25pt}
	\begin{align*}
		#1#2
	\end{align*}
	\vspace{-25pt}
}
%}

%{ 关于常用标识的便捷指令

\newcommand{\Frac}[2]{{\displaystyle{\frac{#1}{#2}}}}
\def\notRightarrow{\Rightarrow\!\!\!\!\!\!\!/\,\,\,\,}%\hspace{-13.5pt}/\hspace{7.5pt}}
\def\notLeftarrow{\Leftarrow\hspace{-13.5pt}/\hspace{7.5pt}}
\def\notLongrightarrow{\Longrightarrow\!\!\!\!\!\!\!\!/\,\,\,\,\,}
\def\notLongleftarrow{\Longleftarrow\!\!\!\!\!\!\!\!/\,\,\,\,\,}
\def\notLongleftrightarrow{\Longleftrightarrow\!\!\!\!\!\!\!\!/\,\,\,\,\,}

\def\Nbb{{\mathbb{N}}}
\def\Zbb{{\mathbb{Z}}}
\def\Qbb{{\mathbb{Q}}}
\def\Fbb{{\mathbb{F}}}
\def\Rbb{{\mathbb{R}}}
\def\Cbb{{\mathbb{C}}}
\def\Nbp{{\mathbb{N}^+}}

\newcommand{\Backslash}[1][\big]{#1\backslash}
\def\nonzeroFbb{\Fbb${\envFontA$\backslash$}${\def\envFont{\envFontB}\zeroSubs}}
\newcommandx{\nonzero}[2][1=\envFontA,2=\envFontB]{${#1$\backslash$}${\Bra[#2]{0}}}
\def\mmid{\text{\envFontA|}}
\def\d{{\textup{\tgnr d}}}
\def\i{{\textup{\tgnr i}}}

\def\Dim{{\textup{\tgnr dim}}} % For dim(...)
\def\Deg{{\textup{\tgnr deg}}} % For deg(...)
\def\range{{\textup{\tgnr range}}\,}
\def\null{{\textup{\tgnr null}}\,}
\def\card{{\textup{\tgnr card}}\,}
\def\Spn{{\textup{\tgnr span}}\,}
\def\Real{{\textup{\tgnr Re}}\,} % For  Re x
\def\Imaginary{{\textup{\tgnr Im}}\,} % For  Im x
\def\REAL{\textup{\tgnr Re}} % For  Re(...)
\def\IMAGINARY{\textup{\tgnr Im}} % For Im(...)
\newcommand{\Span}[2][\Par]{{\textup{\tgnr span}}#1{#2}}
\newcommand{\Lm}[2][\Par]{\mathcal{L}#1{#2}}
\newcommandx{\LmQxx}[4][2=U,3=X,4=\XSlash]{#1#4{^{#2}_{\!#3}}}
\newcommand{\Mt}[2][\Par]{\mathcal{M}#1{#2}}
\def\Mneg{\mathcal{M}^{-1}}
\def\Po{{\mathcal{P}}}
\newcommand{\PoF}[2][\Par]{\Po_{\!\!{#2}}#1{\Fbb}}
\newcommand{\PoR}[2][\Par]{\Po_{\!\!{#2}}#1{\Rbb}}
\newcommand{\PoC}[2][\Par]{\Po_{\!\!{#2}}#1{\Cbb}}
\def\PoFi{\PoF{\,}}
\def\PoRi{\PoR{\,}}
\def\PoCi{\PoC{\,}}
\def\mathC{C}
\def\apostrophe{\prime}
\def\upapostrophe{\!\!\!\apostrophe\,}
\def\BulletPoint{{\small\bullet}}
\def\BulletPointX{\BulletPoint \,\hspace{1pt}}
\def\bullpt{{\tiny\bullet}}

\def\OR{{\large O{\footnotesize R} }}
\def\Or{{\large O{\footnotesize R.} }}
\def\Solution{{\tgbfx\large S\footnotesize{OLUS:}}\,\,\,}
\def\NOTE{\tgnr\large N{\footnotesize OTE}}
\def\NOTEFOR{{\tgnr\large N{\footnotesize OTE} F{\footnotesize OR}}}
\def\NEWTHEOREM{{\tgnr\large N{\footnotesize EW} T{\footnotesize HEO}}}
\def\NOTICE{{\tgnr\large N{\footnotesize OTICE}\;}}
\def\COMMENT{{\tgnr\large C{\footnotesize OMMENT}\;}}
\def\COROLLARY{{\tgnr\large C{\footnotesize ORO}}}
\def\TIPS{{\tgnr\large T{\footnotesize IPS}}}
\newcommand{\TIPSN}[1]{{\tgnr\large T{\footnotesize IPS {\large(#1)}}}}
\def\Tips{{\tgbfx\large T{\footnotesize IPS}:}}
\newcommand{\TipsN}[1]{{\tgbfx\large T{\footnotesize IPS {\large#1\,}}:}}
\def\IndentTips{{\Blind{\Tips \,\,\,}}}
\newcommand{\IndentTipsN}[1]{{\Blind{\TipsN{#1}\,\,\,}}}
\def\IndentComment{{\Blind{\Comment \,\,\,}}}
\def\IndentCorollary{{\Blind{\Corollary \,\,\,}}}
\def\IndentNote{{\Blind{\Note \,\,\,}}}
\def\IndentSolution{{\Blind{\Solution}}}
\newcommand{\parSol}[1]{\par#1\IndentSolution{}}
\def\parCom{\par\IndentComment{}}
\def\parCor{\par\IndentCorollary{}}
\def\parNot{\par\IndentNote{}}
\def\parExa{\par\Blind{\AExa}{}}
\def\Existns{\Sbra[2pt]{{\tgsl Existns}}}
\def\Uniqnes{\Sbra[2pt]{{\tgsl Uniqnes}}}

\newcommand{\Onumber}[1]{\Largebfxx{#1}\hspace{-2pt}}
\newcommand{\NoteFor}[1]{\Largebfx{N{\small OTE} F{\small OR} #1:}}
\newcommand{\NoteForSmall}[1]{{\tgbfx\large N{\footnotesize OTE} F{\footnotesize OR} #1:}}
\newcommand{\NewNotation}{{\tgbfx\large N{\footnotesize EW} N{\footnotesize OTA}:}}
\newcommand{\NewTheorem}{{\tgbfx\large N{\footnotesize EW} T{\footnotesize HEO}:}}
\newcommand{\Comment}{{\tgbfx\large C\footnotesize{OMMENT:}}}
\newcommand{\Example}{{\tgbfx\large E\footnotesize{XA:}}}
\newcommand{\Exercise}[1]{{\tgbfx\large E{\footnotesize XE} {#1}\hspace{2pt}:}}
\newcommand{\Corollary}{{\tgbfx\large C{\footnotesize ORO}:}}
\newcommand{\Note}{{\tgbfx\large N{\footnotesize OTE}:}}
\newcommand{\AComm}{\Comment \,\,\,}
\newcommand{\ACoro}{\Corollary \,\,\,}
\newcommand{\AExa}{\Example \,\,\,}
\newcommand{\ANote}{\Note \,\,\,}


\def\ChEnd{\rightline{\Largebfx{E{\small NDED}}}\par\vspace{6pt}}
\newcommand{\PfEnd}[1][-18pt]{{\large\vspace{#1}\par\hfill$\square$\par}}
\newcommandx{\SepLine}[2][1=0pt,2={}]{{\vspace{-5pt}\par
		#2\tiny \_\,\_\,\_\,\_\,\_\,\_\,\_\,\_\,\_\,\_\,\_\,\_\,\_\,\_\,\_\,\_\,\_\,\_\,\_\,\_\,\_\,\_\,\_\,\_\,\_\,\_\,\_\,\_\,\_\,\_\,\_\,\_\,\_\,\_\,\_\,\_\,\_\,\_\,\_\,\_\,\_\,\_\,\_\,\_\,\_\,\_\,\_\,\_\,\_\,\_\,\_\,\_\,\_\,\_\,\_\,\_\,\_\,\_\,\_\,\_\,\_\,\_\,\_\,\_\,\_\,\_\,\_\,\_\,\_\,\_\,\_\_\,\_\,\_\,\_\,\_\,\_\,\_\,\_\,\_\,\_\,\_\,\_\,\_\,\_\,\_\,\_\,\_\,\_\,\_\,\_\,\_\,\_\,\_\,\_\,\_\,\_\,\_\,\_\,\_\,\_\,\_\,\_\,\_\,\_\,\_\,\_\,\_\,\_\,\_\,\_\,\_\,\_\,\_\,\_\,\_\,\_\,\_\,\_\,\_\,\_\,\_\,\_\,\_\,\_\,\_\,\_\,\_\,\_\,\_\,\_\,\_\,\_\,\_\,\_\,\_\,\_\,\_\,\_\,\_\,\_\,\_}\vspace{5pt}\vspace{#1}\par}
\newcommand{\ChDecl}[3]{{\huge\tgbfxx\hypertarget{#1}{#2}}{#3}{\vspace{5pt}}}
\newcommand\hLk[2]{\hyperlink{#1}{#2}}
\newcommand\Lch[2]{\hLk{Ch#1}{#2}}
\newcommand\TXT[1]{\textup{#1}}
\newcommand{\Blind}[1]{\textcolor{gray!2!yellow!1}{#1}}
%}

%\orMode[setup]{pdf version}[space]{print version}

%{ 关于调试、版本切换
\def\dbsp{\bullet}
\def\dbsp{} % 方便调试段落缩进
\newcommandx{\orMode}[5][1=\hMath{l}{}{},2=\hMath{l}{}{},4=-18pt]{{\normalsize
% If you want the debug version.
	%{\qquad$#1{$#3$}$}\vspace{#4}\par{\qquad$#2{$#5$}$}
% If you want the pdf version.
	{\qquad$#1{$#3$}$}
% If you want the print version.
	%{\qquad$#2{$#5$}$}
}}
%}

