% Copyright (C) 2024 Songbingzhi628. This work is licensed under Creative Commons Attribution-NonCommercial-ShareAlike 4.0 International License.
% Email: 13012057210@163.com

\ChDecl{Ch6A}{6.A}{}

\vspace{4pt}

\ProblemB[]{
	(a) \,$\Dvertsq{u+v}=\Dvertsq{u}+\Dvertsq{v}+2\Real\Ang{u,v}.$ \; $\Dvertsq{u+\i\,v}=\Dvertsq{u}+\Dvertsq{v}+2\Imaginary\Ang{u,v}.$\TextB{}
	(b) \,$\aXMid{\Dvert{u}-\Dvert{v}}\leqslant\Dvert{u-v}.$ \;Equa $\Longleftrightarrow u=cv,\;c>0.$ \,Where $u,v\neq0.$\TextB{}
	(c) \,$\aXMid{\Dvert{v}-1}=\BigDvert{v-v\big/\Dvert{v}}\leqslant\Dvert{v-u}$ \,if\, $\Dvert{u}=1.$ \;Equa $\Longleftrightarrow u=v\big/\Dvert{v}.$\TextB{}
	(d) \,$\aXMid{\Dvertsq{u}-\Dvertsq{v}}=\innerA{u+v,u-v}\leqslant\Dvert{u+v}\:\Dvert{u-v}\leqslant\Dvertsq{u}+\Dvertsq{v}={}${\Large$\frac{\:1\:}{2}$}$\BigSbra{\Dvertsq{u+v}+\Dvertsq{u-v}}.$\vspace{-3pt}\TextB{}
}\SepLine

\Anchor{6A21}\ProblemN{21}{
	\TextA{Implement the corres inner prod from a norm $\Dvert{{\cdot}}:U\rightarrow\Interval{[}{)}{0,\infty}$ satisfying [6.22].\vspace{2pt}}
%	\PrePa {\,}$\Dvert{u}=0\Longleftrightarrow u=0.$ \;(b) \,$\Dvert{au}=\aMid{a}\Dvert{u}.$ \;(c) \,\uline{$\Dvert{u+w}\leqslant\Dvert{u}+\Dvert{w}.$}\TextA{}
}If $\Fbb=\Rbb.$ Define $\Ang{u,v}={}${\Large$\frac{\:1\:}{4}$}$\BigPar{\Dvertsq{u+v}-\Dvertsq{u-v}}=\Ang{v,u}.$ \;Before we start:\def\Lbreak{1pt}\def\LBREAK{6pt}\def\Endl{\vspace{\Lbreak}\parSol{}}\def\ENDL{\vspace{\LBREAK}\parSol{}}\vspace{2pt}\parSol{}
\LX{1} $\Ang{u,v}=-\Ang{{-u,v}}=-\Ang{u,-v}.$\Endl{}
\LX{2} $\Ang{u+v,v}={}${\Large$\frac{\:1\:}{4}$}$\BigSbra{\Dvertsq{u+v+v}+\Dvertsq{{-u+v+v}}-\Dvertsq{{-u+v+v}}-\Dvertsq{u+v-v}}$\Endl{}
\Blind{\LX{2}} $\Blind{\Ang{u+v,v}}={}${\Large$\frac{\:1\:}{2}$}$\BigSbra{\BigPar{\Dvertsq{u}+\Dvertsq{2v}}-\BigPar{\Dvertsq{{-u+v}}+\Dvertsq{v}}}$\Endl{}
\Blind{\LX{2}} $\Blind{\Ang{u+v,v}}={}${\Large$\frac{\:1\:}{4}$}$\BigSbra{2\Dvertsq{u}+2\Dvertsq{v}+4\,\Dvertsq{v}-2\Dvertsq{u-v}}={}${\Large$\frac{\:1\:}{4}$}$\BigPar{\Dvertsq{u+v}-\Dvertsq{u-v}}+\Dvertsq{v}.$\ENDL{}
\LX{3} $\Ang{u,\,2v}={}${\Large$\frac{\:1\:}{4}$}$\BigSbra{\Dvertsq{u+v+v}-\Dvertsq{u-v-v}}$\Endl{}
\Blind{\LX{3}} $\Blind{\Ang{u,\,2v}}={}${\Large$\frac{\:1\:}{4}$}$\BigSbra{\Dvertsq{u+v+v}+\Dvertsq{u+v-v}-\Dvertsq{u+v-v}-\Dvertsq{u-v-v}}$\Endl{}
\Blind{\LX{3}} $\Blind{\Ang{u,\,2v}}={}${\Large$\frac{\:1\:}{2}$}$\BigSbra{\BigPar{\Dvertsq{u+v}+\Dvertsq{v}}-\BigPar{\Dvertsq{u-v}+\Dvertsq{v}}}=2\,\Ang{u,v}.$\ENDL{}
{\tgbf Add:} \,$\Ang{u+w,v}=\Ang{u,v}+\Ang{w,v}.$\Endl{}
We show \:$\Dvertsq{u+w+v}-\Dvertsq{u+w-v}=\Dvertsq{u+v}+\Dvertsq{w+v}-\Dvertsq{u-v}-\Dvertsq{w-v}.$\Endl{}
$RHS={}${\Large$\frac{\:1\:}{2}$}$\BigPar{\Dvertsq{u+w+2v}+\Dvertsq{u-w}}-{}${\Large$\frac{\:1\:}{2}$}$\BigPar{\Dvertsq{u+w-2v}+\Dvertsq{u-w}}=2\,\Ang{u+w,2v}=LHS.$\ENDL{}
%$\Blind{RHS}={}${\Large$\frac{\:1\:}{2}$}$\BigPar{\Dvertsq{u+w+v+v}-\Dvertsq{u+w+v-v}+\Dvertsq{u+w-v+v}-\Dvertsq{u+w-v-v}}$\Endl{}
%$\Blind{RHS}=2\,\Ang{u+w+v,v}-2\,\Ang{u+w-v,-v}$\Endl{}
%$\Blind{RHS}=2\BigPar{\Ang{u+w,v}+\Ang{v,v}}-2\BigPar{\Ang{u+w,-v}+\Ang{{-v,-v}}}=2\,\Ang{u+w,v}+2\,\Ang{u+w,v}=LHS.$\ENDL{}
{\tgbf Homo:} \,$\Ang{\lambda u,v}=\lambda\Ang{u,v}.$ \;True by add if $\lambda\in\Nbb,$ and then by (1) if $\lambda\in\Zbb.$\Endl{}
Note that by add, $\;n\cdot\Ang{n^{-1}u,v}=\Ang{u,v}$ for $n\in\Nbp.$ Thus the case for $\lambda\in\Qbb^+$ holds, so for $\Qbb$.\Endl{}
We show the case for $\lambda\in\Rbb.$ \,By def, $\exists\,!\,\Par{a_n}{_{n=0}^\infty}\in\Qbb{^{\infty}}$ suth $\lim_{n\rightarrow\infty}a_n=\lambda.$\Endl{}
$4\,\lambda\:\!\Ang{u,v}=4\:\!\lim_{n\rightarrow\infty}a_n\:\!\Ang{u,v}=4\:\!\lim_{n\rightarrow\infty}\Ang{a_nu,v}=\lim_{n\rightarrow\infty}\!\BigSbra{\Dvertsq{a_nu+v}-\Dvertsq{a_nu-v}}.$\Endl{}
To show $\,\lim_{n\rightarrow\infty}\Dvert{a_nu+v}=\Dvert{\lambda u+v},$ so then $\lambda\:\!\Ang{u,v}=\Ang{\lambda u,v}.$\Endl{}
\NOTICE that $\Dvert{u\pm v}\leqslant\Dvert{u}+\Dvert{v}\Longrightarrow\aXMid{\Dvert{u}-\Dvert{v}}\leqslant\Dvert{u\pm v}.$\Endl{}
Thus $\aXMid{{\lim_{n\rightarrow\infty}\Dvert{a_nu+v}-\Dvert{\lambda u+v}}}\leqslant\BigDvert{{\lim_{n\rightarrow\infty}a_nv-\lambda v}}=0.$\vspace{6pt}\parSol{}
If $\Fbb=\Cbb.$ Define $\Ang{u,v}=R\Par{u,v}+\i\,I\Par{u,v}.$\Endl{}
Where $R\Par{u,v}={}${\Large$\frac{\:1\:}{4}$}$\BigPar{\Dvertsq{u+v}-\Dvertsq{u-v}}$ \,and $I\Par{u,v}=R\Par{u,\:\!\i\,v}={}${\Large$\frac{\:1\:}{4}$}$\BigPar{\Dvertsq{u+\i\,v}-\Dvertsq{u-\i\,v}}.$\vspace{3pt}\parSol{}
{\tgbf Conjug Symm:} \,$\Ang{u,v}=R\Par{u,v}+\i\,I\Par{u,v}=R\Par{v,u}-\i\,I\Par{v,u}=\overline{\Ang{v,u}}$\Endl{}
Note that $R\Par{u,v}=R\Par{v,u}$ and $R\Par{v,\:\!\i\,u}=R\Par{\i\,u,\:\!v}.$ Thus we show $-I\Par{u,v}=I\Par{v,u}.$\Endl{}
Which equiv $\Dvertsq{u-\i\,v}-\Dvertsq{u+\i\,v}=\BigDvertsq{\i\;\!\Par{{-\i\,u-v}}}-\BigDvertsq{\i\;\!\Par{{-\i\,u+v}}}=\Dvertsq{\i\,u+v}-\Dvertsq{\i\,u-v}.$\vspace{2pt}\Endl{}
{\tgbf Homo:} \,$\Ang{\lambda u,v}=\lambda\Ang{u,v}.$ \;True if $\lambda\in\Rbb.$ \,We show the case for $\lambda=\i.$\Endl{}
$\Ang{\i\,u,\:\!v}={}${\Large$\frac{\:1\:}{4}$}$\BigSbra{\Dvertsq{\i\,u+v}-\Dvertsq{\i\,u-v}+\i\,\BigPar{\Dvertsq{\i\,u+\i\,v}-\Dvertsq{\i\,u-\i\,v}}}$\Endl{}
$\Blind{\Ang{\i\,u,\:\!v}}={}${\Large$\frac{\:1\:}{4}$}$\BigSbra{\Dvertsq{u-\i\,v}-\Dvertsq{u+\i\,v}+\i\,\BigPar{\Dvertsq{u+v}-\Dvertsq{u-v}}}$\Endl{}
$\Blind{\Ang{\i\,u,\:\!v}}=\i{}${\Large$\frac{\:1\:}{4}$}$\BigSbra{{-\i}\,\Dvertsq{u-\i\,v}+\i\,\Dvertsq{u+\i\,v}+\BigPar{\Dvertsq{u+v}-\Dvertsq{u-v}}}{{}=\i\,\Ang{u,v}}$\PfEnd
\SepLine

\Anchor{6A3}\ProblemN{3}{
	\TextA{Supp $\Fbb=\Rbb,V\neq\zeroSubs.$ Replace the positivity cond in [6.3] with $\exists\,v\in V,\,\Ang{v,v}>0.$}
	\TextA{Show this does not change the inner prods from $V\times V$ to $\Rbb.$}
}Supp $w\in V$ with $\Ang{w,w}>0.$ Asum $\exists\,u\in V$ with $\Ang{u,u}<0.$\parSol{}
Define $p\Par{x}=\Ang{u+xw,\,u+xw}=\Ang{w,w}\,x^2+2\Ang{u,w}\,x+\Ang{u,u}\Rightarrow$ two disti zeros.\parSol{}
Supp $\Ang{u+\lambda w,\,u+\lambda w}=0\Rightarrow u+\lambda w=0\Rightarrow\Ang{u,u}=\lambda^2\Ang{{-w,-w}}\geqslant 0,$ ctradic.\PfEnd
\SepLine

\Anchor{6A6}\ProblemN{6}{
	\TextA{Supp $u,v\in V.$ Prove $\Dvert{u}\leqslant\Dvert{u+av}$ for all $a\in\Fbb\Rightarrow\Ang{u,v}=0.$}
}Becs $\Dvertsq{u}\leqslant\Dvertsq{u+av}.$ Let $\Ang{u-cv,\,cv}=0\Rightarrow\Dvertsq{u-cv}=\Ang{u,\,u-cv}=\Dvertsq{u}-\overline{c}\Ang{u,v}.$\parSol{}
Thus $\Dvertsq{u}\leqslant\Dvertsq{u-cv}=\Dvertsq{u}-\innerAsq{u,v}{\;\!\XSlash\;\!}\Dvertsq{v}.$\PfEnd\vspace{4pt}\parSol{}
\Or $\Dvertsq{u}\leqslant\Dvertsq{u}+\aMidsq{a}\,\Dvertsq{v}+2\REAL\:\overline{a}\:\!\Ang{u,v}\Rightarrow-2\REAL\:\overline{a}\:\!\Ang{u,v}\leqslant\aMidsq{a}\,\Dvertsq{v}$ for all $a\in\Fbb.$\parSol{}
Let $a=-\Ang{u,v}\Rightarrow2\:\!\innerA{u,v}{^2}\leqslant\innerA{u,v}{^2}\,\Dvertsq{v}.$ If $\Ang{u,v}\neq0,$ then $2\leqslant\Dvertsq{v};$ might not be true.\PfEnd
\SepLine

\Anchor{6AT1}\ProblemBX{\TipsN{1}}{
	\TextB{Supp $u,v\in V,$ $\Dvertsq{xu+yv}=\aMidsq{x}\:\!\Dvertsq{u}+\aMidsq{y}\:\!\Dvertsq{v}$ for $x,y\in\Fbb.$ Prove $\Ang{u,v}=0.$}
}Becs $\REAL\BigPar{x\overline{y}\,\Ang{u,v}}=0.$ Take $\Par{x,y}=\Par{1,1}$ and $\Par{\i,1}.$\quad\Or By Exe (6), immed.\PfEnd
\SepLine

%\Anchor{6A4e19}\ProblemBnoor{4E 19}{
%	\TextA{Supp $T\in\Lm{V},\,\lambda$ is eigval. Prove $\aMidsq{\lambda}\leqslant\sum_{j=1}^n\sum_{k=1}^n\aMidsq{A_{j,\;\!k}},$ \FontNorm\tgnr where $A=\Mt[\BigPar]{T,\Par{v_1,\dots,v_n}}.$\vspace{3pt}}
%}Let $\Ang{a_1v_1+\dots+a_nv_n,b_1v_1+\dots+b_nv_n}=a_1\overline{b_1}+\dots+a_n\overline{b_n}.$ Let $v$ be eigval corres $\lambda$ with $\Dvert{v}=1.$\vspace{1pt}\parSol{}
%Becs $v=a_1v_1+\dots+a_nv_n\Rightarrow Tv=\sum_{k=1}^na_k\sum_{j=1}^nA_{j,\;\!k}v_j=\sum_{j=1}^n\BigSbra{\sum_{k=1}^na_kA_{j,\;\!k}}v_j.$\vspace{3pt}\parSol{}
%\又 Each $\Dvert{v_j}=1,\:\Ang{v_j,v_k}=0\Rightarrow\Dvertsq{Tv}=\sum_{j=1}^n\Big|\sum_{k=1}^na_kA_{j,\;\!k}\Big|{^2}=\aMidsq{\lambda}.$\vspace{3pt}\parSol{}
%Note that $\aXMidsq{a_1A_{j,1}+\dots+a_nA_{j,\;\!n}}\leqslant\BigDvertsq{\Par{a_1,\dots,a_n}}\cdot\BigDvertsq{\Par{A_{j,1},\dots,A_{j,\;\!n}}}=\sum_{k=1}^n\aMidsq{A_{j,\;\!k}}.$\PfEnd
%\SepLine

\Anchor{6AT2}\ProblemBX{\TipsN{2}}{
	\TextB{Supp $A\in\FbbP{m,\;\!n}.$ Prove $\Dvertsq{Ax}\leqslant\sum_{j=1}^m\sum_{k=1}^n\aMidsq{A_{j,\;\!k}}\cdot\Dvertsq{x}$ for all $x\in\FbbP{m,1}.$\vspace{2pt}}
}$\Dvertsq{Ax}=\Dvertsq{A_{\cdot,1}x_1+\dots+A_{\cdot,n}x_n}=\sum_{j=1}^m\aMidsq{x_1A_{j,1}+\dots+x_nA_{j,\;\!n}}\leqslant\sum_{j=1}^m\Dvertsq{A_{j,\cdot}}\cdot\Dvertsq{x}.$\PfEnd
\SepLine

\Anchor{6A4e23}\ProblemBnoor{4E 23}{
	\TextA{Supp $v_1,\dots,v_m\in V,$ each $\Dvert{v_k}\leqslant 1.$ \,Show $\exists\,a_k=\pm1,\;\Dvert{a_1v_1+\dots+a_mv_m}\leqslant\sqrt{m}.$}
}We use induc on $m.$ (i) $m=1.$ Immed. (ii) $m>1.$ Asum it holds for smaller $m.$\parSol{}
Let $u=a_1v_1,\,w=a_2v_2+\dots+a_mv_m\Rightarrow\Dvertsq{u}\leqslant1,\Dvertsq{w}\leqslant{m-1}.$\parSol{}
Then $\Dvertsq{u+w}+\Dvertsq{u-w}\leqslant2m.$ \;\;\Or $\Dvert{u+w}\cdot\Dvert{u-w}\leqslant m.$\PfEnd
\SepLine

\Anchor{6A'1}\ProblemB{
	\TextB{Supp $u,v_1,\dots,v_n$ are non0 in $V$ suth each $\Ang{v_i,u}>0$ and $\Ang{v_i,v_j}\leqslant0$ \,for $i\neq j.$}
	\TextB{Show $\Par{v_1,\dots,v_n}$ liney indep.}
}(i) Asum $v_1=cv_2.$ Then $\Ang{cv_2,u}>0\Rightarrow c>0,$ while $\Ang{v_1,v_1}=c\Ang{v_2,v_1}\geqslant0\Rightarrow c\leqslant0.$ ctradic.\parSol{}
(ii) Asum $\Par{v_1,\dots,v_{n-1}}$ liney indep. Asum $v_n=c_1v_1+\dots+c_{n-1}v_{n-1}.$\parSol{\Hii}
Then $\Ang{v_n,u}=c_1\Ang{v_1,u}+\dots+c_{n-1}\Ang{v_{n-1},u}>0.$ \;Thus we can choose all $c_k\in\Rbb.$\parSol{\Hii}
Write $c_1v_1+\dots+c_nv_n=0,\:c_n=-1.$ \;Let $P=\Bra{i:c_i\geqslant0},\,N=\Bra{i:c_i<0}.$\vspace{2pt}\parSol{\Hii}
Then $\sum_{j\,\in\,P}c_jv_j=\sum_{k\,\in\,N}-c_kv_k\Longrightarrow 0\leqslant\BigAng{{\sum_{j\,\in\,P}c_jv_j,\:\sum_{k\,\in\,N}-c_kv_k}}=\sum-c_j\:\!c_k\!\:\Ang{v_j,v_k}\leqslant0.$\vspace{3pt}\parSol{\Hii}
While $\BigAng{{\sum_{j\,\in\,P}c_jv_j,\,u}},\:\BigAng{{\sum_{k\,\in\,N}-c_kv_k,\,u}}\geqslant0,$ where the equas hold $\Longleftrightarrow$ all $c_i=0.$\PfEnd
\SepLine

\vfill\ChDecl{Ch6B}{6.B}{}

\vspace{4pt}

\Anchor{6B14}\ProblemN{14}{
	\TextA{Supp $\Par{e_1,\dots,e_m}$ orthon, each $v_j\in V$ suth $\Dvert{e_j-v_j}<{}${\Large$\frac{\;1}{\sqrt{m}\;}$}. Show $\Par{v_1,\dots,v_m}$ liney indep.}
}Let $a_1v_1+\dots+a_mv_m=0.$\vspace{3pt}\parSol{}
$\sum_{j=1}^m\aMidsq{a_j}=\BigDvertsq{{\sum_{j=1}^ma_j\Par{e_j-v_j}}}\leqslant\BigSbra{{\sum_{j=1}^m\aMid{a_j}}\cdot\Dvert{e_j-v_j}}{^2}\leqslant\BigDvertsq{\Par{\aMid{a_j}}{_{j=1}^m}}\cdot\BigDvertsq{\BigPar{\Dvert{e_j-v_j}}{_{j=1}^m}}.$\PfEnd\vspace{6pt}
%\AComm If each $\Dvert{e_j-v_j}=1\Big/\!\sqrt{m}$, then $\Par{v_1,\dots,v_m}$ might be liney dep even if $m\neq1.$\parCom
\AExa Let $v_j=e_j-\Par{e_1+\dots+e_m}\Big/m\Rightarrow\Dvert{e_j-v_j}{^2}=1\big/m.$ \;Note that $v_1+\dots+v_m=0.$\PfEnd
\SepLine

\ProblemB[]{
	For orthog non0 $\Par{e_1,\dots,e_m},$ if $v=a_1e_1+\dots+a_me_m\Rightarrow\Ang{v,e_k}=a_k\Dvertsq{e_k},$ \;$v={}${\Large$\frac{\:\SmallAng{v,\:e_1}\:}{\;\SmallDvertsq{e_1}}$}$\,e_1+\dots+{}${\Large$\frac{\:\SmallAng{v,\:e_m}\:}{\;\SmallDvertsq{e_m}}$}$\,e_m.$\TextB{}
	Now $\Dvertsq{v}={}${\Large$\frac{\:\SmallinnerAsq{v,\:e_1}\:}{\SmallDvertsq{e_1}}$}${}+\dots+{}${\Large$\frac{\:\SmallinnerAsq{v,\:e_m}\:}{\SmallDvertsq{e_m}}$}$.$ Replace each $e_k$ with $\Dvert{e_k}{^{\,-1}}e_k,$ now $\Par{e_1,\dots,e_m}$ is a orthon list.
%	\TextB{\vspace{6pt}}
%	Exe (2) holds only for orthon lists.
	\TextB{\vspace{0pt}}
}\SepLine

\Anchor{6B'1}\ProblemB{
	\TextA{Supp $\Par{e_1,\dots,e_m}$ orthog, $v\in V.$ \,Show $\sum_{k=1}^m\!\BigPar{1-\Dvertsq{e_k}}\,\innerAsq{v,e_k}\leqslant\Dvertsq{v}-\sum_{k=1}^m\innerAsq{v,e_k}.$\vspace{4pt}}
}Let $u=\Ang{v,e_1}\,e_1+\dots+\Ang{v,e_m}\,e_m\Rightarrow\Dvertsq{u}=\sum_{k=1}^n\innerAsq[{\:\!\Dvert{e_k}\,}]{v,e_k},\,\Ang{u,v}=\sum_{k=1}^n\innerAsq{v,e_k}.$\vspace{3pt}\parSol{}
Let $\Dvertsq{v-u}=\Dvertsq{v}+\Dvertsq{u}-\Ang{v,u}-\Ang{u,v}=\Dvertsq{v}+\sum_{k=1}^n\!\BigPar{\Dvertsq{e_k}-2}\,\innerAsq{v,e_k}\geqslant0.$\PfEnd\vspace{4pt}
\Anchor{6B2}\ACoro If orthon, $\Ang{u,\,v-u}=0\Rightarrow\Dvertsq{v}=\Dvertsq{u}+\Dvertsq{v-u}.$\vspace{1pt}\parCor
Bessel's Inequa: $\sum_{k=1}^m\innerAsq{v,e_k}\leqslant\Dvertsq{v}.$ \:\Sbra{Exe (2)} Equa $\Longleftrightarrow v\in\Span{e_1,\dots,e_m}.$
\SepLine\pagebreak

\Anchor{6B4e9}\ProblemBnoor{4E 9}{
	\TextA{Supp $\Par{e_1,\dots,e_m}$ is the result of applying [6.31]\vspace{1pt}}
	\TextA{to a liney indep $\Par{v_1,\dots,v_m}$ in $V.$ \,Show each $\Ang{v_j,\:\!e_j}>0.$}
}Let $\;f_j=v_j-\Ang{v_j,\,e_1}\,e_1-\dots-\Ang{v_j,\,e_{j-1}}\,e_{j-1}.$\vspace{2pt}\parSol{}
Becs $\;\Dvert{\,f_j}\,\Ang{v_j,\,e_j}=\Ang{v_j,\,f_j}=\Dvertsq{v_j}-\innerAsq{v_j,\,e_1}-\dots-\innerAsq{v_j,\,e_{j-1}}\geqslant0,$ by Bessel's Inequa.\vspace{2pt}\parSol{}
If $\,\Ang{v_j,\,f_j}=0,$ then by Exe (2), $v_j\in\Span{e_1,\dots,e_{j-1}}=\Span{v_1,\dots,v_{j-1}}.$ 又 $\Dvert{\,f_j}\neq0.$\PfEnd\vspace{3pt}
\Anchor{6B9}\Anchor{6B4e13}\ANote Supp $\Par{v_1,\dots,v_m}$ liney dep. Let $j$ be the largest suth $\Par{v_1,\dots,v_{j-1}}$ liney indep.\parNot
Apply [6.31]. Now $v_j\in\Span{v_1,\dots,v_{j-1}}=\Span{e_1,\dots,e_{j-1}}\Rightarrow f_j=0.$
\SepLine

\Anchor{6BT}\BulletPointX\Tips \,\,\,Supp $\Par{v_1,\dots,v_m}$ liney indep in $V.$ Get the corres orthon $\Par{e_1,\dots,e_m}$ via [6.31].\TextB{}
{\IndentTips}Let $S=\Bra{\lambda\in\Fbb:\aMid{\lambda}=1}.$ \,Supp orthon $\Par{u_1,\dots,u_m}$ suth each $\Span{u_1,\dots,u_k}=\Span{v_1,\dots,v_k}.$\vspace{1pt}\TextB{}
{\IndentTips}We show it equals $\Par{c_1\:\!e_1,\dots,c_m\:\!e_m}$ for some $\Par{c_1,\dots,c_m}\in S^m$ by induc on $k.$\vspace{1pt}\TextB{}
{\IndentTips}(i) $k=1.$ $\Span{e_1}=\Span{u_1}\Rightarrow u_1=\Ang{u_1,\:\!e_1}\,e_1,$ 又 $\innerA{u_1,\:\!e_1}=1.$ Let $c_1=\Ang{u_1,\:\!e_1}.$\vspace{1pt}\TextB{}
{\IndentTips}(ii) $k>1.$ Asum each $\aMid{c_i}=1$ and $c_i\;\!e_i=u_i$ for $i\in\;\!\!\Bra{1,\dots,k-1}.$\vspace{1pt}\TextB{}
{\IndentTips\Hii}Becs $u_k=\Ang{u_k,\:\!e_1}\:\!e_1+\dots+\Ang{u_k,\:\!e_k}\:\!e_k.$ 又 $\Ang{u_i,u_k}=0=c_i\;\!\Ang{e_i,u_k}.$ \,Simlr, $c_k=\Ang{u_k,\:\!e_k}.$\PfEnd
\SepLine

\Anchor{6B4e10}\ProblemBnoor{4E 10}{
	\TextB{Supp $\Par{v_1,\dots,v_m}$ liney indep. Explain why the
	orthon list produced by [6.31]}
	\TextB{is the only orthon $\Par{e_1,\dots,e_m}$ suth each $\Ang{v_k,\:\!e_k}>0$ and $\Span{v_1,\dots,v_k}=\Span{e_1,\dots,e_k}.$}
}Supp such orthon $B_1=\Par{e_1,\dots,e_m}.$ Apply [6.31] to get $B_2=\Par{u_1,\dots,u_m}.$ We show $B_1=B_2.$\parSol{}
{\tgbf Step 1}\;\; Becs $e_1=c\:\!v_1\Rightarrow c=\Ang{v_1,\:\!e_1}>0,$ 又 $\aMid{c}=1\big/\Dvert{v_1}.$ Immed.\parSol{}
{\tgbf Step k}\;\: Let \,$v_k=a_1\:\!e_1+\dots+a_k\:\!e_k\Rightarrow$ each $a_j=\Ang{v_k,\:\!e_j}.$ Let \,$f_k=v_k-a_1\:\!e_1-\dots-a_{k-1}\:\!e_{k-1}.$\vspace{1pt}\parSol{}
\Blind{{\tgbf Step k}\;\: }Becs $e_k=\;\!f_k\Big/\;\!\!a_k\Rightarrow\Dvertsq{\,f_k}\!\Big/\aMidsq{a_k}=1\Rightarrow\aMid{a_k}=\Dvert{\,f_k}.$ 又 $a_k=\Ang{v_k,\:\!e_k}>0\Rightarrow a_k=\Dvert{\,f_k}.$\PfEnd
\SepLine

\Anchor{6B10}\ProblemN{10}{
	\TextA{Supp $\Fbb=\Rbb,$ $\Par{v_1,\dots,v_m}$ is liney indep in $V.$}
	\TextA{Prove $\exists$ exactly $2^m$ orthon lists spans $\Span{v_1,\dots,v_m}.$}
}Using induc on $m.$ (i) $m=1.$ Let $e_1=\pm v_1\big/\Dvert{v_1}.$ (ii) $m>1.$ Asum it holds for $\Par{m-1}.$\vspace{1pt}\parSol{}
Get $2^{m-1}$ orthon lists corres $\Par{v_1,\dots,v_{m-1}}.$ Fix one as $\Par{e_1,\dots,e_{m-1}}$ and apply [6.31] at step $m.$\vspace{2pt}\parSol{}
Supp $\Par{e_1,\dots,e_{m-1},e_m'}$ is also orthon. \NOTICE that $e_m'=\Ang{e_m',\,e_m}\,e_m.$ \;So $\innerA{e_m',\,e_m}=1.$\vspace{2pt}\parSol{}
Let $e_m'=-e_m.$ Sum it up, we have $2^{m-1}\times 2=2^m$ orthon lists. \qquad\Or By \TIPS, immed.\PfEnd
\SepLine

\Anchor{6B11}\ProblemN{11}{
	\TextA{Supp $V\neq0,$ and $\Ang{\cdot,\cdot}{_1},\Ang{\cdot,\cdot}{_2}$ are inner prods suth $\Ang{v,w}{_1}=0\Longleftrightarrow\Ang{v,w}{_2}=0.$\vspace{1pt}}
	\TextA{Prove $\exists\,c>0,\:\Ang{v,w}{_1}=c\:\!\Ang{v,w}{_2}$ for all $v,w\in V.$}
}Fix non0 $v_1,v_2\in V.$ Define $\varphi_1,\psi_1\in V\apostrophe$ by $\varphi_1:v\mapsto\Ang{v,v_1}{_1},\;\psi_1:v\mapsto\Ang{v,v_2}{_1}.$ Simlr for $\varphi_2,\psi_2.$\parSol{}
Becs $\Ang{v,v_1}{_1}=0\Longleftrightarrow\Ang{v_1,v}{_2}=0.$ By (3.B.30), let $c_1=\Ang{v_1,v_1}{_1}\Big/\Ang{v_1,v_1}{_2}>0\Rightarrow\varphi_1=c_1\varphi_2.$\parSol{}
Simlr, let $c_2=\Ang{v_2,v_2}{_1}\Big/\Ang{v_2,v_2}{_2}\Rightarrow\psi_1=c_2\psi_2.$ Choose $v_1=v_2$ so that $c=c_1=c_2.$\vspace{3pt}\parSol{}
For any $v_1'\in V,$ get $c_1'$ simlr. Becs $\Ang{v,v_1}{_1}=c_1\:\!\Ang{v,v_1}{_2}$ while $\Ang{v,v_1'}{_1}=c_1'\:\!\Ang{v,v_1'}{_2}.$\vspace{1pt}\parSol{}
Now $c_1\Ang{v_1',v_1}{_2}=\Ang{v_1',v_1}{_1}=\overline{\Ang{v_1,v_1'}{_1}}=\overline{c_1'\:\!\Ang{v_1,v_1'}{_2}}\Rightarrow c_1=c_1'.$ Simlr for $c_2=c_2'.$\PfEnd\vspace{2pt}\parSol{}
\Or For any $v_1',v_2'\in V,$ get $c_1'=c_2'$ simlr. Becs $c_2\Ang{v_1',v_2}{_2}=\Ang{v_1',v_2}{_1}=c_1'\Ang{v_1',v_2}{_2}.$\PfEnd\vspace{6pt}\parSol{}
\Or Define $c_v=\Ang{v,v}{_1}\Big/\Ang{v,v}{_2}$ for all non0 $v\in V.$ Fix non0 $u,v\in V.$\parSol{}
Let $c=\Ang{u,v}{_2}\Big/\Ang{v,v}{_2}\Rightarrow\Ang{u-cv,v}{_1}=\Ang{u-cv,v}{_2}=0\Rightarrow\Ang{u,v}{_1}=c\Ang{v,v}{_1}=c_v\Ang{u,v}{_2}.$\vspace{1pt}\parSol{}
Rev the roles of $u,v\Rightarrow c_v\Ang{u,v}{_2}={\Ang{u,v}{_1}}=\overline{\Ang{v,u}{_1}}=\overline{c_u\Ang{v,u}{_2}}=c_u\Ang{u,v}{_2}\Rightarrow c_v=c_u.$\PfEnd
\SepLine\pagebreak

\Anchor{6B12}\ProblemN{12}{
	\TextA{Supp $V$ is finide. Let $\Ang{\cdot,\cdot}{_1}$ and $\Ang{\cdot,\cdot}{_2}$ be inner prods with corres norms $\Dvert{{\cdot}}{_1}$ and $\Dvert{{\cdot}}{_2}.$\vspace{1pt}}
	\TextA{Prove $\exists\,c>0,\:\Dvert{v}{_1}\leqslant c\Dvert{v}{_2}$ \,for all $v\in V.$}
}Let $B_V=\Par{e_1,\dots,e_n}$ be orthon wrto $\Ang{\cdot,\cdot}{_2}.$ Supp $v=a_1e_1+\dots+a_ne_n.$\vspace{1pt}\parSol{}
\NOTICE that $\Dvert{v}{_1}\leqslant\Dvert{a_1e_1}{_1}+\dots+\Dvert{a_ne_n}{_1}\leqslant\max\!\Bra{\Dvert{e_1}{_1},\dots,\Dvert{e_n}{_1}}\cdot\BigPar{\aMid{a_1}+\dots+\aMid{a_n}}.$\vspace{3pt}\parSol{}
\又 $\aMid{a_1}+\dots+\aMid{a_n}\leqslant n\cdot\max\!\Bra{\aMid{a_1},\dots,\aMid{a_n}}\leqslant n\cdot\sqrt{\aMidsq{a_1}+\dots+\aMidsq{a_n}}\leqslant n\cdot\Dvert{v}{_2}.$\PfEnd
\SepLine

%\Anchor{6B13}\ProblemN{13}{
%	\TextA{Supp $\Par{v_1,\dots,v_m}$ liney indep in $V.$ Show $\exists\,w\in V$ suth each $\Ang{w,\:\!v_j}>0.$}
%}Using induc on $m.$ (i) $m=1.$ Let $w=v_1.$ (ii) $m>1.$ Asum it holds for $\Par{m-1}.$\parSol{}
%By asum, $\exists\,w'\in\Span{v_1,\dots,v_{m-1}}$ suth each $\Ang{w',\:\!v_k}>0$ for $k\in\;\!\!\Bra{1,\dots,m-1}.$\parSol{}
%Apply [6.31] to get the corres $\Par{e_1,\dots,e_m}.$ \;Let $w=w'+a\:\!e_m.$\parSol{}
%Becs each $\Ang{e_m,\:\!v_k}=0\Rightarrow\Ang{w,\:\!v_k}=\Ang{w',\:\!v_k}>0$ for $k\in\;\!\!\Bra{1,\dots,m-1}.$\parSol{}
%Note that $\Ang{e_m,\:\!v_m}\neq0.$ Hence $\exists\,a\in\Fbb,\:\Ang{w,\:\!v_m}=\Ang{w'+a\:\!e_m,\:\!v_m}=\Ang{w',\:\!v_m}+a\:\!\Ang{e_m,\:\!v_m}>0.$\PfEnd\vspace{4pt}\parSol{}
%\Or We show $\exists\,w\in V$ suth each $\Ang{w,\:\!v_j}=\Ang{v_j,\:\!w}=1.$ Let $U=\Span{v_1,\dots,v_m}.$\parSol{}
%Define $\varphi\in U\apostrophe$ by each $\varphi\Par{v_j}=1.$ \;Becs $\exists\,!\,w\in U,$ each $\varphi\Par{v_j}=\Ang{v_j,\:\!w}.$\PfEnd
%\SepLine

\Anchor{6B4e19}\ProblemBnoor{4E 19}{
	\TextA{Supp $B_V=\Par{v_1,\dots,v_n}.$ Prove $\exists\,B_V'=\Par{u_1,\dots,u_n}$ suth $\Ang{v_j,u_k}=\delta_{j,\;\!k}.$}
}Let $\Par{\varphi_1,\dots,\varphi_n}$ be the corres dual bss of $B_V.$ \;Becs $\exists\,!\,u_k\in V,\,\varphi_k\Par{v}=\Ang{v,u_k}$ for all $v\in V.$\parSol{}
Then $\varphi_k\Par{v_j}=\delta_{j,\;\!k}=\Ang{v_j,u_k}.$ \;Now let \,$a_1u_1+\dots+a_nu_n=0\Rightarrow$ each $\Ang{v_j,0}=a_j.$\PfEnd
\SepLine

\Anchor{6B16}\ProblemN{16}{
	\TextA{Supp $\Fbb=\Cbb,V$ finide, non0 $T\in\Lm{V},$ all eigvals have abs vals less than $1.$\vspace{1pt}}
	\TextA{Let $\epsilon>0.$ Prove $\exists\,m\in\Nbp,\:\Dvert{T^mv}\leqslant\epsilon\Dvert{v}$ \,{\tgsc for all v \ensuremath{\boldsymbol{\in}} V}.}\vspace{1pt}
}Let $\Ang{\cdot,\cdot}{_V}$ be the inner prod on $V,$ and $\Dvert{{\cdot}}{_V}$ be the corres norm on $V.$\parSol{}
Using Euclid inner prod $\Ang{\cdot,\cdot}$ and the corres norm $\Dvert{{\cdot}}$ on $\CbbP{n,1}$ id with $\CbbP{n}.$\parSol{}
Supp $A=\Mt{T}$ up-trig wrto orthon $B_V=\Par{e_1,\dots,e_n}.$\parSol{}
Then $\forall v=x_1e_1+\dots+x_ne_n\in V,\Dvert{v}{_V}=\Dvert{x}.$ Now we show $\Dvert{A^mx}\leqslant\epsilon\Dvert{x}$ for all $x\in\CbbP{n,1}.$\parSol{}
Define $D,N\in\CbbP{n,\;\!n}$ by $D_{j,\;\!k}=\delta_{j,\:\!k}A_{j,\;\!k},\:N=A-D.$ Then $N$ is nilp with $N^p=0\neq N^{p-1}.$\parSol{}
Let $\rho=\max\!\Bra{\aMid{D_{1,1}},\dots,\aMid{D_{n,n}}}\Rightarrow0\leqslant\rho<1,$ and each $\Dvert{D^{\,k}x}\leqslant\rho^{\,k}\Dvert{x}\leqslant\Dvert{x}.$\vspace{3pt}\parSol{}
Let $M=\sum_{j=1}^n\sum_{k=1}^n\aMidsq{N_{j,\;\!k}}.$ \,{By \Sbra{6.A \TIPSN{2}}}, $\Dvert{Nx}\leqslant M\Dvert{x}\Rightarrow\Dvert{N^{k}x}\leqslant M^k\Dvert{x}.$\vspace{5pt}\parSol{}
Hence $\Dvert{A^{p+q}x}=\BigDvert{b_0D^{\,p+q}x+\dots+b_kD^{\,p+q-k}N^{k}x+\dots+b_{p-1}D^{\,q+1}N^{p-1}x}$\vspace{3pt}\parSol{}
\Blind{So that $\Dvert{A^{p+q}x}$}$\leqslant\!\BigSbra{{b_0\rho^{\,p+q}+\dots+b_k\rho^{\,p+q-k}M^k+b_{p-1}\rho^{\,q+1}M^{p-1}}}\Dvert{x}.$\vspace{4pt}\parSol{}
Where each $b_j=\mathC_{p+q}^j\leqslant\Par{p+q}{^j}\leqslant\Par{p+q}{^{p-1}}$ for $j\in\;\!\!\Bra{0,\dots,p-1}.$\vspace{1pt}\parSol{}
Let $\sigma=\max\!\Bra{1,M,\dots,M^{p-1}}.$ 又 $\max\!\Bra{\rho^{\,p+q},\dots,\rho^{\,q+1}}=\rho^{\,q+1}.$\parSol{}
Now $\Dvert{A^{p+q}x}\leqslant\Par{p+q}{^{p-1}}\rho^{\,q+1}\sigma\Dvert{x}.$ \;Note that as $q\rightarrow\infty,$ \,$\Par{p+q}{^{p-1}}\rho^{\,q+1}\rightarrow0.$\PfEnd
\SepLine
\ChEnd

\ChDecl{Ch6C}{6.C}{}

\vspace{4pt}

\Anchor{6CT1}\ProblemBX{\TipsN{1}}{
	\TextA{Supp $V$ finide, $T\in\Lm{V},$ and all vecs in $\null T$ orthog to all vecs in $\range T.$}
	\TextA{Prove $\Par{\null T}{^\perp}=\range T.$}
}Becs $\range T\subseteq\Par{\null T}{^\perp}.$ 又 $\null T\cap\range T=\zeroSubs.$ \,By \Sbra{1.C \TIPSN{1}}.\PfEnd\parSol{}
\Or $\forall v\in\Par{\null T}{^\perp},\exists\,!\,\Par{u,w}\in\null T\times\range T,\Ang{u+w,u}=\Ang{u,u}=0\Rightarrow v\in\range T.$\PfEnd
\SepLine

\Anchor{6C8}\ProblemN{8}{
	\TextA{Supp $V$ is finide, $P^2=P\in\Lm{V},$ $\Dvert{Pv}\leqslant\Dvert{v}$ for all $v\in V.$ Prove $\range P=\Par{\null P}{^\perp}.$}
}$\Dvert{w}=\Dvert{Pv}\leqslant\BigDvert{Pv+\Par{v-Pv}}=\Dvert{w+u},$ where $w=Pv,u\in\null P.$ \;Supp non0 $u\in\null P.$\parSol{}
$\forall a\in\Fbb,\;\Dvert{w}\leqslant\Dvert{w+au},$ \Or $\Dvert{Pv}=\Dvert{P\Par{Pv+au}}\leqslant\Dvert{Pv+au}.$ \;Thus $\Ang{Pv,v-Pv}=0.$\PfEnd
\SepLine

\Anchor{6C10}\ProblemN{10}{
	\TextA{Supp $V$ finide, $U$ a subsp, $T\in\Lm{V},$ and $P_{\!U}T=TP_{\!U}.$ Prove $U$ and $U^\perp$ invard $T.$}
}(a) $P_{\!U}TP_{\!U}=TP_{\!U}P_{\!U}=TP_{\!U}.$ \,(b) $P_{U^\perp}TP_{U^\perp}=\Par{I-P_{\!U}}T\Par{I-P_{\!U}}=T\Par{I-P_{\!U}}{^2}=TP_{U^\perp}.$\vspace{2pt}\PfEnd\parSol{}
\Or (a) $\range T\mmid_U=\range TP_{\!U}=\range P_{\!U}T\subseteq U.$\parSol{}
\Blind{\Or}(b) $\range T\mmid_{U^\perp}=\range T\Par{I-P_{\!U}}=\range\Par{I-P_{\!U}}T\subseteq U^\perp.$\PfEnd\parSol{}
\Blind{\Or}\AComm The trick $T=\Par{P_{\!U}\mmid_{\range T}}{^{-1}}TP_{\!U}$ is invalid.
\SepLine

\Anchor{6CT2}\BulletPointX\TipsN{2}\,\,\,Supp $U$ finide subsp of $V,v\in V,\varphi\in U\apostrophe:u\mapsto\Ang{u,v}.$\TextB{}
{\IndentTipsN{2}}Then $\exists\,!\,w\in U,\varphi\Par{u}=\Ang{u,w}=\Ang{u,v}$ for all $u\in U\Rightarrow v-w\in U^\perp.$ Now $w=P_{\!U}v.$
\SepLine

%\Anchor{6CT3}\BulletPointX\TipsN{3}\,\,\,Supp $e_1,\dots,e_n\in V$ with each $\Dvert{e_k}=1,$ and for all $v\in V,\:\Dvert{v}{^2}=\innerAsq{v,\,e_1}+\dots+\innerAsq{v,\,e_n}.$\TextB{}
%{\IndentTipsN{3}}Then each $\Dvert{e_j}{^2}=\sum_{j\neq k}^n\innerAsq{e_j,\,e_k}+\innerAsq{e_j,\,e_j}\Rightarrow$ each $\innerAsq{e_j,e_k}=0.$\TextB{}
%{\IndentTipsN{3}}And by (6.B.2), $V=\Span{e_1,\dots,e_n}.$ \;\Or Becs $\forall v\in\Span{e_1,\dots,e_n}{^\perp},\Dvert{v}{^2}=0.$
%\SepLine

\Anchor{6CT3}\ProblemBX{\TipsN{3}}{
	\TextA{Supp $U,W$ finide subsps of $V.$ Prove $W^\perp\subseteq U^\perp\Rightarrow U\subseteq W.$}
}$v\in U\Rightarrow\forall x\in W^\perp\subseteq U^\perp,\:\Ang{v,x}=0\Rightarrow v\in\Par{W^\perp}{^\perp}=W,$ by [6.51].\PfEnd
\SepLine

\Anchor{6CT4}\ProblemBX{\TipsN{4}}{
	\TextA{Supp $U,W$ subsps of $V.$ Prove $\BigPar{U+W}{^\perp}=U^\perp\cap W^\perp,\,\BigPar{U\cap W}{^\perp}=U^\perp+W^\perp.$}
}(a) Supp $v\in V.$ Then $\forall u+w\in U+W,\Ang{u+w,v}=0\Longleftrightarrow\forall u\in U,w\in W,\Ang{u,v}=\Ang{u,w}=0.$\vspace{2pt}\parSol{}
(b) $v_1\in U^\perp,v_2\in W^\perp\Rightarrow\forall x\in U\cap W,\Ang{v_1+v_2,x}=0.$ Thus $U^\perp+W^\perp\subseteq\BigPar{U\cap W}{^\perp}.$\vspace{1pt}\parSol{\Hb}
\Or $U\cap W\subseteq U,W\Rightarrow\Par{U\cap W}{^\perp}\supseteq U^\perp+W^\perp.$\vspace{2pt}\parSol{\Hb}
\!\Sbra[3pt]{{\tgsl Req Finide}} \;By (a), $\Dim\BigPar{U^\perp+W^\perp}=\dim V-\Dim\Par{U\cap W}.$\PfEnd\vspace{4pt}
\AExa Not true if infinide. Let $U=\Bra{\Par{x_1,0,\cdots,0,x_{2k-1},0,\cdots}\in\FbbP{\infty}},W=\Bra{{\sum_{k=1}^\infty a_k\Par{e_{2n}+\text{\Large\ensuremath{\frac{\:1\:}{n}}}\:\!e_{2n-1}}}}.$\vspace{2pt}\parExa{}
Then $U\cap W=\zeroSubs\Rightarrow\Par{U\cap W}{^\perp}=\FbbP{\infty}.$ While $U^\perp+W^\perp\not\ni\BigPar{0,1,0,\text{\Large\ensuremath{\frac{\:1\:}{2}}},0,\cdots,0,\text{\Large\ensuremath{\frac{\:2\:}{k}}},0,\cdots}.$
\SepLine

\ChEnd

\vfill\ChDecl{Ch7A}{7.A,B,C \& 7.D[4E]}{\quad{\ANote {\FontSmall $V$ denotes a finide vecsp over $\Fbb.$\qquad7.D[4E]处结合了3e的9.B节。}}}%An Exe marked by $\blacksquare$ is true if infinide or partially finide.

\vspace{6pt}

\Anchor{7AT}\ProblemN{A.\Tips}{
	\TextA{Supp $T\in\Lm{V}$ is normal. Prove $p\Par{T}$ is normal for any $p\in\PoFi.$}
}Becs $T$ normal $\Rightarrow T^j$ normal. We use induc on $\deg p=k.$\parSol{}
(i) $k=0,1.$ \;$p\Par{z}=c_0+c_1z\Rightarrow p\Par{T}{^*}p\Par{T}=\Par{\overline{c_0}I+\overline{c_1}T^*}\Par{c_0I+c_1T}=p\Par{T}p\Par{T}{^*}.$\parSol{}
(ii) $k>1.$ \;Asum true for smaller deg. Let $q\Par{z}=c_0+c_1z+\dots+c_{k-1}z^{k-1}\Rightarrow p\Par{z}=q\Par{z}+c_kz^k.$\parSol{\Hii}
Then $p\Par{T}{^*}p\Par{T}=\Sbra{q\Par{T}{^*}+\overline{c_k}\Par{T^k}{^*}}\Sbra{q\Par{T}+c_kT^k}=p\Par{T}p\Par{T}{^*}.$\PfEnd
\SepLine

%\Anchor{7A3}\ProblemN{3}{
%	\TextA{Supp $T\in\Lm{V}$ and $U$ is a subsp of $V.$ Prove $U$ invard $T\Longleftrightarrow U^\perp$ invard $T^*.$}
%}If $U$ invard. Then $\forall u\in U,\forall w\in U^\perp,\:\Ang{Tu,w}=0=\Ang{u,T^*w}\Rightarrow T^*w\in U^\perp.$ Rev the roles.\PfEnd
%\SepLine

%\Anchor{7A4e19}\ProblemBnoor{4E 19}{
%	\TextA{Supp $T\in\Lm{V}$ and $\Dvert{T^*v}\leqslant\Dvert{Tv}$ for all $v\in V.$ Prove $T$ is normal.}
%}Let orthon $B_V=\Par{e_1,\dots,e_n}.$ By Exe (4E 5), $\sum_{i=1}^n\Dvertsq{T^*e_i}=\sum_{i=1}^n\Dvertsq{Te_i}.$\parSol{}
%Becs each $\Dvertsq{T^*e_i}\leqslant\Dvertsq{Te_i}.$ Note that this orhon $B_V$ is arb.\PfEnd
%\SepLine

%\Anchor{7A11}\ProblemB{
%	\TextB{Supp $P^2=P=P_{\!U}\in\Lm{V}$ for some subsp $U$ of $V.$ Prove $P$ is self-adj.}
%}$\forall\Par{u,w},\Par{x,y}\in U\times U^\perp,\:\BigAng{P\Par{u+w},x+y}=\Ang{u+w,x}=\BigAng{u+w,P^*\Par{x+y}}.$\PfEnd
%\SepLine

\Anchor{7A17}\ProblemN{A.17}{
	\TextA{Supp $T\in\Lm{V}$ is normal. Prove each $\null T^k=\null T$ \,and\, $\range T^k=\range T.$}
}Becs $\range T=\Par{\null T^*}{^\perp}=\Par{\null T}{^\perp}\Rightarrow T\mmid_{\range T}$ is inje. Thus $\null T^k=\null T.$\parSol{}
And $\range T^2=\range T\mmid_{\range T}=\range T=\range T\Rightarrow\range T^{k-1}\mmid_{\range T}=\range T^{k-1}\mmid_{\range T^2}.$\PfEnd\vspace{4pt}\parSol{}
\Or $v\in\null T^{k+1}\Rightarrow T^kv\in\null T=\null T^*\Rightarrow0=\Ang{T^*T^kv,T^{k-1}v}=\Ang{T^kv,T^kv}\Rightarrow v\in\null T^k.$\parSol{}
Note that $T$ normal $\Rightarrow T^k$ normal. Then $\range T^k=\Par{\null T^k}{^\perp}=\Par{\null T}{^\perp}=\range T.$\PfEnd
\SepLine

\Anchor{7A4e28}\ProblemBnoor{4E A.28}{
	\TextA{Supp $T\in\Lm{V}$ is normal. Prove the min of $T$ is not a multi of any $\Par{z-\lambda}{^2}.$}
}Supp the min of $T$ is $p\Par{z}=\Par{z-\lambda}{^k}q\Par{z}$ with $k\geqslant1$ and $q\Par{\lambda}\neq0.$\parSol{}
Then $p\Par{T}v=0\Rightarrow q\Par{T}v\in\Null\Par{T-\lambda I}{^k}=\Null\Par{T-\lambda I}\Rightarrow k=1.$\PfEnd\vspace{2pt}\parSol{}
\Or By {\TIPS} and \Sbra{8.B \TIPSN{4}}. Factoriz the min of $T\Rightarrow$ each liney factor has expo $1.$\PfEnd\vspace{2pt}\parSol{}
\Or Becs $\Range\Par{T-\lambda I}=\Null\Par{T-\lambda I}{^\perp}\Rightarrow V=\Null\Par{T-\lambda I}\oplus\range\Par{T-\lambda I}$ for all $\lambda\in\Fbb.$\parSol{}
By (5.C.5). If $\Fbb=\Rbb,$ then apply to $T_{\!\Cbb}.$ Now every liney factor has expo $1.$\PfEnd
\SepLine

%\Anchor{8C16}\Anchor{7A4e24}\ProblemBnoor{4E 24 {\OR} 8.C.16}{
%	\TextA{Supp $T\in\Lm{V}$ with the min $p.$ Prove the min $s$ of $T$ is $\overline{p}.$}
%}$\forall q\in\PoFi,\,q\Par{T}=0\Longleftrightarrow q\Par{T}{^*}=\overline{q}\Par{T^*}=0.$ \,{\OR} $\overline{q}\Par{T}=0\Longleftrightarrow \overline{q}\Par{T}{^*}=q\Par{T^*}=0.$\parSol{}
%Thus $\overline{p}$ is a multi of $s,$ and $\overline{s}$ is a multi of $p.$\PfEnd
%\SepLine

%\Anchor{8B4e10}\ProblemBnoor{4E 8.B.10}{
%	\TextA{Supp $\Fbb=\Cbb,\Par{e_1,\dots,e_n}$ is an orthon bss of $V$, $T\in\Lm{V}.$}
%	\TextA{Let $\lambda_1,\dots,\lambda_n$ be eigvals with repeti due to multies.\vspace{1pt}}
%	\TextA{Prove $\aMidsq{\lambda_1}+\dots+\aMidsq{\lambda_n}\leqslant\Dvertsq{Te_1}+\dots+\Dvertsq{Te_n}.$}
%}Let $A=\Mt{T}$ up-trig wrto orthon $B_V=\Par{\,f_1,\dots,f_n}.$ Then $\sum\Dvertsq{Te_j}=\sum\Dvertsq{T^*\:\!f_j}=\sum\Dvertsq{T\:\!f_j}.$\parSol{}
%%Becs $\dim\Null\Par{T-\lambda I}{^k}=\dim\Null\Par{T^*-\overline{\lambda}I}{^k}.$
%\SepLine

%\Anchor{7A4e2}\ProblemBnoor{4E 2}{
%	\TextA{Supp $T\in\Lm{V,W}.$ Prove $T=0\Longleftrightarrow T^*=0\Longleftrightarrow T^*T=0\Longleftrightarrow TT^*=0.$}
%}$T=0\Rightarrow\range T^*=\zeroSubs,\null T^*=W\Rightarrow T^*=0.$ Convly rev the roles.\parSol{}
%$T^*T=0\Rightarrow T^*\mmid_{\range T}=T^*\mmid_{\SmallPar{\null T^*}{^\perp}}=0\Rightarrow T^*=0.$ Simlr for $TT^*=0\Rightarrow T=0.$\PfEnd\vspace{3pt}\parSol{}
%\Or $T=0\Rightarrow\forall w\in W,\forall v\in V,\Ang{v,T^*w}=\Ang{Tv,w}=0\Rightarrow\range T^*\subseteq V^\perp.$\parSol{}
%$T^*T=0\Rightarrow\forall v\in V,\Ang{Tv,Tv}=\Ang{v,T^*Tv}=0\Rightarrow T=0.$\PfEnd
%\SepLine
%\ChEnd
%\pagebreak
%\vspace{8pt}

%\ChDecl{Ch7B}{7.B}{\quad{\ANote {\FontSmall $V$ denotes a finide vecsp over $\Fbb.$}}}

%\vspace{4pt}

\Anchor{7B14}\ProblemN{B.14}{
	\TextA{Supp $\Fbb=\Rbb,$ $T\in\Lm{V}.$ Prove $T$ diag $\Rightarrow$ self-adj wrto some $\Ang{\cdot,\cdot}{_V}.$}
}Let eigvecs $B_V=\Par{e_1,\dots,e_n}$ be orthon wrto $\Ang{e_j,\:\!e_k}{_V}=\delta_{j,\:\!k}.$ \,Becs $\Mt{T}=\Mt{T}{^t}=\Mt{T^*}.$\PfEnd\vspace{2pt}
\Anchor{7B4e24}\ANote (a) $\Mt{T}=\overline{\Mt{T}}{^t}$ wrto some $B_V\Longleftrightarrow T$ self-adj wrto some $\Ang{\cdot,\cdot}{_V}$ $\Longleftrightarrow$ diag.\parNot
(b) $\Mt{T}\overline{\Mt{T}}{^t}=\overline{\Mt{T}}{^t}\Mt{T}$ wrto some $B_V\Longleftrightarrow T$ normal wrto some $\Ang{\cdot,\cdot}{_V}$ $\Longleftrightarrow$ diag on $\Cbb.$
\SepLine

%\Anchor{7B7}\ProblemN{7}{
%	\TextA{Supp $\Fbb=\Cbb,T\in\Lm{V}$ is normal and $T^9=T^8.$ Prove $T^2=T$ is self-adj.}
%}Becs $T^8\Par{T-I}=0\Rightarrow0,1\in\Rbb$ are all possible eigvals.\PfEnd\parSol{}
%\Or $\range T^*=\range T=\Par{\null T}{^\perp}\Rightarrow T\Par{T-I}=0.$ 又 $Tv=Tv+\Par{v-Tv}\Rightarrow T=P_{\range T}.$\PfEnd\parSol{}
%\Or $\exists$ orthon $B_V=\Par{e_1,\dots,e_n}$ of eigvecs with corres $\lambda_1,\dots,\lambda_n.$\parSol{}
%Now $\lambda_j^8e_j=T^8e_j=T^9e_j=\lambda_j^9e_j\Rightarrow\lambda_j=0$ or $1\in\Rbb.$ 又 Each $T^2e_j=\lambda_j^2e_j=\lambda_je_j=Te_j.$\PfEnd
%\SepLine
\pagebreak
\Anchor{7B4e8}\ProblemBnoor{4E B.8}{
	\TextA{Supp $\Fbb=\Cbb,T\in\Lm{V}.$ Prove each eigvec of $T$ is an eigvec of $T^*\Rightarrow T$ is normal.}
}Supp $v$ is eigvec of $T$ corres $\lambda$ and of $T^*$ corres $\mu.$\parSol{}
Then $\lambda\Dvertsq{v}=\Ang{Tv,v}=\Ang{v,T^*v}=\overline{\mu}\Dvertsq{v}\Rightarrow\lambda=\overline{\mu}.$\parSol{}
Thus each $E\Par{\lambda,T}=E\Par{\overline{\lambda},T^*}$ invard $T,T^*\Rightarrow E\Par{\lambda,T}{^\perp}=E\Par{\overline{\lambda},T^*}{^\perp}$ invard $T^*,T.$\parSol{}
Let $W=\bigcap_{\lambda\,\in\,\Fbb}E\Par{\lambda,T}{^\perp}$ invard $T,T^*.$ By \Sbra{6.C \TIPSN{4}}, no eigvals of $T\mmid_W,T^*\mmid_W\Rightarrow W=\zeroSubs.$\PfEnd\vspace{4pt}\parSol{}
\Or $\exists$ orthon $B_V=\Par{e_1,\dots,e_n}$ suth $\Mt{T}$ up-trig $\Rightarrow\overline{A}{^t}=\Mt{T^*}$ low-trig.\parSol{}
(i) Now $Te_1=A_{1,1}e_1\Rightarrow\overline{A_{1,1}}e_1+\dots+\overline{A_{1,\;\!n}}e_n=T^*e_1\Rightarrow A_{1,2}=\dots=A_{1,\;\!n}=0.$\parSol{}
(ii) Asum $\Par{A_{1,2}\;\cdots\;A_{1,\;\!n}}=\dots=\Par{A_{k-1,\;\!k}\;\cdots\;A_{k-1,\;\!n}}=0.$ 又 $A$ is up-trig.\vspace{1pt}\parSol{\Hii}
Then $Te_{k}=A_{k,\;\!k}e_{k}\Rightarrow\overline{A_{k,\;\!k}}e_1+\dots+\overline{A_{k,\;\!n}}e_n=T^*e_{k}\Rightarrow A_{k,\;\!k+1}=\dots=A_{k,\;\!n}=0.$\PfEnd
\SepLine

%\Anchor{7B4e9}\ProblemBnoor{4E 9}{
%	\TextA{Supp $\Fbb=\Cbb,T\in\Lm{V}$ is normal. Prove $\exists\,p\in\PoFi,\,T^*=p\Par{T}.$}
%}Let $\lambda_1,\dots,\lambda_m$ are disti eigvals of $T.$ \,By Exe (4.5), $\exists\,!\,p\in\PoF{m},$ each $p\Par{\lambda_k}=\overline{\lambda_k}.$ \,Immed.
%Let $p\Par{z}=\sum_{j=1}^m{}${\Large\envFontSmall[\footnotesize]\def\SmallPar{\Par}$\frac{\Par{z\,-\,\lambda_1}\,\cdots\,\Par{z\,-\,\lambda_{j-1}}\Par{z\,-\,\lambda_{j+1}}\,\cdots\,\Par{z\,-\,\lambda_m}}{\Par{\lambda_j\,-\,\lambda_1}\,\cdots\,\Par{\lambda_j\,-\,\lambda_{j-1}}\Par{\lambda_j\,-\,\lambda_{j+1}}\,\cdots\,\Par{\lambda_j\,-\,\lambda_m}}$}${\,}\overline{\lambda_j}\Rightarrow$ each $p\Par{\lambda_j}=\overline{\lambda_j}.$\vspace{3pt}\parSol{}
%\PfEnd
%\SepLine

\Anchor{7B4e12}\ProblemBnoor{4E B.12}{
	\TextA{Supp $\Fbb=\Cbb,T\in\Lm{V}$ is normal, $S\in\Lm{V}$ and $ST=TS.$ Prove $ST^*=T^*S.$}
}Let $B_V=\Par{e_1,\dots,e_n}$ be orthon eigvecs of $T$ corres $\lambda_1,\dots,\lambda_n.$\parSol{}
Becs each $E\Par{\lambda_k,T}=E\Par{\overline{\lambda_k},T^*}$ invard $S\Rightarrow ST^*e_k=\overline{\lambda_k}Se_k=T^*Se_k.$ \;\Or Becs $T^*=p\Par{T}.$\PfEnd
\SepLine

\Anchor{7B4e20}\ProblemBnoor{4E B.20}{
	\TextA{Supp $\Fbb=\Cbb,T\in\Lm{V}$ is normal and $U$ invarspd $T.$}
	\TextA{Prove {\tgnr\large(a)} $U^\perp$ invard $T,$ \;{\tgnr\large(b)} $\Par{T\mmid_U}{^*}=T^*\mmid_U\in\Lm{U},$ \;{\tgnr\large(c)} $T\mmid_U,T\mmid_{U^\perp}$ normal.}
}By \Sbra{5.A \TIPSN{3}}, and apply [6.31] to each $E\Par{\lambda_k,T\mmid_U},$ let $B_U=\Par{e_1,\dots,e_m}$ be orthon eigvecs.\parSol{}
Let $B_V=\Par{e_1,\dots,e_n}$ be orthon eigvecs. Then $B_{U^\perp}=\Par{e_{m+1},\dots,e_n},$ invard $T.$ And $U$ invard $T^*.$\vspace{2pt}\parSol{}
(b) $\forall u,v\in U,\:\BigAng{v,\Par{T\mmid_U}{^*}u}=\BigAng{T\mmid_U v,u}=\BigAng{v,T^*\mmid_U u}\Rightarrow\Sbra{\Par{T\mmid_U}{^*}-\Par{T^*\mmid_U}}\Par{u}\in U\cap U^\perp.$\vspace{2pt}\parSol{}
(c) $\forall u\in U,\:\BigDvert{T\mmid_U u}=\BigDvert{T^*\mmid_U u}=\BigDvert{\Par{T\mmid_U}{^*}u}.$ \;\Or $\Par{T\mmid_U}\Par{T\mmid_U}{^*}=T^*T\mmid_U=T^*\mmid_UT\mmid_U.$\PfEnd\vspace{3pt}
%\Or Let orthon $B_U=\Par{e_1,\dots,e_m}\Rightarrow B_V=\Par{e_1,\dots,e_n}\Rightarrow B_{U^\perp}=\Par{e_{m+1},\dots,e_n}.$ Let $A=\Mt{T,B_V}.$\parSol{}
%Let $A$ be block matrix {\small$\begin{pmatrix}X &\hspace{-4pt} Y\\[-2pt] 0 &\hspace{-4pt} Z\end{pmatrix}$}, where $X\in\FbbP{m,\;\!m},Y\in\FbbP{m,\:\!n-m},Z\in\FbbP{n-m,\:\!n-m}.$\parSol{}
%(a) Becs $\sum_{j=1}^m\Dvertsq{X_{\cdot,j}}=\sum_{j=1}^m\Dvertsq{Te_j}=\sum_{j=1}^m\Dvertsq{T^*e_j}=\sum_{j=1}^m\BigDvertsq{\overline{X_{j,\cdot}}}+\sum_{j=1}^m\BigDvertsq{\overline{Y_{j,\cdot}}}\Rightarrow Y=0.$\vspace{2pt}\parSol{\Ha}
%Apply to $T^*\Rightarrow U$ invard $T^*.$ \Or Becs $\Mt{T^*,B_V}=\overline{A}{^t}={}${\small$\begin{pmatrix}\overline{X}{^t} &\hspace{-4pt} 0\\[-2pt] 0 &\hspace{-4pt} \overline{Z}{^t}\end{pmatrix}$}.\parSol{}
%(b) $\Mt{T\mmid_U,B_V}=X\Rightarrow\Mt[\BigPar]{\Par{T\mmid_U}{^*},B_V}=\overline{X}{^t}.$ 又 $\Mt{T^*\mmid_U,B_V}=\overline{X}{^t}.$\PfEnd\vspace{4pt}
\AComm See [9.30] without using [7.24] and the hypo $\Fbb=\Cbb.$\par
\Anchor{7B13}\ANote Another proof of [7.24]: Induc step: For $\dim V>1.$ Asum it holds for smaller dim.\parNot
Let $u$ be an eigvec with $\Dvert{u}=1.$ Let $B_U=\Par{u}\Rightarrow U$ invard $T,$ so is $U^\perp\Rightarrow T\mmid_{U^\perp}$ normal.\parNot
By asum, $\exists$ orthon $B_{U^\perp}$ of eigvecs of $T\mmid_{U^\perp}.$ Now $B_V=B_U\cup B_{U^\perp}$ of orthon eigvecs.\PfEnd
\SepLine
%\vspace{8pt}\pagebreak

%\ChDecl{Ch7C}{7.C \& 7.D [4E]}{\quad{\ANote {\FontSmall $V$ denotes a finide vecsp over $\Fbb.$\qquad7.D[4E]处结合了3e的9.B节。}}}

%\vspace{4pt}

\Anchor{7CNSRI}\ProblemBX[]{\NoteForSmall{Square Root of Id}}{
	Supp $T\in\Lm{\FbbP{2}}$ and $T^2=I.$ Let $\Mt{T}={}${\small$\begin{pmatrix}a &\hspace{-6pt} b\\[-2pt] c &\hspace{-6pt} d\end{pmatrix}$} wrto std bss.\TextB{}
	(a) If $T$ is self-adj $\Longleftrightarrow b=c$. Then $ab+bd=0,\,a^2+b^2=b^2+d^2=1.$\TextB{}
	\Ha $\aMid{a}=\aMid{d}=1,\:\!b=0,$ \;\OR \,$a=\pm\sqrt{1-b^2}=-d,\,b\neq0,$ \;\OR \,$b=\pm\sqrt{1-a^2}=\pm\sqrt{1-d^2}\neq0,\:a=-d.$\TextB{}
	\Ha If $\Fbb=\Rbb,\,\aMid{b}<1,$ then $\Mt{T}={}${\small$\begin{pmatrix}\cos\alpha &\hspace{-6pt} \Blind{-}\sin\alpha\\[-2pt] \sin\alpha &\hspace{-6pt} -\cos\alpha\end{pmatrix}$}, {\FontSmall and $T\Par{r\cos\beta,\:r\sin\beta}=\BigPar{r\cos\!\Par{\alpha-\beta},\:r\sin\!\Par{\alpha-\beta}}$}.\TextB{\vspace{6pt}}
	\Anchor{7D12}(b) If $T$ is not self-adj and $T\neq\pm I.$ \,Then by (4E 5.B.11), $a=-d,\,a^2+bc=1.$\TextB{}
	\Hb If $a=-d=\pm 1,$ then $bc=0,$ and if $\Dvert{Te_1}\neq\Dvert{Te_2}$ or $\Ang{Te_1,Te_2}\neq0,$ then $T$ is not an isomet.\TextB{\vspace{-3pt}}
}\SepLine

%\Anchor{7C4}\ProblemN{4}{
%	\TextA{Supp $1<n\in\Nbb.$ Show $\exists\,A\in\FbbP{n,\;\!n}$ with all ents posi numbers and $A=\overline{A}{^t},$}
%	\TextA{while wrto std bss $\Mt{T}=A$ but $T\in\Lm{\FbbP{n}}$ is not posi.}
%}
%\SepLine

\Anchor{7C20}\ProblemN{C.20}{
	\TextA{Supp $T\in\Lm{V}$ and orthon $B_V=\Par{e_1,\dots,e_n}.$}
	\TextA{Supp $v_1,\dots,v_n\in V$ and each $\Ang{Te_k,\:\!e_j}=\Ang{v_k,v_j}.$ Prove $T$ posi.\vspace{0pt}}
%$\overline{\Ang{v_j,v_k}}=\Ang{v_k,v_j}=\Ang{Te_k,\:\!e_j}=\overline{\Ang{Te_j,\:\!e_k}}\Rightarrow\Ang{Te_k,\:\!e_j}=\Ang{e_k,\:\!Te_j}\Rightarrow\Mt{T}=\overline{\Mt{T}}{^t}.$\vspace{1pt}\parSol{}
%\Or $\forall v=\sum_{k=1}^na_ke_k,\:w=\sum_{j=1}^nb_je_j,\;\Ang{Tv,w}=\sum_{k=1}^n\sum_{j=1}^na_k\overline{b_j}\:\!\Ang{e_k,\:\!Te_j}=\Ang{v,Tw}.$\vspace{2pt}\parSol{}
%Let $v=\sum_{k=1}^na_ke_k.$ Then $\Ang{Tv,v}=\sum_{k=1}^n\sum_{j=1}^na_k\overline{a_j}\:\!\Ang{Te_k,\:\!e_j}.$ Note that $\Ang{Te_k,\:\!e_j}=\overline{\Ang{Te_j,\:\!e_k}}.$\vspace{2pt}\parSol{}
%Hence each $a_k\overline{a_j}\:\!\Ang{Te_k,\:\!e_j}+a_j\overline{a_k}\:\!\Ang{Te_j,\:\!e_k}=2\REAL\:a_k\overline{a_j}\:\!\Ang{Te_k,\:\!e_j}.$ 又 Each $\Ang{Te_k,\:\!e_k}=\Ang{v_k,\:\!v_k}\geqslant 0.$\PfEnd\vspace{6pt}\parSol{}
}Define $R\in\Lm{V}:e_k\mapsto v_k\Rightarrow\Ang{Te_k,\:\!e_j}=\Ang{Re_k,\:\!Re_j}=\Ang{R^*Re_k,\:\!e_j}\Rightarrow\Mt{T,B_V}=\Mt{R^*R,B_V}.$\PfEnd
\SepLine

\Anchor{7C22}\ProblemN{C.22}{
	\TextA{Supp $T$ posi, $u\in V$ with $\Dvert{u}=1$ suth $\Dvert{Tu}\geqslant\Dvert{Tv}$ for all $v\in V$ with $\Dvert{v}=1.$}
	\TextA{Show $u$ is eigvec corres the largest eigval of $T.$}
}Supp orthon eigvecs $B_V=\Par{e_1,\dots,e_n}$ corres $\lambda_1\geqslant\dots\geqslant\lambda_n.$ \;Let $u=\sum_{k=1}^nc_ke_k\Rightarrow\sum_{k=1}^n\aMidsq{c_k}=1.$\vspace{1pt}\parSol{}
Supp $v=\sum_{j=1}^na_je_j$ and $\Dvert{v}=1.$ Then $\Dvertsq{Tv}=\sum_{j=1}^n\aMidsq{\lambda_j}\aMidsq{a_j}\leqslant\aMidsq{\lambda_1}.$ Simlr, $\Dvertsq{Tu}\leqslant\aMidsq{\lambda_1}.$\vspace{1pt}\parSol{}
\又 $\Dvertsq{Tv}=\aMidsq{\lambda_1}\Longleftrightarrow v=a_1e_1+\dots+a_Je_J,$ where $\lambda_1=\cdots=\lambda_J>\lambda_{J+1},$ if $\lambda_n\neq\lambda_1;$ \,othws $J=n.$\vspace{1pt}\parSol{}
Hence $\sum_{k=1}^n\aMidsq{\lambda_k}\aMidsq{c_k}=\Dvertsq{Tu}=\sum_{k=1}^n\aMidsq{\lambda_1}\aMidsq{c_k}\Rightarrow\sum_{k=J}^n\!\Sbra{\aMidsq{\lambda_1}-\aMidsq{\lambda_k}}\cdot\aMidsq{c_k}=0.$\PfEnd
\SepLine

\Anchor{7C23}\ProblemN{C.23}{
	\TextA{Supp $\Ang{\cdot,\cdot}{_1},\Ang{\cdot,\cdot}{_2}$ are inner prods on $V.$ Prove $\exists$ inv posi $T\in\Lm{V},\;\Ang{u,v}{_2}=\Ang{Tu,v}{_1}.$}
}Let $\Par{e_1,\dots,e_n},\Par{\,f_1,\dots,f_n}$ be orthon bses wrto $\Ang{\cdot,\cdot}{_2},\Ang{\cdot,\cdot}{_1}.$ Define $R\in\Lm{V}$ by $Re_k=f_k.$\parSol{}
$\forall u=\sum_{i=1}^nx_i\,e_i,\:v=\sum_{i=1}^ny_i\,e_i,\;\Ang{u,v}{_2}=x_1\:\!\overline{y_1}+\dots+x_n\:\!\overline{y_n}=\Ang{Ru,Rv}{_1}=\Ang{R^*Ru,v}{_1}.$\PfEnd
\SepLine

\ChDecl{Ch7D}{}{}

\Anchor{7D2}\ProblemN{D.2}{
	\TextA{Supp $T\in\Lm{V,W},\,\Ang{Tu,Tv}=0$ for all orthog $u,v\in V.$ Prove $\exists$ isomet $S,\,T=\lambda S.$}
}Supp orthog $B_V=\Par{v_1,\dots,v_n}\Rightarrow\Par{Tv_1,\dots,Tv_n}$ is orthog.\parSol{}
Let $g_k=Tv_k\big/\Dvert{Tv_k},\,e_k=v_k\big/\Dvert{v_k}.$ \,Define isomet $S\in\Lm{V,W}:e_k\mapsto g_k.$\parSol{}
Let $\lambda_k=\Dvert{Tv_k}\Big/\Dvert{v_k}.$ Then $S^*:g_k\mapsto e_k\Rightarrow S^*\Par{Tv_k}=\Dvert{Tv_k}\,e_k=\lambda_kv_k.$\parSol{}
\NOTICE that $v_1$ is arb. Simlr to (4E 3.A.11). Hence $S^*T=\lambda I\Rightarrow T=\lambda S.$\PfEnd\vspace{3pt}\parSol{}
\Or Let orthon $B_V=\Par{e_1,\dots,e_n}.$ Becs $\Ang{u+v,\,u-v}=\Dvertsq{u}-\Dvertsq{v}.$\parSol{}
Now $0=\Ang{e_1+e_k,\,e_1-e_k}=\Ang{Te_1+Te_k,\,Te_1-Te_k}\Rightarrow$ each $\Dvert{Te_k}=\lambda.$ Supp $\lambda\neq0.$\parSol{}
Let $S=\lambda^{-1}T.$ Becs $\Ang{e_j,\,e_k}=0\Rightarrow\Ang{Te_j,\,Te_k}=\Ang{\lambda Se_j,\,\lambda Se_k}=0\Longleftrightarrow\Ang{Se_j,\,Se_k}=0.$\PfEnd
\SepLine

\Anchor{7D5}\ProblemN{D.5}{
	\TextA{Supp $S\in\Lm{V}.$ Prove $S$ self-adj and unit $\Longleftrightarrow\exists\,P_{\!U},\,S=2P_{\!U}-I.$}
}Supp $S$ self-adj and unit. Then $V=E\Par{1,S}\oplus E\Par{{-1},S},\,E\Par{1,S}=E\Par{{-1},S}{^\perp}\Rightarrow S=2P_{\!U}-I.$\parSol{}
\Or $S^2=I.$ \,Let $P={}${\Large$\frac{\:1\:}{2}$}$\Par{S+I}\Rightarrow P^2=P$ self-adj $\Rightarrow\range P=\Par{\null P}{^\perp}=U.$\vspace{3pt}\parSol{}
Supp $S=2P_{\!U}-I\Rightarrow S$ self-adj. Then $\forall u\in U,\,Su=u,$ and $\forall w\in U^\perp,\,Sw=-w.$\parSol{}
$\BigDvertsq{S\Par{u+w}}=\Dvertsq{u+w}.$ \Or $S^2\Par{u+w}=u+w\Rightarrow S^{-1}=S=S^*.$ \Or Apply to a orthon $B_V.$\PfEnd
\SepLine

\Anchor{7DT}\ProblemN{D.\Tips}{
	\TextA{Supp $T\in\Lm{V},$ each eigval of $T_{\!\Cbb}$ has abs val $1.$}
	\TextA{Supp $\Dvert{Tv}\leqslant\Dvert{v}$ for all $v\in V.$ Prove $T$ unit.}
}Supp Jordan $\Mt{T_{\!\Cbb}}$ wrto $B_1=\Par{u_1+\i\,v_1,\dots,u_n+\i\,v_n}\Rightarrow\Mt{T,B_1}=\Mt{T,B_2},$\parSol{}
where $B_2=\Par{x_1+\i\,y_1,\dots,x_n+\i\,y_n}$ with each $x_k+\i\,y_k=\BigPar{\!\sqrt{\Dvertsq{u_k}+\Dvertsq{v_k}}}{^{-1}}\Par{u_k+\i\,v_k}.$\parSol{}
Becs $\BigDvertsq{T_{\!\Cbb}\Par{u+\i\,v}}=\Dvertsq{Tu}+\Dvertsq{Tv}\leqslant\Dvertsq{u+\i\,v}.$ By Exe (D.9), $T_{\!\Cbb}$ is unit.\parSol{}
Consider $\Mt{T}$ wrto $B_V=\Par{v_1,u_1,\cdots,v_n,u_n}$ and by [9.36].\PfEnd
\SepLine

\Anchor{7D11}\ProblemN{D.11}{
	\TextA{Supp $S\in\Lm{V},$ and $\Bra{Sv:v\in\bigodot\!}=\Bra{v\in V:\Dvert{v}\leqslant 1}=\bigodot.$ Prove $S$ is unit.}
}\NOTICE that $\BigDvert{S\Par{\Dvert{v}{^{-1}}v}}\leqslant 1\Rightarrow\Dvert{Sv}\leqslant\Dvert{v}$ for all $v\in\bigodot.$\parSol{}
Asum $S$ not inv. Then $\exists\,w\in V\Backslash\range S,\,\Dvert{w}{^{-1}}w\not\in\Bra{Sv:v\in\bigodot\!}=\bigodot.$ Ctradic.\parSol{}
\ANote If $v\neq0,\,Sv=0\in\bigodot,$ \uline{then} $v\in\bigodot.$ \;\uline{Wrong} becs only $\,a\,\Dvert{v}{^{-1}}v\in\bigodot,$ where $0\leqslant a\leqslant1.$\vspace{4pt}\parSol{}
Now $\forall v\in V\nonzero,\,S\Sbra{\Dvert{Sv}{^{-1}}v}\in\bigodot\Longleftrightarrow\Dvert{Sv}{^{-1}}v\in\bigodot\Rightarrow\Dvert{Sv}{^{-1}}\Dvert{v}\leqslant1.$\PfEnd\vspace{2pt}\parSol{}
\Or \NOTICE that \,$\bigodot{_{\!\Cbb}}=\Bra{u+\i\,v\in V_{\!\Cbb}:\Dvertsq{u}+\Dvertsq{v}\leqslant1}=\Bra{Su+\i\,Sv:u,v\in V,\Dvertsq{u}+\Dvertsq{v}\leqslant 1}.$\parSol{}
We show each eigval of $S_{\Cbb}$ has abs val $1.$ Then done by \TIPS.\parSol{}
Asum $\aMid{\lambda}<1$ and $\lambda$ is eigval of $S_{\Cbb}$ with $u+\i\,v$ and $\Dvertsq{u}+\Dvertsq{v}=1.$\parSol{}
Then $S_{\Cbb}\Sbra{\lambda^{-1}\Par{u+\i\,v}}=u+\i\,v\in\bigodot{_{\!\Cbb}}$ while $\lambda^{-1}\Par{u+\i\,v}\not\in\bigodot{_{\!\Cbb}}.$ Ctradic.\PfEnd
\SepLine
\ChEnd

%\Anchor{7D15}\ProblemN{15}{
%	\TextA{Supp $T\in\Lm{V}$ unit and $\Par{T-I}$ inv. Prove $\Par{T+I}\Par{T-I}{^{-1}}$ is skew.}
%}$\Sbra{\Par{T+I}\Par{T-I}{^{-1}}}{^*}=\Par{T^{-1}+I}\Par{T^{-1}-I}{^{-1}}=-\Par{T+I}\Par{T-I}{^{-1}}$\parSol{}
%$\Longleftrightarrow\Par{T^{-1}+I}\Par{T-I}=T-T^{-1}=-\Par{T^{-1}-T}=-\Par{T^{-1}-I}\Par{T+I}.$\PfEnd
%\SepLine

%\Anchor{7D16}\ProblemN{16}{
%	\TextA{Supp $\Fbb=\Cbb,T\in\Lm{V}$ self-adj. Prove $1$ not eigval of the unit $\Par{T+\i\:\!I}\Par{T-\i\:\!I}{^{-1}}.$}
%}(a) \NOTICE that $\forall\;\!$inv $S\in\Lm{V},\exists\,q_1,q_2\in\PoFi,\,S^{-1}=q_1\Par{S+\i I}=q_2\Par{S-\i I}.$\parSol{\Ha}
%Write $\Par{T+\i\:\!I}\Par{T-\i\:\!I}{^{-1}}v=\Par{T-\i\:\!I}{^{-1}}\Par{T+\i\:\!I}\;\!v=v\Rightarrow v=0.$\vspace{2pt}\parSol{}
%(b) $\Sbra{\Par{T+\i\:\!I}\Par{T-\i\:\!I}{^{-1}}}{^*}=\Par{T+\i\:\!I}{^{-1}}\Par{T-\i\:\!I}=\Sbra{\Par{T+\i\:\!I}\Par{T-\i\:\!I}{^{-1}}}{^{-1}}.$\vspace{2pt}\parSol{\Hb}
%\Or Let orthon $B_V=\Par{e_1,\dots,e_n}$ be eigvecs of $T$ wrto $\lambda_1,\dots,\lambda_n\in\Rbb.$\parSol{\Hb}
%Then each $\Par{T-\i\:\!I}{^{-1}}\Par{T+\i\:\!I}\;\!e_k=\Par{\lambda_k-\i}{^{-1}}\Par{\lambda_k+\i}\;\!e_k.$\PfEnd
%\SepLine
\pagebreak

\ChDecl{Ch7E}{7.E [4E] \& 7.F [4E]}{\quad{\ANote {\FontSmall $V,W$ are finide {\tgbfxx non0} vecsps over $\Fbb.$}}}

\vspace{6pt}

%\Anchor{7E2}\ProblemN{E.2}{
%	\TextA{Supp $T\in\Lm{V,W},s>0.$}
%	\TextA{Prove $s$ is singval $\Rightarrow\exists$ non0 $v\in V,w\in W$ suth $Tv=sw,\,T^*w=sv.$}
%}Supp $s^2$ is eigval of $T^*T$ with eigvec $v.$ Let $Tv\big/s=w\Rightarrow T^*w=sv.$\PfEnd
%\AComm If $s=0.$ Let $V=\FbbP{2}.$ Let 
%\SepLine

\Anchor{7E1}\ProblemN{E.1}{
	\TextA{Supp $T\in\Lm{V,W}.$ Show $T=0\Longleftrightarrow$ all singvals are $0.$}
}(a) $T=0\Longleftrightarrow T^*=0\Longleftrightarrow T^*T=0\Rightarrow$ all singvals are $0.$\parSol{}
(b) all singvals are $0\Longleftrightarrow T^*T$ nilp. Becs $T^*T$ diag $\Rightarrow T^*T=0=T.$\PfEnd\vspace{2pt}\parSol{}
\Or Supp $T$ has $N$ posi singvals. Now $N=0\Longleftrightarrow\dim\range T=0\Longleftrightarrow T=0.$\PfEnd
\SepLine

\Anchor{7E4}\ProblemN{E.4}{
	\TextA{Supp $T\in\Lm{V,W},$ and $s_1,s_n$ are the max and min of singvals.\vspace{1pt}}
	\TextA{Prove $\Bra{\Dvert{Tv}:v\in V,\Dvert{v}=1}=\Interval{[}{]}{s_n,\,s_1}.$\vspace{2pt}}
}Get the SVD $\Par{e_1,\dots,e_m},\Par{\,f_1,\dots,f_m}$ in $V,W.$ \,We show $\forall s\in\;\!\!\Interval{[}{]}{s_n,\,s_1},\exists\,v\in V,\,\Dvert{Tv}=s.$\vspace{1pt}\parSol{}
Say $v=x\:\!e_1+y\:\!e_n$ with (I) $\Dvertsq{v}=x^2+y^2=1,$ (II) $\Dvertsq{Tv}=s_1^2\,x^2+s_n^2\,y^2=s^2.$\vspace{1pt}\parSol{}
If $s_1=s_n.$ Done. \,Supp $s_1>s_n.$ Then \,$s_1^2-s^2=\Par{s_1^2-s_n^2}\,y^2,$ \,and \:$s^2-s_n^2=\Par{s_1^2-s_n^2}\,x^2.$\PfEnd\vspace{3pt}
\AComm $T$ is a scalar multi of an isomet $\Longleftrightarrow\Bra{\Dvert{Tv}:v\in V,\Dvert{v}=1}=\Bra{s_1}.$
%Supp $s_1>s_n.$ $\forall v\in V,\,\Dvertsq{Tv}=s_1^2\,\innerAsq{v,\:\!e_1}+\dots+s_m^2\,\innerAsq{v,\:\!e_m}\leqslant s_1^2\:\!\BigPar{\innerAsq{v,\:\!e_1}+\dots+\innerAsq{v,\:\!e_m}}.$\vspace{1pt}\parSol{}
%By Bessel's Inequa, $\Dvertsq{Tv}\leqslant s_1^2\,\Dvertsq{v}.$ \;Thus \,$LHS\subseteq\Interval{[}{]}{0,\,s_1}.$\vspace{2pt}\parSol{}
%If $s_n>0\Rightarrow \dim V=n=m.$ Then $\Dvertsq{Tv}\geqslant s_1^2\:\!\BigPar{\innerAsq{v,\:\!e_1}+\dots+\innerAsq{v,\:\!e_m}}=s_1^2\,\Dvertsq{v}.$\parSol{}
%Supp $s\in\Interval{[}{]}{s_n,\,s_1}.$ We show $\exists\,v\in V$ with $\Dvert{v}=1$ suth $\Dvert{Tv}=s.$\PfEnd
\SepLine

%\Anchor{7E8}\ProblemN{8}{
%	\TextA{Supp $T\in\Lm{V,W},s_1\geqslant\cdots\geqslant s_m>0,$ and $\Par{e_1,\dots,e_m},\Par{\,f_1,\dots,f_m}$ orthon in $V,W.$}
%	\TextA{$\forall v\in V,\:Tv=s_1\:\!\Ang{v,\:\!e_1}\,f_1+\dots+s_m\:\!\Ang{v,\:\!e_m}\,f_m.$ Prove $s_1,\dots,s_m$ are the posi singvals of $T.$}
%}Becs $\range T=\Span{\,f_1,\dots,f_m}\Rightarrow T$ has $\dim\range T=m$ posi singvals.\PfEnd\vspace{2pt}\parSol{}
%\Or $\forall\Par{v,w}\in V\times W,\:\Ang{Tv,w}=\sum s_j\:\!\Ang{v,\:\!e_j}\:\!\Ang{\,f_j,w}=\Ang{v,T^*w}\Rightarrow T^*w=\sum s_j\:\!\Ang{w,\,f_j}\,e_j.$\parSol{}
%$\Rightarrow T^*Tv=\sum s_j^2\,\Ang{v,\:\!e_j}\,e_j.$ \;Get an orthon $B_V=\Par{e_1,\dots,e_m,\,g_1,\dots,e_p}\Rightarrow T^*Te_k=s_k^2\:\!e_k,T^*Tg_j=0.$\PfEnd
%\SepLine

\Anchor{7E11}\ProblemN{E.11}{
	\TextA{Supp $T\in\Lm{V}$ is posi, $B_V=\Par{v_1,\dots,v_n}$ is orthon, and $s_1,\dots,s_n$ are the singvals.\vspace{1pt}}
	\TextA{Prove $\Ang{Tv_1,\;\!v_1}+\dots+\Ang{Tv_n,\;\!v_n}=s_1+\dots+s_n.$\vspace{2pt}}
}$\Ang{Tv_k,\;\!v_k}=\BigAng{\sqrt{T}v_k,\:\!\sqrt{T}v_k}=\BigDvertsq{\;\!\!\sqrt{T}v_k}.$ Note that $\sqrt{T}=\sqrt{T^*}=\BigPar{\!\!\:\sqrt{T}}{^*}$ is posi.\parSol{}
\NOTICE that \,$s_1,\dots,s_n$ are the eigvals of $\sqrt{T}{^*}\sqrt{T}=T\Longrightarrow\sqrt{s_1},\dots,\sqrt{s_n}$ are the singvals of $\sqrt{T}.$\vspace{1pt}\parSol{}
Get the SVD $\Par{e_1,\dots,e_n},\Par{\,f_1,\dots,f_n}.$ By (4E 7.A.5), $\uline{\sum\BigDvertsq{\;\!\!\sqrt{T}v_k}=\sum\BigDvertsq{\;\!\!\sqrt{T}e_k}=\sum s_k}.$\PfEnd
%Becs $A=\Mt{T}=\Mt{T}{^*}$ wrto $B_V,$ and each $\Ang{Tv_k,\:\!v_k}=A_{k,\;\!k}=\lambda_k\Rightarrow T$ diag wrto $B_V.$\parSol{}
%Becs $T$ posi $\Rightarrow\aMidsq{\lambda_k}=\lambda_k^2.$ 又 $T^*Tv_k=T^2v_k=\lambda_k^2\:\!v_k=s_k^2\:\!v_k.$\PfEnd\vspace{2pt}\parSol{}
\SepLine

%\Anchor{7E13}\ProblemN{13}{
%	\TextA{Prove $T_{\!1},T_{\!2}\in\Lm{V,W}$ same singvals $\Longleftrightarrow\exists$ unit $S_1\in\Lm{W},S_2\in\Lm{V},\:T_{\!1}=S_1T_{\!2}S_2.$}
%}(a) Get the SVD $\Par{e_k}{_{k=1}^p},\Par{\,f_k}{_{k=1}^p}$ and $\Par{g_k}{_{k=1}^p},\Par{h_k}{_{k=1}^p}$ for $T_{\!1},T_{\!2}$ respectly.\parSol{\Ha}
%Extend up to orthon bss with $\dim V=n,\dim W=m.$\, Define $S_2\Par{e_k}=g_k,\,S_1\Par{h_k}=f_k.$\parSol{\Ha}
%Thus $\forall v=\sum\Ang{v,\:\!e_k}\,e_k\in V,\:T_{\!2}S_2v=\sum s_k\:\!\Ang{v,\:\!e_k}\,h_k\Rightarrow S_1T_{\!2}T_2=\sum s_k\:\!\Ang{v,\:\!e_k}\,f_k=T_{\!1}.$\vspace{2pt}\parSol{}
%(b) $T_{\!1}^*T_{\!1}=S_2^*T_{\!2}^*T_{\!2}S_2\Rightarrow T_{\!1}^*T_{\!1}-\lambda I=S_2^*\Par{T_{\!2}^*T_{\!2}-\lambda I}S_2\Rightarrow\dim E\Par{\lambda,T_{\!1}^*T_{\!1}}=\dim E\Par{\lambda,T_{\!2}^*T_{\!2}}.$
%\SepLine

\Anchor{7E17}\ProblemN{E.17}{
	\TextA{Supp $T\in\Lm{V}.$ Prove $T$ self-adj $\Longleftrightarrow T^\dagger$ self-adj.\hfill\FontNorm\tgnr By Exe (E.16), immed.}
}Let $\lambda_1,\dots,\lambda_m$ be disti eigvals of $T$ with $\lambda_1=0$ if any. Let $U=\Par{\null T}{^\perp}.$\parSol{}
$m_T\Par{z}=\Par{z-\lambda_1}\cdots\Par{z-\lambda_m}\Longleftrightarrow m_{T|_U}=m_T$ if $T$ inje, othws $m_{T|_U}\Par{z}=\Par{z-\lambda_2}\cdots\Par{z-\lambda_m}$\parSol{}
$\Longleftrightarrow$ the min of $\Par{T\mmid_U}{^{-1}}$ is $\Par{z-\lambda_1^{-1}}\cdots\Par{z-\lambda_m^{-1}}$ if $T$ inje, and is $\Par{z-\lambda_2^{-1}}\cdots\Par{z-\lambda_m^{-1}}$ othws\parSol{}
$\Longleftrightarrow m_{T^\dagger}$ is the min of $\Par{T\mmid_U}{^{-1}}$ if $T$ inje, othws $m_{T^\dagger}\Par{z}=z\Par{z-\lambda_2^{-1}}\cdots\Par{z-\lambda_m^{-1}}.$\PfEnd
\SepLine

\ChDecl{Ch7F}{}{}

%\Anchor{7F1}\ProblemN{1}{
%	\TextA{Prove $\aXMid{\Dvert{S}-\Dvert{T}}\leqslant\Dvert{S-T}$ for $S,T\in\Lm{V,W}.$}
%}$\Dvert{T}=\Dvert{S-T-S}\leqslant\Dvert{S-T}+\Dvert{S}$ \,and\, $\Dvert{S}=\Dvert{S-T+T}\leqslant\Dvert{S-T}+\Dvert{T}.$\PfEnd\vspace{3pt}\parSol{}
%\Or Let $v\in V$ be suth $\Dvert{\Par{S-T}v}=\Dvert{S-T}\Rightarrow\aXMid{\Dvert{Sv}-\Dvert{Tv}}\leqslant\Dvert{Sv-Tv}.$\parSol{}
%Supp $\Dvert{Sv}\geqslant\Dvert{Tv}\Rightarrow\aXMid{\Dvert{Sv}-\Dvert{Tv}}\leqslant\Dvert{S}-\Dvert{Tv}.$ Becs $-\Dvert{Tv}\geqslant-\Dvert{T}.$\parSol{}
%Now 
%\SepLine

\Anchor{7F3}\ProblemN{F.3}{
	\TextA{Supp $T\in\Lm{V,W}$ and $v\in V.$ Prove $\Dvert{Tv}=\Dvert{T}\,\Dvert{v}\Longleftrightarrow T^*Tv=\Dvertsq{T}v.$}
}Let $s_1=\dots=s_j\geqslant\cdots\geqslant s_n$ be the singvals with the SVD bses $\Par{e_1,\dots,e_n},\Par{\,f_1,\dots,f_n}.$\parSol{}
\NOTICE that $\Dvertsq{Tv}=s_1^2\,\innerAsq{v,\:\!e_1}+\dots+s_n^2\,\innerAsq{v,\:\!e_n}=s_1^2\;\!\Dvertsq{v}$\parSol{}
\Blind{\NOTICE that }$\Longleftrightarrow v\in\Span{e_1,\dots,e_j}\Longleftrightarrow T^*Tv=\sum s_i^2\,\Ang{v,\:\!e_i}\,e_i=\sum s_1^2\,\Ang{v,\:\!e_i}\,e_i.$\PfEnd\vspace{4pt}\par\quad
\Or Supp $T^*Tv=\Dvertsq{T}v.$ Then $\Dvertsq{T}\,\Dvert{v}=\Dvert{T^*Tv}\leqslant\Dvert{T}\,\Dvert{Tv}\Rightarrow\Dvert{T}\,\Dvert{v}\leqslant\Dvert{Tv}.$\par\quad
Supp $\Dvert{Tv}=\Dvert{T}\,\Dvert{v}.$ Then $\BigDvertsq{T^*Tv-\Dvertsq{T}v}=\Dvertsq{T^*Tv}+\Dvert{T}{^4}\Dvertsq{v}-2\Real\BigAng{T^*Tv,\:\!\Dvertsq{T}v}\leqslant0.$\par\quad
Becs $\Dvertsq{T^*Tv}\leqslant\Dvertsq{T}\Dvertsq{Tv}=\Dvertsq{T}\Dvertsq{T}\Dvertsq{v},$ and $\BigAng{T^*Tv,\:\!\Dvertsq{T}v}=\BigAng{Tv,\:\!\Dvertsq{T}Tv}=\Dvertsq{T}\Dvertsq{Tv}.$\PfEnd
\SepLine

\Anchor{7F5}\ProblemN{F.5}{
	\TextA{Supp $U$ is finide inner prodsp, $T\in\Lm{U,V},S\in\Lm{U,W}.$ Prove $\Dvert{ST}\leqslant\Dvert{S}\,\Dvert{T}.$}
}Take $v$ suth $\Dvert{ST}=\Dvert{STv}\leqslant\Dvert{S}\,\Dvert{Tv}\leqslant\Dvert{S}\,\Dvert{T}\,\Dvert{v}\leqslant\Dvert{S}\,\Dvert{T}.$\PfEnd\vspace{3pt}\parSol{}
\Or $\Dvert{STv}\leqslant\BigDvert{S${\Large$\frac{\:Tv\:}{\SmallDvert{Tv}}$}$}\,\Dvert{Tv}\leqslant\BigDvert{S${\Large$\frac{\:Tv\:}{\SmallDvert{Tv}}$}$}\,\Dvert{T}\leqslant\Dvert{S}\,\Dvert{T}.$\PfEnd
\SepLine

\Anchor{7F19}\ProblemN{F.19}{
	\TextA{Prove $\Dvert{T^*T}=\Dvertsq{T}.$\hfill\FontNorm\tgnr $\Dvert{T}=\Dvert{\;\!\!\sqrt{T^*T}\;\!}=\sqrt{\Dvert{T^*T}}.$ \,\Or By Exe (F.2), $\Dvert{T^*T}=s_1^2.$\Blind{\quad}\PfEnd}
}$\forall v\in V,\:\Dvert{Tv}=\BigDvert{\;\!\!\sqrt{T^*T}v}\leqslant\BigDvert{\;\!\!\sqrt{T^*T}\;\!}\,\Dvert{v}\Rightarrow\Dvert{T}\leqslant\BigDvert{\;\!\!\sqrt{T^*T}\;\!}.$ 又 $\Dvert{T^*T}\leqslant\Dvert{T^*}\,\Dvert{T}.$\PfEnd\vspace{2pt}\parSol{}
\Or $T=S\sqrt{T^*T},\sqrt{T^*T}=S^*T.$ Simlr.\PfEnd
\SepLine

\Anchor{7F20}\ProblemN{F.20}{
	\TextA{Supp $T\in\Lm{V}$ normal. Prove $\Dvert{T^k}=\Dvert{T}{^k}.$}
}Let $B_V=\Par{e_1,\dots,e_n}$ be orthon eigvecs of $T^*T$ with corres eigvals $s_1^2,\dots,s_n^2.$\parSol{}
Becs $\Par{T^k}{^*}T^ke_j=\Par{T^*T}{^k}e_j=s_j^{2k}e_j.$ \;Now $\max\!\Bra{s_1,\dots,s_n}{^k}=\max\!\Bra{s_1^k,\dots,s_n^k}.$\PfEnd\vspace{3pt}\parSol{}
\Or Note that $\Par{T^*T}{_\Cbb}$ and $T^*T$ have the same eigvals
$\Rightarrow\Dvert{T_{\!\Cbb}}{_\Cbb}=\Dvert{T}.$\parSol{}
Let $B_{V_{\!\Cbb}}=\Par{e_1,\dots,e_n}$ be orthon eigvecs of $T_{\!\Cbb}$ corres eigvals $\lambda_1,\dots,\lambda_n$ of $T^k$ corres $\lambda_1^k,\dots,\lambda_n^k.$\parSol{}
By Exe (F.2), $\Dvert{T_{\!\Cbb}^k}=\max\!\Bra{\aMid{\lambda_j^k}}=\max\!\Bra{\aMid{\lambda_j}}{^k}=\Dvert{T_{\!\Cbb}}{^k}.$\PfEnd\vspace{2pt}\parSol{}
\Or Becs singvals of $T$ are the abs of eigvals in $\Cbb.$ \,Let $T_{\!\Cbb}v=\Dvert{T}v$ with $v\in V_{\!\Cbb}$ and $\Dvert{v}=1.$\parSol{}
Then $T_{\!\Cbb}^kv=\Dvert{T}{^k}v\Rightarrow\Dvert{T}{^k}=\BigDvert{T_{\!\Cbb}^kv}\leqslant\Dvert{T_{\!\Cbb}^k}.$ \;又 By Exe (F.5), $\Dvert{T_{\!\Cbb}^k}\leqslant\Dvert{T_{\!\Cbb}}{^k}$.\PfEnd
\SepLine

%\Anchor{7F26}\ProblemN{F.26}{
%	\TextA{Supp $T\in\Lm{V}.$ Prove $\exists\,!$ unit $S\in\Lm{V},\,T=S\sqrt{T^*T}\Rightarrow T$ inv.}
%}
%\SepLine


%\Anchor{7F27}\ProblemN{F.27}{
%	\TextA{Supp $T\in\Lm{V},s_1,\dots,s_n$ are singvals with the SVD bses $\Par{e_1,\dots,e_n},\Par{\,f_1,\dots,f_n}.$}
%	\TextA{Define $S\in\Lm{V}:e_k\mapsto f_k.$ \;Show for all $E\in\Lm{V}$ unit, $\Dvert{T-E}\geqslant\Dvert{T-S}.$}
%}We show $\exists\,v\in V,\,\Dvert{Tv-Ev}\geqslant\Dvert{T-S}=\max\!\Bra{\aMid{s_1-1},\dots,\aMid{s_n-1}}=\aMid{s_M-1}.$\parSol{}
%Note that $\Dvert{Tv-Ev}\geqslant\aXMid{\Dvert{Tv}-1}$ for $\Dvert{v}=1.$ \;Let $v=e_M\Rightarrow\Dvert{T-E}\geqslant\aMid{s_M-1}.$\PfEnd
%\SepLine

\Anchor{7F28}\ProblemN{F.28}{
	\TextA{Supp $T\in\Lm{V}.$ Prove $\exists$ unit $S\in\Lm{V},\,T=\sqrt{TT^*}\,S.$\hfill\FontNorm\tgnr Let $T^*=S\sqrt{TT^*}.$\Blind{\quad}\PfEnd}
}Supp $s_1,\dots,s_m$ are the posi singvals of $T$ with SVD $\Par{e_1,\dots,e_m},\Par{\,f_1,\dots,f_m}.$\vspace{1pt}\parSol{}
Becs $T^*Te_k=s_k^2\:\!e_k,\;f_k=Te_k\big/s_k\Rightarrow TT^*\,f_k=TT^*Te_k\big/s_k=s_kTe_k=s_k^2\;f_k\Rightarrow\sqrt{TT^*}\:f_k=s_k\:f_k.$\vspace{1pt}\parSol{}
$\dim E\BigPar{s_k,\sqrt{T^*T}}\leqslant\dim E\BigPar{s_k,\sqrt{TT^*}}.$ Apply revly. Thus $\dim E\BigPar{0,\sqrt{T^*T}}=\dim E\BigPar{0,\sqrt{TT^*}}.$\vspace{1pt}\parSol{}
Get orthon bses of $E\BigPar{0,\sqrt{T^*T}}$ and $E\BigPar{0,\sqrt{TT^*}}.$ Forming two orthon bses of $V.$\parSol{}
Define $S\in\Lm{V}$ by $Se_j=\;\!f_j.$ \quad\ANote The same $S$ in $T=S\sqrt{T^*T}.$\PfEnd\vspace{4pt}
%\vspace{2pt}\parSol{}
%\Or Extend to orthon bses $\Par{e_1,\dots,e_n},\Par{\,f_1,\dots,f_n}.$ Define $Se_j=\;\!f_j\Rightarrow\sqrt{TT^*}\,Se_k=s_k\:f_k=Te_k.$\PfEnd\vspace{4pt}
\Anchor{7F29}\ACoro $T=S\sqrt{T^*T}\Rightarrow T^*=\sqrt{T^*T}\,S^*\Longrightarrow TT^*=ST^*TS^*.$\parCor
\Or $T=S\sqrt{T^*T}=\sqrt{TT^*}\,S\Rightarrow \sqrt{T^*T}=S^*\sqrt{TT^*}\,S\Longrightarrow T^*T=S^*TT^*S.$
\SepLine

%\Anchor{7F30}\ProblemN{F.30}{
%	\TextA{Supp $T\in\Lm{V},$ and $S\in\Lm{V}$ unit suth $ST$ posi. Prove $ST=\sqrt{T^*T}.$}
%}\NOTICE that $T^*T=STT^*S^*=\Par{ST}{^2}.$ \;By the uniqnes.
%\PfEnd\vspace{2pt}\parSol{}
%\Or Let $B_V=\Par{e_1,\dots,e_n}$ be orthon eigvecs of $ST$ with corres eigvals $\lambda_1,\dots,\lambda_n.$\parSol{}
%Becs $ST=T^*S^*\Rightarrow T=S^*T^*S^*.$ \;Let each \:$f_k=S^*e_k\Rightarrow T^*\,f_k=\lambda_k\:\!e_k.$\parSol{}
%Thus $T^*Te_k=T^*S^*STe_k=\lambda_k^2\:\!e_k\Rightarrow\sqrt{T^*T}\,e_k=\lambda_k\:\!e_k=STe_k.$\PfEnd
%\SepLine

\Anchor{7F31}\ProblemN{F.31}{
	\TextA{Supp $T\in\Lm{V}$ self-adj. Or supp $\Fbb=\Cbb$ and $T\in\Lm{V}$ normal.\vspace{1pt}}
	\TextA{Prove $\exists$ unit $S$ with $T=S\sqrt{T^*T}$ suth $S,\sqrt{T^*T}$ diag wrto same orthon bss.}
}Becs $T=S\sqrt{T^*T}=\sqrt{TT^*}\,S=\sqrt{T^*T}\,S.$ \;If $\Fbb=\Cbb,$ then done. But othws\:?\parSol{}
Let $B_V=\Par{e_1,\dots,e_n}$ be orthon eigvecs of $T$ with corres $\lambda_1,\dots,\lambda_n,$ which are real if self-adj.\parSol{}
Note that $T^*e_k=\overline{\lambda_k}e_k\Rightarrow T^*Te_k=\aMidsq{\lambda_k}e_k.$ \;Define $S\in\Lm{V}$ by $Se_k=\aMid{\lambda_k}{^{-1}}\lambda_k\:\!e_k.$\PfEnd
\SepLine

\Anchor{7F8}\ProblemN{F.8}{
	\TextA{Supp $S\in\Lm{V}$ inv. Prove if $T\in\Lm{V}$ and $\Dvert{S-T}<1\big/\Dvert{S^{-1}},$ then $T$ inv.}
}Note that $1\big/\Dvert{S^{-1}}=s_n,$ where $s_1\geqslant\cdots\geqslant s_n$ are the singvals of $T.$\parSol{}
Becs $s_n=\min\!\Bra{\Dvert{S-T}:\dim\range T=0,1,\dots,n-1}>\Dvert{S-T}\Rightarrow\dim\range T=n.$\PfEnd\vspace{2pt}\parSol{}
\Or $v\neq0,{}$\uline{$Tv=0\Rightarrow\Dvert{v}=\Dvert{S^{-1}Sv}$}${}\leqslant\Dvert{S^{-1}}\:\Dvert{Sv}={}$\uline{$\Dvert{S^{-1}}\:\Dvert{\Par{S-T}v}$}.\PfEnd
\SepLine

\Anchor{7F14}\ProblemN{F.14}{
	\TextA{Supp $U,W$ subsps of $V$ suth $\Dvert{P_{\!U}-P_{\!W}}<1.$ Prove $\dim U=\dim W.$}
}Note that $1=s_m=\min\!\Bra{\Dvert{P_{\!U}-T}:T\in\Lm{V}\text{ and }\dim\range T=0,1,\dots,m-1}.$\parSol{}
Thus $\dim\range P_{\!W}\geqslant\dim\range P_{\!U}.$ \;Apply revly, done.\PfEnd\vspace{2pt}\parSol{}
\Or Becs $P_{\!U}=I-P_{\!U^\perp}$ \OR $P_{\!W}=I-P_{\!W^\perp}\Rightarrow P_{\!W}-P_{\!U}=P_{\!W}+P_{\!U^\perp}-I$\, \OR \,$P_{\!U}-P_{\!W}=P_{\!U}+P_{\!W^\perp}-I.$\parSol{}
\uline{By Exe (F.8), $P_{\!W}+P_{\!U^\perp}$ and $P_{\!U}+P_{\!W^\perp}$ are inv} $\Rightarrow\zeroSubs=\Null\Par{P_{\!W}+P_{\!U^\perp}}\supseteq\null P_{\!W}\cap\null P_{\!U^\perp}.$\parSol{}
And $V=\Range\Par{P_{\!U}+P_{\!W^\perp}}\subseteq\range P_{\!U}+\range P_{\!W^\perp}=U+W^\perp.$\PfEnd
\SepLine

\Anchor{7F22}\ProblemN{F.22}{
	\TextA{Supp $T\in\Lm{V,W}.$ Let $n=\dim V$ and $s_1\geqslant\cdots\geqslant s_n$ are the singvals.}
	\TextA{Prove $s_{n-k+1}=\min\!\Bra{\Dvert{T\mmid_U}:U\text{ is subsp of }V,\,\dim U=k}$ \,for $1\leqslant k\leqslant n.$\vspace{1pt}}
}Get the SVD bses $\Par{e_1,\dots,e_n},\Par{\,f_1,\dots,f_n}.$ Let $U=\Span{e_{n-k+1},\dots,e_n}\Rightarrow\Dvert{T\mmid_U}=s_{n-k+1}.$\vspace{1pt}\parSol{}
Supp $U$ is a $k$\hspace{1pt}-\hspace{1pt}dim subsp of $V.$ \,\NOTICE that $\uline{\Dvert{T\mmid_U}=\BigDvert{T\mmid_UP_{\!U}}=\BigDvert{T-T\mmid_{U^\perp}P_{\!U^\perp}}}\geqslant s_{n-k+1}.$\PfEnd
%$\Dvert{T\mmid_U}\leqslant\Dvert{T\mmid_UP_{\!U}}=\Dvert{TP_{\!U}}=\Dvert{T-T\mmid_{U^\perp}P_{\!U^\perp}}.$
\SepLine\pagebreak

\Anchor{7F13}\ProblemN{F.13}{
	\TextA{Supp $S,T\in\Lm{V}$ are posi. Show $\Dvert{S-T}\leqslant\max\!\Bra{\Dvert{S},\Dvert{T}}\leqslant\Dvert{S+T}.$}
}$\Dvert{S+T}=\BigDvertsq{\;\!\!\sqrt{S+T}\;\!}\geqslant\BigDvertsq{\;\!\!\sqrt{S+T}\,v}=\Ang{Sv+Tv,v}=\Ang{Sv,v}+\Ang{Tv,v}=\BigDvertsq{\;\!\!\sqrt{S}\,v}+\BigDvertsq{\;\!\!\sqrt{T}\,v}.$\parSol{}
Let $v\in V$ suth $\BigDvertsq{\;\!\!\sqrt{S}\,v}=\BigDvertsq{\;\!\!\sqrt{S}\;\!}=\Dvert{S}\Rightarrow\Dvert{S+T}\geqslant\Dvert{S}.$ \;Simlr, $\Dvert{S+T}\geqslant\Dvert{T}.$\vspace{2.5pt}\parSol{}
%\Or By \TIPS, $S,T,S+T-T,S+T-S$ posi $\Rightarrow$ $\Dvert{T}\leqslant\Dvert{S+T-T+T},$ and $\Dvert{S}\leqslant\Dvert{S+T-S+S}.$\vspace{3pt}\parSol{}
Denote $\,\max\!\Bra{\aMid{\lambda}:\lambda\text{ is eigval of }T}\,$ by $\,\aMid{\lambda_M}{_T}.$ We show $\aMid{\lambda_M}{_{S-T}}\leqslant\max\!\Bra{\aMid{\lambda_M}{_S},\aMid{\lambda_M}{_T}}.$\parSol{}
Note that $\Dvert{T}\:\!I-T$ is posi, so is $R=\Dvert{T}I-T+S=\Dvert{T}\:\!I-\Par{T-S}.$\parSol{}
Thus $\Dvert{T}-\aMid{\lambda_M}{_R}=\aMid{\lambda_M}{_{S-T}}\leqslant\Dvert{T}.$ \;Simlr, $\Dvert{T}-\aMid{\lambda_M}{_{\,\SmallDvert{S}\:\!I\,-\,\SmallPar{S-T}}}=\aMid{\lambda_M}{_{T-S}}\leqslant\Dvert{S}.$\PfEnd
\SepLine

%\Anchor{7F9}\ProblemN{F.9}{
	%	\TextA{Supp $T\in\Lm{V}.$ Prove $\forall\epsilon>0,\exists$ inv $S\in\Lm{V}$ suth $0<\Dvert{T-S}<\epsilon.$}
	%}Supp $Tv=\sum s_k\:\!\Ang{v,\:\!e_k}\:f_k.$ Define $Sv=\sum\Par{s_k-\delta}\:\!\Ang{v,\:\!e_k}\:f_k,$ where $\delta\in\Interval{(}{)}{0,\epsilon}.$\PfEnd
%\SepLine

%\Anchor{7F10}\ProblemN{F.10}{
	%	\TextA{Supp $\dim V>1,T\in\Lm{V}$ not inv.}
	%	\TextA{Prove $\forall\epsilon>0,\exists$ non-inv $S\in\Lm{V}$ suth $0<\Dvert{T-S}<\epsilon.$}
	%}Supp $Tv=\sum_{k=1}^n s_k\:\!\Ang{v,\:\!e_k}\:f_k.$ Define $Sv=\sum_{k=1}^{n-1}\Par{s_k-\delta}\:\!\Ang{v,\:\!e_k}\:f_k,$ where $\delta\in\Interval{(}{)}{0,\epsilon}.$\PfEnd
%\SepLine

\Anchor{7F11}\ProblemN{F.11}{
	\TextA{Supp $\Fbb=\Cbb,T\in\Lm{V}.$ Prove $\forall\epsilon>0,\exists$ diag $S$ suth $0<\Dvert{T-S}<\epsilon.$}
	%	We show $\exists\,R\in\Lm{V}$ suth $T+R$ has $\dim V$ disti eigvals, and $0<\Dvert{R}<\epsilon.$\parSol{}
}Supp $A=\Mt{T}$ up-trig wrto orthon $B_V=\Par{e_1,\dots,e_n}.$ Let $\lambda_k=A_{k,\;\!k}.$\parSol{}
Becs $T$ has finily many eigvals, $\exists\,\delta\in\Interval{(}{)}{0,\:\!\epsilon\big/n},$ $\nexists$ disti $j,k$ suth $\lambda_j+j\delta=\lambda_k+k\delta.$\parSol{}
Thus define $Re_k=k\delta\:\!e_k\Rightarrow T+R$ has $n$ disti eigvals, while $R$ posi and $0<\Dvert{R}<\delta.$\PfEnd
\SepLine

\Anchor{7F16}\ProblemN{F.16}{
	\TextA{Supp $S\in\Lm{V}$ posi inv. Prove $\exists\,\delta>0,\forall\:\!$self-adj $T$ suth $\Dvert{S-T}<\delta,$ \,$T$ posi.}
}Let $\delta$ be the smallest singval of posi inv $S.$ Then $\forall v\in V,\Ang{Sv,v}=\Dvertsq{\;\!\!\sqrt{S}v}\geqslant\delta\:\!\Dvertsq{v}.$\parSol{}
Supp $T$ self-adj and $0<\Dvert{S-T}<\delta=\min\!\Bra{\Dvert{S-R}:\dim\range R\leqslant\dim V-1}\Rightarrow T$ inv.\vspace{1pt}\parSol{}
Then $\aXMid{\Ang{\Par{S-T}v,v}}\leqslant\Dvert{S-T}\:\Dvertsq{v}<\delta\:\!\Dvertsq{v}$ \,for all non0 $v\in V.$\vspace{2pt}\parSol{}
Asum $\exists\,v\in V,\Ang{Tv,v}<0\Rightarrow\delta\:\!\Dvertsq{v}\leqslant\aXMid{\Ang{Sv,v}}\leqslant\aXMid{\Ang{Sv-Tv,v}}<\delta\:\!\Dvertsq{v}.$\vspace{1pt}\parSol{}
\Or $\Ang{Tv,v}=\Ang{Sv,v}+\BigAng{\Par{T-S}v,v}\geqslant\delta\:\!\Dvertsq{v}-\BigAng{\Par{S-T}v,v}>0$ for $v\neq0.$\PfEnd
\SepLine
\ChEnd\pagebreak

{\small 少了关于无限维内积空间及其算子的基本内容,LADR第七章学起来明显更快更平常,没有前半部分许多题目去掉有限维假设后的悬疑和激动时刻。也许研究无限维内积空间有硬性的分析学知识门槛,所以这本书的性质不允许过多涉及这方面。
%同样遗憾的是,由于微积分的知识门槛,我放弃了第10章B节最后一小部分和相应习题(这部分在4E中被删掉了,所以我可以找借口说它是3E中不完善的地方)。
}\vspace{12pt}

\ChDecl{Ch10A}{10.A}{\quad{\ANote {\FontSmall $V$ denotes a finide non0 vecsp over $\Fbb.$ \qquad For (10.B), see [4E] Chapter 9.}}}%An Exe marked by $\blacksquare$ is true if infinide or partially finide.

\vspace{4pt}

%\Anchor{10A12}\ProblemN{12}{
%	\TextA{Supp $V$ is inner prodsp, $P=P_{\!U},$ and $S\in\Lm{V}$ is posi. Prove $\Tr\Par{SP}\geqslant 0.$}
%}Let orthon $B_U=\Par{e_1,\dots,e_m},B_{U^\perp}=\Par{\,f_1,\dots,f_n}.$\parSol{}
%Becs $\Tr\Par{SP}=\sum\!\!\:\Ang{SPe_i,e_i}+\sum\!\!\:\Ang{SP\:\!f_j,\,f_j}=\sum\!\!\:\Ang{Se_i,e_i}\geqslant0.$\PfEnd
%\SepLine

%\Anchor{10A'1}\ProblemB{
%	\TextA{Supp $\dim V=n,$ and $S,T\in\Lm{V}$ are posi inv. Prove $\Tr\Par{ST}\geqslant0.$}
%}Let $A=\Mt{S},B=\Mt{T}$ wrto the same bss.\parSol{}
%Let $A=Q^*Q,B=R^*R,$ where the up-trig $Q,R$ have only posi ents on diag.
%\SepLine

\Anchor{10A17}\ProblemN{17}{
	\TextA{Supp $T\in\Lm{V}$ suth \:\!$\Tr\Par{ST}=0$ \,for all $S\in\Lm{V}.$ \,Prove $T=0.$}
}Let $S=T^*\Rightarrow\Tr\Par{T^*T}=s_1^2+\dots+s_n^2=0.$ \;By (4E 7.E.1) \OR Exe (11).\PfEnd\vspace{2pt}\parSol{}
\Or Asum $T\neq0\Rightarrow\exists$\;\!non0 $v\in V$ suth $Tv\neq0.$ Extend $v=v_1$ to $B_V=\Par{v_1,\dots,v_n}.$\parSol{}
Supp $Tv_1=A_{1,1}v_1+\dots+A_{n,1}v_n\Rightarrow\exists\,A_{j,1}\neq0.$ Define $S\in\Lm{V}$ by each $Sv_k=\delta_{j,\:\!k}\:\!v_1.$\parSol{}
Now $S=E_{j,1}\Rightarrow\Mt{ST}={\mEnt{1,\,j}\Mt{T}}\Longrightarrow0=\Tr\Par{ST}=A_{j,1}.$ Ctradic.\PfEnd\vspace{2pt}\parSol{}
\Or Asum $\exists\,Tv\neq0.$ Extend to $B_V=\Par{Tv,u_1,\dots,u_n}.$ Define $S\in\Lm{V}:Tv\mapsto v,\,u_i\mapsto 0.$\parSol{}
Now $\Par{TS}\Par{Tv}=Tv,\Par{TS}\Par{u_i}=0\Rightarrow\Tr\Par{TS}=1.$ Ctradic.\PfEnd
\SepLine

\Anchor{10A4e10}\ProblemBnoor{4E 8.D.10}{
	\TextB{Supp $\tau\in\BigPar{\Lm{V}}\apostrophe,\tau\Par{I}=\dim V,$ and $\tau\Par{ST}=\tau\Par{TS}.$ Prove $\tau=\Tr.$}
}$\tau\Par{E_{i,\;\!j}}=\tau\Par{E_{x,\;\!j}E_{i,\;\!x}}=\tau\Par{E_{i,\;\!x}E_{x,\;\!j}}=\delta_{i,j}\tau\Par{E_{x,\;\!x}}.$ \,又 $I=E_{1,1}+\dots+E_{n,n}\Rightarrow n=n\tau\Par{E_{x,\:\!x}}.$\PfEnd
\SepLine

\Anchor{10A4e12}\ProblemBnoor{4E 8.D.12}{
	\TextB{Supp $V,W$ are finide inner prodsps. Define $\Ang{\cdot,\cdot}$ on $\Lm{V,W}$ by $\Ang{S,T}=\Tr\Par{ST^*}.$\vspace{2pt}}
	\TextB{Let orthon $B_V=\Par{e_1,\dots,e_n},B_W=\Par{\,f_1,\dots,f_m}.$ Show $\Ang{S,T}=\sum_{j=1}^m\sum_{k=1}^n\Mt{S}{_{j,\;\!k}}\overline{\Mt{T}{_{j,\;\!k}}}.$\vspace{3pt}}
}Becs $\Mt{ST^*}{_{j,\;\!k}}=\Mt{S}{_{j,\;\!r}}\Mt{T^*}{_{r\!\!\:,\;\!k}}=\Mt{S}{_{j,\;\!r}}\overline{\Mt{T}{_{k,\;\!r}}}.$ Take $k=j.$\PfEnd\vspace{4pt}\parSol{}
\Or Define $E_{k,\;\!j}\Par{e_x}=\delta_{k,\;\!x}\:f_j,\;R_{j,\;\!k}\Par{\,f_y}=\delta_{j,\;\!y}e_k,$ and $Q_{j,\;\!k}\Par{\,f_y}=\delta_{j,\;\!y}\:f_k.$ \;See {\NOTEFOR} [3.60].\vspace{2pt}\parSol{}
%Supp $T=\sum_{j=1}^m\sum_{k=1}^nA_{j,\;\!k}E_{k,\;\!j}\in\Lm{V,W}\Rightarrow A=\Mt{T}$ wrto $B_V,B_W.$\vspace{2pt}\parSol{}
Becs \,$E_{k,\;\!j}^*=R_{j,\;\!k},$ and $\Ang{E_{k,\;\!l},E_{i,\;\!j}}=\Tr\Par{E_{k,\;\!l}R_{j,\;\!i}}=\Tr\Par{\delta_{i,\:\!k}Q_{j,\;\!l}}=\delta_{i,\:\!k}\delta_{j,\:\!l}.$\vspace{2pt}\parSol{}
%Hence $B_{\!\Lm[\SmallPar]{V,\,W}}=\Bra{E_{k,\;\!j}:1\leqslant k\leqslant n,\text{ and }1\leqslant j\leqslant m}$ is orthon.\vspace{3pt}\parSol{}
Now $\BigAng{{\sum A_{j,\;\!k}E_{k,\;\!j}\;\!,\;\sum C_{x,\;\!y}E_{y,\;\!x}}}=\sum A_{j,\;\!k}\overline{C_{j,\;\!k}}$\PfEnd
%=\Tr\BigPar{{\sum A_{j,\;\!k}E_{k,\;\!j}\:\sum \overline{C_{x,\;\!y}}R_{x,\;\!y}}}.
%Becs $\sum A_{j,\;\!k}E_{k,\;\!j}\:\sum \Par{C^*}{_{y,\;\!x}}R_{x,\;\!y}=\sum\sum A_{j,\;\!k}\Par{C^*}{_{y,\;\!x}}\,\delta_{k,\;\!y}\,Q_{x,\;\!j}=\sum\sum A_{j,\;\!k}\Par{C^*}{_{k,\;\!x}}\,Q_{x,\;\!j}.$ Take $x=j.$\PfEnd
\SepLine


\vfill\ChDecl{Ch4e9}{[4E] 9}{\quad{\ANote {\FontSmall $V,W$ denote finide non0 vecsps over $\Fbb.$}}}%An Exe marked by $\blacksquare$ is true if infinide or partially finide.

\vspace{6pt}

\Anchor{Ch4e9A}
\Anchor{4e9A2}\ProblemN{A.2}{
	\TextA{Supp $\beta$ is biliney on $V$ of dim $n.$ Prove $\exists\,\varphi_k,\tau_k\in V\apostrophe,\,\beta\Par{u,v}=\sum_{k=1}^n\varphi_k\Par{u}\;\!\tau_k\Par{v}.$}
}$\forall v\in V,\exists\,\varphi_v\in V\apostrophe$ suth $\varphi_v:\:\!u\mapsto\beta\Par{u,v}.$ \,Define such $\varphi_1,\dots,\varphi_n$ for a $B_V=\Par{v_1,\dots,v_n}.$\parSol{}
$\forall u,v\in V,\,\beta\Par{u,v}=\beta\BigPar{u,\sum a_iv_i}=\sum a_i\,\beta\Par{u,v_i}=\sum a_i\:\!\varphi_i\Par{u}.$ Get the dual bss of $B_V.$\PfEnd
%\Or $\forall u\in V,\,V\apostrophe\ni\tau_u:\:\!v\mapsto\beta\Par{u,v}.$ \,Define such $\tau_1,\dots,\tau_n$ for a $B_V=\Par{u_1,\dots,u_n}.$\parSol{}
%Now $\forall u,v\in V,\,\beta\Par{u,v}=\beta\BigPar{\sum a_iu_i,v}=\sum a_i\,\beta\Par{u_i,v}=\sum a_i\:\!\tau_i\Par{v}.$ Simlr.\PfEnd\vspace{2pt}\parSol{}
\SepLine

%\Anchor{4e9A3}\ProblemN{A.3}{
%	\TextA{Supp $\beta\in\Par{V\times V}\apostrophe$ is biliney. Prove $\beta=0.$}
%}$\beta\Par{u,x}+\beta\Par{v,y}+\beta\Par{u,y}+\beta\Par{v,x}=\beta\Par{u+v,x+y}=\beta\Par{u,x}+\beta\Par{v,y}=\beta\Par{u,y}+\beta\Par{v,x}=0.$\par\quad
%$2\beta\Par{u,v}=\beta\Par{2u,2v}=4\beta\Par{u,v}=0.$ \;\Or $\beta\Par{2u,2v}=\beta\Par{2u,v}=\beta\Par{u,2v}\Rightarrow\beta\Par{0,v}=\beta\Par{u,0}=0.$\PfEnd
%\SepLine

\Anchor{4e9A4}\ProblemN{A.4}{
	\TextA{Supp $V$ is real inner prodsp, $\beta$ is biliney on $V.$ Show $\exists\,!\,T\in\Lm{V},\,\beta\Par{u,v}=\Ang{Tu,v}.$}
}Let orthon $B_V=\Par{e_1,\dots,e_n}\Rightarrow\beta\Par{u,v}=\sum\beta\Par{u,e_i}\Ang{e_i,v}=\BigAng{{\sum\beta\Par{u,e_i}\;\!e_i},\:\!v}.$\parSol{}
Supp $\beta\Par{u,v}=\Ang{Tu,v}=\sum\Ang{Tu,e_i}\:\!\Ang{e_i,v}.$ Then each $\Ang{Tu,e_i}=\beta\Par{u,e_i}\Rightarrow Tu=\sum\beta\Par{u,e_i}\;\!e_i.$\PfEnd
%\vspace{3pt}\parSol{}
%\Or Define $Te_j=\sum\beta\Par{e_j,e_i}\;\!e_i.$ Then $\Ang{Te_j,e_k}=\beta\Par{e_j,e_k}.$\parSol{}
%Thus $\Ang{Tu,v}=\sum a_jb_k\Ang{Te_j,e_k}=\sum a_jb_k\;\!\beta\Par{e_j,e_k}=\beta\Par{u,v}.$\PfEnd
\SepLine

\Anchor{4e9B1}\ProblemN{B.1}{
	\TextA{Supp $m\in\Nbp.$ Show $\dim V^{\SmallPar{m}}=\dim V^m.$}
}Let $B_V=\Par{e_1,\dots,e_n}$ with corres $B_{V\apostrophe}=\Par{\varphi_1,\dots,\varphi_n}.$ Supp $\alpha$ is $m$-liney.\vspace{1pt}\parSol{}
$\forall v_j=\sum_{i=1}^n\varphi_i\Par{v_j}\:\!e_i,\;\alpha\Par{v_1,\dots,v_m}=\sum_{i_1=1}^n\cdots\sum_{i_m=1}^n\,\varphi_{i_1}\:\!\!\Par{v_1}\cdots\:\!\varphi_{i_m}\:\!\!\Par{v_m}\;\alpha\Par{e_{i_1},\dots,e_{i_m}}.$\vspace{2pt}\parSol{}
Define $m$-liney \,$\alpha_I\Par{v_1,\dots,v_m}=\varphi_{i_1}\:\!\!\Par{v_1}\cdots\:\!\varphi_{i_m}\:\!\!\Par{v_m},$ \,for each $I=\Par{i_1,\dots,i_m}\in\Gamma=\Bra{1,\dots,n}{^m}.$\vspace{2pt}\parSol{}
Hence $\alpha=\sum_{I\,\in\,\Gamma}\,\alpha\Par{e_{i_1},\dots,e_{i_m}}\Sbra{\alpha_I}\in\Span{\alpha_I}{_{I\,\in\,\Gamma}}.$ \,Supp $\exists\,c_I\in\Fbb,\,I\in\Gamma$ \,suth\, $\sum_{\,I\,\in\,\Gamma}\,c_I\,\alpha_I=0.$\vspace{2pt}\parSol{}
Note that for each $J=\Par{j_1,\dots,j_m}\in\Gamma\:\!\!,\:\alpha_I\Par{e_{j_1},\dots,e_{j_m}}=\delta_{I,J}\Rightarrow 0\Par{e_{j_1},\dots,e_{j_m}}=c_J.$\PfEnd\vspace{4pt}\parSol{}
\Or We show $V^{\SmallPar{m}}=\mB\Par{V,\cdots,V}$ iso to $\FbbP{n,\dots,n}$ via liney $\alpha\mapsto\Mt{\alpha}.$ Supp $A\in\FbbP{n,\dots,n}.$\parSol{}
Define $\alpha_A\in\mB\Par{V,\cdots,V}$ by $\alpha_A\BigPar{{\sum_{i=1}^nc_{i,1}\:\!e_i,\cdots,\sum_{i=1}^nc_{i,\;\!m}\;\!e_i}}=\sum_{I\,\in\,\Gamma}c_{j_1,1}\cdots c_{j_m,\;\!m}\,A_I.$\vspace{1pt}\parSol{}
Let $\Mt{\alpha}\in\FbbP{n,\dots,n}$ with each $\Mt{\alpha}{_{j_1,\dots,\,j_m}}=\alpha\Par{e_{j_1},\dots,e_{j_m}}.$ \,Becs $\alpha_{\Mt[\SmallPar]{\alpha}}=\alpha.$\PfEnd
\SepLine\pagebreak

%\Anchor{4e9B3}\ProblemN{B.3}{
%	\TextA{Supp $\alpha$ is $m$-liney with $\alpha\Par{v_1,\dots,v_j,v_{j+1},\dots,v_m}=0$ \,for $v_j=v_{j+1}.$ \;Show $\alpha\in V^{\SmallPar{m}}_{\alt}.$}
%}Becs $0=\alpha\Par{v_1,\dots,v_j+v_{j+1},v_j+v_{j+1},\dots,v_m}=\alpha\Par{v_1,\dots,v_{j+1},v_j,\dots,v_m}+\alpha\Par{v_1,\dots,v_j,v_{j+1},\dots,v_m}.$\parSol{}
%Supp $m>2$ and $u_1,\dots,u_m\in V$ suth $\exists\,u_j=u_k$ with $1\leqslant j<k-1\leqslant m-1.$\parSol{}
%Observe $\Par{u_1,\dots,u_j,u_{j+1},\dots,u_k,\dots,u_m}.$ By swapping $\Par{u_j,u_{j+1}},\dots,\Par{u_j,u_{k-1}}$ in order.\PfEnd
%\SepLine

%\Anchor{4e9C2}\ProblemN{C.2}{
%	\TextA{Supp $A\in\FbbP{n,\;\!n},B\in\FbbP{n-1,n-1},\,A_{k,1}=\delta_{k,1},\,B_{j,\;\!k}=A_{j+1,\;\!k+1}.$ \;Show $\det A=A_{1,1}\det B.$}
%}$\det A={\sum_{J\,\in\,\perm n}}\Par{\sign J}\,A_{j_1,1}\cdots A_{j_n,\;\!n},$ with $J=\Par{j_1,\dots,j_n}$ and $j_1=1\Rightarrow j_2,\dots,j_n\neq 1.$\PfEnd\vspace{4pt}\parSol{}
%\Or Let $C={}${\scriptsize$\begin{pmatrix}\SmallPar{1} &\hspace{-6pt} 0\\[-2pt] 0 &\hspace{-6pt} B\end{pmatrix}$}${}\Rightarrow\Det\Par{A_{\cdot,1}\;\cdots\;A_{\cdot,n}}=A_{1,1}\Det\BigPar{I_{\cdot,1}\;\,C_{\cdot,2}\;\,\cdots\;\,C_{\cdot,n}}=A_{1,1}\det C.$\PfEnd
%\SepLine

\Anchor{4e9C2}\Anchor{4e9C5}\ProblemN{C.2,5}{
	\TextA{Supp square $B,C$ on diag of block up-trig $A.$ Prove $\det A=\Par{{\det B}}\Par{{\det C}}.$}
}If the cols of $B$ are liney dep, then the cols of $A$ are liney dep, done. Now supp $B$ is inv.\vspace{1pt}\parSol{}
Let $A={}${\scriptsize$\begin{pmatrix}B &\hspace{-6pt} D\\[-2pt] 0 &\hspace{-6pt} C\end{pmatrix}$}$,\:P={}${\scriptsize$\begin{pmatrix}B &\hspace{-6pt} 0\\[-2pt] 0 &\hspace{-6pt} C\end{pmatrix}$}$,\:Q={}${\scriptsize$\begin{pmatrix}0 &\hspace{-6pt} D\\[-2pt] 0 &\hspace{-6pt} 0\end{pmatrix}$}$.$ Supp $A\in\FbbP{n,\:\!n},B\in\FbbP{m,\:\!m},C\in\FbbP{p,\;\!p}.$\vspace{-2pt}\parSol{}
Now $\Det\Par{A_{\cdot,1}\;\cdots\;A_{\cdot,\;\!n}}=\Det\BigBigPar{P_{\!\cdot,1}\;\cdots\;P_{\!\cdot,\;\!m}\;\,\Par{P+Q}{_{\cdot,\;\!m+1}}\;\cdots\;\Par{P+Q}{_{\cdot,\;\!n}}}.$\vspace{1pt}\parSol{}
Becs each col of $D$ is in $\col B=\FbbP{m}\Rightarrow$ each $Q_{\cdot,m+k}\in\Span{P_{\!\cdot,1},\dots,P_{\!\cdot,\;\!m}}\Rightarrow\det A=\det P.$\PfEnd
%\vspace{4pt}\parSol{}
%\Or Let $\Par{e_1,\dots,e_n}$ be std bses of $\FbbP{n}.$ Let $U=\Span{e_1,\dots,e_m},W=\Span{e_{m+1},\dots,e_n}.$\parSol{}
%Define $T\in\Lm{\FbbP{n}}:e_k\mapsto A_{\cdot,k}\Rightarrow T\mmid_U\in\Lm{U},$ and $\Mt{T\mmid_U}=B.$ \,Define $S\in\Lm{W}:e_{m+j}\mapsto C_{\cdot,j}$\parSol{}
%Let $\alpha$ be a $n$-liney form with $\alpha\Par{e_1,\dots,e_n}=1.$ Define $\beta\Par{v_1,\dots,v_m}=\alpha\Par{v_1,\dots,v_m,Te_{m+1},\dots,Te_n}.$\parSol{}
\SepLine

%\Anchor{4e9C9}\ProblemN{C.9}{
%	\TextA{Supp $V$ is on $\Rbb$ of even dim, $T\in\Lm{V},$ and $\det T<0.$ Prove $T$ has at least two disti eigvals.}
%}Let $\lambda_1,\dots,\lambda_m$ be disti eigvals of $T_{\!\Cbb},$ and each $\lambda_k$ of multy $m_k.$\parSol{}
%Now $\det T=\det T_{\!\Cbb}<0\Rightarrow\exists\,\lambda_k<0,$ and $m_k$ is odd $\Rightarrow\prod_{i\neq k}\Par{z-\lambda_i}{^{m_i}}$ is of odd deg.\PfEnd
%\SepLine

%\Anchor{4e9C16}\ProblemN{C.16}{
%	\TextA{Supp $T\in\Lm{V}.$ Define $g:\Fbb\rightarrow\Fbb$ by $g\Par{x}=\Det\Par{I+xT}.$ Show $g\apostrophe\Par{0}=\tr T.$}
%}Supp $T_{\!\Cbb}$ up-trig wrto $B_V$ and $\lambda_1,\dots,\lambda_n$ are the diag ents.\parSol{}
%Then \,$\det\Mt{I+\delta T_{\!\Cbb},B_V}=\Par{1+\delta\lambda_1}\cdots\Par{1+\delta\lambda_n}=1+\delta\Par{\lambda_1+\dots+\lambda_n}+\cdots+\delta^n\:\!\lambda_1\!\cdots\lambda_n$\PfEnd\vspace{2pt}\parSol{}
%\Or Let $p$ be the char of $-T.$ \;Then $\forall x\in\nonzeroFbb,$\parSol{}
%$\Det\Par{I+xT}=x^n\Det\Par{x^{-1}I+T}=x^np\Par{x^{-1}}=1-\tr\Par{{-T}}\cdot x+\cdots+\Par{{-1}}{^n}\Det\Par{{-T}}\cdot x^n.$\PfEnd
%\SepLine

%\Anchor{4e9C19}\ProblemN{C.19}{
%	\TextA{Supp $V$ is inner prodsp, $B_V=\Par{e_1,\dots,e_n}$ is orthon, and $T\in\Lm{V}$ is posi.\vspace{1pt}}
%	\TextA{Prove $\det T\leqslant\prod_{k=1}^n\Ang{Te_k,e_k},$ and if $T$ inv, then equa $\Longleftrightarrow B_V$ are eigvecs.\vspace{3pt}}
%}$\det T=\BigPar{{\det\sqrt{T}}}\BigPar{{\det\sqrt{T}}}\leqslant\prod_{k=1}^n\BigDvertsq{\;\!\!\sqrt{T}e_k}=\prod_{k=1}^n\Ang{Te_k,e_k}.$ \;Let $\Mt[\BigPar]{\!\sqrt{T},B_V}=QR$ factoriz.\vspace{2pt}\parSol{}
%Then $R$ diag $\Longleftrightarrow\det\sqrt{T}=\prod_{k=1}^n\BigDvert{\;\!\!\sqrt{T}e_k}\Longleftrightarrow\Mt{T,B_V}=R^*Q^*QR=R^*R$ diag.\PfEnd\vspace{2pt}\parSol{}
%\Or Let $B=\Mt[\BigPar]{\!\sqrt{T},B_V}.$ Then equa $\Longleftrightarrow\BigPar{{\!\sqrt{T}e_1,\dots,\sqrt{T}e_n}}$ orthog $\Longleftrightarrow B^*B=\Mt{T,B_V}$ diag.\PfEnd
%\SepLine

\newcommandx{\mKf}[2]{{\mathcal{K}\!}_{#1}^{\;\TinyPar{#2}}}

\Anchor{4e9C21}\ProblemN{C.22}{
	\TextA{Supp $n\in\Nbp,$ and $\delta:\CbbP{n,\;\!n}\rightarrow\Cbb$ suth $\delta\Par{AB}=\delta\Par{A}\cdot\delta\Par{B},$ and $\delta\Par{D}=D_{1,1}\!\cdots D_{n,n},$}
	\TextA{for all $A,B\in\CbbP{n,\;\!n}$ and all diag $D\in\CbbP{n,\;\!n}.$ \;Prove $\delta={\det}.$}
}Define $\mKf{i,\,j}{\alpha}=I+\alpha\:\!\mEnt{i,\,j}.$ \;We show $\delta\BigPar{\mKf{i,\,j}{\alpha}}=1+\delta_{i,\;\!j}\;\!\alpha.$ \,Supp $i\neq j.$\vspace{2pt}\parSol{}
\NOTICE that $\mKf{i,\,j}{\alpha}\,\mKf{i,\,j}{\alpha}=\Par{I+\alpha\,\mEnt{i,\,j}}{^2}=I+2\;\!\alpha\,\mEnt{i,\,j}$\vspace{1pt}\parSol{}
\Blind{\NOTICE that }$=\Sbra{I+\mEnt{j,\,j}+2\;\!\alpha\,\mEnt{i,\,j}}-\frac{\,1\,}{2}\:\!\mEnt{j,\,j}\Sbra{I+\mEnt{j,\,j}+2\;\!\alpha\,\mEnt{i,\,j}}$\vspace{1pt}\parSol{}
\Blind{\NOTICE that }$=\Par{I-\frac{\,1\,}{2}\:\!\mEnt{j,\,j}}\Sbra{\Par{I+\alpha\,\mEnt{i,\,j}}\Par{I+\mEnt{j,\,j}}}=\mKf{j,\,j}{1/2}\,\mKf{i,\,j}{\alpha}\,\mKf{j,\,j}{1}.$\vspace{2pt}\parSol{}
\又 $\mKf{i,\,j}{\alpha}\,\mKf{i,\,j}{{-\alpha}}=I\Rightarrow\delta\BigPar{\mKf{i,\,j}{\alpha}}\neq0.$ \,Thus $\delta\BigPar{\mKf{i,\,j}{\alpha}}=\delta\BigPar{\mKf{j,\,j}{1/2}\,\mKf{j,\,j}{1}}=1.$\vspace{3pt}\parSol{}
Supp $A\in\CbbP{n,\;\!n}.$ Define $T\in\Lm{\CbbP{n}}:e_k\mapsto A_{\cdot,k}\Rightarrow\Mt{T}=A$ wrto std bses.\parSol{}
Write $A=\Mt{I,Tv\rightarrow e}\Mt{T,v\rightarrow Tv}\Mt{I,e\rightarrow v}.$ Thus $A$ not inv $\Rightarrow\delta\Par{A}=0.$ \;Supp $A$ inv.\parSol{}
Supp $T$ up-trig wrto $\Par{u_1,\dots,u_n}\Rightarrow C=\Mt{T,u\rightarrow u}=\Mt{I,e\rightarrow u}\:\!\Mt{T,e\rightarrow e}\:\!\Mt{I,u\rightarrow e}.$\parSol{}
Then $A\mKf{i,\,j}{\alpha}=A+\alpha\:\!A\:\!\mEnt{i,\,j},$ which is the result of adding $i^\text{th}$ col times $\alpha$ to the $j^\text{th}$ col.\parSol{}
{\tgbfx Step 1.} Let $c_1=C_{1,2}\big/C_{1,1}.$ Then let $R_1=C\:\!\mKf{1,\,2}{{-c_1}}.$ Now $\Par{R_1}{_{1,2}}=0.$\parSol{}
{\tgbfx Step k.} Let $c_k=\BigPar{C_{1,\;\!k+1}\big/C_{1,1},\dots,C_{k,\;\!k+1}\big/C_{k,\;\!k}}.$ Then let $R_{k+1}=R_k\:\!\mKf{1,\;\!k+1}{{-c_{1,\;\!k}}}\!\cdots\mKf{k,\;\!k+1}{{-c_{k,\;\!k}}}.$\parSol{}
We stop at step $\Par{n-1}.$ \,Becs each $\delta\Par{R_{k+1}}=\delta\Par{R_k}.$\PfEnd
\SepLine

%\Anchor{4e9D1}\ProblemN{D.1}{
%	\TextA{Supp $v\in V,w\in W.$ Prove $v\otimes w=0\Rightarrow v=0$ or $w=0.$}
%}$\forall\varphi\in V\apostrophe,\tau\in W\apostrophe,\,\varphi\Par{v}\,\tau\Par{w}=0.$\parSol{}
%Asum $v\neq0\Rightarrow\exists\,\varphi\Par{v}\neq0\Rightarrow w=0.$ Simlr for $w\neq0.$\PfEnd\parSol{}
%\Or We show ctrapos. Supp $v,w\neq0\Rightarrow\exists\,\varphi\Par{v}=\tau\Par{w}=1.$\PfEnd
%\SepLine

%\Anchor{4e9D3}\ProblemN{D.3}{
%	\TextA{Supp $\Par{v_1,\dots,v_m}$ liney indep in $V,$ and $w_1,\dots,w_m\in W,$ $v_1\otimes w_1+\dots+v_m\otimes w_m=0.$ Prove $w_1=\dots=w_m=0.$}
%}Extend to $B_V=\Par{v_1,\dots,v_n}.$ Let the corres $B_{V\apostrophe}=\Par{\varphi_1,\dots,\varphi_n}\Rightarrow\forall\tau\in W\apostrophe,$ each $\sum_{k=1}^m\Par{v_k\otimes w_k}\Par{\varphi_j,\tau}=\tau\Par{w_j}=0\Rightarrow w_j=0.$\PfEnd
%\SepLine

%\ProblemB{
%	\TextB{Supp $S\in\Lm{V},T\in\Lm{W},$ \FontNorm$B_V=\Par{e_1,\dots,e_n},\,B_W=\Par{\,f_1,\dots,f_m},\;A=\Mt{S,B_V},B=\Mt{T,B_W}.$\vspace{1pt}}
%	\Anchor{4e9D9}\PrePa\TextB{Prove $\exists\,!\,\mA\in\Lm{V\otimes W}:v\otimes w\mapsto Sv\otimes Tw.$\FontNorm\tgnr\hfill Denote it by $S\otimes T.$}
%	\Anchor{4e9D10}\PrePb\TextB{Prove $S\otimes T$ inv $\Longleftrightarrow S,T$ inv.\FontNorm\tgnr\hfill$\Par{S\otimes T}{^{-1}}=S^{-1}\otimes T^{-1}.$}
%	\Anchor{4e9D11}\PrePc\TextB{Supp $V,W$ are inner prodsps. Prove $\Par{S\otimes T}{^*}=S^*\otimes T^*.$}
%}(a) Define $\mA\Par{e_j\otimes\:\!f_k}=Se_j\otimes T\:\!f_k.$ \;Using liney map lemma.\parSol{\Ha}
%\Or Define biliney $\Gamma:V\times W\rightarrow V\otimes W$ by $\Gamma:\Par{v,w}\mapsto Sv\otimes Tw.$ \;Let $\mA=\hat{\Gamma}\in\Lm{V\otimes W}.$\vspace{2pt}\parSol{}
%(b) Supp $S$ or $T$ not inv. Let some $B_{\null S}\subseteq B_V\Rightarrow Se_i=0\Rightarrow\mA\Par{e_i,w}=0\otimes Tw=0.$ Simlr for $T.$\vspace{3pt}\parSol{}
%(c) $\BigAng{Sv\otimes Tw,\,u\otimes x}=\Ang{Sv,u}\Ang{Tw,x}=\Ang{v,S^*u}\Ang{w,T^*x}=\BigAng{v\otimes w,S^*u\otimes T^*x}.$\vspace{3pt}\parSol{\Hc}
%\Or $\mA\Par{v\otimes w}={\sum_{j=1}^n\sum_{k=1}^m}\Par{a_jb_k}\Par{Se_j\otimes T\:\!f_k}={\sum_{j=1}^n\sum_{k=1}^m}\Par{a_jb_k}\sum_{x=1}^n\sum_{y=1}^mA_{x,\;\!j}B_{y,\;\!k}\Par{e_x\otimes\:\!f_y}.$\vspace{2pt}\parSol{\Hc}
%$\BigAng{Sv\otimes Tw,\,{u\otimes r}}={\sum_{j=1}^n\sum_{k=1}^m}\Par{a_jb_k}\sum_{x=1}^n\sum_{y=1}^mA_{x,\;\!j}\:\!\overline{c_x}\,B_{y,\;\!k}\:\!\overline{d_y}$\vspace{2pt}\parSol{\Hc}
%$=\BigAng{v\otimes w,\,\mA^*\Par{u\otimes r}}=\BigAng{v\otimes w,\,{\sum_{x=1}^n\sum_{y=1}^m}\Par{c_x\:\!d_y}\sum_{j=1}^n\sum_{k=1}^m\overline{A_{x,\;\!j}B_{y,\;\!k}}\Par{e_j\otimes\:\!f_k}}.$\PfEnd
%\SepLine

\Anchor{4e9DNE6}\ProblemBX[]{\NoteForSmall{Exe (D.6)}}{
	Let $R_r=\Bra{v_1\otimes w_1+\dots+v_{\;\!\!r}\otimes w_{\;\!\!r}:\BigPar{\Par{v_1,\dots,v_{\;\!\!r}},\Par{w_1,\dots,w_{\;\!\!r}}}\in V^r\times W^r}.$\vspace{1pt}\TextB{}
	Let $V=\FbbP{m},W=\FbbP{n}.$ Let $M_r=\Bra{A\in\FbbP{m,n}:\rank A\leqslant r}.$ Id $v\otimes w$ with $v\:\!w^t.$\TextB{}
	Then $\Par{v_1\:\!w_1^t+\dots+v_{\;\!\!r}\:\!w_{\;\!\!r}^t}{_{\cdot,k}}=w_1[k]\cdot v_1+\dots+w_{\;\!\!r}[k]\cdot v_{\;\!\!r}\in\Span{v_1,\dots,v_{\;\!\!r}}.$\vspace{1pt}\TextB{}
	Simlr $\Par{v_1\:\!w_1^t+\dots+v_{\;\!\!r}\:\!w_{\;\!\!r}^t}{_{j,\cdot}}=v_1[\:\!j]\cdot w_1^t+\dots+v_{\;\!\!r}[\:\!j]\cdot w_{\;\!\!r}^t\in\Span{w_1,\dots,w_{\;\!\!r}}.$\vspace{1pt}\TextB{}
	Let $N=\min\!\Bra{m,n}.$ Then $R_1,\dots,R_{N-1}\neq V\otimes W$ id with $\FbbP{m,n}=M_N.$\vspace{0pt}\TextB{}
	Immed $R_N=M_N$ via $v\otimes w=v\:\!w^t,$ with a bss of the smaller vecsp.\par\vspace{3pt}
	\BulletPointX\AComm For $r\in\!\!\;\Bra{1,\dots,N-1},$ $R_r$ is not a subsp of $V\otimes W,$ as $M_r$ is not a subsp of $\FbbP{m,n},$\parCom{\IndentB}
	as the set of non-inv $T\in\Lm{V,W}$ is not a subsp of $\Lm{V,W}.$\TextB{\vspace{-3pt}}
}\SepLine

%\textsl{\normalsize 
%}
%\pagebreak
%
%\ChDecl{}{\Largebfx{Extra Exes}}{\quad{\ANote {\FontSmall 题目源自网络。}}}
%
%\vspace{4pt}
%
%\ProblemB{
%	\TextB{Supp $V$ is inner prodsp, and $T\in\Lm{V},T^2=0.$}
%	\TextB{Let $S=T^*T+TT^*.$ Prove $V=\range T\oplus\range T^*\oplus\null S.$}
%}$\forall Tv,T^*w,\,\Ang{Tv,T^*w}=0.$ Thus $\range T\cap\range T^*=\zeroSubs.$\parSol{}
%$\forall u\in\null S,\,\Ang{Su,u}=\Dvertsq{T^*u}+\Dvertsq{Tu}.$ Thus $\null S\subseteq\null T\cap\null T^*.$\parSol{}
%$\forall u\in\null S,\forall\Par{Tv,T^*w}\in\range T\times\range T^*,\,\Ang{u,Tv+T^*w}=\Ang{T^*u,v}+\Ang{Tu,w}=0.$ Thus $\range T\oplus\range T^*\subseteq\BigPar{\null S}{^\perp}.$\parSol{}
%Becs $\range S\subseteq\BigPar{\null S}{^\perp}$ and $\range S\oplus\null S=V.$
%\SepLine
%
%\ChEnd\pagebreak
%
%\ChDecl{}{\Largebfx{Simpler Representa of Matrix Multi Using Tensor Prod}}
%
%\vspace{4pt}
%
%\BulletPointX Let $A\in\FbbP{m,p},C\in\FbbP{p,n},V=\FbbP{1,p},W=\FbbP{p,1}.$ Let $A=\Par{A_{1,\cdot},\dots,A_{m,\cdot}}\in V^m,$ and $C=\Par{C_{\cdot,1},\dots,C_{\cdot,n}}\in W^n.$\TextB{}
%Let $\Par{e_1,\dots,e_p},\Par{\,f_1,\dots,\,f_p}$ be std bses of $V,W.$\TextB{}
%Then $AC=A\otimes C=\sum A_{j,r}C_{r,k}\:\!\Par{e_j\otimes\;\!f_k}.$
%\SepLine


\ChEnd