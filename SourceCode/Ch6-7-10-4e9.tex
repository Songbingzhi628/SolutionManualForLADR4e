% Copyright (C) 2024 Songbingzhi628. This work is licensed under Creative Commons Attribution-NonCommercial-ShareAlike 4.0 International License.
% Email: 13012057210@163.com

\ChDecl{Ch6A}{6.A}{}

\vspace{4pt}

\ProblemB[]{
	(a) \,$\Dvertsq{u+v}=\Dvertsq{u}+\Dvertsq{v}+2\Real\Ang{u,v}.$ \; $\Dvertsq{u+\i\,v}=\Dvertsq{u}+\Dvertsq{v}+2\Imaginary\Ang{u,v}.$\TextB{}
	(b) \,$\aXMid{\Dvert{u}-\Dvert{v}}\leqslant\Dvert{u-v}.$ \;Equa $\Longleftrightarrow u=cv,\;c>0.$ \,Where $u,v\neq0.$\TextB{}
	(c) \,$\aXMid{\Dvert{v}-1}=\BigDvert{v-v\big/\Dvert{v}}\leqslant\Dvert{v-u}$ \,if\, $\Dvert{u}=1.$ \;Equa $\Longleftrightarrow u=v\big/\Dvert{v}.$\TextB{}
	(d) \,$\aXMid{\Dvertsq{u}-\Dvertsq{v}}=\innerA{u+v,u-v}\leqslant\Dvert{u+v}\,\Dvert{u-v}\leqslant\Dvertsq{u}+\Dvertsq{v}={}${\Large$\frac{\:1\:}{2}$}$\BigSbra{\Dvertsq{u+v}+\Dvertsq{u-v}}.$\vspace{-3pt}\TextB{}
}\SepLine

\Anchor{6A21}\ProblemN{21}{
	\TextA{Implement the corres inner prod from a norm $\Dvert{{\cdot}}:U\rightarrow\Interval{[}{)}{0,\infty}$ satisfying [6.22].\vspace{3pt}}
%	\PrePa {\,}$\Dvert{u}=0\Longleftrightarrow u=0.$ \;(b) \,$\Dvert{au}=\aMid{a}\Dvert{u}.$ \;(c) \,\uline{$\Dvert{u+w}\leqslant\Dvert{u}+\Dvert{w}.$}\TextA{}
}If $\Fbb=\Rbb.$ Define $\Ang{u,v}={}${\Large$\frac{\:1\:}{4}$}$\BigPar{\Dvertsq{u+v}-\Dvertsq{u-v}}=\Ang{v,u}.$ \;Before we start:\def\Lbreak{1pt}\def\LBREAK{6pt}\def\Endl{\vspace{\Lbreak}\parSol{}}\def\ENDL{\vspace{\LBREAK}\parSol{}}\vspace{2pt}\parSol{}
\LX{1} $\Ang{u,v}=-\Ang{{-u,v}}=-\Ang{u,-v}.$\Endl{}
\LX{2} $\Ang{u+v,v}={}${\Large$\frac{\:1\:}{4}$}$\BigSbra{\Dvertsq{u+v+v}+\Dvertsq{{-u+v+v}}-\Dvertsq{{-u+v+v}}-\Dvertsq{u+v-v}}$\Endl{}
\Blind{\LX{2}} $\Blind{\Ang{u+v,v}}={}${\Large$\frac{\:1\:}{2}$}$\BigSbra{\BigPar{\Dvertsq{u}+\Dvertsq{2v}}-\BigPar{\Dvertsq{{-u+v}}+\Dvertsq{v}}}$\Endl{}
\Blind{\LX{2}} $\Blind{\Ang{u+v,v}}=4\,\Ang{v,v}+2\BigPar{\Dvertsq{u}+\Dvertsq{v}}-2\,\Dvertsq{u-v}=\Ang{u,v}+\Ang{v,v}.$\ENDL{}
\LX{3} $\Ang{u,\,2v}={}${\Large$\frac{\:1\:}{4}$}$\BigSbra{\Dvertsq{u+v+v}-\Dvertsq{u-v-v}}$\Endl{}
\Blind{\LX{3}} $\Blind{\Ang{u,\,2v}}={}${\Large$\frac{\:1\:}{4}$}$\BigSbra{\Dvertsq{u+v+v}+\Dvertsq{u+v-v}-\Dvertsq{u+v-v}-\Dvertsq{u-v-v}}$\Endl{}
\Blind{\LX{3}} $\Blind{\Ang{u,\,2v}}={}${\Large$\frac{\:1\:}{2}$}$\BigSbra{\BigPar{\Dvertsq{u+v}+\Dvertsq{v}}-\BigPar{\Dvertsq{u-v}+\Dvertsq{v}}}=2\,\Ang{u,v}.$\ENDL{}
{\tgbf Add:} \,$\Ang{u+w,v}=\Ang{u,v}+\Ang{w,v}.$\Endl{}
We show \:$\Dvertsq{u+w+v}-\Dvertsq{u+w-v}=\Dvertsq{u+v}+\Dvertsq{w+v}-\Dvertsq{u-v}-\Dvertsq{w-v}.$\Endl{}
$RHS={}${\Large$\frac{\:1\:}{2}$}$\BigPar{\Dvertsq{u+w+2v}+\Dvertsq{u-w}}-{}${\Large$\frac{\:1\:}{2}$}$\BigPar{\Dvertsq{u+w-2v}+\Dvertsq{u-w}}=2\,\Ang{u+w,2v}=LHS.$\ENDL{}
%$\Blind{RHS}={}${\Large$\frac{\:1\:}{2}$}$\BigPar{\Dvertsq{u+w+v+v}-\Dvertsq{u+w+v-v}+\Dvertsq{u+w-v+v}-\Dvertsq{u+w-v-v}}$\Endl{}
%$\Blind{RHS}=2\,\Ang{u+w+v,v}-2\,\Ang{u+w-v,-v}$\Endl{}
%$\Blind{RHS}=2\BigPar{\Ang{u+w,v}+\Ang{v,v}}-2\BigPar{\Ang{u+w,-v}+\Ang{{-v,-v}}}=2\,\Ang{u+w,v}+2\,\Ang{u+w,v}=LHS.$\ENDL{}
{\tgbf Homo:} \,$\Ang{\lambda u,v}=\lambda\Ang{u,v}.$ \;True by add if $\lambda\in\Nbb,$ and then by (1) if $\lambda\in\Zbb.$\Endl{}
Note that by add, $\;n\cdot\Ang{n^{-1}u,v}=\Ang{u,v}$ for $n\in\Nbp.$ Thus the case for $\lambda\in\Qbb^+$ holds, so for $\Qbb$.\Endl{}
We show the case for $\lambda\in\Rbb.$ \,By def, $\exists\,!\,\Par{a_n}{_{n=0}^\infty}\in\Qbb^{\infty}$ suth $\lim_{n\rightarrow\infty}a_n=\lambda.$\Endl{}
$4\,\lambda\:\!\Ang{u,v}=4\:\!\lim_{n\rightarrow\infty}a_n\:\!\Ang{u,v}=4\:\!\lim_{n\rightarrow\infty}\Ang{a_nu,v}=\lim_{n\rightarrow\infty}\!\BigSbra{\Dvertsq{a_nu+v}-\Dvertsq{a_nu-v}}.$\Endl{}
To show $\,\lim_{n\rightarrow\infty}\Dvert{a_nu+v}=\Dvert{\lambda u+v},$ so then $\lambda\:\!\Ang{u,v}=\Ang{\lambda u,v}.$\Endl{}
\NOTICE that $\Dvert{u\pm v}\leqslant\Dvert{u}+\Dvert{v}\Longrightarrow\aXMid{\Dvert{u}-\Dvert{v}}\leqslant\Dvert{u\pm v}.$\Endl{}
Thus $\aXMid{{\lim_{n\rightarrow\infty}\Dvert{a_nu+v}-\Dvert{\lambda u+v}}}\leqslant\BigDvert{\lim_{n\rightarrow\infty}a_nv-\lambda v}=0.$\vspace{6pt}\parSol{}
If $\Fbb=\Cbb.$ Define $\Ang{u,v}=R\Par{u,v}+\i\,I\Par{u,v}.$\Endl{}
Where $R\Par{u,v}={}${\Large$\frac{\:1\:}{4}$}$\BigPar{\Dvertsq{u+v}-\Dvertsq{u-v}}$ \,and $I\Par{u,v}=R\Par{u,\:\!\i\,v}={}${\Large$\frac{\:1\:}{4}$}$\BigPar{\Dvertsq{u+\i\,v}-\Dvertsq{u-\i\,v}}.$\vspace{3pt}\parSol{}
{\tgbf Conjug Symm:} \,$\Ang{u,v}=R\Par{u,v}+\i\,I\Par{u,v}=R\Par{v,u}-\i\,I\Par{v,u}=\overline{\Ang{v,u}}$\Endl{}
Note that $R\Par{u,v}=R\Par{v,u}$ and $R\Par{v,\:\!\i\,u}=R\Par{\i\,u,\:\!v}.$ Thus we show $-I\Par{u,v}=I\Par{v,u}.$\Endl{}
Which equiv $\Dvertsq{u-\i\,v}-\Dvertsq{u+\i\,v}=\BigDvertsq{\i\:\!\Par{{-\i\,u-v}}}-\BigDvertsq{\i\:\!\Par{{-\i\,u+v}}}=\Dvertsq{\i\,u+v}-\Dvertsq{\i\,u-v}.$\vspace{2pt}\Endl{}
{\tgbf Homo:} \,$\Ang{\lambda u,v}=\lambda\Ang{u,v}.$ \;True if $\lambda\in\Rbb.$ \,We show the case for $\lambda=\i.$\Endl{}
$\Ang{\i\,u,\:\!v}={}${\Large$\frac{\:1\:}{4}$}$\BigSbra{\Dvertsq{\i\,u+v}-\Dvertsq{\i\,u-v}+\i\,\BigPar{\Dvertsq{\i\,u+\i\,v}-\Dvertsq{\i\,u-\i\,v}}}$\Endl{}
$\Blind{\Ang{\i\,u,\:\!v}}={}${\Large$\frac{\:1\:}{4}$}$\BigSbra{\Dvertsq{u-\i\,v}-\Dvertsq{u+\i\,v}+\i\,\BigPar{\Dvertsq{u+v}-\Dvertsq{u-v}}}$\Endl{}
$\Blind{\Ang{\i\,u,\:\!v}}=\i{}${\Large$\frac{\:1\:}{4}$}$\BigSbra{{-\i}\,\Dvertsq{u-\i\,v}+\i\,\Dvertsq{u+\i\,v}+\BigPar{\Dvertsq{u+v}-\Dvertsq{u-v}}}{{}=\i\,\Ang{u,v}}$\PfEnd
\SepLine

\Anchor{6A3}\ProblemN{3}{
	\TextA{Supp $\Fbb=\Rbb,V\neq\zeroSubs.$ Replace the positivity cond in [6.3] with $\exists\,v\in V,\,\Ang{v,v}>0.$}
	\TextA{Show this does not change the inner prods from $V\times V$ to $\Rbb.$}
}Supp $w\in V$ with $\Ang{w,w}>0.$ Asum $\exists\,u\in V$ with $\Ang{u,u}<0.$\parSol{}
Define $p\Par{x}=\Ang{u+xw,\,u+xw}=\Ang{w,w}\,x^2+2\Ang{u,w}\,x+\Ang{u,u}\Rightarrow$ two disti zeros.\parSol{}
Supp $\Ang{u+\lambda w,\,u+\lambda w}=0\Rightarrow u+\lambda w=0\Rightarrow\Ang{u,u}=\lambda^2\Ang{{-w,-w}}\geqslant 0,$ ctradic.\PfEnd
\SepLine

\Anchor{6A6}\ProblemN{6}{
	\TextA{Supp $u,v\in V.$ Prove $\Dvert{u}\leqslant\Dvert{u+av}$ for all $a\in\Fbb\Rightarrow\Ang{u,v}=0.$}
}Becs $\Dvertsq{u}\leqslant\Dvertsq{u+av}.$ Let $\Ang{u-cv,\,cv}=0\Rightarrow\Dvertsq{u-cv}=\Ang{u,\,u-cv}=\Dvertsq{u}-\overline{c}\Ang{u,v}.$\parSol{}
Thus $\Dvertsq{u}\leqslant\Dvertsq{u-cv}=\Dvertsq{u}-\innerAsq{u,v}{\;\!\XSlash\;\!}\Dvertsq{v}.$\PfEnd\vspace{4pt}\parSol{}
\Or $\Dvertsq{u}\leqslant\Dvertsq{u}+\aMidsq{a}\,\Dvertsq{v}+2\REAL\:\overline{a}\:\!\Ang{u,v}\Rightarrow-2\REAL\:\overline{a}\:\!\Ang{u,v}\leqslant\aMidsq{a}\,\Dvertsq{v}$ for all $a\in\Fbb.$\parSol{}
Let $a=-\Ang{u,v}\Rightarrow2\:\!\innerA{u,v}{^2}\leqslant\innerA{u,v}{^2}\,\Dvertsq{v}.$ If $\Ang{u,v}\neq0,$ then $2\leqslant\Dvertsq{v};$ might not be true.\PfEnd
\SepLine

\Anchor{6AT1}\ProblemBX{\TipsN{1}}{
	\TextB{Supp $u,v\in V,$ $\Dvertsq{xu+yv}=\aMidsq{x}\:\!\Dvertsq{u}+\aMidsq{y}\:\!\Dvertsq{v}$ for $x,y\in\Fbb.$ Prove $\Ang{u,v}=0.$}
}Becs $\REAL\BigPar{x\overline{y}\,\Ang{u,v}}=0.$ Take $\Par{x,y}=\Par{1,1}$ and $\Par{\i,1}.$\quad\Or By Exe (6), immed.\PfEnd
\SepLine

%\Anchor{6A4e19}\ProblemBnoor{4E 19}{
%	\TextA{Supp $T\in\Lm{V},\,\lambda$ is eigval. Prove $\aMidsq{\lambda}\leqslant\sum_{j=1}^n\sum_{k=1}^n\aMidsq{A_{j,k}},$ \FontNorm\tgnr where $A=\Mt[\BigPar]{T,\Par{v_1,\dots,v_n}}.$\vspace{3pt}}
%}Let $\Ang{a_1v_1+\dots+a_nv_n,b_1v_1+\dots+b_nv_n}=a_1\overline{b_1}+\dots+a_n\overline{b_n}.$ Let $v$ be eigval corres $\lambda$ with $\Dvert{v}=1.$\vspace{1pt}\parSol{}
%Becs $v=a_1v_1+\dots+a_nv_n\Rightarrow Tv=\sum_{k=1}^na_k\sum_{j=1}^nA_{j,k}v_j=\sum_{j=1}^n\BigSbra{\sum_{k=1}^na_kA_{j,k}}v_j.$\vspace{3pt}\parSol{}
%又 Each $\Dvert{v_j}=1,\:\Ang{v_j,v_k}=0\Rightarrow\Dvertsq{Tv}=\sum_{j=1}^n\Big|\sum_{k=1}^na_kA_{j,k}\Big|{^2}=\aMidsq{\lambda}.$\vspace{3pt}\parSol{}
%Note that $\aXMidsq{a_1A_{j,1}+\dots+a_nA_{j,n}}\leqslant\BigDvertsq{\Par{a_1,\dots,a_n}}\cdot\BigDvertsq{\Par{A_{j,1},\dots,A_{j,n}}}=\sum_{k=1}^n\aMidsq{A_{j,k}}.$\PfEnd
%\SepLine

\Anchor{6AT2}\ProblemBX{\TipsN{2}}{
	\TextB{Supp $A\in\Fbb^{m,n}.$ Prove $\Dvertsq{Ax}\leqslant\sum_{j=1}^m\sum_{k=1}^n\aMidsq{A_{j,k}}\cdot\Dvertsq{x}$ for all $x\in\Fbb^{m,1}.$\vspace{2pt}}
}$\Dvertsq{Ax}=\Dvertsq{A_{\cdot,1}x_1+\dots+A_{\cdot,n}x_n}=\sum_{j=1}^m\aMidsq{x_1A_{j,1}+\dots+x_nA_{j,n}}\leqslant\sum_{j=1}^m\Dvertsq{A_{j,\cdot}}\cdot\Dvertsq{x}.$\PfEnd
\SepLine

\Anchor{6A4e23}\ProblemBnoor{4E 23}{
	\TextA{Supp $v_1,\dots,v_m\in V,$ each $\Dvert{v_k}\leqslant 1.$ \,Show $\exists\,a_k=\pm1,\;\Dvert{a_1v_1+\dots+a_mv_m}\leqslant\sqrt{m}.$}
}We use induc on $m.$ (i) $m=1.$ Immed. (ii) $m>1.$ Asum it holds for smaller $m.$\parSol{}
Let $u=a_1v_1,\,w=a_2v_2+\dots+a_mv_m\Rightarrow\Dvertsq{u}\leqslant1,\Dvertsq{w}\leqslant{m-1}.$\parSol{}
Then $\Dvertsq{u+w}+\Dvertsq{u-w}\leqslant2m.$ \;\;\Or $\Dvert{u+w}\cdot\Dvert{u-w}\leqslant m.$\PfEnd
\SepLine

\Anchor{6A'1}\ProblemB{
	\TextB{Supp $u,v_1,\dots,v_n$ are non0 in $V$ suth each $\Ang{v_i,u}>0$ and $\Ang{v_i,v_j}\leqslant0$ \,for $i\neq j.$}
	\TextB{Show $\Par{v_1,\dots,v_n}$ liney indep.}
}(i) Asum $v_1=cv_2.$ Then $\Ang{cv_2,u}>0\Rightarrow c>0,$ while $\Ang{v_1,v_1}=c\Ang{v_2,v_1}\geqslant0\Rightarrow c\leqslant0.$ ctradic.\parSol{}
(ii) Asum $\Par{v_1,\dots,v_{n-1}}$ liney indep. Asum $v_n=c_1v_1+\dots+c_{n-1}v_{n-1}.$\parSol{}
Then $\Ang{v_n,u}=c_1\Ang{v_1,u}+\dots+c_{n-1}\Ang{v_{n-1},u}>0.$ \;Thus each $c_k\in\Rbb.$\parSol{}
Write $c_1v_1+\dots+c_nv_n=0,\:c_n=-1.$ \;Let $P=\Bra{i:c_i\geqslant0},\,N=\Bra{i:c_i<0}.$\vspace{1pt}\parSol{}
Then $\sum_{j\in P}c_jv_j=\sum_{k\in N}-c_kv_k\Longrightarrow 0\leqslant\BigAng{{\sum_{j\in P}c_jv_j,\,\sum_{k\in N}-c_kv_k}}=\sum_{j\in P}\sum_{k\in N}-c_j\:\!c_k\!\:\Ang{v_j,v_k}\leqslant0.$\vspace{2pt}\parSol{}
While $\BigAng{{\sum_{j\in P}c_jv_j,u}},\BigAng{{\sum_{k\in N}-c_kv_k,u}}\geqslant0,$ where the equas hold $\Longleftrightarrow$ all $c_i=0.$\PfEnd
\SepLine

\vfill\ChDecl{Ch6B}{6.B}{}

\vspace{4pt}

\Anchor{6B14}\ProblemN{14}{
	\TextA{Supp $\Par{e_1,\dots,e_m}$ orthon, each $v_j\in V$ suth $\Dvert{e_j-v_j}<{}${\Large$\frac{\;1}{\sqrt{m}\;}$}. Show $\Par{v_1,\dots,v_m}$ liney indep.}
}Let $a_1v_1+\dots+a_mv_m=0.$\vspace{3pt}\parSol{}
$\sum_{j=1}^m\aMidsq{a_j}=\BigDvertsq{{\sum_{j=1}^ma_j\Par{e_j-v_j}}}\leqslant\BigSbra{{\sum_{j=1}^m}\aMid{a_j}\cdot\Dvert{e_j-v_j}}{^2}\leqslant\BigDvertsq{\Par{\aMid{a_j}}{_{j=1}^m}}\cdot\BigDvertsq{\BigPar{\Dvert{e_j-v_j}}{_{j=1}^m}}.$\PfEnd\vspace{6pt}
%\AComm If each $\Dvert{e_j-v_j}=1\Big/\!\sqrt{m}$, then $\Par{v_1,\dots,v_m}$ might be liney dep even if $m\neq1.$\parCom
\AExa Let $v_j=e_j-\Par{e_1+\dots+e_m}\Big/m\Rightarrow\Dvert{e_j-v_j}{^2}=1\big/m.$ \;Note that $v_1+\dots+v_m=0.$\PfEnd
\SepLine

\ProblemB[]{
	For orthog $\Par{e_1,\dots,e_m}$ and $v=a_1e_1+\dots+a_me_m,$ becs $\Ang{v,e_k}=a_k\Dvertsq{e_k},$ \;$v={}${\Large$\frac{\:\SmallAng{v,\:e_1}\:}{\;\SmallDvertsq{e_1}}$}$\,e_1+\dots+{}${\Large$\frac{\:\SmallAng{v,\:e_m}\:}{\;\SmallDvertsq{e_m}}$}$\,e_m.$\TextB{}
	Now $\Dvertsq{v}={}${\Large$\frac{\:\SmallinnerAsq{v,\:e_1}\:}{\SmallDvertsq{e_1}}$}${}+\dots+{}${\Large$\frac{\:\SmallinnerAsq{v,\:e_m}\:}{\SmallDvertsq{e_m}}$}$.$ Replace each $e_k$ with $\Dvert{e_k}{^{\,-1}}e_k,$ now $\Par{e_1,\dots,e_m}$ is a orthon list.
%	\TextB{\vspace{6pt}}
%	Exe (2) holds only for orthon lists.
	\TextB{\vspace{0pt}}
}\SepLine

\Anchor{6B'1}\ProblemB{
	\TextA{Supp $\Par{e_1,\dots,e_m}$ orthog, $v\in V.$ \,Show $\sum_{k=1}^m\!\BigPar{1-\Dvertsq{e_k}}\,\innerAsq{v,e_k}\leqslant\Dvertsq{v}-\sum_{k=1}^m\innerAsq{v,e_k}.$\vspace{4pt}}
}Let $u=\Ang{v,e_1}\,e_1+\dots+\Ang{v,e_m}\,e_m\Rightarrow\Dvertsq{u}=\sum_{k=1}^n\innerAsq[{\:\!\Dvert{e_k}\,}]{v,e_k},\,\Ang{u,v}=\sum_{k=1}^n\innerAsq{v,e_k}.$\vspace{3pt}\parSol{}
Let $\Dvertsq{v-u}=\Dvertsq{v}+\Dvertsq{u}-\Ang{v,u}-\Ang{u,v}=\Dvertsq{v}+\sum_{k=1}^n\!\BigPar{\Dvertsq{e_k}-2}\,\innerAsq{v,e_k}\geqslant0.$\PfEnd\vspace{4pt}
\Anchor{6B2}\ACoro If orthon, $\Ang{u,\,v-u}=0\Rightarrow\Dvertsq{v}=\Dvertsq{u}+\Dvertsq{v-u}.$\vspace{1pt}\parCor
Bessel's Inequa: $\sum_{k=1}^m\innerAsq{v,e_k}\leqslant\Dvertsq{v}.$ \:\Sbra{Exe (2)} Equa $\Longleftrightarrow v\in\Span{e_1,\dots,e_m}.$
\SepLine\pagebreak

\Anchor{6B4e9}\ProblemBnoor{4E 9}{
	\TextA{Supp $\Par{e_1,\dots,e_m}$ is the result of applying [6.31]\vspace{1pt}}
	\TextA{to a liney indep $\Par{v_1,\dots,v_m}$ in $V.$ \,Show each $\Ang{v_j,e_j}>0.$}
}Let $\;f_j=v_j-\Ang{v_j,\,e_1}\,e_1-\dots-\Ang{v_j,\,e_{j-1}}\,e_{j-1}.$\vspace{2pt}\parSol{}
Becs $\;\Dvert{\,f_j}\,\Ang{v_j,\,e_j}=\Ang{v_j,\,f_j}=\Dvertsq{v_j}-\innerAsq{v_j,\,e_1}-\dots-\innerAsq{v_j,\,e_{j-1}}\geqslant0,$ by Bessel's Inequa.\vspace{2pt}\parSol{}
If $\,\Ang{v_j,\,f_j}=0,$ then by Exe (2), $v_j\in\Span{e_1,\dots,e_{j-1}}=\Span{v_1,\dots,v_{j-1}}.$ 又 $\Dvert{\,f_j}\neq0.$\PfEnd\vspace{3pt}
\Anchor{6B9}\Anchor{6B4e13}\ANote Supp $\Par{v_1,\dots,v_m}$ liney dep. Let $j$ be the largest suth $\Par{v_1,\dots,v_{j-1}}$ liney indep.\parNot
Apply [6.31]. Now $v_j\in\Span{v_1,\dots,v_{j-1}}=\Span{e_1,\dots,e_{j-1}}\Rightarrow f_j=0.$
\SepLine

\Anchor{6BT}\BulletPointX\Tips \,\,\,Supp $\Par{v_1,\dots,v_m}$ liney indep in $V.$ Get the corres orthon $\Par{e_1,\dots,e_m}$ via [6.31].\TextB{}
{\IndentTips}Let $S=\Bra{\lambda\in\Fbb:\aMid{\lambda}=1},$ and $S^m$ be the collec of maps $\Bra{1,\dots,m}\rightarrow S.$\TextB{}
{\IndentTips}Supp orthon $\Par{u_1,\dots,u_m}$ suth each $\Span{u_1,\dots,u_k}=\Span{v_1,\dots,v_k}.$\vspace{1pt}\TextB{}
{\IndentTips}We show it equals $\BigBigPar{c\Par{1}e_1,\dots,c\Par{m}e_m}$ for some $c\in S^m$ by induc on $k.$\vspace{1pt}\TextB{}
{\IndentTips}(i) $k=1.$ $\Span{e_1}=\Span{u_1}\Rightarrow u_1=\Ang{u_1,\,e_1}\,e_1,$ 又 $\innerA{u_1,\,e_1}=1.$ Let $c\Par{1}=\Ang{u_1,\,e_1}.$\vspace{1pt}\TextB{}
{\IndentTips}(ii) $k>1.$ Asum each $\aMid{c\Par{i}}=1$ and $c\Par{i}e_i=u_i$ for $i\in\!\Bra{1,\dots,k-1}.$\vspace{1pt}\TextB{}
{\IndentTips\Hii}$u_k=\Ang{u_k,\:\!e_1}\:\!e_1+\dots+\Ang{u_k,\:\!e_k}\:\!e_k.$ 又 $\Ang{u_j,u_k}=0=c\Par{j}\Ang{e_j,u_k}$ for $j\neq k.$ \,Simlr, $c\Par{k}=\Ang{u_k,\,e_k}.$
\SepLine

\Anchor{6B4e10}\ProblemBnoor{4E 10}{
	\TextB{Supp $\Par{v_1,\dots,v_m}$ liney indep. Explain why the
	orthon list produced by [6.31]}
	\TextB{is the only orthon $\Par{e_1,\dots,e_m}$ suth each $\Ang{v_k,\,e_k}>0$ and $\Span{v_1,\dots,v_k}=\Span{e_1,\dots,e_k}.$}
}Fix one $k.$ Let \,$v_k=a_1e_1+\dots+a_ke_k\Rightarrow$ each $a_j=\Ang{v_k,e_j}.$ Let \,$f_k=v_k-a_1e_1-\dots-a_{k-1}e_{k-1}.$\vspace{1pt}\parSol{}
Then $e_k=f_k\Big/\;\!\!a_k\Rightarrow\Ang{\,f_k,\,f_k}\big/a_k^2=1\Rightarrow a_k=\pm\Dvert{\,f_k}.$ 又 $a_k>0.$ \:Hence each $e_k=f_k\Big/\Dvert{\,f_k}.$\PfEnd\vspace{4pt}\parSol{}
\Or By \TIPS. Get $\Par{e_1',\dots,e_m'}$ from $\Par{v_1,\dots,v_m}$ via [6.31].\vspace{1pt}\parSol{}
Becs each $\BigAng{v_k,\,c\Par{k}e_k'}>0,$ while $\Ang{v_k,\,e_k'}>0.$ Thus $c\Par{k}=\REAL\Sbra{c\Par{k}}>0\Rightarrow c\Par{k}=1.$\PfEnd
\SepLine

\Anchor{6B10}\ProblemN{10}{
	\TextA{Supp $\Fbb=\Rbb,$ $\Par{v_1,\dots,v_m}$ is liney indep in $V.$}
	\TextA{Prove $\exists$ exactly $2^m$ orthon lists spans $\Span{v_1,\dots,v_m}.$}
}Using induc on $m.$ (i) $m=1.$ Let $e_1=\pm v_1\big/\Dvert{v_1}.$ (ii) $m>1.$ Asum it holds for $\Par{m-1}.$\vspace{1pt}\parSol{}
Get $2^{m-1}$ orthon lists corres $\Par{v_1,\dots,v_{m-1}}.$ Fix one as $\Par{e_1,\dots,e_{m-1}}$ and apply [6.31] at step $m.$\vspace{2pt}\parSol{}
Supp $\Par{e_1,\dots,e_{m-1},e_m'}$ is also orthon. \NOTICE that $e_m'=\Ang{e_m',\,e_m}\,e_m.$ \;So $\innerA{e_m',\,e_m}=1.$\vspace{2pt}\parSol{}
Let $e_m'=-e_m.$ Sum it up, we have $2^{m-1}\times 2=2^m$ orthon lists. \qquad\Or By \TIPS, immed.\PfEnd
\SepLine

\Anchor{6B11}\ProblemN{11}{
	\TextA{Supp $V\neq0,$ and $\Ang{\cdot,\cdot}{_1},\Ang{\cdot,\cdot}{_2}$ are inner prods suth $\Ang{v,w}{_1}=0\Longleftrightarrow\Ang{v,w}{_2}=0.$\vspace{1pt}}
	\TextA{Prove $\exists\,c>0,\:\Ang{v,w}{_1}=c\Ang{v,w}{_2}$ for all $v,w\in V.$}
}Fix non0 $v_1,v_2\in V.$ Define $\varphi_1,\psi_1\in V\apostrophe$ by $\varphi_1:v\mapsto\Ang{v_1,v}{_1},\;\psi_1:v\mapsto\Ang{v,v_2}{_1}.$ Simlr for $\varphi_2,\psi_2.$\parSol{}
Becs $\Ang{v_1,v}{_1}=0\Longleftrightarrow\Ang{v_1,v}{_2}=0.$ By (3.B.30), let $c_1=\Ang{v_1,v_1}{_1}\Big/\Ang{v_1,v_1}{_2}>0\Rightarrow\varphi_1=c_1\varphi_2.$\parSol{}
Simlr, let $c_2=\Ang{v_2,v_2}{_1}\Big/\Ang{v_2,v_2}{_2}\Rightarrow\psi_1=c_2\psi_2.$ Choose $v_1=v_2$ so that $c=c_1=c_2.$\vspace{3pt}\parSol{}
For any $v_1'\in V,$ get $c_1'$ simlr. Becs $\Ang{v_1,v}{_1}=c_1\Ang{v_1,v}{_2}$ while $\Ang{v_1',v}{_1}=c_1'\Ang{v_1',v}{_2}.$\vspace{1pt}\parSol{}
Now $c_1\Ang{v_1,v_1'}{_2}=\Ang{v_1,v_1'}{_1}=\overline{\Ang{v_1',v_1}{_1}}=\overline{c_1'\Ang{v_1',v_1}{_2}}\Rightarrow c_1=c_1'.$ Simlr for $c_2=c_2'.$\vspace{3pt}\parSol{}
\Or For any $v_1',v_2'\in V,$ get $c_1'=c_2'$ simlr. Becs $c_2\Ang{v_1',v_2}{_2}=\Ang{v_1',v_2}{_1}=c_1'\Ang{v_1',v_2}{_2}.$\PfEnd\vspace{6pt}\parSol{}
\Or Define $c_v=\Ang{v,v}{_1}\Big/\Ang{v,v}{_2}$ for all non0 $v\in V.$ Fix non0 $u,v\in V.$\parSol{}
Let $c=\Ang{u,v}{_2}\Big/\Ang{v,v}{_2}\Rightarrow\Ang{u-cv,v}{_1}=\Ang{u-cv,v}{_2}=0\Rightarrow\Ang{u,v}{_1}=c\Ang{v,v}{_1}=c_v\Ang{u,v}{_2}.$\vspace{1pt}\parSol{}
Rev the roles of $u,v\Rightarrow c_v\Ang{u,v}{_2}={\Ang{u,v}{_1}}=\overline{\Ang{v,u}{_1}}=\overline{c_u\Ang{v,u}{_2}}=c_u\Ang{u,v}{_2}\Rightarrow c_v=c_u.$\PfEnd
\SepLine\pagebreak

\Anchor{6B12}\ProblemN{12}{
	\TextA{Supp $V$ is finide. Let $\Ang{\cdot,\cdot}{_1}$ and $\Ang{\cdot,\cdot}{_2}$ be inner prods with corres norms $\Dvert{{\cdot}}{_1}$ and $\Dvert{{\cdot}}{_2}.$\vspace{1pt}}
	\TextA{Prove $\exists\,c>0,\:\Dvert{v}{_1}\leqslant c\Dvert{v}{_2}$ \,for all $v\in V.$}
}Let $B_V=\Par{e_1,\dots,e_n}$ be orthon wrto $\Ang{\cdot,\cdot}{_2}.$ Supp $v=a_1e_1+\dots+a_ne_n.$\vspace{1pt}\parSol{}
\NOTICE that $\Dvert{v}{_1}\leqslant\Dvert{a_1e_1}{_1}+\dots+\Dvert{a_ne_n}{_1}\leqslant\max\!\Bra{\Dvert{e_1}{_1},\dots,\Dvert{e_n}{_1}}\cdot\BigPar{\aMid{a_1}+\dots+\aMid{a_n}}.$\vspace{3pt}\parSol{}
又 $\aMid{a_1}+\dots+\aMid{a_n}\leqslant n\cdot\max\!\Bra{\aMid{a_k}:1\leqslant k\leqslant n}\leqslant n\cdot\sqrt{\aMidsq{a_1}+\dots+\aMidsq{a_n}}\leqslant n\cdot\Dvert{v}{_2}.$\PfEnd
\SepLine

\Anchor{6B13}\ProblemN{13}{
	\TextA{Supp $\Par{v_1,\dots,v_m}$ liney indep in $V.$ Show $\exists\,w\in V$ suth each $\Ang{w,\:\!v_j}>0.$}
}Using induc on $m.$ (i) $m=1.$ Let $w=v_1.$ (ii) $m>1.$ Asum it holds for $\Par{m-1}.$\parSol{}
By asum, $\exists\,w'\in\Span{v_1,\dots,v_{m-1}}$ suth each $\Ang{w',\:\!v_k}>0$ for $k\in\Bra{1,\dots,m-1}.$\parSol{}
Apply [6.31] to get the corres $\Par{e_1,\dots,e_m}.$ \;Let $w=w'+a\:\!e_m.$\parSol{}
Becs each $\Ang{e_m,\:\!v_k}=0\Rightarrow\Ang{w,\:\!v_k}=\Ang{w',\:\!v_k}>0$ for $k\in\Bra{1,\dots,m-1}.$\parSol{}
Note that $\Ang{e_m,\:\!v_m}\neq0.$ Hence $\exists\,a\in\Fbb,\:\Ang{w,\:\!v_m}=\Ang{w'+a\:\!e_m,\:\!v_m}=\Ang{w',\:\!v_m}+a\:\!\Ang{e_m,\:\!v_m}>0.$\PfEnd\vspace{4pt}\parSol{}
\Or We show $\exists\,w\in V$ suth each $\Ang{w,\:\!v_j}=\Ang{v_j,\:\!w}=1.$ Let $U=\Span{v_1,\dots,v_m}.$\parSol{}
Define $\varphi\in U\apostrophe$ by $\varphi\Par{v_j}=1.$ Becs $\exists\,!\,w\in U,$ each $\varphi\Par{v_j}=\Ang{v_j,\:\!w}.$\PfEnd
\SepLine

\Anchor{6B4e19}\ProblemBnoor{4E 19}{
	\TextA{Supp $B_V=\Par{v_1,\dots,v_n}.$ Prove $\exists\,B_V'=\Par{u_1,\dots,u_n}$ suth $\Ang{v_j,u_k}=\delta_{j,k}.$}
}Let $\Par{\varphi_1,\dots,\varphi_n}$ be the corres dual bss of $B_V.$ By [6.42], $\exists\,!\,u_k\in V,\varphi_k\Par{v}=\Ang{v,u_k}$ for all $v\in V.$\parSol{}
Then $\varphi_k\Par{v_j}=\delta_{j,k}=\Ang{v_j,u_k}.$ Supp $a_1u_1+\dots+a_nu_n=0\Rightarrow$ each $\Ang{v_j,0}=0=a_j.$\PfEnd
\SepLine

\Anchor{6B16}\ProblemN{16}{
	\TextA{Supp $\Fbb=\Cbb,V$ finide, non0 $T\in\Lm{V},$ all eigvals have absolute vals less than $1.$\vspace{1pt}}
	\TextA{Let $\epsilon>0.$ Prove $\exists\,m\in\Nbp,\:\Dvert{T^mv}\leqslant\epsilon\Dvert{v}$ \,{\tgsc for all v \ensuremath{\boldsymbol{\in}} V}.}\vspace{1pt}
}Let $\Ang{\cdot,\cdot}{_V}$ be the inner prod on $V,$ and $\Dvert{{\cdot}}{_V}$ be the corres norm on $V.$\parSol{}
Using Euclid inner prod $\Ang{\cdot,\cdot}$ and the corres norm $\Dvert{{\cdot}}$ on $\Cbb^{n,1}$ id with $\Cbb^n.$\parSol{}
Supp $A=\Mt{T}$ up-trig wrto orthon $B_V=\Par{e_1,\dots,e_n}.$\parSol{}
Then $\forall v=x_1e_1+\dots+x_ne_n\in V,\Dvert{v}{_V}=\Dvert{x}.$ Now we show $\Dvert{A^mx}\leqslant\epsilon\Dvert{x}$ for all $x\in\Cbb^{n,1}.$\parSol{}
Define $D,N\in\Cbb^{n,n}$ by $D_{j,k}=\delta_{j,k}A_{j,k},\:N=A-D.$ Then $N$ is nilp with $N^p=0\neq N^{p-1}.$\parSol{}
Let $\rho=\max\!\Bra{\aMid{D_{1,1}},\dots,\aMid{D_{n,n}}}\Rightarrow0\leqslant\rho<1,$ and each $\Dvert{D^{\,k}x}\leqslant\rho^{\,k}\Dvert{x}\leqslant\Dvert{x}.$\vspace{1pt}\parSol{}
Let $M=\sum_{j=1}^n\sum_{k=1}^n\aMidsq{N_{j,k}}.$ We show $\Dvert{N^mx}\leqslant M^m\Dvert{x}$ by induc. (i) $m=1.$ By \Sbra{6.A \TIPSN{2}}.\vspace{1pt}\parSol{}
(ii) $m>1.$ Asum it holds for $\Par{m-1}.$ Then $\Dvert{NN^{m-1}x}\leqslant M\cdot\Dvert{N^{m-1}x}\leqslant M^m\Dvert{x}.$\vspace{3pt}\parSol{}
Hence $\Dvert{A^{p+q}x}=\BigDvert{b_0D^{\,p+q}x+\dots+b_kD^{\,p+q-k}N^{k}x+\dots+b_{p-1}D^{\,q+1}N^{p-1}x}$\vspace{3pt}\parSol{}
\Blind{So that $\Dvert{A^{p+q}x}$}$\leqslant\BigSbra{{b_0\rho^{\,p+q}+\dots+b_k\rho^{\,p+q-k}M^k+b_{p-1}\rho^{\,q+1}M^{p-1}}}\Dvert{x}.$\vspace{2pt}\parSol{}
Where each $b_j=\mathC_{p+q}^j\leqslant\Par{p+q}{^j}$ for $j\in\Bra{0,\dots,p-1}\Rightarrow$ each $b_j\leqslant\Par{p+q}{^{p-1}}.$\vspace{1pt}\parSol{}
Let $\sigma=\max\!\Bra{1,M,\dots,M^{p-1}}.$ 又 $\max\!\Bra{\rho^{\,p+q},\dots,\rho^{\,q+1}}=\rho^{\,q+1}.$\parSol{}
Now $\Dvert{A^{p+q}x}\leqslant\Par{p+q}{^{p-1}}\rho^{\,q+1}\sigma\Dvert{x}.$ \;Note that as $q\rightarrow\infty,$ \,$\Par{p+q}{^{p-1}}\rho^{\,q+1}\rightarrow0.$\PfEnd
\SepLine
\ChEnd

\pagebreak

\ChDecl{Ch6C}{6.C}{}

\vspace{4pt}

\Anchor{6CT1}\ProblemBX{\TipsN{1}}{
	\TextA{Supp $V$ finide, $T\in\Lm{V},$ and all vecs in $\null T$ ortho to all vecs in $\range T.$}
	\TextA{Prove $\Par{\null T}{^\perp}=\range T.$}
}Note that $\range T\subseteq\Par{\null T}{^\perp}=\Bra{v\in V:v\text{ ortho to all vecs in }\null T}.$\parSol{}
Then becs $\null T\cap\range T=\zeroSubs\Rightarrow V=\null T\oplus\range T.$ Now by \Sbra{1.C \TIPSN{2}}.\parSol{}
\Or $\forall v\in\Par{\null T}{^\perp},\exists\,!\,\Par{u,w}\in\null T\times\range T,\Ang{u+w,u}=\Ang{u,u}=0\Rightarrow v\in\range T.$\PfEnd
\SepLine

\Anchor{6C8}\ProblemN{8}{
	\TextA{Supp $V$ is finide, $P^2=P\in\Lm{V},$ $\Dvert{Pv}\leqslant\Dvert{v}$ for all $v\in V.$ Prove $\range P=\Par{\null P}{^\perp}.$}
}$\Dvert{w}=\Dvert{Pv}\leqslant\BigDvert{Pv+\Par{v-Pv}}=\Dvert{w+u},$ where $w=Pv,u\in\null P.$ \;Supp non0 $u\in\null P.$\parSol{}
$\forall a\in\Fbb,\;\Dvert{w}\leqslant\Dvert{w+au},$ \Or $\Dvert{Pv}=\Dvert{P\Par{Pv+au}}\leqslant\Dvert{Pv+au}.$ \;Thus $\Ang{Pv,v-Pv}=0.$\PfEnd
\SepLine

\Anchor{6C10}\ProblemN{10}{
	\TextA{Supp $V$ finide, $U$ a subsp, $T\in\Lm{V},$ and $P_UT=TP_U.$ Prove $U$ and $U^\perp$ invard $T.$}
}(a) $P_UTP_U=TP_UP_U=TP_U.$ \,(b) $P_{U^\perp}TP_{U^\perp}=\Par{I-P_U}T\Par{I-P_U}=T\Par{I-P_U}{^2}=TP_{U^\perp}.$\vspace{2pt}\PfEnd\parSol{}
\Or (a) $\range T\mmid_U=\range TP_U=\range P_UT\subseteq U.$\parSol{}
\Blind{\Or}(b) $\range T\mmid_{U^\perp}=\range T\Par{I-P_U}=\range\Par{I-P_U}T\subseteq U^\perp.$\PfEnd\parSol{}
\Blind{\Or}\AComm The trick $T=\Par{P_U\mmid_{\range T}}{^{-1}}TP_U$ is invalid.
\SepLine

\Anchor{6CT2}\BulletPointX\TipsN{2}\,\,\,Supp $U$ finide subsp of $V,v\in V,\varphi\in U\apostrophe:u\mapsto\Ang{u,v}.$\TextB{}
{\IndentTipsN{2}}Then $\exists\,!\,w\in U,\varphi\Par{u}=\Ang{u,w}=\Ang{u,v}$ for all $u\in U\Rightarrow v-w\in U^\perp.$ Now $w=P_Uv.$
\SepLine

%\Anchor{6CT3}\BulletPointX\TipsN{3}\,\,\,Supp $e_1,\dots,e_n\in V$ with each $\Dvert{e_k}=1,$ and for all $v\in V,\:\Dvert{v}{^2}=\innerAsq{v,\,e_1}+\dots+\innerAsq{v,\,e_n}.$\TextB{}
%{\IndentTipsN{3}}Then each $\Dvert{e_j}{^2}=\sum_{j\neq k}^n\innerAsq{e_j,\,e_k}+\innerAsq{e_j,\,e_j}\Rightarrow$ each $\innerAsq{e_j,e_k}=0.$\TextB{}
%{\IndentTipsN{3}}And by (6.B.2), $V=\Span{e_1,\dots,e_n}.$ \;\Or Becs $\forall v\in\Span{e_1,\dots,e_n}{^\perp},\Dvert{v}{^2}=0.$
%\SepLine

\ChEnd
\vspace{8pt}

\vfill\ChDecl{Ch7A}{7.A}{\quad{\ANote {\FontSmall $V$ denotes a finide vecsp over $\Fbb.$}}}%An Exe marked by $\blacksquare$ is true if infinide or partially finide.

\vspace{6pt}

%\Anchor{7A3}\ProblemN{3}{
%	\TextA{Supp $T\in\Lm{V}$ and $U$ is a subsp of $V.$ Prove $U$ invard $T\Longleftrightarrow U^\perp$ invard $T^*.$}
%}If $U$ invard. Then $\forall u\in U,\forall w\in U^\perp,\:\Ang{Tu,w}=0=\Ang{u,T^*w}\Rightarrow T^*w\in U^\perp.$ Rev the roles.\PfEnd
%\SepLine

%\Anchor{7A4e19}\ProblemBnoor{4E 19}{
%	\TextA{Supp $T\in\Lm{V}$ and $\Dvert{T^*v}\leqslant\Dvert{Tv}$ for all $v\in V.$ Prove $T$ is normal.}
%}Let orthon $B_V=\Par{e_1,\dots,e_n}.$ By Exe (4E 5), $\sum_{i=1}^n\Dvertsq{T^*e_i}=\sum_{i=1}^n\Dvertsq{Te_i}.$\parSol{}
%Becs each $\Dvertsq{T^*e_i}\leqslant\Dvertsq{Te_i}.$ Note that this orhon $B_V$ is arb.\PfEnd
%\SepLine

%\Anchor{7A11}\Anchor{7A4e20}\ProblemB{
%	\TextB{Supp $P^2=P=P_U\in\Lm{V}$ for some subsp $U$ of $V.$ Prove $P$ is self-adj.}
%}$\forall\Par{u,w},\Par{x,y}\in U\times U^\perp,\:\BigAng{P\Par{u+w},x+y}=\Ang{u+w,x}=\BigAng{u+w,P^*\Par{x+y}}.$\PfEnd
%\SepLine

\Anchor{7A4e28}\ProblemBnoor{4E 28}{
	\TextA{Supp $T\in\Lm{V}$ is normal. Prove the min of $T$ is not a multi of any $\Par{z-\lambda}{^2}.$}
}Note that each $\Par{T-\lambda I}$ is normal $\Rightarrow$ each $\Null\Par{T-\lambda I}{^k}=\Null\Par{T-\lambda I}.$ By (8.B.5).\PfEnd\vspace{3pt}\parSol{}
\Or By Exe (16), $V=\Null\Par{T-\lambda I}\oplus\range\Par{T-\lambda I}$ for all $\lambda\in\Fbb.$\parSol{}
By (5.C.5). If $\Fbb=\Rbb.$ Then by [4.17] for the min of $T_{\!\Cbb}.$ \;\Or By Exe (17), \Sbra{8.B \TIPSN{4}}.\PfEnd\vspace{3pt}\parSol{}
We prove the ctrapos. Supp the min of $T$ is $p\Par{z}=\Par{z-\lambda}{^2}q\Par{z}.$ Let $s\Par{z}=\Par{z-\lambda}q\Par{z}.$\parSol{}
Then $\exists\,v\in V,\,s\Par{T}v\neq0\Rightarrow q\Par{T}v\in\Null\Par{T-\lambda I}{^2}\Backslash\Null\Par{T-\lambda I}.$\PfEnd
\SepLine

%\Anchor{8C16}\Anchor{7A4e24}\ProblemBnoor{4E 24 {\OR} 8.C.16}{
%	\TextA{Supp $T\in\Lm{V}$ with the min $p.$ Prove the min $s$ of $T$ is $\overline{p}.$}
%}$\forall q\in\PoFi,\,q\Par{T}=0\Longleftrightarrow q\Par{T}{^*}=\overline{q}\Par{T^*}=0.$ \,{\OR} $\overline{q}\Par{T}=0\Longleftrightarrow \overline{q}\Par{T}{^*}=q\Par{T^*}=0.$\parSol{}
%Thus $\overline{p}$ is a multi of $s,$ and $\overline{s}$ is a multi of $p.$\PfEnd
%\SepLine

\Anchor{7A17}\ProblemN{17}{
	\TextA{Supp $T\in\Lm{V}$ is normal. Prove each $\null T^k=\null T$ \,and\, $\range T^k=\range T.$}
}Becs $\range T=\Par{\null T^*}{^\perp}=\Par{\null T}{^\perp}\Rightarrow T\mmid_{\range T}$ is inje. Thus $\null T^k=\null T.$\parSol{}
And $\range T^2=\range T\mmid_{\range T}=\range T=\range T\Rightarrow\range T^{k-1}\mmid_{\range T}=\range T^{k-1}\mmid_{\range T^2}.$\PfEnd\vspace{4pt}\parSol{}
\Or $v\in\null T^{k+1}\Rightarrow T^kv\in\null T=\null T^*\Rightarrow0=\Ang{T^*T^kv,T^{k-1}v}=\Ang{T^kv,T^kv}\Rightarrow v\in\null T^k.$\parSol{}
Note that $T$ normal $\Rightarrow T^k$ normal. Then $\range T^k=\Par{\null T^k}{^\perp}=\Par{\null T}{^\perp}=\range T.$\PfEnd
\SepLine

%\Anchor{7A4e2}\ProblemBnoor{4E 2}{
%	\TextA{Supp $T\in\Lm{V,W}.$ Prove $T=0\Longleftrightarrow T^*=0\Longleftrightarrow T^*T=0\Longleftrightarrow TT^*=0.$}
%}$T=0\Rightarrow\range T^*=\zeroSubs,\null T^*=W\Rightarrow T^*=0.$ Convly rev the roles.\parSol{}
%$T^*T=0\Rightarrow T^*\mmid_{\range T}=T^*\mmid_{\SmallPar{\null T^*}{^\perp}}=0\Rightarrow T^*=0.$ Simlr for $TT^*=0\Rightarrow T=0.$\PfEnd\vspace{3pt}\parSol{}
%\Or $T=0\Rightarrow\forall w\in W,\forall v\in V,\Ang{v,T^*w}=\Ang{Tv,w}=0\Rightarrow\range T^*\subseteq V^\perp.$\parSol{}
%$T^*T=0\Rightarrow\forall v\in V,\Ang{Tv,Tv}=\Ang{v,T^*Tv}=0\Rightarrow T=0.$\PfEnd
%\SepLine
\ChEnd
\pagebreak

\ChDecl{Ch7B}{7.B}{}

\vspace{4pt}

\ProblemN{14}{
	\TextA{Supp $\Fbb=\Rbb,$ $T\in\Lm{V}.$ Prove $T$ diag $\Rightarrow$ self-adj wrto some $\Ang{\cdot,\cdot}{_V}.$}
}Let $B_V=\Par{e_1,\dots,e_n}$ of eigvecs. \,Define $\Ang{e_j,\:\!e_k}{_V}=\delta_{j,k}.$ \;So that $B_V$ orthon. \;\Or By ctrapos.\PfEnd\vspace{5pt}
\Anchor{7B4e24}\ANote (a) $\Mt{T}=\overline{\Mt{T}}{^t}$ wrto some $B_V\Longleftrightarrow T$ self-adj wrto some $\Ang{\cdot,\cdot}{_V}$ $\Longleftrightarrow$ diag.\parNot
(b) $\Mt{T}\overline{\Mt{T}}{^t}=\overline{\Mt{T}}{^t}\Mt{T}$ wrto some $B_V\Longleftrightarrow T$ normal wrto some $\Ang{\cdot,\cdot}{_V}$ $\Longleftrightarrow$ diag on $\Cbb.$
\SepLine

\ProblemN{7}{
	\TextA{Supp $\Fbb=\Cbb,T\in\Lm{V}$ is normal and $T^9=T^8.$ Prove $T^2=T$ is self-adj.}
}Becs $T^8\Par{T-I}=0\Rightarrow0,1\in\Rbb$ are all possible eigvals.\PfEnd\parSol{}
\Or $\range T^*=\range T=\Par{\null T}{^\perp}\Rightarrow T\Par{T-I}=0.$ 又 $Tv=Tv+\Par{v-Tv}\Rightarrow T=P_{\range T}.$\PfEnd\parSol{}
\Or $\exists$ orthon $B_V=\Par{e_1,\dots,e_n}$ of eigvecs with corres $\lambda_1,\dots,\lambda_n.$\parSol{}
Now $\lambda_j^8e_j=T^8e_j=T^9e_j=\lambda_j^9e_j\Rightarrow\lambda_j=0$ or $1\in\Rbb.$ 又 Each $T^2e_j=\lambda_j^2e_j=\lambda_je_j=Te_j.$\PfEnd
\SepLine

\ProblemBnoor{4E 8}{
	\TextA{Supp $\Fbb=\Cbb,T\in\Lm{V}.$ Prove each eigvec of $T$ is an eigvec of $T^*\Rightarrow T$ is normal.}
}Supp $v$ is eigvec of $T$ corres $\lambda$ and of $T^*$ corres $\mu.$\parSol{}
Then $\lambda\Dvertsq{v}=\Ang{Tv,v}=\Ang{v,T^*v}=\overline{\mu}\Dvertsq{v}\Rightarrow\lambda=\overline{\mu}.$\parSol{}
Thus each $E\Par{\lambda,T}=E\Par{\overline{\lambda},T^*}$ invard $T,T^*,$ so is $E\Par{\lambda,T}{^\perp}=E\Par{\overline{\lambda},T^*}.$\parSol{}
Let $W=\bigcap_{\lambda\,\in\,\Fbb}E\Par{\lambda,T}{^\perp}$ invard $T,T^*.$ No eigvals of $T\mmid_W,T^*\mmid_W\Rightarrow W=\zeroSubs.$ By (3.F.22).\PfEnd
\SepLine

\ProblemBnoor{4E 9}{
	\TextA{Supp $\Fbb=\Cbb,T\in\Lm{V}$ is normal. Prove $T^*=p\Par{T}.$}
}
\SepLine

\ProblemBnoor{4E 12}{
	\TextA{}
}
\SepLine

\ProblemBnoor{4E 20}{
	\TextA{Supp $T\in\Lm{V}$ is normal and $U$ invarspd $T.$}
	\TextA{Prove {\tgnr\large(a)} $U^\perp$ invard $T,$ \;{\tgnr\large(b)} $\Par{T\mmid_U}{^*}=T^*\mmid_U\in\Lm{U},$ \;{\tgnr\large(c)} $T\mmid_U,T\mmid_{U^\perp}$ normal.}
}Let $\lambda_1,\dots,\lambda_m$ be disti eigvals. By \Sbra{5.A \TIPSN{3}}, $U=E\Par{\lambda_1,T\mmid_U}\oplus\cdots\oplus E\Par{\lambda_m,T\mmid_U}.$\parSol{}
Let orthon $B_V=\Par{e_1,\dots,e_n}$ of eigvecs suth $B_U=\Par{e_1,\dots,e_m}\Rightarrow B_{U^\perp}=\Par{e_{m+1},\dots,e_n}.$\parSol{}
(a) Now $U^\perp$ invard $T.$ Apply to $T^*\Rightarrow U$ invard $T^*.$\parSol{}
(b) $\forall u,v\in U,\BigAng{v,\Par{T\mmid_U}{^*}u}=\Ang{T\mmid_U v,u}=\Ang{v,T*\mmid_U u}\Rightarrow\BigPar{\Par{T\mmid_U}{^*}-T^*\mmid_U}u\in U\cap U^\perp.$\parSol{}
(c) $\forall u\in U,\Dvert{T\mmid_U u}=\Dvert{T^*\mmid_U u}=\BigDvert{\Par{T\mmid_U}{^*}u}.$ Apply to invarsp $U^\perp.$\parSol{\Hc}
\Or $T\mmid_U\Par{T\mmid_U}{^*}=TT^*\mmid_U=T^*T\mmid_U=T^*\mmid_UT\mmid_U=\Par{T\mmid_U}{^*}T\mmid_U.$\PfEnd\vspace{2pt}
\ANote Another proof of [7.24]: Induc step: For $\dim V>1.$ Asum it holds for smaller dim.\parNot
Let $u$ be an eigvec with $\Dvert{u}=1.$ Let $B_U=\Par{u}\Rightarrow U$ invard $T,$ so is $U^\perp\Rightarrow T\mmid_{U^\perp}$ normal.\parNot
By asum, $\exists$ orthon $B_{U^\perp}$ of eigvecs of $T\mmid_{U^\perp}.$ Now $B_V=B_U\cap B_{U^\perp}$ of orthon eigvecs.\PfEnd
\SepLine
\ChEnd