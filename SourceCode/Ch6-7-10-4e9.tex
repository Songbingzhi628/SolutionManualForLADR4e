% Copyright (C) 2024 Songbingzhi628. This work is licensed under Creative Commons Attribution-NonCommercial-ShareAlike 4.0 International License.
% Email: 13012057210@163.com

\ChDecl{Ch6A}{6.A}{}

\vspace{4pt}

\ProblemB[]{
	(a) $\Dvertsq{u+v}=\Dvertsq{u}+\Dvertsq{v}+2\Real\Ang{u,v}.$ \; $\Dvertsq{u+\i v}=\Dvertsq{u}+\Dvertsq{v}+2\Imaginary\Ang{u,v}.$\TextB{}
	(b) $\aXMid{\Dvert{u}-\Dvert{v}}\leqslant\Dvert{u-v}.$ \,Equa $\Longleftrightarrow u=cv,c>0,$ where $u,v\neq0.$\TextB{}
	(c) $\aXMid{\Dvert{v}-1}=\BigDvert{v-v\big/\Dvert{v}}\leqslant\Dvert{v-u}$ if $\Dvert{u}=1.$ \,Equa $\Longleftrightarrow u=v\big/\Dvert{v}.$\TextB{}
	(d) $\aXMid{\Dvertsq{u}-\Dvertsq{v}}=\innerA{u+v,u-v}\leqslant\Dvert{u+v}\,\Dvert{u-v}\leqslant\Dvertsq{u}+\Dvertsq{v}={}${\Large$\frac{\:1\:}{2}$}$\Sbra{\Dvertsq{u+v}+\Dvertsq{u-v}}.$\vspace{-3pt}\TextB{}
}\SepLine

\Anchor{6A21}\ProblemN{21}{
	\TextA{Implement the corres inner prod from a norm $\Dvert{\cdot}:U\rightarrow\Interval{[}{)}{0,\infty}$ satisfying [6.22].}
	\PrePa $\Dvert{u}=0\Longleftrightarrow u=0.$ \;(b) $\Dvert{au}=\aMid{a}\Dvert{u}.$ \;(c) $\Dvert{u+w}\leqslant\Dvert{u}+\Dvert{w}.$\TextA{}
}If $\Fbb=\Rbb.$ Define $\Ang{u,w}={}${\Large$\frac{\:1\:}{4}$}$\BigPar{\Dvertsq{u+w}-\Dvertsq{u-w}}.$\vspace{2pt}\parSol{}
Add in 1st slot: $\Ang{u+v,w}=$
\SepLine

\Anchor{6A3}\ProblemN{3}{
	\TextA{Supp $\Fbb=\Rbb,V\neq\zeroSubs.$ Replace the positivity cond in [6.3] with $\exists\,v\in V,\,\Ang{v,v}>0.$}
	\TextA{Show this does not change the inner prods from $V\times V$ to $\Rbb.$}
}Supp $w\in V$ with $\Ang{w,w}>0.$ Asum $\exists\,u\in V$ with $\Ang{u,u}<0.$\parSol{}
Define $p\Par{x}=\Ang{u+xw,u+xw}=\Ang{w,w}\,x^2+2\Ang{u,w}\,x+\Ang{u,u}\Rightarrow$ two disti zeros.\parSol{}
Supp $\Ang{u+\lambda w,u+\lambda w}=0\Rightarrow u+\lambda w=0\Rightarrow\Ang{u,u}=\lambda^2\Ang{-w,-w}\geqslant 0,$ ctradic.\PfEnd
\SepLine

\Anchor{6A6}\ProblemN{6}{
	\TextA{Supp $u,v\in V.$ Prove $\Dvert{u}\leqslant\Dvert{u+av}$ for all $a\in\Fbb\Rightarrow\Ang{u,v}=0.$}
}Becs $\Dvertsq{u}\leqslant\Dvertsq{u+av}.$ Let $\Ang{u-cv,cv}=0\Rightarrow\Dvertsq{u-cv}=\Ang{u,u-cv}=\Dvertsq{u}-\overline{c}\Ang{u,v}.$\parSol{}
Thus $\Dvertsq{u}\leqslant\Dvertsq{u-cv}=\Dvertsq{u}-\innerA{u,v}{\;\!\XSlash\;\!}\Dvertsq{v}.$\PfEnd\vspace{4pt}\parSol{}
\Or $\Dvertsq{u}\leqslant\Dvertsq{u}+\aMidsq{a}\Dvertsq{v}+2\Real\overline{a}\Ang{u,v}\Rightarrow-2\Real\overline{a}\Ang{u,v}\leqslant\aMidsq{a}\Dvertsq{v}$ for all $a\in\Fbb.$\parSol{}
Let $a=-\Ang{u,v}\Rightarrow2\innerA{u,v}{^2}\leqslant\innerA{u,v}{^2}\Dvertsq{v}.$ If $\Ang{u,v}\neq0,$ then $2\leqslant\Dvertsq{v};$ might not be true.\PfEnd
\SepLine

\Anchor{6A'1}\ProblemB{
	\TextB{Supp $u,v\in V$ and for all $x,y\in\Fbb,\:\Dvertsq{xu+yv}=\aMidsq{x}\:\!\Dvertsq{u}+\aMidsq{y}\:\!\Dvertsq{v}.$ Prove $\Ang{u,v}=0.$}
}By Exe (6), $\Dvertsq{u}\leqslant\Dvertsq{u}+\aMidsq{a}\:\!\Dvertsq{v}=\Dvertsq{u+a\:\!v}$ for all $a\in\Fbb.$\PfEnd
\SepLine

\Anchor{6A4e19}\ProblemBnoor{4E 19}{
	\TextA{Supp $T\in\Lm{V},\,\lambda$ is eigval. Prove $\aMidsq{\lambda}\leqslant\sum_{j=1}^n\sum_{k=1}^n\aMidsq{A_{j,k}},$ \FontNorm\tgnr where $A=\Mt[\BigPar]{T,\Par{v_1,\dots,v_n}}.$\vspace{3pt}}
}Let $\Ang{a_1v_1+\dots+a_nv_n,b_1v_1+\dots+b_nv_n}=a_1\overline{b_1}+\dots+a_n\overline{b_n}.$ Let $v$ be eigval corres $\lambda$ with $\Dvert{v}=1.$\vspace{1pt}\parSol{}
Becs $v=a_1v_1+\dots+a_nv_n\Rightarrow Tv=\sum_{k=1}^na_k\sum_{j=1}^nA_{j,k}v_j=\sum_{j=1}^n\BigSbra{\sum_{k=1}^na_kA_{j,k}}v_j.$\vspace{3pt}\parSol{}
又 Each $\Dvert{v_j}=1,\:\Ang{v_j,v_k}=0\Rightarrow\Dvertsq{Tv}=\sum_{j=1}^n\Big|\sum_{k=1}^na_kA_{j,k}\Big|{^2}=\aMidsq{\lambda}.$\vspace{3pt}\parSol{}
Note that $\aXMidsq{a_1A_{j,1}+\dots+a_nA_{j,n}}\leqslant\BigDvertsq{\Par{a_1,\dots,a_n}}\cdot\BigDvertsq{\Par{A_{j,1},\dots,A_{j,n}}}=\sum_{k=1}^n\aMidsq{A_{j,k}}.$\PfEnd
\SepLine

\Anchor{6A4e23}\ProblemBnoor{4E 23}{
	\TextA{Supp $v_1,\dots,v_m\in V,$ each $\Dvert{v_k}\leqslant 1.$ \,Show $\exists\,a_k=\pm1,\;\Dvert{a_1v_1+\dots+a_mv_m}\leqslant\sqrt{m}.$}
}We use induc on $m.$ (i) $m=1.$ Immed. (ii) $m>1.$ Asum it holds for smaller $m.$\parSol{}
Let $u=a_1v_1,\,w=a_2v_2+\dots+a_mv_m\Rightarrow\Dvertsq{u}\leqslant1,\Dvertsq{w}\leqslant{m-1}.$\parSol{}
Then $\Dvertsq{u+w}+\Dvertsq{u-w}\leqslant2m.$ \;\;\Or $\Dvert{u+w}\cdot\Dvert{u-w}\leqslant m.$\PfEnd
\SepLine

\ChEnd\pagebreak

\ChDecl{Ch6B}{6.B}{}

\vspace{4pt}

\ProblemB[]{
	For orthog $\Par{e_1,\dots,e_m}$ and $v=a_1e_1+\dots+a_me_m,$ becs $\Ang{v,e_k}=a_k\Dvertsq{e_k},$ \;$v={}${\Large$\frac{\:\SmallAng{v,\:e_1}\:}{\;\SmallDvertsq{e_1}}$}$\,e_1+\dots+{}${\Large$\frac{\:\SmallAng{v,\:e_m}\:}{\;\SmallDvertsq{e_m}}$}$\,e_m.$\TextB{}
	Now $\Dvertsq{v}={}${\Large$\frac{\:\SmallinnerAsq{v,\:e_1}\:}{\SmallDvertsq{e_1}}$}${}+\dots+{}${\Large$\frac{\:\SmallinnerAsq{v,\:e_m}\:}{\SmallDvertsq{e_m}}$}$.$ Replace each $e_k$ with $\Dvert{e_k}{^{\,-1}}e_k,$ now $\Par{e_1,\dots,e_m}$ orthon.\TextB{\vspace{6pt}}
	Exe (2) holds only for orthon lists.\TextB{\vspace{-4pt}}
}\SepLine


\Anchor{6B'1}\ProblemB{
	\TextA{Supp $\Par{e_1,\dots,e_m}$ orthog, $v\in V.$ Show $\sum_{k=1}^m\!\BigPar{2-\Dvertsq{e_k}}\,\innerAsq{v,e_k}\leqslant\Dvertsq{v}.$\vspace{2pt}}
}Let $u=\Ang{v,e_1}\,e_1+\dots+\Ang{v,e_m}\,e_m\Rightarrow\Dvertsq{u}=\sum_{k=1}^n\innerAsq[{\:\!\Dvert{e_k}\,}]{v,e_k},\,\Ang{u,v}=\sum_{k=1}^n\innerAsq{v,e_k}.$\vspace{2pt}\parSol{}
Let $\Dvertsq{v-u}=\Dvertsq{v}+\Dvertsq{u}-\Ang{v,u}-\Ang{u,v}=\Dvertsq{v}+\sum_{k=1}^n\!\BigPar{\Dvertsq{e_k}-2}\,\innerAsq{v,e_k}\geqslant0.$\PfEnd\vspace{4pt}
\Anchor{6B2}\ACoro If orthon, $\Ang{u,\,v-u}=0\Rightarrow\Dvertsq{v}=\Dvertsq{u}+\Dvertsq{v-u}.$\parCor
Bessel's Inequa: $\sum_{k=1}^m\innerAsq{v,e_k}\leqslant\Dvertsq{v}.$ \:\Sbra{Exe (2)} Equa $\Longleftrightarrow v\in\Span{e_1,\dots,e_m}.$
\SepLine

\Anchor{6B14}\ProblemN{14}{
	\TextA{Supp $\Par{e_1,\dots,e_n}$ is orthon bss of $V$ and each $v_j\in V$ suth $\Dvert{e_j-v_j}<{}${\Large$\frac{\;1}{\sqrt{n}\;}$}.}
	\TextA{Prove $B_V=\Par{v_1,\dots,v_n}.$}
}
\SepLine

\Anchor{6B4e9}\ProblemBnoor{4E 9}{
	\TextA{Supp $\Par{e_1,\dots,e_m}$ is the result of applying [6.31]}
	\TextA{to a liney indep $\Par{v_1,\dots,v_m}$ in $V.$ \,Show each $\Ang{v_j,e_j}>0.$}
}Let $f_j=v_j-\Ang{v_j,\,e_1}\,e_1-\dots-\Ang{v_j,\,e_{j-1}}\,e_{j-1}.$\vspace{1pt}\parSol{}
Becs $\;\Dvert{f_j}\,\Ang{v_j,\,e_j}=\Ang{v_j,\,f_j}=\Dvertsq{v_j}-\innerAsq{v_j,\,e_1}-\dots-\innerAsq{v_j,\,e_{j-1}}\geqslant0,$ by Bessel's Inequa.\vspace{1pt}\parSol{}
If $\Ang{v_j,\,f_j}=0,$ then by Exe (2), $v_j\in\Span{e_1,\dots,e_{j-1}}=\Span{v_1,\dots,v_{j-1}}.$ Now $\Dvert{f_j}>0.$\PfEnd\vspace{2pt}
\Anchor{6B9}\Anchor{6B4e13}\ANote Supp $\Par{v_1,\dots,v_m}$ liney dep. Let $j$ be the largest suth $\Par{v_1,\dots,v_{j-1}}$ liney indep.\parNot
Apply [6.31]. Now $v_j\in\Span{v_1,\dots,v_{j-1}}=\Span{e_1,\dots,e_{j-1}}\Rightarrow f_j=0.$
\SepLine

\Anchor{6B4e10}\ProblemBnoor{4E 10}{
	\TextB{Supp $\Par{v_1,\dots,v_m}$ liney indep. Explain why the
	orthon list produced by [6.31]}
	\TextB{is the only orthon $\Par{e_1,\dots,e_m}$ suth each $\Ang{v_k,\,e_k}>0$ and $\Span{v_1,\dots,v_k}=\Span{e_1,\dots,e_k}.$}
}Fix one $k.$ Let $v_k=a_1e_1+\dots+a_ke_k\Rightarrow$ each $a_j=\Ang{v_k,e_j}.$ Let $f_k=v_k-a_1e_1-\dots-a_{k-1}e_{k-1}.$\parSol{}
Then $e_k=f_k\Big/\!a_k\Rightarrow\Ang{f_k,\,f_k}\big/a_k^2=1\Rightarrow a_k=\pm\Dvert{f_k}.$ 又 $a_k>0.$ \:Hence each $e_k=f_k\Big/\Dvert{f_k}.$\PfEnd
\SepLine

\Anchor{6B10}\ProblemN{10}{
	\TextA{Supp $\Fbb=\Rbb,$ $\Par{v_1,\dots,v_m}$ is liney indep in $V.$}
	\TextA{Prove $\exists$ exactly $2^m$ orthon $\Par{e_1,\dots,e_m}$ suth $\Span{e_1,\dots,e_m}=\Span{v_1,\dots,v_m}.$}
}Using induc on $m.$ (i) $m=1.$ Let $e_1=\pm v_1\big/\Dvert{v_1}.$ (ii) $m>1.$ Asum it holds for $\Par{m-1}.$\parSol{}
Get $2^{m-1}$ orthon corres $\Par{v_1,\dots,v_{m-1}}.$ Fix one $\Par{e_1,\dots,e_{m-1}}$ and apply [6.31] for $v_m$ to get $e_m.$\parSol{}
Supp $\Par{e_1,\dots,e_{m-1},e_m'}$ is also orthon. \NOTICE that $e_m'=\Ang{e_m',e_m}e_m.$ So $\innerA{e_m',e_m}=1.$\vspace{1pt}\parSol{}
Let $e_m'=-e_m.$ Sum it up, we have $2^{m-1}\times 2=2^m$ orthon lists.\PfEnd
\SepLine

\Anchor{6B11}\ProblemN{11}{
	\TextA{Supp $V\neq0,$ and $\Ang{\cdot,\cdot}{_1},\Ang{\cdot,\cdot}{_2}$ are inner prods suth $\Ang{v,w}{_1}=0\Longleftrightarrow\Ang{v,w}{_2}=0.$}
	\TextA{Prove $\exists\,c>0,\:\Ang{v,w}{_1}=c\Ang{v,w}{_2}$ for all $v,w\in V.$}
}Fix non0 $v_1,v_2\in V.$ Define $\varphi_1,\psi_1\in V\apostrophe$ by $\varphi_1:v\mapsto\Ang{v_1,v}{_1},\;\psi_1:v\mapsto\Ang{v,v_2}{_1}.$ Simlr for $\varphi_2,\psi_2.$\parSol{}
Becs $\Ang{v_1,v}{_1}=0\Longleftrightarrow\Ang{v_1,v}{_2}=0.$ By (3.B.30), let $c_1=\Ang{v_1,v_1}{_1}\Big/\Ang{v_1,v_1}{_2}>0\Rightarrow\varphi_1=c_v\varphi_2.$\parSol{}
Simlr, let $c_2=\Ang{v_2,v_2}{_1}\Big/\Ang{v_2,v_2}{_2}\Rightarrow\psi_1=c_2\psi_2.$ Choose $v_1=v_2$ so that $c_1=c_2.$\PfEnd\vspace{4pt}\parSol{}
\Or Define $c_v=\Ang{v,v}{_1}\Big/\Ang{v,v}{_2}$ for all non0 $v\in V.$ Fix non0 $u,v\in V.$\parSol{}
Let $c=\Ang{u,v}{_2}\Big/\Ang{v,v}{_2}\Rightarrow\Ang{u-cv,v}{_1}=\Ang{u-cv,v}{_2}=0\Rightarrow\Ang{u,v}{_1}=c\Ang{v,v}{_1}=c_v\Ang{u,v}{_2}.$\parSol{}
Rev the roles of $u,v\Rightarrow c_v\Ang{u,v}{_2}={\Ang{u,v}{_1}}=\overline{\Ang{v,u}{_1}}=\overline{c_u\Ang{v,u}{_2}}=c_u\Ang{u,v}{_2}\Rightarrow c_v=c_u.$\PfEnd
\SepLine

\Anchor{6B12}\ProblemN{12}{
	\TextA{Supp $V$ is finide and $\Ang{\cdot,\cdot}{_1},\Ang{\cdot,\cdot}{_2}$ are inner prods with corres norms $\Dvert{\cdot}{_1},\Dvert{\cdot}{_2}.$}
	\TextA{Prove $\exists\,c>0,\:\Dvert{v}\leqslant c\Dvert{v}$ for all $v\in V.$}
}
\SepLine

\Anchor{6B4e17}\ProblemBnoor{4E 17}{
	\TextA{Supp $\Fbb=\Cbb,V$ finide, $T\in\Lm{V},$ $1$ is the only eigval,}
	\TextA{and $\Dvert{Tv}\leqslant\Dvert{v}$ for all $v\in V.$ Show $T=I.$}
}
\SepLine

\Anchor{6B13}\ProblemN{13}{
	\TextA{Supp $\Par{v_1,\dots,v_m}$ is liney indep in $V.$ Show $\exists\,w\in V,$ each $\Ang{w,v_j}>0.$}
}
\SepLine

\Anchor{6B4e18}\ProblemBnoor{4E 18}{
	\TextA{Supp $\Par{v_1,\dots,v_m}$ liney indep. Show $\exists\,w\in V,$ each $\Ang{w,v_k}=1.$}
}
\SepLine

\Anchor{6B4e19}\ProblemBnoor{4E 19}{
	\TextA{Supp $B_V=\Par{v_1,\dots,v_n}.$ Prove $\exists\,B_V'=\Par{u_1,\dots,u_n}$ suth $\Ang{v_j,u_k}=\delta_{j,k}.$}
}Let $\Par{\varphi_1,\dots,\varphi_n}$ be the corres dual bss of $B_V.$ By [6.42], $\exists\,!\,u_k\in V,\varphi_k\Par{v}=\Ang{v,u_k}$ for all $v\in V.$\parSol{}
Then $\varphi_k\Par{v_j}=\delta_{j,k}=\Ang{v_k,u_k}.$ Supp $a_1u_1+\dots+a_nu_n=0\Rightarrow$ each $\Ang{v_j,0}=0=a_j.$\PfEnd
\SepLine

\Anchor{6B16}\ProblemN{16}{
	\TextA{Supp $\Fbb=\Cbb,V$ finide, $T\in\Lm{V},$ all eigvals $\lambda$ have absolute vals less than $1.$}
	\TextA{Let $\varepsilon>0.$ Prove $\exists\,m\in\Nbp,\:\Dvert{T^mv}\leqslant\varepsilon\Dvert{v}$ for all $v\in V.$}
}
\SepLine

\ChEnd\pagebreak

\ChDecl{Ch6C}{6.C}{}

\ChEnd\pagebreak

\ChDecl{Ch7A}{7.A}{}

\ChEnd\pagebreak

\ChDecl{Ch7B}{7.B}{}

\ChEnd